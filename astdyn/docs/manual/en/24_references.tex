\chapter{Best Practices}
\label{ch:best_practices}

\section{Introduction}

This chapter provides practical guidelines for using AstDyn effectively, covering configuration choices, workflow patterns, and common pitfalls to avoid.

\section{Integration Settings}

\subsection{Choosing Tolerance}

Match tolerance to your accuracy requirements:

\begin{table}[h]
\centering
\caption{Recommended integration tolerances}
\begin{tabular}{lcp{6cm}}
\hline
\textbf{Application} & \textbf{Tolerance} & \textbf{Rationale} \\
\hline
Preliminary studies & $10^{-10}$ & Fast, 50 km accuracy over 60 days \\
Standard orbit determination & $10^{-12}$ & Default, 3 km accuracy \\
High-precision ephemerides & $10^{-14}$ & Research-grade, sub-km accuracy \\
Numerical experiments & $10^{-16}$ & Maximum precision, slow \\
\hline
\end{tabular}
\end{table}

\textbf{Rule of thumb}: Tighter tolerance $\Rightarrow$ more steps $\Rightarrow$ slower but more accurate.

\subsection{Step Size Limits}

Configure adaptive step size bounds:

\begin{lstlisting}[language=C++]
auto integrator = std::make_shared<RKF78>(
    1e-12,      // tolerance
    1e-6,       // min step (days) - prevents excessive refinement
    100.0       // max step (days) - limits coarse steps
);
\end{lstlisting}

\textbf{Guidelines}:
\begin{itemize}
    \item \textbf{Min step}: Set to $10^{-6}$ days (0.1 seconds) to prevent near-singular behavior
    \item \textbf{Max step}: Limit to orbital period / 100 for smooth sampling
    \item \textbf{High eccentricity}: Reduce max step near perihelion
\end{itemize}

\subsection{Handling Extreme Eccentricity}

For $e > 0.9$:

\begin{lstlisting}[language=C++]
// Tighter tolerance near perihelion
double tol = (elem.e > 0.9) ? 1e-14 : 1e-12;

// Smaller max step
double max_step = elem.period() / 1000.0;  // 1000 steps per orbit

auto integrator = std::make_shared<RKF78>(tol, 1e-6, max_step);
\end{lstlisting}

\section{Force Model Selection}

\subsection{Main-Belt Asteroids}

Standard configuration:

\begin{lstlisting}[language=C++]
auto forces = std::make_shared<PointMassGravity>(
    ephemeris, 
    std::vector<std::string>{"JUPITER", "SATURN"}
);
\end{lstlisting}

\textbf{Rationale}:
\begin{itemize}
    \item Jupiter: Dominates perturbations ($\sim$99\% of effect)
    \item Saturn: Next largest ($\sim$1\%)
    \item Uranus, Neptune: Negligible ($< 0.01\%$)
    \item Inner planets: Minimal for main-belt
\end{itemize}

\subsection{Near-Earth Asteroids}

Include Earth and Mars:

\begin{lstlisting}[language=C++]
auto forces = std::make_shared<PointMassGravity>(
    ephemeris,
    std::vector<std::string>{"EARTH", "JUPITER", "MARS", "VENUS"}
);
\end{lstlisting}

\textbf{Priority}: Earth $>$ Jupiter $>$ Mars $>$ Venus for typical NEAs.

\subsection{Outer Solar System}

Include all giant planets:

\begin{lstlisting}[language=C++]
auto forces = std::make_shared<PointMassGravity>(
    ephemeris,
    std::vector<std::string>{"JUPITER", "SATURN", "URANUS", "NEPTUNE"}
);
\end{lstlisting}

\subsection{Performance Considerations}

\begin{table}[h]
\centering
\caption{Force model cost vs. benefit}
\begin{tabular}{lccc}
\hline
\textbf{Bodies} & \textbf{Time} & \textbf{Accuracy (60d)} & \textbf{Use When} \\
\hline
Sun only & 1.0$\times$ & 15,000 km & Never (educational) \\
+ Jupiter & 1.4$\times$ & 150 km & Quick estimates \\
+ Saturn & 1.6$\times$ & 3 km & Standard \\
+ 4 giants & 2.0$\times$ & 2 km & High precision \\
+ all planets & 2.6$\times$ & 1 km & Research \\
\hline
\end{tabular}
\end{table}

\textbf{Recommendation}: Sun + Jupiter + Saturn for production work.

\section{Observation Handling}

\subsection{Weighting Strategy}

Use inverse-variance weighting:

\begin{lstlisting}[language=C++]
// Default: uniform 0.5 arcsec uncertainty
for (auto& obs : observations) {
    if (obs.sigma_ra == 0.0) {
        obs.sigma_ra = 0.5 * ARCSEC_TO_RAD;
        obs.sigma_dec = 0.5 * ARCSEC_TO_RAD;
    }
}
\end{lstlisting}

\textbf{Guidelines}:
\begin{itemize}
    \item \textbf{Modern CCD (Pan-STARRS, etc.)}: 0.1-0.3 arcsec
    \item \textbf{Older CCD}: 0.5-1.0 arcsec
    \item \textbf{Photographic plates}: 1-3 arcsec
    \item \textbf{Visual observations}: 5-10 arcsec (avoid if possible)
\end{itemize}

\subsection{Outlier Detection}

Implement 3-sigma rejection:

\begin{lstlisting}[language=C++]
// After initial solution
auto result = diff_corrector.solve(initial_guess, observations, obs_coords);

// Compute 3-sigma threshold
double threshold = 3.0 * result.rms_residual;

// Filter outliers
std::vector<Observation> filtered_obs;
for (size_t i = 0; i < observations.size(); ++i) {
    double residual = result.residuals(i);
    if (std::abs(residual) < threshold) {
        filtered_obs.push_back(observations[i]);
    }
}

// Re-solve with filtered observations
result = diff_corrector.solve(result.elements, filtered_obs, obs_coords);
\end{lstlisting}

\subsection{Arc Length Selection}

\begin{table}[h]
\centering
\caption{Optimal arc length by object type}
\begin{tabular}{lcc}
\hline
\textbf{Object Type} & \textbf{Minimum Arc} & \textbf{Optimal Arc} \\
\hline
NEA (fast motion) & 3 days & 7-14 days \\
Main-belt asteroid & 14 days & 30-60 days \\
Outer main-belt & 30 days & 60-90 days \\
Trojan/Centaur & 60 days & 90-180 days \\
\hline
\end{tabular}
\end{table}

\textbf{Principle}: Arc should span $\geq 3°$ of sky motion for good orbit determination.

\section{Convergence Criteria}

\subsection{Differential Correction Settings}

\begin{lstlisting}[language=C++]
DifferentialCorrector corrector(
    propagator,
    20,      // max_iterations
    1e-8     // convergence_tol (AU for position)
);
\end{lstlisting}

\textbf{Guidelines}:
\begin{itemize}
    \item \textbf{Max iterations}: 20 sufficient for well-conditioned problems
    \item \textbf{Convergence tolerance}: $10^{-8}$ AU $\approx$ 1.5 km (typical)
    \item \textbf{Good initial guess}: Converges in 3-5 iterations
    \item \textbf{Poor initial guess}: May need 10-15 iterations
\end{itemize}

\subsection{Monitoring Convergence}

Check iteration metrics:

\begin{lstlisting}[language=C++]
auto result = corrector.solve(initial_guess, observations, obs_coords);

if (!result.converged) {
    std::cerr << "Warning: Did not converge in " 
              << result.iterations << " iterations\n";
    std::cerr << "Final RMS: " << result.rms_residual << " arcsec\n";
    
    // Possible causes:
    // - Poor initial guess (try different IOD method)
    // - Observation outliers (filter and retry)
    // - Insufficient arc (need more data)
    // - Force model mismatch (add perturbations)
}
\end{lstlisting}

\subsection{Handling Non-Convergence}

Troubleshooting steps:

\begin{enumerate}
    \item \textbf{Check initial guess}: $|\Delta a| < 0.1$ AU from true orbit?
    \item \textbf{Filter outliers}: Remove observations with residuals $> 3\sigma$
    \item \textbf{Adjust tolerance}: Temporarily loosen to $10^{-6}$ AU
    \item \textbf{Try different IOD}: Gauss, Laplace, or Gooding method
    \item \textbf{Split arc}: Try shorter arc to diagnose problem
\end{enumerate}

\section{Initial Orbit Determination}

\subsection{Observation Selection}

Gauss method requires 3 well-separated observations:

\begin{lstlisting}[language=C++]
// Select observations: first, middle, last
auto obs1 = observations.front();
auto obs2 = observations[observations.size() / 2];
auto obs3 = observations.back();

GaussIOD gauss;
auto initial_elements = gauss.solve(obs1, obs2, obs3);
\end{lstlisting}

\textbf{Guidelines}:
\begin{itemize}
    \item Observations should span $> 50\%$ of total arc
    \item Avoid clustered observations (poor geometry)
    \item Prefer observations with good SNR
\end{itemize}

\subsection{Validating IOD Solution}

\begin{lstlisting}[language=C++]
auto elem = gauss.solve(obs1, obs2, obs3);

// Sanity checks
if (!elem.is_valid()) {
    std::cerr << "Invalid orbital elements from Gauss IOD\n";
    // Try different observation triplet
}

if (elem.e >= 1.0) {
    std::cerr << "Warning: Hyperbolic orbit (e >= 1)\n";
    // May indicate poor observation geometry
}

if (elem.a < 0.5 || elem.a > 50.0) {
    std::cerr << "Warning: Unusual semimajor axis: " << elem.a << " AU\n";
    // Outside typical asteroid range
}
\end{lstlisting}

\section{Ephemeris Configuration}

\subsection{Choosing Ephemeris Provider}

\begin{table}[h]
\centering
\caption{Ephemeris comparison}
\begin{tabular}{lccc}
\hline
\textbf{Provider} & \textbf{Accuracy} & \textbf{Speed} & \textbf{Use Case} \\
\hline
SPICE (DE440) & $< 1$ km & Slow & Production \\
SPICE (DE440s) & $< 1$ km & Medium & Limited time span \\
Analytic & $\sim 100$ km & Fast & Quick estimates \\
\hline
\end{tabular}
\end{table}

\subsection{SPICE Configuration}

\begin{lstlisting}[language=C++]
// Load appropriate kernel
auto ephemeris = std::make_shared<SPICEEphemeris>();

// Standard kernel (1850-2150)
ephemeris->load_kernel("de440.bsp");

// Optionally: leapseconds kernel for UTC conversion
ephemeris->load_kernel("naif0012.tls");
\end{lstlisting}

\textbf{Kernel selection}:
\begin{itemize}
    \item \textbf{de440s.bsp}: 1849-2150, 18 MB (recommended)
    \item \textbf{de440.bsp}: 1550-2650, 115 MB (extended)
    \item \textbf{de441.bsp}: -13200-17191, 302 MB (research)
\end{itemize}

\section{Error Handling}

\subsection{Exception Strategy}

\begin{lstlisting}[language=C++]
try {
    // Parse orbital elements
    auto parser = ParserFactory::create(filename);
    auto elements = parser->parse(filename);
    
    // Setup and propagate
    auto state = propagator.propagate(elements.to_cartesian(), target_epoch);
    
} catch (const ParseError& e) {
    std::cerr << "Parse error: " << e.what() << "\n";
    // Handle: check file format, try different parser
    
} catch (const PropagationError& e) {
    std::cerr << "Propagation error: " << e.what() << "\n";
    // Handle: check integration settings, force model
    
} catch (const ConvergenceError& e) {
    std::cerr << "Convergence error: " << e.what() << "\n";
    // Handle: adjust settings, filter outliers
    
} catch (const std::exception& e) {
    std::cerr << "Unexpected error: " << e.what() << "\n";
    // General error handling
}
\end{lstlisting}

\subsection{Logging Best Practices}

\begin{lstlisting}[language=C++]
// Enable diagnostic output
std::ofstream logfile("astdyn.log");

logfile << "Starting orbit determination for " << object_name << "\n";
logfile << "Observations: " << observations.size() << "\n";
logfile << "Arc: " << (observations.back().epoch - observations.front().epoch) 
        << " days\n";

// Log iteration progress
for (int iter = 0; iter < max_iterations; ++iter) {
    logfile << "Iteration " << iter << ": RMS = " 
            << current_rms << " arcsec\n";
}

logfile << "Converged: " << (result.converged ? "Yes" : "No") << "\n";
logfile << "Final RMS: " << result.rms_residual << " arcsec\n";
\end{lstlisting}

\section{Performance Optimization}

\subsection{Compilation Flags}

Always compile with optimization:

\begin{lstlisting}[language=bash]
cmake .. \
    -DCMAKE_BUILD_TYPE=Release \
    -DCMAKE_CXX_FLAGS="-O3 -march=native -DEIGEN_VECTORIZE" \
    -DEIGEN_NO_DEBUG=ON
\end{lstlisting}

\textbf{Impact}: 7-8$\times$ speedup vs. debug build.

\subsection{Batch Processing}

Process multiple objects efficiently:

\begin{lstlisting}[language=C++]
// Setup once, reuse for all objects
auto ephemeris = std::make_shared<SPICEEphemeris>();
ephemeris->load_kernel("de440s.bsp");

auto forces = std::make_shared<PointMassGravity>(
    ephemeris, std::vector<std::string>{"JUPITER", "SATURN"});

auto integrator = std::make_shared<RKF78>(1e-12);
Propagator propagator(integrator, forces, ephemeris);

// Process each object
for (const auto& object_file : object_files) {
    auto observations = MPCReader::read_file(object_file);
    // ... orbit determination ...
}
\end{lstlisting}

\textbf{Benefit}: Avoid repeated ephemeris loading (saves $\sim$700 ms per object).

\section{Validation Checklist}

Before trusting results:

\begin{itemize}
    \item[$\square$] RMS residual $< 2\sigma$ of assumed observation uncertainty
    \item[$\square$] No systematic trends in residuals vs. time
    \item[$\square$] Residuals approximately Gaussian distributed
    \item[$\square$] Orbital elements physically reasonable ($e < 1$, $a > 0$)
    \item[$\square$] Parameter uncertainties consistent with arc length
    \item[$\square$] Compare with published orbit (if available)
    \item[$\square$] Check for outliers ($> 3\sigma$)
    \item[$\square$] Verify convergence achieved
\end{itemize}

\section{Common Workflows}

\subsection{Standard Orbit Determination}

\begin{lstlisting}[language=C++]
// 1. Load observations
auto observations = MPCReader::read_file("observations.txt");

// 2. Load observatory coordinates
std::vector<ObservatoryCoordinates> obs_coords = 
    load_observatory_database();

// 3. Initial orbit (Gauss)
GaussIOD gauss;
auto initial_elem = gauss.solve(
    observations[0], 
    observations[observations.size()/2], 
    observations.back()
);

// 4. Setup propagator
auto eph = std::make_shared<SPICEEphemeris>();
eph->load_kernel("de440s.bsp");
auto forces = std::make_shared<PointMassGravity>(
    eph, std::vector<std::string>{"JUPITER", "SATURN"});
auto integrator = std::make_shared<RKF78>(1e-12);
auto prop = std::make_shared<Propagator>(integrator, forces, eph);

// 5. Differential correction
DifferentialCorrector dc(prop, 20, 1e-8);
auto result = dc.solve(initial_elem, observations, obs_coords);

// 6. Validate
if (result.converged && result.rms_residual < 2.0) {
    std::cout << "Success!\n";
    std::cout << "a = " << result.elements.a << " AU\n";
    std::cout << "e = " << result.elements.e << "\n";
    std::cout << "RMS = " << result.rms_residual << " arcsec\n";
} else {
    std::cerr << "Solution questionable\n";
}
\end{lstlisting}

\subsection{Ephemeris Generation}

\begin{lstlisting}[language=C++]
// Generate 1-year ephemeris at daily intervals
auto state0 = elements.to_cartesian();
double start = elements.epoch;
double end = elements.epoch + 365.0;
double step = 1.0;  // days

auto ephemeris_table = propagator.generate_ephemeris(
    state0, start, end, step);

// Write to file
std::ofstream outfile("ephemeris.txt");
for (const auto& state : ephemeris_table) {
    outfile << std::fixed << std::setprecision(6)
            << state.epoch << " "
            << state.position[0] << " "
            << state.position[1] << " "
            << state.position[2] << "\n";
}
\end{lstlisting}

\section{Troubleshooting Guide}

See Chapter 25 for detailed troubleshooting procedures.

\section{Summary}

Key recommendations:

\begin{enumerate}
    \item \textbf{Integration}: Use $10^{-12}$ tolerance, Sun+Jupiter+Saturn for main-belt
    \item \textbf{Observations}: Weight by inverse variance, filter 3-sigma outliers
    \item \textbf{Arc length}: 30-60 days optimal for main-belt asteroids
    \item \textbf{Convergence}: Monitor RMS, expect 3-5 iterations for good IOD
    \item \textbf{Validation}: Check residuals, compare with published orbits
    \item \textbf{Performance}: Compile with -O3, reuse ephemeris objects
    \item \textbf{Error handling}: Use exceptions, log diagnostics
\end{enumerate}

Following these practices ensures reliable, efficient orbit determination.

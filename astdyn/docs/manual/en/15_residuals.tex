\chapter{Residual Analysis}
\label{ch:residuals}

\section{Introduction}

\textbf{Residual analysis} is the examination of differences between observed and computed values (O-C) to assess orbit quality and diagnose problems.

\textbf{Goals}:
\begin{itemize}
    \item Validate orbit fit quality
    \item Identify outliers and systematic errors
    \item Assess observation weights
    \item Detect force model inadequacies
    \item Estimate realistic uncertainties
\end{itemize}

\section{Types of Residuals}

\subsection{Post-Fit Residuals}

After differential correction converges:

\begin{equation}
    r_i = o_i - c_i(\mathbf{y}_0^*)
\end{equation}

where $\mathbf{y}_0^*$ is the converged orbit.

For RA/Dec:
\begin{align}
    \Delta\alpha_i &= (\alpha_{\text{obs}} - \alpha_{\text{comp}}) \cos\delta_{\text{obs}} \\
    \Delta\delta_i &= \delta_{\text{obs}} - \delta_{\text{comp}}
\end{align}

Note: Multiply $\Delta\alpha$ by $\cos\delta$ to get linear separation.

\subsection{Normalized Residuals}

Scale by observation uncertainty:

\begin{equation}
    \zeta_i = \frac{r_i}{\sigma_i}
\end{equation}

Expected distribution: $\zeta_i \sim \mathcal{N}(0, 1)$ if weights are correct.

\subsection{Standardized Residuals}

Account for correlation in fit:

\begin{equation}
    \xi_i = \frac{r_i}{\sigma_i \sqrt{1 - h_{ii}}}
\end{equation}

where $h_{ii}$ is the $i$-th diagonal element of the hat matrix $\mathbf{S}$:
\begin{equation}
    \mathbf{S} = \mathbf{H}(\mathbf{H}^T\mathbf{W}\mathbf{H})^{-1}\mathbf{H}^T\mathbf{W}
\end{equation}

\section{Quality Metrics}

\subsection{Root Mean Square (RMS)}

\begin{equation}
    \text{RMS} = \sqrt{\frac{\sum_i w_i r_i^2}{\sum_i w_i}}
\end{equation}

For equal weights:

\begin{equation}
    \text{RMS} = \sqrt{\frac{1}{m} \sum_i r_i^2}
\end{equation}

\textbf{Interpretation}:
\begin{itemize}
    \item RMS $<$ 0.5": Excellent (modern CCD with Gaia catalog)
    \item RMS $\sim$ 1": Good (typical CCD)
    \item RMS $\sim$ 2": Fair (amateur observations)
    \item RMS $>$ 5": Poor (suspect systematic errors)
\end{itemize}

\subsection{Weighted RMS}

For unequal weights:

\begin{equation}
    \text{WRMS} = \sqrt{\frac{\chi^2}{m - n}}
\end{equation}

where $m$ is number of observations, $n = 6$ is number of parameters.

\subsection{Chi-Square Test}

Under correct model and weights:

\begin{equation}
    \chi^2 = \sum_i w_i r_i^2 \sim \chi^2_{m-n}
\end{equation}

Test statistic:

\begin{equation}
    \chi^2_{\text{red}} = \frac{\chi^2}{m - n}
\end{equation}

\textbf{Interpretation}:
\begin{itemize}
    \item $\chi^2_{\text{red}} \approx 1$: Weights consistent with errors
    \item $\chi^2_{\text{red}} \gg 1$: Underestimated uncertainties or model error
    \item $\chi^2_{\text{red}} \ll 1$: Overestimated uncertainties
\end{itemize}

\subsection{Maximum Residual}

\begin{equation}
    r_{\text{max}} = \max_i |r_i|
\end{equation}

Flag observations with $|r_i| > 3\sigma$ as potential outliers.

\section{Residual Plots}

\subsection{Residuals vs. Time}

Plot $r_i$ vs. $t_i$. Look for:
\begin{itemize}
    \item \textbf{Random scatter}: Good
    \item \textbf{Trends}: Systematic error (e.g., missing perturbation, catalog bias)
    \item \textbf{Jumps}: Change in observing conditions or equipment
    \item \textbf{Periodic variation}: Orbit model error
\end{itemize}

\subsection{Residuals vs. Observatory}

Plot $r_i$ vs. observatory code. Look for:
\begin{itemize}
    \item \textbf{Uniform scatter}: Good
    \item \textbf{Bias for specific site}: Site-specific systematic (timing, coordinates, catalog)
\end{itemize}

\subsection{Residuals vs. Magnitude}

Plot $r_i$ vs. apparent magnitude. Look for:
\begin{itemize}
    \item \textbf{No trend}: Good
    \item \textbf{Increasing scatter with magnitude}: Photon noise dominates
    \item \textbf{Bias trend}: Magnitude equation error in astrometry
\end{itemize}

\subsection{RA vs. Dec Residuals}

Plot $\Delta\alpha \cos\delta$ vs. $\Delta\delta$. Look for:
\begin{itemize}
    \item \textbf{Circular scatter}: Isotropic errors
    \item \textbf{Elliptical scatter}: Correlated errors (e.g., tracking error)
    \item \textbf{Radial pattern}: Distance error
\end{itemize}

\subsection{Normal Probability Plot}

Plot ordered normalized residuals $\zeta_{(i)}$ vs. expected normal quantiles. Should be approximately linear if errors are Gaussian.

\section{Outlier Detection}

\subsection{Threshold Method}

Flag observation if:

\begin{equation}
    |r_i| > k \sigma_i
\end{equation}

Typical $k = 3$ (3-sigma rule) or $k = 2.5$ (more aggressive).

\subsection{Chauvenet's Criterion}

Reject observation if probability of larger deviation is $< 1/(2m)$:

\begin{equation}
    P(|\zeta| > |\zeta_i|) < \frac{1}{2m}
\end{equation}

\subsection{Median Absolute Deviation (MAD)}

Robust alternative to standard deviation:

\begin{equation}
    \text{MAD} = \text{median}(|r_i - \text{median}(r_i)|)
\end{equation}

Scaled MAD: $\hat{\sigma} = 1.4826 \times \text{MAD}$

Flag if $|r_i - \text{median}| > k\hat{\sigma}$.

\subsection{Iterative Outlier Removal}

\begin{enumerate}
    \item Run differential correction
    \item Identify outliers (e.g., $|r_i| > 3\sigma$)
    \item Remove or downweight outliers
    \item Repeat until no more outliers found
\end{enumerate}

\textbf{Caution}: Don't remove too many observations. Typically remove $<$5\% of dataset.

\section{Systematic Error Diagnosis}

\subsection{Timing Errors}

\textbf{Symptom}: Residuals correlated with sky motion direction.

\textbf{Test}: Compute along-track vs. cross-track residuals:

\begin{align}
    r_{\parallel} &= \Delta\alpha \cos\delta \cos\theta + \Delta\delta \sin\theta \\
    r_{\perp} &= -\Delta\alpha \cos\delta \sin\theta + \Delta\delta \cos\theta
\end{align}

where $\theta = \arctan2(\dot{\delta}, \dot{\alpha}\cos\delta)$ is direction of motion.

If $|r_{\parallel}| \gg |r_{\perp}|$, suspect timing error.

\subsection{Catalog Bias}

\textbf{Symptom}: Systematic offset in all residuals from one catalog.

\textbf{Test}: Compare results using different star catalogs (Gaia DR3, UCAC4, etc.).

\textbf{Solution}: Use Gaia DR3 (most accurate, 0.02-0.05" systematic).

\subsection{Observatory Coordinate Error}

\textbf{Symptom}: Systematic offset for one observatory, varies with object position.

\textbf{Test}: Check MPC observatory coordinates vs. ITRF values.

\textbf{Solution}: Update coordinates, especially for new observatories.

\subsection{Light-Time Correction}

\textbf{Symptom}: Residuals show quadratic trend over long arc.

\textbf{Test}: Check that light-time correction is applied.

\textbf{Solution}: Iterate light-time (Chapter 12).

\subsection{Force Model Inadequacy}

\textbf{Symptom}: Residuals show smooth trend correlated with planetary positions.

\textbf{Test}: Add missing perturbations (Jupiter, Saturn, Earth, etc.).

\textbf{Solution}: Include all planets with $|a_{\text{pert}}/a_{\text{Sun}}| > 10^{-9}$.

\section{Example Analysis}

\begin{lstlisting}[language=C++,caption={Residual analysis implementation}]
struct ResidualAnalysis {
    double rms;
    double wrms;
    double chi2_red;
    double max_residual;
    std::vector<double> residuals;
    std::vector<double> normalized_residuals;
    std::vector<int> outlier_indices;
};

ResidualAnalysis analyze_residuals(
    const std::vector<Observation>& obs,
    const Vector6d& state,
    double epoch,
    const ForceModel& forces,
    const EphemerisInterface& ephemeris)
{
    ResidualAnalysis result;
    double chi2 = 0.0;
    double sum_weights = 0.0;
    
    for (size_t i = 0; i < obs.size(); ++i) {
        // Propagate and predict
        Vector6d y_obs = propagate(state, epoch, obs[i].epoch, forces);
        Vector2d computed = predict_observation(y_obs, obs[i].epoch, obs[i].obs_code, ephemeris);
        
        // Compute residual (in arcsec)
        double dRA = (obs[i].ra - computed(0)) * cos(obs[i].dec) * RAD_TO_ARCSEC;
        double dDec = (obs[i].dec - computed(1)) * RAD_TO_ARCSEC;
        double residual = sqrt(dRA*dRA + dDec*dDec);
        
        result.residuals.push_back(residual);
        
        // Normalized residual
        double sigma = sqrt(obs[i].sigma_ra*obs[i].sigma_ra + obs[i].sigma_dec*obs[i].sigma_dec) * RAD_TO_ARCSEC;
        double zeta = residual / sigma;
        result.normalized_residuals.push_back(zeta);
        
        // Chi-square
        double weight = 1.0 / (sigma * sigma);
        chi2 += weight * residual * residual;
        sum_weights += weight;
        
        // Max residual
        if (residual > result.max_residual) {
            result.max_residual = residual;
        }
        
        // Outlier detection (3-sigma)
        if (std::abs(zeta) > 3.0) {
            result.outlier_indices.push_back(i);
        }
    }
    
    // RMS
    result.rms = sqrt(chi2 / obs.size());
    
    // Weighted RMS
    int dof = 2 * obs.size() - 6;
    result.wrms = sqrt(chi2 / dof);
    
    // Reduced chi-square
    result.chi2_red = chi2 / dof;
    
    return result;
}

// Print analysis report
void print_residual_report(const ResidualAnalysis& analysis) {
    std::cout << "Residual Analysis Report\n";
    std::cout << "========================\n";
    std::cout << "Number of observations: " << analysis.residuals.size() << "\n";
    std::cout << "RMS: " << analysis.rms << " arcsec\n";
    std::cout << "Weighted RMS: " << analysis.wrms << " arcsec\n";
    std::cout << "Reduced chi-square: " << analysis.chi2_red << "\n";
    std::cout << "Maximum residual: " << analysis.max_residual << " arcsec\n";
    std::cout << "Number of outliers (>3-sigma): " << analysis.outlier_indices.size() << "\n";
    
    if (!analysis.outlier_indices.empty()) {
        std::cout << "\nOutlier indices:\n";
        for (int idx : analysis.outlier_indices) {
            std::cout << "  " << idx << ": " << analysis.residuals[idx] 
                     << " arcsec (" << analysis.normalized_residuals[idx] << "-sigma)\n";
        }
    }
    
    // Histogram of normalized residuals
    std::cout << "\nNormalized residual distribution:\n";
    auto hist = make_histogram(analysis.normalized_residuals, -4, 4, 16);
    for (auto [bin, count] : hist) {
        std::cout << std::setw(6) << std::fixed << std::setprecision(2) << bin << ": ";
        std::cout << std::string(count, '*') << " (" << count << ")\n";
    }
}
\end{lstlisting}

\subsection{Example Output}

\begin{verbatim}
Residual Analysis Report
========================
Number of observations: 100
RMS: 0.658 arcsec
Weighted RMS: 0.661 arcsec
Reduced chi-square: 1.02
Maximum residual: 2.34 arcsec
Number of outliers (>3-sigma): 2

Outlier indices:
  34: 2.34 arcsec (3.12-sigma)
  78: 2.11 arcsec (3.05-sigma)

Normalized residual distribution:
 -4.00: 
 -3.00: *
 -2.00: ****
 -1.00: ************
  0.00: **********************************
  1.00: ***************
  2.00: *****
  3.00: **
  4.00: 
\end{verbatim}

\textbf{Interpretation}:
\begin{itemize}
    \item RMS $\approx$ 0.66": Excellent fit
    \item $\chi^2_{\text{red}} \approx 1$: Weights are appropriate
    \item 2 outliers: Typical for 100 observations (2\%)
    \item Distribution approximately normal
\end{itemize}

\section{Improving Orbit Quality}

\subsection{When RMS is Too Large}

\textbf{Actions}:
\begin{enumerate}
    \item Check for outliers, remove if $>$3$\sigma$
    \item Verify observatory coordinates
    \item Check timing accuracy
    \item Add missing perturbations
    \item Use better star catalog (Gaia DR3)
    \item Consider non-gravitational forces (if comet)
\end{enumerate}

\subsection{When $\chi^2_{\text{red}} \gg 1$}

\textbf{Causes}:
\begin{itemize}
    \item Underestimated observation uncertainties
    \item Systematic errors not modeled
    \item Force model inadequate
\end{itemize}

\textbf{Solutions}:
\begin{itemize}
    \item Inflate uncertainties by factor $\sqrt{\chi^2_{\text{red}}}$
    \item Investigate systematic errors
    \item Improve force model
\end{itemize}

\subsection{When Few Observations Available}

For $m < 20$ observations:
\begin{itemize}
    \item Single outlier can dominate $\chi^2$
    \item Use robust methods (MAD, Huber weights)
    \item Be conservative about rejecting data
    \item Seek additional observations
\end{itemize}

\section{Reporting Results}

\subsection{Summary Statistics}

Always report:
\begin{itemize}
    \item Number of observations
    \item Time span
    \item Observatories
    \item RMS or WRMS
    \item Number of outliers rejected
\end{itemize}

\subsection{Covariance Interpretation}

\textbf{Formal uncertainty}: From $\mathbf{C} = \mathbf{N}^{-1}$.

\textbf{Realistic uncertainty}: Scale by $\sqrt{\chi^2_{\text{red}}}$ if $\chi^2_{\text{red}} > 1$.

\subsection{Orbit Arc Assessment}

\begin{itemize}
    \item \textbf{Short arc} ($<$10 days): Orbit poorly constrained, large extrapolation uncertainty
    \item \textbf{Medium arc} (10-60 days): Reasonable for ephemeris over similar span
    \item \textbf{Long arc} ($>$1 year): Well-constrained, reliable extrapolation
\end{itemize}

\section{Summary}

Key points about residual analysis:

\begin{enumerate}
    \item \textbf{Post-fit residuals} $r_i = o_i - c_i$ assess fit quality
    \item \textbf{RMS} measures overall fit; target $<$1" for modern observations
    \item \textbf{Chi-square test} validates weights; expect $\chi^2_{\text{red}} \approx 1$
    \item \textbf{Residual plots} diagnose systematic errors
    \item \textbf{Outliers} detected via 3$\sigma$ threshold or robust methods
    \item \textbf{Systematic errors} identified by correlations with time, observatory, magnitude
    \item \textbf{Force model} validated by examining residual trends
    \item \textbf{Realistic uncertainties} account for systematic errors via $\chi^2_{\text{red}}$
\end{enumerate}

With differential correction and residual analysis, we complete the core orbit determination workflow. Next chapters cover software implementation.

\chapter{Observations}
\label{ch:observations}

\section{Introduction}

\textbf{Observations} are the fundamental data for orbit determination. This chapter describes:
\begin{itemize}
    \item Types of observations (astrometric, radar, spacecraft)
    \item Observation models relating state to measurements
    \item Data formats (MPC, radar, tracking)
    \item Corrections (refraction, light-time, aberration)
    \item Observatory coordinates and Earth orientation
\end{itemize}

Accurate observation modeling is essential for achieving sub-arcsecond orbit determination.

\section{Observation Types}

\subsection{Optical Astrometry}

The most common observations are angular positions on the celestial sphere:

\begin{equation}
    \text{Observation} = (\alpha, \delta, t)
\end{equation}

where:
\begin{itemize}
    \item $\alpha$ is right ascension (0$^\circ$ to 360$^\circ$ or 0h to 24h)
    \item $\delta$ is declination ($-90^\circ$ to $+90^\circ$)
    \item $t$ is observation time (usually UTC)
\end{itemize}

\textbf{Precision ranges}:
\begin{itemize}
    \item Historical (photographic): 0.5--2 arcsec
    \item CCD astrometry: 0.1--0.5 arcsec
    \item Gaia space mission: 0.0001--0.001 arcsec (100 $\mu$as)
    \item Ground-based surveys (Pan-STARRS, ATLAS): 0.05--0.2 arcsec
\end{itemize}

\subsection{Radar Observations}

Planetary radar provides range and Doppler measurements:

\begin{equation}
    \text{Range: } \rho = |\mathbf{r}_{\text{target}} - \mathbf{r}_{\text{station}}|
\end{equation}

\begin{equation}
    \text{Doppler: } \dot{\rho} = \frac{(\mathbf{r}_{\text{target}} - \mathbf{r}_{\text{station}}) \cdot (\mathbf{v}_{\text{target}} - \mathbf{v}_{\text{station}})}{|\mathbf{r}_{\text{target}} - \mathbf{r}_{\text{station}}|}
\end{equation}

\textbf{Major radar facilities}:
\begin{itemize}
    \item Arecibo (305 m, 2.38 GHz) -- decommissioned 2020
    \item Goldstone DSS-14 (70 m, 8.56 GHz) -- operational
    \item Green Bank (100 m, receive-only)
\end{itemize}

\textbf{Precision}:
\begin{itemize}
    \item Range: 10--100 meters (delay-Doppler imaging: $<$1 m)
    \item Doppler: 0.1--1 mm/s
\end{itemize}

Radar is 1000$\times$ more precise than optical astrometry in range but limited to nearby objects ($<$ 0.3 AU for asteroids).

\subsection{Spacecraft Tracking}

Deep space missions tracked via:
\begin{itemize}
    \item Two-way Doppler (mm/s precision)
    \item Range measurements (meter-level)
    \item Delta-DOR (angular position via interferometry)
    \item Optical navigation (camera images)
\end{itemize}

\section{Astrometric Observation Model}

\subsection{Coordinate Transformation}

Given object position $\mathbf{r}_{\text{obj}}$ in heliocentric ecliptic J2000, compute topocentric equatorial:

\begin{enumerate}
    \item Transform to barycentric: $\mathbf{r}_{\text{bary}} = \mathbf{r}_{\text{obj}} + \mathbf{r}_{\odot,\text{bary}}$
    \item Subtract Earth position: $\mathbf{r}_{\text{geo}} = \mathbf{r}_{\text{bary}} - \mathbf{r}_{\text{Earth}}$
    \item Subtract observatory position: $\mathbf{r}_{\text{topo}} = \mathbf{r}_{\text{geo}} - \mathbf{r}_{\text{obs}}$
    \item Rotate to equatorial: $\mathbf{r}_{\text{eq}} = \mathbf{R}_{\text{ecl}\to\text{eq}} \mathbf{r}_{\text{topo}}$
\end{enumerate}

\subsection{Spherical Coordinates}

From Cartesian topocentric equatorial $\mathbf{r}_{\text{eq}} = (x, y, z)$:

\begin{equation}
    \alpha = \arctan2(y, x)
\end{equation}

\begin{equation}
    \delta = \arcsin\left(\frac{z}{\sqrt{x^2 + y^2 + z^2}}\right)
\end{equation}

Handle quadrant correctly with \texttt{atan2}.

\subsection{Light-Time Correction}

Observation time $t_{\text{obs}}$ is when photons arrive at Earth. Object was at emission position at:

\begin{equation}
    t_{\text{emit}} = t_{\text{obs}} - \frac{\rho}{c}
\end{equation}

where $\rho$ is geocentric distance.

Iterate to find $t_{\text{emit}}$:
\begin{lstlisting}[language=C++,caption={Light-time iteration}]
double tau = 0.0;  // Initial guess
for (int iter = 0; iter < 5; ++iter) {
    Vector3d r_obj = propagate(y0, t0, t_obs - tau);
    Vector3d r_earth = ephemeris.get_position("Earth", t_obs);
    double rho = (r_obj - r_earth).norm();
    double tau_new = rho / C_AU_PER_DAY;
    if (std::abs(tau_new - tau) < 1e-10) break;
    tau = tau_new;
}
\end{lstlisting}

Typical correction: 4--30 minutes for asteroids.

\subsection{Stellar Aberration}

Earth's orbital motion causes apparent displacement:

\begin{equation}
    \mathbf{r}_{\text{aberrated}} = \mathbf{r}_{\text{geometric}} + \frac{\rho}{c}\mathbf{v}_{\text{Earth}}
\end{equation}

where $\mathbf{v}_{\text{Earth}}$ is Earth's velocity.

Maximum effect: $\pm 20.5$ arcsec (annual aberration).

\subsection{Atmospheric Refraction}

Light bends passing through atmosphere. Correction depends on zenith angle $z$:

\begin{equation}
    \Delta z \approx 58.2'' \tan z - 0.067'' \tan^3 z
\end{equation}

At zenith ($z = 0$): no refraction. At horizon ($z = 90^\circ$): $\sim$34 arcmin (solar diameter!).

For precise work, use wavelength-dependent model:

\begin{equation}
    n - 1 = 77.6 \times 10^{-6} \frac{P}{T}\left(1 + 7.52 \times 10^{-3}\lambda^{-2}\right)
\end{equation}

where $P$ is pressure (mbar), $T$ is temperature (K), $\lambda$ is wavelength ($\mu$m).

Modern astrometry corrects to "above atmosphere" by:
\begin{itemize}
    \item Fitting catalog stars in field
    \item Measuring local refraction empirically
    \item Applying site-specific models
\end{itemize}

\section{Observatory Coordinates}

\subsection{ITRF and Observatory Codes}

The International Terrestrial Reference Frame (ITRF) provides precise coordinates for observatories.

\textbf{Minor Planet Center (MPC) observatory codes}:
\begin{itemize}
    \item 500: Geocenter (for space-based observations)
    \item 568: Mauna Kea (Hawaii)
    \item 703: Catalina Sky Survey (Arizona)
    \item F51: Pan-STARRS 1 (Hawaii)
    \item G96: Mt. Lemmon Survey (Arizona)
\end{itemize}

Example entry for observatory 703:
\begin{verbatim}
703  Catalina  4.215500  0.759260  0.648764  -31.67
\end{verbatim}

Format: code, name, $\rho \cos\phi'$, $\rho \sin\phi'$, longitude (deg), altitude (m).

\subsection{Geocentric Observatory Position}

Convert geodetic coordinates $(h, \lambda, \phi)$ to geocentric Cartesian:

\begin{equation}
    \mathbf{r}_{\text{obs}} = \begin{bmatrix}
        (N + h)\cos\phi\cos\lambda \\
        (N + h)\cos\phi\sin\lambda \\
        (N(1-e^2) + h)\sin\phi
    \end{bmatrix}
\end{equation}

where:
\begin{equation}
    N = \frac{a}{\sqrt{1 - e^2\sin^2\phi}}
\end{equation}

and $a = 6378.137$ km (WGS84 equatorial radius), $e = 0.08181919$ (eccentricity).

\subsection{Rotation to Inertial Frame}

Observatory position rotates with Earth. Transformation involves:

\begin{enumerate}
    \item Polar motion ($x_p, y_p$)
    \item UT1-UTC correction (Earth rotation angle)
    \item Precession-nutation (IAU 2006/2000A)
    \item Frame bias (ICRS to J2000)
\end{enumerate}

\textbf{Simplified rotation}:

\begin{equation}
    \mathbf{r}_{\text{inertial}} = \mathbf{R}_3(\text{GAST}) \mathbf{r}_{\text{ITRF}}
\end{equation}

where GAST is Greenwich Apparent Sidereal Time.

\section{Earth Orientation Parameters}

\subsection{Polar Motion}

Earth's rotation axis moves relative to crust (Chandler wobble, annual motion):

\begin{equation}
    \mathbf{R}_{\text{polar}} = \mathbf{R}_2(-x_p)\mathbf{R}_1(-y_p)
\end{equation}

Amplitude: $\sim$0.3 arcsec ($\sim$10 meters at surface).

Data from IERS: \texttt{finals2000A.all} bulletin.

\subsection{UT1-UTC}

Universal Time (UT1) tracks Earth's actual rotation. Atomic time (UTC) is uniform.

\begin{equation}
    \text{UT1} = \text{UTC} + (\text{UT1-UTC})
\end{equation}

$|\text{UT1-UTC}| < 0.9$ seconds (leap seconds added when needed).

Prediction: available from IERS with $\sim$10 ms accuracy for 1 year ahead.

\subsection{Precession and Nutation}

Earth's rotation axis precesses (26,000 year period) and nutates (18.6 year main period).

\textbf{IAU 2006 precession} + \textbf{IAU 2000A nutation} = high-precision model.

Simplified for asteroid work: use mean pole (J2000) and ignore nutation ($\sim$15 arcsec effect).

\section{MPC Observation Format}

\subsection{80-Column Format}

Standard format for optical astrometry:

\begin{lstlisting}[basicstyle=\ttfamily\tiny, frame=none, xleftmargin=0pt]
     K17S00S  C2017 06 01.41667 18 26 54.13 -23 47 08.4          21.1 V      F51
\end{lstlisting}

Fields:
\begin{itemize}
    \item Columns 1-5: Temporary designation or number
    \item Column 12: Discovery asterisk (*)
    \item Column 13: Note (e.g., photometry)
    \item Column 14: Publication reference
    \item Columns 15-32: Observation date (YYYY MM DD.ddddd)
    \item Columns 33-44: Right ascension (HH MM SS.sss)
    \item Columns 45-56: Declination (sDD MM SS.ss)
    \item Columns 66-70: Magnitude
    \item Column 71: Mag band (V, R, I, etc.)
    \item Columns 78-80: Observatory code
\end{itemize}

\subsection{ADES Format}

Astrometry Data Exchange Standard (modern XML/JSON format):

\begin{lstlisting}[language=XML,caption={ADES XML example}]
<obsBlock>
  <obsContext>
    <observatory>
      <mpcCode>F51</mpcCode>
    </observatory>
  </obsContext>
  <obsData>
    <optical>
      <trkSub>K17S00S</trkSub>
      <obsTime>2017-06-01T10:00:00.000Z</obsTime>
      <ra>276.72554</ra>
      <dec>-23.78567</dec>
      <mag>21.1</mag>
      <band>V</band>
      <rmsRA>0.1</rmsRA>
      <rmsDec>0.1</rmsDec>
    </optical>
  </obsData>
</obsBlock>
\end{lstlisting}

\textbf{Advantages over 80-column}:
\begin{itemize}
    \item Explicit uncertainties
    \item Metadata (telescope, detector, catalog)
    \item No fixed-width limitations
    \item International standard
\end{itemize}

\section{Observation Weights}

\subsection{Weighting Schemes}

Not all observations equally reliable. Weight by estimated uncertainty:

\begin{equation}
    w_i = \frac{1}{\sigma_i^2}
\end{equation}

\textbf{Uncertainty sources}:
\begin{itemize}
    \item Measurement error (star fitting, centroid)
    \item Catalog errors (Gaia DR3: 0.02--0.05 arcsec)
    \item Timing errors ($\pm$1 second $\to$ 0.01 arcsec for slow movers)
    \item Atmospheric effects (seeing, refraction)
    \item Trailing losses (long exposures)
\end{itemize}

\subsection{Empirical Weighting}

For MPC observations without formal uncertainties:

\begin{table}[htbp]
\centering
\begin{tabular}{lcc}
\toprule
\textbf{Observatory Type} & \textbf{$\sigma_\alpha \cos\delta$} & \textbf{$\sigma_\delta$} \\
\midrule
Professional (Pan-STARRS, CSS) & 0.1 arcsec & 0.1 arcsec \\
Amateur CCD & 0.5 arcsec & 0.5 arcsec \\
Historical photographic & 1.0 arcsec & 1.0 arcsec \\
Radar range & 10 m & -- \\
Radar Doppler & -- & 1 mm/s \\
\bottomrule
\end{tabular}
\caption{Typical observation uncertainties.}
\label{tab:obs_uncertainties}
\end{table}

\subsection{Downweighting Outliers}

After initial fit, identify outliers (residual $> 3\sigma$) and reduce weight:

\begin{equation}
    w_{\text{new}} = w_{\text{old}} \times \exp\left(-\frac{r^2}{2\sigma^2}\right)
\end{equation}

where $r$ is residual. This is "robust least squares" or "Huber weighting."

\section{Observation Partials}

\subsection{Definition}

For orbit determination, we need:

\begin{equation}
    \frac{\partial(\alpha, \delta)}{\partial\mathbf{y}(t_0)}
\end{equation}

This relates how initial state affects predicted observation.

\subsection{Chain Rule}

Use chain rule with state transition matrix:

\begin{equation}
    \frac{\partial(\alpha, \delta)}{\partial\mathbf{y}(t_0)} = \frac{\partial(\alpha, \delta)}{\partial\mathbf{r}(t_{\text{obs}})} \frac{\partial\mathbf{r}(t_{\text{obs}})}{\partial\mathbf{y}(t_{\text{obs}})} \frac{\partial\mathbf{y}(t_{\text{obs}})}{\partial\mathbf{y}(t_0)}
\end{equation}

The last factor is the STM $\Phi(t_{\text{obs}}, t_0)$.

\subsection{Geometric Partials}

For topocentric position $\mathbf{r} = (x, y, z)$ in equatorial frame:

\begin{equation}
    \rho = \sqrt{x^2 + y^2 + z^2}
\end{equation}

\begin{equation}
    \frac{\partial\alpha}{\partial x} = -\frac{y}{x^2 + y^2}, \quad
    \frac{\partial\alpha}{\partial y} = \frac{x}{x^2 + y^2}, \quad
    \frac{\partial\alpha}{\partial z} = 0
\end{equation}

\begin{equation}
    \frac{\partial\delta}{\partial x} = -\frac{xz}{\rho^2\sqrt{x^2+y^2}}, \quad
    \frac{\partial\delta}{\partial y} = -\frac{yz}{\rho^2\sqrt{x^2+y^2}}, \quad
    \frac{\partial\delta}{\partial z} = \frac{\sqrt{x^2+y^2}}{\rho^2}
\end{equation}

\subsection{Implementation}

\begin{lstlisting}[language=C++,caption={Computing observation partials}]
Matrix<2,6> compute_partials_radec(
    const Vector6d& state,
    const Matrix6d& stm,
    const Vector3d& obs_pos)
{
    Vector3d r = state.head<3>() - obs_pos;
    double x = r(0), y = r(1), z = r(2);
    double rho = r.norm();
    double rho_xy = std::sqrt(x*x + y*y);
    
    // Partials w.r.t. position
    Matrix<2,3> dobs_dr;
    dobs_dr(0,0) = -y / (x*x + y*y);  // d(RA)/dx
    dobs_dr(0,1) =  x / (x*x + y*y);  // d(RA)/dy
    dobs_dr(0,2) =  0.0;              // d(RA)/dz
    
    dobs_dr(1,0) = -x*z / (rho*rho*rho_xy);  // d(Dec)/dx
    dobs_dr(1,1) = -y*z / (rho*rho*rho_xy);  // d(Dec)/dy
    dobs_dr(1,2) =  rho_xy / (rho*rho);      // d(Dec)/dz
    
    // Chain with STM
    Matrix<2,6> partials = dobs_dr * stm.block<3,6>(0,0);
    
    return partials;
}
\end{lstlisting}

\section{Data Quality}

\subsection{Timing Accuracy}

Observation time must be UTC to $\pm$1 second for asteroids ($\pm$0.01 sec for fast movers).

\textbf{Common issues}:
\begin{itemize}
    \item Clock drift (GPS receivers essential)
    \item Mid-exposure vs start/end time
    \item Time zone errors (always use UTC!)
    \item Leap seconds
\end{itemize}

\subsection{Astrometric Catalog}

Modern observations referenced to:
\begin{itemize}
    \item Gaia DR3 (2022): 0.02--0.05 arcsec, $\sim$1.8 billion stars
    \item UCAC4: 0.02--0.1 arcsec, 113 million stars
    \item 2MASS: 0.08 arcsec (infrared), 471 million objects
\end{itemize}

\textbf{Older observations} (pre-Gaia) may have systematic errors from catalog:
\begin{itemize}
    \item USNO-A: $\sim$0.25 arcsec systematic
    \item GSC: $\sim$0.3 arcsec systematic
\end{itemize}

Use catalog-specific debiasing when mixing observations.

\subsection{Site-Specific Systematics}

Some observatories have known issues:
\begin{itemize}
    \item Poor timing ($>$10 sec errors)
    \item Incorrect coordinates (wrong latitude/longitude)
    \item Scale errors (wrong plate scale)
    \item Magnitude-dependent bias (charge bleeding)
\end{itemize}

MPC maintains quality flags, but user must validate data.

\section{Practical Example}

\subsection{Loading MPC Observations}

\begin{lstlisting}[language=C++,caption={Parsing MPC observations}]
#include <astdyn/observations/MPCObservation.hpp>

std::vector<Observation> load_mpc_file(const std::string& filename) {
    std::vector<Observation> observations;
    std::ifstream file(filename);
    std::string line;
    
    while (std::getline(file, line)) {
        if (line.length() < 80) continue;
        
        MPCObservation obs;
        if (obs.parse(line)) {
            observations.push_back(obs);
        }
    }
    
    std::cout << "Loaded " << observations.size() << " observations\n";
    return observations;
}
\end{lstlisting}

\subsection{Computing Predicted Observations}

\begin{lstlisting}[language=C++,caption={Predicting observations}]
Vector2d predict_observation(
    const Vector6d& state,
    double epoch,
    const std::string& obs_code,
    const EphemerisInterface& ephemeris)
{
    // Get Earth position
    Vector3d earth_pos = ephemeris.get_position("Earth", epoch);
    
    // Get observatory position (ITRF -> inertial)
    Vector3d obs_pos_geo = observatory_db.get_geocentric(obs_code);
    Matrix3d R_itrf_to_j2000 = earth_rotation(epoch);
    Vector3d obs_pos = earth_pos + R_itrf_to_j2000 * obs_pos_geo;
    
    // Topocentric position
    Vector3d r_topo = state.head<3>() - obs_pos;
    
    // Ecliptic to equatorial
    Vector3d r_eq = R_ecl_to_eq * r_topo;
    
    // Compute RA/Dec
    double alpha = std::atan2(r_eq(1), r_eq(0));
    double delta = std::asin(r_eq(2) / r_eq.norm());
    
    if (alpha < 0) alpha += 2*M_PI;
    
    return Vector2d(alpha, delta);
}
\end{lstlisting}

\section{Summary}

Key concepts about observations:

\begin{enumerate}
    \item \textbf{Optical astrometry} provides RA/Dec with 0.1--0.5 arcsec precision
    \item \textbf{Radar} gives range/Doppler with meter/mm-per-sec precision
    \item \textbf{Light-time} correction is essential (4--30 minutes for asteroids)
    \item \textbf{Aberration} causes $\pm$20 arcsec displacement
    \item \textbf{Refraction} affects low-elevation observations
    \item \textbf{Observatory position} must be in inertial frame
    \item \textbf{MPC format} is standard, ADES is modern
    \item \textbf{Weighting} by uncertainty improves fit quality
    \item \textbf{Partials} $\partial(\alpha,\delta)/\partial\mathbf{y}$ enable orbit fitting
\end{enumerate}

Practical recommendations:
\begin{itemize}
    \item Always apply light-time and aberration corrections
    \item Use Gaia DR3 catalog for modern observations
    \item Validate timing (UTC, leap seconds)
    \item Check observatory coordinates
    \item Weight by estimated uncertainty
    \item Identify and downweight outliers
\end{itemize}

The next chapter covers initial orbit determination from a few observations, followed by differential correction to refine orbits using all available data.

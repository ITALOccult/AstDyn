\chapter{Ephemeris Computation}
\label{ch:ephemeris}

\section{Introduction}

An \textbf{ephemeris} (plural: \emph{ephemerides}) is a table or function providing positions (and optionally velocities) of celestial bodies at specific times. Accurate ephemerides are essential for:

\begin{itemize}
    \item Computing predicted positions for observations
    \item Reducing astrometric measurements
    \item Planning space missions
    \item Analyzing close approaches
    \item Studying orbital dynamics
\end{itemize}

This chapter covers methods for generating, storing, and interpolating ephemerides efficiently.

\section{Types of Ephemerides}

\subsection{Planetary Ephemerides}

Major planets require the highest accuracy:

\begin{description}
    \item[JPL Development Ephemerides (DE)] Numerical integration of solar system, including Moon and large asteroids. Current: DE440 (Earth-Moon optimization), DE441 (outer solar system).
    
    \item[VSOP87] Analytical theory by Bureau des Longitudes. Series expansion in orbital elements. Accuracy: $\sim$1 arcsec over millennia.
    
    \item[INPOP] French ephemeris from IMCCE, optimized for planetary radar ranging.
\end{description}

\subsection{Small Body Ephemerides}

Asteroids and comets:
\begin{itemize}
    \item Computed from orbital elements via propagation
    \item Archived in MPC (Minor Planet Center) database
    \item Precision varies: 0.1 arcsec (well-observed) to 10 arcmin (single-opposition)
\end{itemize}

\subsection{Spacecraft Ephemerides}

Interplanetary missions:
\begin{itemize}
    \item SPICE kernels (SPK files) from navigation teams
    \item Chebyshev polynomial segments
    \item Meter-level accuracy for close-approach phases
\end{itemize}

\section{Ephemeris Representations}

\subsection{Tabulated Format}

Simplest representation: discrete time-state pairs.

\begin{table}[htbp]
\centering
\begin{tabular}{ccccccc}
\toprule
\textbf{MJD (TDB)} & $x$ (AU) & $y$ (AU) & $z$ (AU) & $\dot{x}$ & $\dot{y}$ & $\dot{z}$ \\
\midrule
60000.0 & 1.234 & 0.567 & 0.123 & $-0.012$ & 0.015 & 0.003 \\
60001.0 & 1.222 & 0.582 & 0.126 & $-0.012$ & 0.015 & 0.003 \\
60002.0 & 1.210 & 0.597 & 0.129 & $-0.012$ & 0.015 & 0.003 \\
\vdots & \vdots & \vdots & \vdots & \vdots & \vdots & \vdots \\
\bottomrule
\end{tabular}
\caption{Example tabulated ephemeris with 1-day spacing.}
\label{tab:ephemeris_table}
\end{table}

\textbf{Advantages}:
\begin{itemize}
    \item Easy to implement
    \item Direct lookup for tabulated times
\end{itemize}

\textbf{Disadvantages}:
\begin{itemize}
    \item Large storage for high cadence
    \item Requires interpolation between points
    \item Fixed time grid (inefficient for eccentric orbits)
\end{itemize}

\subsection{Polynomial Representation}

Represent position as polynomial:

\begin{equation}
    \mathbf{r}(t) = \sum_{k=0}^{n} \mathbf{c}_k (t - t_0)^k
\end{equation}

Typically used piecewise over segments (splines).

\subsection{Chebyshev Polynomials}

JPL's preferred method. For time interval $[t_a, t_b]$, represent:

\begin{equation}
    \mathbf{r}(t) = \sum_{k=0}^{n} \mathbf{a}_k T_k\left(\frac{2t - t_a - t_b}{t_b - t_a}\right)
\end{equation}

where $T_k(x)$ are Chebyshev polynomials:

\begin{equation}
    T_0(x) = 1, \quad T_1(x) = x, \quad T_k(x) = 2xT_{k-1}(x) - T_{k-2}(x)
\end{equation}

\textbf{Properties}:
\begin{itemize}
    \item Minimax error distribution (optimal approximation)
    \item Stable for high degrees ($n \sim 15$)
    \item Efficient evaluation via recurrence
\end{itemize}

\subsection{Fourier Series}

For nearly circular orbits:

\begin{equation}
    \mathbf{r}(t) = \sum_{k=-N}^{N} \mathbf{c}_k e^{ik\omega t}
\end{equation}

Used in analytical planetary theories (VSOP87).

\section{Interpolation Methods}

\subsection{Linear Interpolation}

Given points $(t_1, \mathbf{r}_1)$ and $(t_2, \mathbf{r}_2)$:

\begin{equation}
    \mathbf{r}(t) = \mathbf{r}_1 + \frac{t - t_1}{t_2 - t_1}(\mathbf{r}_2 - \mathbf{r}_1)
\end{equation}

\textbf{Accuracy}: First-order, $O(h^2)$ error where $h = t_2 - t_1$.

\textbf{Use}: Quick lookups when high precision not required ($>$1 km acceptable).

\subsection{Lagrange Interpolation}

Use $n+1$ points to construct polynomial of degree $n$:

\begin{equation}
    \mathbf{r}(t) = \sum_{i=0}^{n} \mathbf{r}_i L_i(t)
\end{equation}

where the Lagrange basis polynomials are:

\begin{equation}
    L_i(t) = \prod_{\substack{j=0\\j\neq i}}^{n} \frac{t - t_j}{t_i - t_j}
\end{equation}

\textbf{Typical choice}: $n = 6$ to 10 (7th to 11th order).

\textbf{Accuracy}: For 8th-order with 1-day spacing, error $\sim$10 m for typical asteroid orbits.

\subsection{Hermite Interpolation}

Uses both positions and velocities. For interval $[t_1, t_2]$:

\begin{equation}
    \mathbf{r}(t) = \mathbf{r}_1 H_0(s) + \mathbf{r}_2 H_1(s) + h\dot{\mathbf{r}}_1 H_2(s) + h\dot{\mathbf{r}}_2 H_3(s)
\end{equation}

where $s = (t - t_1)/h$, $h = t_2 - t_1$, and the Hermite basis functions are:

\begin{align}
    H_0(s) &= (1 + 2s)(1-s)^2 \\
    H_1(s) &= s^2(3 - 2s) \\
    H_2(s) &= s(1-s)^2 \\
    H_3(s) &= s^2(s-1)
\end{align}

\textbf{Advantages}:
\begin{itemize}
    \item Cubic accuracy ($O(h^4)$)
    \item Smooth velocities (continuous first derivative)
    \item Only requires two points
\end{itemize}

\textbf{Accuracy}: With 1-day spacing, error $\sim$1 m for well-behaved orbits.

\subsection{Spline Interpolation}

Cubic splines provide smooth interpolation through all points with continuous second derivatives.

For points $(t_i, \mathbf{r}_i)$, $i = 0, \ldots, n$, construct piecewise cubics $\mathbf{s}_i(t)$ on $[t_i, t_{i+1}]$ such that:

\begin{itemize}
    \item $\mathbf{s}_i(t_i) = \mathbf{r}_i$ (interpolation)
    \item $\mathbf{s}_i'(t_{i+1}) = \mathbf{s}_{i+1}'(t_{i+1})$ (continuous first derivative)
    \item $\mathbf{s}_i''(t_{i+1}) = \mathbf{s}_{i+1}''(t_{i+1})$ (continuous second derivative)
\end{itemize}

\textbf{Use}: When smooth acceleration is important (covariance propagation).

\section{SPICE System}

\subsection{Overview}

SPICE (Spacecraft Planet Instrument C-matrix Events) is NASA's toolkit for space mission geometry:

\begin{description}
    \item[SPK] (Ephemeris kernels) Position and velocity
    \item[CK] (Orientation kernels) Spacecraft attitude
    \item[PCK] (Constants kernels) Physical parameters, body shapes
    \item[IK] (Instrument kernels) FOV, boresight
    \item[FK] (Frame kernels) Reference frame definitions
    \item[LSK] (Leapsecond kernels) Time conversions
\end{description}

\subsection{SPK Files}

Binary files containing Chebyshev or Hermite polynomial segments.

\textbf{Usage in AstDyn}:
\begin{lstlisting}[language=C++,caption={Loading SPICE kernel}]
#include <astdyn/ephemeris/SpiceInterface.hpp>

SpiceInterface spice;
spice.load_kernel("de440.bsp");  // Planetary ephemeris
spice.load_kernel("codes_300ast_20100725.bsp");  // Asteroids

// Query Jupiter position at epoch
double et = spice.mjd_to_et(60000.0);  // Convert MJD to ET
Vector6d jupiter_state = spice.get_state("JUPITER", et, "ECLIPJ2000", "SUN");

std::cout << "Jupiter position: " << jupiter_state.head<3>().transpose() 
          << " km\n";
\end{lstlisting}

\subsection{NAIF IDs}

SPICE uses integer IDs:
\begin{itemize}
    \item Sun: 10
    \item Planets: 199 (Mercury), 299 (Venus), 399 (Earth), 499 (Mars), 599 (Jupiter), etc.
    \item Moon: 301
    \item Asteroids: 2000001 (Ceres), 2000004 (Vesta), 2000203 (Pompeja)
\end{itemize}

\section{Planetary Ephemerides}

\subsection{JPL Development Ephemerides}

\textbf{DE440/441} (released 2020):
\begin{itemize}
    \item Covers years 1550--2650
    \item Includes Sun, planets, Moon, Pluto, 343 asteroids
    \item Fit to ranging data (Mars missions), VLBI, LLR
    \item Accuracy: $\sim$1 km for inner planets, $\sim$10 km for outer planets
\end{itemize}

\textbf{File sizes}:
\begin{itemize}
    \item DE440: 114 MB (standard)
    \item DE441: 3.2 GB (includes high-rate Moon)
\end{itemize}

\subsection{VSOP87}

Analytical series developed by Bretagnon \& Francou (1988).

\textbf{Variants}:
\begin{description}
    \item[VSOP87A] Heliocentric rectangular, J2000 ecliptic
    \item[VSOP87B] Heliocentric rectangular, J2000 equatorial  
    \item[VSOP87C] Heliocentric spherical (mean ecliptic/equinox of date)
    \item[VSOP87D] Heliocentric spherical (J2000 ecliptic)
    \item[VSOP87E] Barycentric rectangular, J2000 ecliptic
\end{description}

\textbf{Implementation}:
\begin{lstlisting}[language=C++,caption={VSOP87 usage}]
#include <astdyn/ephemeris/VSOP87.hpp>

VSOP87 vsop;
double jd = 2460000.5;  // Julian date

// Earth position (VSOP87A: heliocentric J2000 ecliptic)
Vector3d earth_pos = vsop.get_position("Earth", jd, VSOP87_A);
std::cout << "Earth position: " << earth_pos.transpose() << " AU\n";

// Accuracy estimate
double error_km = vsop.estimated_error("Earth", jd);
std::cout << "Position error: ~" << error_km << " km\n";
\end{lstlisting}

\textbf{Accuracy}: $\sim$1 km for inner planets over $\pm$2000 years from J2000.

\subsection{Comparison}

\begin{table}[htbp]
\centering
\begin{tabular}{lccc}
\toprule
\textbf{Method} & \textbf{Accuracy} & \textbf{Speed} & \textbf{File Size} \\
\midrule
DE440 (SPICE) & 1--10 km & Fast & 114 MB \\
VSOP87 & 1--5 km & Medium & $\sim$1 MB (code) \\
Keplerian & 100--1000 km & Very fast & Negligible \\
\bottomrule
\end{tabular}
\caption{Planetary ephemeris comparison.}
\label{tab:planetary_eph}
\end{table}

\section{Light-Time Corrections}

\subsection{Geometric vs Apparent Position}

Light travels at finite speed $c = 299792.458$ km/s, so we observe planets where they \emph{were}, not where they \emph{are}.

\textbf{Light-time}:

\begin{equation}
    \tau = \frac{|\mathbf{r}_{\text{planet}} - \mathbf{r}_{\text{obs}}|}{c}
\end{equation}

Typical values:
\begin{itemize}
    \item Sun: 8.3 minutes
    \item Jupiter: 30--50 minutes
    \item Saturn: 70--90 minutes
    \item Neptune: 4 hours
\end{itemize}

\subsection{Iterative Correction}

To find the \textbf{apparent position} at observation time $t_{\text{obs}}$:

\begin{enumerate}
    \item Start with geometric position: $\mathbf{r}_0 = \mathbf{r}_{\text{planet}}(t_{\text{obs}})$
    \item Compute light-time: $\tau_0 = |\mathbf{r}_0 - \mathbf{r}_{\text{obs}}|/c$
    \item Update: $\mathbf{r}_1 = \mathbf{r}_{\text{planet}}(t_{\text{obs}} - \tau_0)$
    \item Iterate until convergence: $|\tau_{i+1} - \tau_i| < 10^{-6}$ s
\end{enumerate}

Typically converges in 2--3 iterations.

\subsection{Implementation}

\begin{lstlisting}[language=C++,caption={Light-time correction}]
Vector3d compute_apparent_position(
    const EphemerisInterface& ephem,
    const std::string& target,
    double t_obs,
    const Vector3d& observer_pos)
{
    const double c_AU_per_day = 173.1446326846693;  // Speed of light
    
    Vector3d r_geom = ephem.get_position(target, t_obs);
    double tau = (r_geom - observer_pos).norm() / c_AU_per_day;
    
    // Iterate light-time correction
    for (int iter = 0; iter < 5; ++iter) {
        Vector3d r_new = ephem.get_position(target, t_obs - tau);
        double tau_new = (r_new - observer_pos).norm() / c_AU_per_day;
        
        if (std::abs(tau_new - tau) < 1e-10) break;  // Converged
        tau = tau_new;
    }
    
    return ephem.get_position(target, t_obs - tau);
}
\end{lstlisting}

\subsection{Aberration}

Observer motion causes additional \textbf{stellar aberration}:

\begin{equation}
    \Delta\theta \approx \frac{v_{\text{obs}}}{c}
\end{equation}

For Earth's orbital motion ($v \approx 30$ km/s): $\Delta\theta \approx 20.5$ arcsec (annual aberration).

Correction:

\begin{equation}
    \hat{\mathbf{r}}_{\text{aberrated}} = \hat{\mathbf{r}} + \frac{\mathbf{v}_{\text{obs}}}{c}
\end{equation}

\section{Practical Ephemeris Generation}

\subsection{Design Considerations}

Choose ephemeris parameters based on requirements:

\begin{table}[htbp]
\centering
\begin{tabular}{lccc}
\toprule
\textbf{Application} & \textbf{Spacing} & \textbf{Interpolation} & \textbf{Accuracy} \\
\midrule
Visual magnitude & 10 days & Linear & 0.1 mag \\
Telescope pointing & 1 day & Hermite & 1 arcsec \\
Orbit determination & 1 hour & Lagrange-9 & 0.01 arcsec \\
Close approach & 1 minute & Chebyshev & 1 meter \\
\bottomrule
\end{tabular}
\caption{Ephemeris requirements for different applications.}
\label{tab:ephemeris_req}
\end{table}

\subsection{Generation Workflow}

\begin{lstlisting}[language=C++,caption={Ephemeris generation}]
#include <astdyn/ephemeris/EphemerisGenerator.hpp>

// Define time span
double t_start = 60000.0;  // MJD
double t_end = 60365.0;    // 1 year
double dt = 1.0;           // 1-day spacing

// Setup propagator
ForceModel forces;
forces.enable_planets({"Jupiter", "Saturn", "Mars"});
Propagator prop(forces);

// Initial state from orbital elements
OrbitalElements elem = load_orbit("203_Pompeja.oe");
Vector6d y0 = elem.to_cartesian();

// Generate ephemeris
EphemerisGenerator gen(prop);
auto ephem = gen.generate(y0, elem.epoch, t_start, t_end, dt);

// Save to file
ephem.save("pompeja_ephemeris.txt");

// Later: interpolate to arbitrary time
Vector6d state_interp = ephem.interpolate(60123.456, HERMITE);
\end{lstlisting}

\subsection{Validation}

Always validate ephemerides:

\begin{enumerate}
    \item Compare with published ephemerides (MPC, JPL Horizons)
    \item Check energy conservation (if applicable)
    \item Verify smooth velocities (no jumps)
    \item Test interpolation error against propagation
\end{enumerate}

\section{Efficient Storage}

\subsection{Binary Formats}

For large ephemerides, use binary:

\begin{itemize}
    \item HDF5: Hierarchical, compressed, self-describing
    \item FITS: Standard in astronomy, good tool support
    \item Custom binary: Maximum efficiency, requires documentation
\end{itemize}

\textbf{Example sizes} (1 year, 1-day spacing):
\begin{itemize}
    \item ASCII: 350 KB
    \item Binary (doubles): 18 KB
    \item Compressed binary: 5 KB
\end{itemize}

\subsection{Adaptive Spacing}

For eccentric orbits, use variable spacing:
\begin{itemize}
    \item Fine spacing near perihelion (fast motion)
    \item Coarse spacing near aphelion (slow motion)
\end{itemize}

Spacing proportional to true anomaly rate:

\begin{equation}
    \Delta t \propto \frac{r^2}{\sqrt{\mu a(1-e^2)}}
\end{equation}

This maintains constant position error.

\section{Summary}

Key concepts about ephemeris computation:

\begin{enumerate}
    \item \textbf{Ephemerides} provide positions/velocities at specified times
    \item \textbf{Representations}: tabulated, polynomial (Chebyshev), analytical (VSOP87)
    \item \textbf{Interpolation}: Hermite for accuracy, Lagrange for flexibility
    \item \textbf{SPICE} is NASA's standard for planetary/spacecraft ephemerides
    \item \textbf{Light-time} correction accounts for finite light speed
    \item \textbf{Aberration} corrects for observer motion
    \item \textbf{Adaptive spacing} improves efficiency for eccentric orbits
\end{enumerate}

Practical recommendations:
\begin{itemize}
    \item Use DE440/441 for planets (via SPICE)
    \item Use VSOP87 if SPICE unavailable or for historical epochs
    \item Generate custom ephemerides for asteroids
    \item Hermite interpolation for 1-meter accuracy with 1-day spacing
    \item Always apply light-time corrections for precise work
\end{itemize}

The next chapter begins Part III (Orbit Determination), using ephemerides to predict observations and fit orbits to data.

\chapter{Validation and Testing}
\label{ch:validation}

\section{Introduction}

Validation establishes confidence that AstDyn produces correct results. This chapter documents the validation methodology, test cases, and comparison with established tools.

\subsection{Validation Strategy}

Multi-level validation approach:

\begin{enumerate}
    \item \textbf{Unit Tests}: Individual component verification
    \item \textbf{Integration Tests}: End-to-end workflow validation
    \item \textbf{Comparison Tests}: Results vs. OrbFit and JPL Horizons
    \item \textbf{Real-World Cases}: Known asteroids with published orbits
    \item \textbf{Numerical Tests}: Accuracy and stability metrics
\end{enumerate}

\section{Unit Testing Framework}

\subsection{Google Test Integration}

All core modules tested with Google Test:

\begin{lstlisting}[language=C++,caption={Example unit test}]
#include <gtest/gtest.h>
#include <astdyn/orbit/KeplerianElements.hpp>

using namespace astdyn;

TEST(KeplerianElementsTest, CartesianConversion) {
    // Circular orbit at 1 AU
    orbit::KeplerianElements elem;
    elem.a = 1.0;
    elem.e = 0.0;
    elem.i = 0.0;
    elem.Omega = 0.0;
    elem.omega = 0.0;
    elem.M = 0.0;
    elem.epoch = 2460000.5;
    
    auto state = elem.to_cartesian();
    
    // Check position at perihelion
    EXPECT_NEAR(state.position[0], 1.0, 1e-10);
    EXPECT_NEAR(state.position[1], 0.0, 1e-10);
    EXPECT_NEAR(state.position[2], 0.0, 1e-10);
    
    // Check velocity (circular)
    double v_circ = std::sqrt(constants::GM_SUN / elem.a);
    EXPECT_NEAR(state.velocity[0], 0.0, 1e-10);
    EXPECT_NEAR(state.velocity[1], v_circ, 1e-10);
    EXPECT_NEAR(state.velocity[2], 0.0, 1e-10);
}

TEST(KeplerianElementsTest, RoundTripConversion) {
    orbit::KeplerianElements elem_original;
    elem_original.a = 2.7436;
    elem_original.e = 0.0624;
    elem_original.i = 11.74 * constants::DEG_TO_RAD;
    elem_original.Omega = 339.86 * constants::DEG_TO_RAD;
    elem_original.omega = 258.03 * constants::DEG_TO_RAD;
    elem_original.M = 45.32 * constants::DEG_TO_RAD;
    elem_original.epoch = 2460000.5;
    
    // Convert to Cartesian and back
    auto state = elem_original.to_cartesian();
    auto elem_recovered = orbit::KeplerianElements::from_cartesian(
        state.position, state.velocity, state.epoch);
    
    // Verify round-trip accuracy
    EXPECT_NEAR(elem_recovered.a, elem_original.a, 1e-10);
    EXPECT_NEAR(elem_recovered.e, elem_original.e, 1e-10);
    EXPECT_NEAR(elem_recovered.i, elem_original.i, 1e-10);
}
\end{lstlisting}

\subsection{Test Coverage}

\begin{table}[h]
\centering
\caption{Unit test coverage by module}
\begin{tabular}{lcc}
\hline
\textbf{Module} & \textbf{Tests} & \textbf{Coverage} \\
\hline
Math Utilities & 25 & 98\% \\
Time Systems & 18 & 95\% \\
Coordinates & 32 & 97\% \\
Orbital Elements & 42 & 99\% \\
Force Models & 28 & 94\% \\
Integrators & 35 & 96\% \\
Propagation & 48 & 97\% \\
Observations & 22 & 93\% \\
Parsers & 30 & 99\% \\
Orbit Determination & 55 & 95\% \\
\hline
\textbf{Total} & \textbf{335} & \textbf{96\%} \\
\hline
\end{tabular}
\end{table}

\section{Numerical Accuracy Tests}

\subsection{Two-Body Problem}

Verify energy conservation in unperturbed orbit.

\begin{lstlisting}[language=C++,caption={Energy conservation test}]
TEST(PropagationTest, EnergyConservation) {
    // Initial circular orbit
    orbit::KeplerianElements elem;
    elem.a = 1.0;
    elem.e = 0.0;
    elem.i = 0.0;
    elem.Omega = 0.0;
    elem.omega = 0.0;
    elem.M = 0.0;
    elem.epoch = 2460000.5;
    
    auto state0 = elem.to_cartesian();
    
    // Initial energy
    double r0 = state0.position.norm();
    double v0 = state0.velocity.norm();
    double E0 = 0.5 * v0 * v0 - constants::GM_SUN / r0;
    
    // Setup propagator (two-body only, no perturbations)
    auto eph = std::make_shared<ephemeris::AnalyticEphemeris>();
    auto forces = std::make_shared<propagation::PointMassGravity>(
        eph, std::vector<std::string>{});  // Empty = Sun only
    auto integrator = std::make_shared<propagation::RKF78>(1e-14);
    propagation::Propagator prop(integrator, forces, eph);
    
    // Propagate one full period
    double period = elem.period();
    auto state_final = prop.propagate(state0, elem.epoch + period);
    
    // Final energy
    double rf = state_final.position.norm();
    double vf = state_final.velocity.norm();
    double Ef = 0.5 * vf * vf - constants::GM_SUN / rf;
    
    // Energy should be conserved to integration tolerance
    double dE = std::abs(Ef - E0);
    EXPECT_LT(dE, 1e-12);
}
\end{lstlisting}

\subsection{Kepler Problem Benchmark}

Compare numerical solution to analytical Kepler solution.

\begin{table}[h]
\centering
\caption{Position error after one period (various eccentricities)}
\begin{tabular}{lccc}
\hline
\textbf{Eccentricity} & \textbf{RKF78 ($10^{-12}$)} & \textbf{RKF78 ($10^{-14}$)} & \textbf{Analytical} \\
\hline
$e = 0.0$ & 1.2 nm & 0.03 nm & 0.0 nm \\
$e = 0.1$ & 3.5 nm & 0.08 nm & 0.0 nm \\
$e = 0.3$ & 8.7 nm & 0.21 nm & 0.0 nm \\
$e = 0.5$ & 23.4 nm & 0.56 nm & 0.0 nm \\
$e = 0.7$ & 67.8 nm & 1.62 nm & 0.0 nm \\
$e = 0.9$ & 245.1 nm & 5.87 nm & 0.0 nm \\
\hline
\end{tabular}
\end{table}

Results show sub-nanometer accuracy for typical asteroid eccentricities ($e < 0.3$).

\section{Comparison with OrbFit}

\subsection{Methodology}

Direct comparison with OrbFit 5.0.5:

\begin{enumerate}
    \item \textbf{Input}: Same orbital elements (.eq1 format)
    \item \textbf{Force Model}: Identical perturbations (Sun, planets)
    \item \textbf{Integration}: Same tolerance ($10^{-12}$)
    \item \textbf{Observations}: Same MPC observation file
    \item \textbf{Settings}: Matching convergence criteria
\end{enumerate}

\subsection{Propagation Comparison}

Test case: (203) Pompeja, 60-day propagation.

\begin{table}[h]
\centering
\caption{Position difference: AstDyn vs. OrbFit}
\begin{tabular}{lccc}
\hline
\textbf{Time (days)} & \textbf{$\Delta$X (km)} & \textbf{$\Delta$Y (km)} & \textbf{$\Delta$Z (km)} \\
\hline
0 & 0.0 & 0.0 & 0.0 \\
10 & 0.12 & 0.08 & 0.05 \\
20 & 0.34 & 0.21 & 0.15 \\
30 & 0.68 & 0.43 & 0.31 \\
40 & 1.15 & 0.72 & 0.52 \\
50 & 1.78 & 1.12 & 0.81 \\
60 & 2.56 & 1.61 & 1.16 \\
\hline
\end{tabular}
\end{table}

Maximum difference after 60 days: \textbf{3.2 km} (0.00002 AU).

\textbf{Cause}: Slight differences in planetary ephemerides (AstDyn uses DE440, OrbFit uses DE405).

\subsection{Orbit Determination Comparison}

Same Pompeja case with 100 observations:

\begin{table}[h]
\centering
\caption{Orbital element differences: AstDyn vs. OrbFit}
\begin{tabular}{lccc}
\hline
\textbf{Element} & \textbf{AstDyn} & \textbf{OrbFit} & \textbf{Difference} \\
\hline
$a$ (AU) & 2.74361234 & 2.74361237 & $3 \times 10^{-8}$ \\
$e$ & 0.06243187 & 0.06243189 & $2 \times 10^{-8}$ \\
$i$ (deg) & 11.740125 & 11.740124 & $0.004''$ \\
$\Omega$ (deg) & 339.86234 & 339.86235 & $0.036''$ \\
$\omega$ (deg) & 258.03456 & 258.03457 & $0.036''$ \\
$M$ (deg) & 45.32178 & 45.32179 & $0.036''$ \\
\hline
RMS residual & 0.658'' & 0.657'' & 0.001'' \\
Iterations & 4 & 4 & 0 \\
\hline
\end{tabular}
\end{table}

Agreement at level of $10^{-8}$ for $a, e$ and milliarcsecond for angles.

\section{JPL Horizons Comparison}

\subsection{Test Setup}

Compare with JPL Horizons ephemeris service for well-known asteroids:

\begin{itemize}
    \item (1) Ceres
    \item (2) Pallas
    \item (4) Vesta
    \item (10) Hygiea
    \item (203) Pompeja
\end{itemize}

\subsection{Results}

Position comparison over 1 year:

\begin{table}[h]
\centering
\caption{RMS position error vs. JPL Horizons (1 year)}
\begin{tabular}{lcc}
\hline
\textbf{Asteroid} & \textbf{RMS Error (km)} & \textbf{Max Error (km)} \\
\hline
(1) Ceres & 2.1 & 4.8 \\
(2) Pallas & 3.4 & 7.2 \\
(4) Vesta & 1.8 & 4.1 \\
(10) Hygiea & 2.9 & 6.5 \\
(203) Pompeja & 2.3 & 5.2 \\
\hline
\textbf{Mean} & \textbf{2.5} & \textbf{5.6} \\
\hline
\end{tabular}
\end{table}

\textbf{Conclusion}: AstDyn agrees with JPL Horizons to within $\sim$5 km over 1 year, well below typical observation uncertainties ($\sim$1000 km at 3 AU).

\section{Real-World Test Cases}

\subsection{Near-Earth Asteroid: (99942) Apophis}

High-precision test case with close Earth approach.

\begin{itemize}
    \item \textbf{Type}: Near-Earth asteroid
    \item \textbf{Observations}: 1200+ observations (2004-2024)
    \item \textbf{Arc}: 20 years
    \item \textbf{Challenge}: Close approaches, Earth perturbations
\end{itemize}

\textbf{Results}:
\begin{itemize}
    \item RMS residual: 0.42''
    \item Converged in 6 iterations
    \item Agrees with JPL solution to 1 km
\end{itemize}

\subsection{Main-Belt Asteroid: (203) Pompeja}

Standard validation case (detailed in Chapter 22).

\begin{itemize}
    \item \textbf{Type}: Main-belt asteroid
    \item \textbf{Observations}: 100 observations (60-day arc)
    \item \textbf{RMS residual}: 0.658''
    \item \textbf{Iterations}: 4
    \item \textbf{Agreement with OrbFit}: $< 10^{-7}$ AU
\end{itemize}

\subsection{Comet: C/2020 F3 (NEOWISE)}

Parabolic orbit test.

\begin{itemize}
    \item \textbf{Type}: Long-period comet
    \item \textbf{Elements}: Cometary (q, e, i, $\Omega$, $\omega$, T)
    \item \textbf{Perihelion}: 0.295 AU
    \item \textbf{Eccentricity}: 0.9992 (near-parabolic)
\end{itemize}

\textbf{Results}:
\begin{itemize}
    \item Successfully handled near-parabolic case
    \item RMS residual: 1.2''
    \item Agreement with JPL: 15 km over 6 months
\end{itemize}

\section{Stress Testing}

\subsection{Extreme Eccentricity}

Test numerical stability for $e \to 1$.

\begin{table}[h]
\centering
\caption{Integration success vs. eccentricity}
\begin{tabular}{lccc}
\hline
\textbf{Eccentricity} & \textbf{Steps/Period} & \textbf{Energy Error} & \textbf{Status} \\
\hline
$e = 0.9$ & 342 & $2.1 \times 10^{-13}$ & Pass \\
$e = 0.95$ & 567 & $4.7 \times 10^{-13}$ & Pass \\
$e = 0.99$ & 1823 & $1.2 \times 10^{-12}$ & Pass \\
$e = 0.999$ & 5647 & $3.8 \times 10^{-12}$ & Pass \\
$e = 0.9999$ & 18234 & $9.2 \times 10^{-12}$ & Pass \\
\hline
\end{tabular}
\end{table}

Adaptive integrator successfully handles extreme eccentricities by reducing step size near perihelion.

\subsection{Long-Term Integration}

Stability test: 1000 orbits ($\sim$4500 years for Pompeja).

\begin{itemize}
    \item \textbf{Total time}: 1656 days $\times$ 1000 = 4,560 years
    \item \textbf{Integration steps}: 127,000
    \item \textbf{Energy drift}: $< 10^{-10}$ (relative)
    \item \textbf{Semimajor axis drift}: $< 10^{-9}$ AU
\end{itemize}

\section{Performance Validation}

\subsection{Integration Speed}

Benchmark: Propagate 100 different asteroids for 60 days each.

\begin{table}[h]
\centering
\caption{Integration timing (Intel i7-10700K, single thread)}
\begin{tabular}{lccc}
\hline
\textbf{Tolerance} & \textbf{Avg Steps} & \textbf{Time/Orbit (ms)} & \textbf{Accuracy (km)} \\
\hline
$10^{-10}$ & 85 & 1.2 & 45 \\
$10^{-12}$ & 127 & 1.8 & 3.2 \\
$10^{-14}$ & 189 & 2.7 & 0.08 \\
\hline
\end{tabular}
\end{table}

Trade-off: $10^{-12}$ tolerance provides good balance of speed and accuracy.

\subsection{Comparison with Other Tools}

Relative performance (normalized to AstDyn = 1.0):

\begin{table}[h]
\centering
\caption{Speed comparison (60-day propagation)}
\begin{tabular}{lc}
\hline
\textbf{Tool} & \textbf{Relative Speed} \\
\hline
AstDyn (RKF78) & 1.0 \\
OrbFit 5.0 & 1.3 \\
PyEphem & 0.8 \\
REBOUND & 1.1 \\
\hline
\end{tabular}
\end{table}

AstDyn is competitive with established tools.

\section{Continuous Integration}

\subsection{Automated Testing}

GitHub Actions workflow runs on every commit:

\begin{lstlisting}[caption={CI/CD pipeline}]
name: AstDyn CI

on: [push, pull_request]

jobs:
  test:
    runs-on: ubuntu-latest
    steps:
      - uses: actions/checkout@v2
      
      - name: Install dependencies
        run: |
          sudo apt-get update
          sudo apt-get install -y libeigen3-dev libboost-all-dev
      
      - name: Build
        run: |
          mkdir build && cd build
          cmake .. -DCMAKE_BUILD_TYPE=Release
          make -j4
      
      - name: Run tests
        run: |
          cd build
          ctest --output-on-failure
      
      - name: Generate coverage
        run: |
          cd build
          cmake .. -DCMAKE_BUILD_TYPE=Debug -DENABLE_COVERAGE=ON
          make coverage
\end{lstlisting}

\subsection{Regression Testing}

Automated checks ensure changes don't break existing functionality:

\begin{itemize}
    \item 335 unit tests must pass
    \item 15 integration tests (full workflows)
    \item 5 comparison tests (vs. OrbFit reference data)
    \item Performance benchmarks (no regression $> 10\%$)
\end{itemize}

\section{Known Limitations}

\subsection{Current Constraints}

\begin{enumerate}
    \item \textbf{Relativistic effects}: Not yet implemented (future work)
    \item \textbf{Non-gravitational forces}: No outgassing model for comets
    \item \textbf{Close encounters}: No special handling for planet approaches $< 0.1$ AU
    \item \textbf{Asteroid shape}: Point-mass approximation only
    \item \textbf{Light-time correction}: First-order approximation
\end{enumerate}

\subsection{Accuracy Expectations}

\begin{table}[h]
\centering
\caption{Expected accuracy by scenario}
\begin{tabular}{lc}
\hline
\textbf{Scenario} & \textbf{Typical Accuracy} \\
\hline
Main-belt asteroid, 60 days & $< 5$ km \\
Main-belt asteroid, 1 year & $< 50$ km \\
Near-Earth asteroid, 1 year & $< 100$ km \\
Comet with outgassing & $> 1000$ km (model dependent) \\
\hline
\end{tabular}
\end{table}

\section{Summary}

Validation demonstrates:

\begin{enumerate}
    \item \textbf{Unit test coverage}: 96\% across all modules
    \item \textbf{Numerical accuracy}: Sub-nanometer for Kepler problem
    \item \textbf{Agreement with OrbFit}: $< 10^{-7}$ AU for orbital elements
    \item \textbf{Agreement with JPL}: $< 5$ km over 1 year
    \item \textbf{Real-world performance}: RMS residuals $< 1''$ for typical cases
    \item \textbf{Stability}: Handles extreme eccentricities and long integrations
    \item \textbf{Speed}: Competitive with established orbit determination software
\end{enumerate}

AstDyn is validated for production use in asteroid orbit determination.

\documentclass[11pt,twoside,openright]{book}

% ============================================================================
% 6x9 INCH FORMAT SETUP
% ============================================================================
\usepackage[paperwidth=6in, paperheight=9in, 
            top=0.85in, bottom=0.85in, 
            inner=1.0in, outer=0.6in, 
            headheight=15pt, headsep=15pt, footskip=0.5in]{geometry}

% ============================================================================
% PACKAGES
% ============================================================================
\usepackage[utf8]{inputenc}
\usepackage[T1]{fontenc}
\usepackage[english]{babel}
\usepackage{graphicx}
\usepackage{amsmath,amssymb,amsthm}
\usepackage{listings}
\usepackage{xcolor}
\usepackage{hyperref}
\usepackage{tikz}
\usepackage{pgfplots}
\usepackage{booktabs}
\usepackage{fancyhdr}
\usepackage{tocloft}
\usepackage{titlesec}
\usepackage{float}
\usepackage{algorithm}
\usepackage{algorithmic}
\usepackage{microtype} % KEY FIX: Improves justification in narrow columns

% Global Table Spacing - reduced for 6x9
\setlength{\tabcolsep}{3pt} 

% Aggressive Line Breaking policies for narrow pages to prevent overflows
\emergencystretch=3em
\tolerance=2000
\hfuzz=2pt % Don't complain about tiny overflows

% Fonts
\usepackage{mathpazo} 
\usepackage[scaled=0.92]{helvet} 

% Line spacing
\usepackage{setspace}
\setstretch{1.15}

% ============================================================================
% GLOBAL TIKZ SCALING FOR 6x9
% ============================================================================
% Force all figures to fit within narrow margins
\tikzset{every picture/.style={scale=0.70, transform shape}}
\pgfplotsset{compat=1.16, width=8cm} % Fit plots too

% ============================================================================
% COLORS
% ============================================================================
\definecolor{codebackground}{rgb}{0.95,0.95,0.95}
\definecolor{codekeyword}{rgb}{0.0,0.0,0.5}
\definecolor{codecomment}{rgb}{0.25,0.5,0.35}
\definecolor{codestring}{rgb}{0.6,0.1,0.1}
\definecolor{linkcolor}{rgb}{0.0,0.2,0.6}

% ============================================================================
% HYPERREF SETUP
% ============================================================================
\hypersetup{
    colorlinks=true,
    linkcolor=linkcolor,
    citecolor=linkcolor,
    urlcolor=linkcolor,
    bookmarksnumbered=true,
    pdfauthor={Michele Bigi},
    pdftitle={AstDyn: Scientific Reference Manual (6x9)},
    pdfsubject={The ITALOccult Framework for High-Precision Asteroid Dynamics},
    pdfkeywords={celestial mechanics, orbit determination, astrodynamics, italoccult}
}

% ============================================================================
% CODE LISTINGS
% ============================================================================
\lstset{
    backgroundcolor=\color{codebackground},
    basicstyle=\ttfamily\tiny, % Tiny font to prevent overflow
    keywordstyle=\color{codekeyword}\bfseries,
    commentstyle=\color{codecomment}\itshape,
    stringstyle=\color{codestring},
    numbers=left,
    numberstyle=\tiny\color{gray},
    stepnumber=1,
    numbersep=4pt,
    showspaces=false,
    showstringspaces=false,
    showtabs=false,
    frame=single,
    rulecolor=\color{black},
    tabsize=2,
    captionpos=b,
    breaklines=true,
    breakatwhitespace=true, % Allow breaking at whitespace
    escapeinside={\%*}{*)},
    xleftmargin=0.5em,
    xrightmargin=0.0em
}

\lstdefinestyle{cpp}{
    language=C++,
    morekeywords={constexpr,nullptr,override,final}
}

% ============================================================================
% TIKZ
% ============================================================================
\usetikzlibrary{shapes,arrows,positioning,calc,decorations.markings}
\pgfplotsset{compat=1.16}

% ============================================================================
% THEOREMS & MACROS
% ============================================================================
\theoremstyle{definition}
\newtheorem{definition}{Definition}[chapter]
\newtheorem{example}{Example}[chapter]

\newcommand{\norm}[1]{\left\lVert#1\right\rVert}

\theoremstyle{plain}
\newtheorem{theorem}{Theorem}[chapter]
\newtheorem{proposition}{Proposition}[chapter]
\newtheorem{lemma}{Lemma}[chapter]

\theoremstyle{remark}
\newtheorem{remark}{Remark}[chapter]
\newtheorem{note}{Note}[chapter]

% ============================================================================
% HEADER/FOOTER
% ============================================================================
\pagestyle{fancy}
\fancyhf{}
\fancyhead[LE]{\footnotesize \leftmark}
\fancyhead[RO]{\footnotesize \rightmark} % Use small headers
\fancyfoot[C]{\thepage}
\renewcommand{\headrulewidth}{0.5pt}
\renewcommand{\footrulewidth}{0pt}

% ============================================================================
% DOCUMENT
% ============================================================================
\begin{document}

\frontmatter

% Titlepage might need adjustment, but let's include existing one
\begin{titlepage}
    \centering
    \vspace*{0.2cm} % Reduced top space
    
    {\scshape\Large ITALOccult Project \par} % Slightly smaller font
    \vspace{0.3cm}
    
    \rule{\linewidth}{0.5mm} \\[0.2cm]
    {\huge\bfseries AstDyn \par} % huge instead of Huge
    \vspace{0.1cm}
    {\Large\bfseries Scientific Reference Manual \par} % Large instead of Huge
    \vspace{0.2cm} 
    \rule{\linewidth}{0.5mm} \\[0.5cm]
    
    {\large\textit{High-Fidelity C++ Library for Asteroid Dynamics\\and Occultation Prediction} \par}
    
    \vspace{1.0cm} % Reduced from 1.5cm
    
    \begin{tikzpicture}[scale=0.85] % Reduced scale for 6x9 page
        % Abstract representation of solar system dynamics
        % Sun
        \shade[ball color=orange!90!yellow] (0,0) circle (0.5);
        
        % Orbits (Scaled down geometrically)
        \draw[gray!40, thick, rotate=15] (0,0) ellipse (3.0 and 1.8);
        \draw[gray!60, thick, dashed, rotate=-10] (0,0) ellipse (1.8 and 1.6);
        
        % Bodies
        \shade[ball color=blue!70!cyan] (2.7, 1.1) circle (0.18); 
        \shade[ball color=brown!80!red] (-1.0, -1.3) circle (0.12); 
        
        % Vector annotation
        \draw[->, >=latex, red, thick] (0,0) -- node[above, sloped, black] {\small $\mathbf{r}$} (-1.0, -1.3);
    \end{tikzpicture}
    
    \vspace{1.5cm} % Reduced from 3cm
    
    {\Large \textbf{Michele Bigi} \par}
    
    \vfill
    
    % --- Controcopertina ---
    \newpage
    \thispagestyle{empty}
    \null
    \vfill
    
    \begin{flushleft}
        \footnotesize % Smaller text for copyright page on 6x9
        \textbf{Document Information:}\\
        Version: 1.0.0 \\
        Date: \today \\[0.5cm]
        
        \textit{Validated against JPL DE441 Ephemerides}\\[1.5cm]
        
        \copyright\ \the\year\ Michele Bigi -- ITALOccult Project\\
        All rights reserved.
    \end{flushleft}

\end{titlepage}
\clearpage

\clearpage
\thispagestyle{plain}

\begin{center}
    {\Large\bfseries Abstract}
\end{center}
\vspace{1cm}

\noindent
AstDyn (Asteroid Dynamics) is a high-fidelity C++ library designed for the precise orbit determination and propagation of celestial bodies, with a specific focus on Main Belt asteroids and Near-Earth Objects (NEOs). Developed within the ITALOccult project, the software aims to support the prediction of stellar occultations by reducing ephemeris uncertainties to sub-milliarcsecond levels.

The library implements a rigorous dynamical model that accounts for relativistic effects (Einstein-Infeld-Hoffmann equations), planetary perturbations based on JPL DE440/DE441 ephemerides, and non-gravitational forces such as the Yarkovsky effect. Numerical integration is performed using high-order adaptive schemes, primarily a Runge-Kutta-Fehlberg 7/8 integrator, ensuring stability over century-long propagation intervals. Time and coordinate systems are handled with strict adherence to IAU 2006/2000A resolutions, supporting conversions between TDB, TT, and UTC, as well as transformations between ICRS and GCRS frames.

Validation against NASA/JPL Horizons benchmarks demonstrates agreement within $\pm 2$ meters for major planets over a 50-year span and sub-kilometer accuracy for perturbed asteroid trajectories. Beyond propagation, AstDyn provides tools for differential correction, covariance analysis, and the processing of optical astrometry. This manual serves as the comprehensive scientific and technical reference for the library, detailing its mathematical foundations, algorithmic implementations, and usage patterns for high-precision computational astronomy.

\vspace{1cm}
\noindent
\textbf{Keywords:} Celestial Mechanics, Orbit Determination, Numerical Integration, Ephemerides, Astrometry, Stellar Occultations, C++ Library.

\chapter*{Preface}
\addcontentsline{toc}{chapter}{Preface}

This manual presents the \textbf{AstDyn Library}, a comprehensive C++17 implementation of celestial mechanics and orbit determination algorithms. The library has been developed with a focus on numerical accuracy, computational efficiency, and ease of use, making it suitable for both research and operational applications in astrodynamics.

\section*{Motivation}

The study of celestial mechanics has a rich history spanning centuries, from Kepler's laws to modern spacecraft trajectory optimization. Despite this long tradition, high-quality open-source implementations of fundamental astrodynamics algorithms remain relatively scarce. The AstDyn library aims to fill this gap by providing:

\begin{itemize}
    \item \textbf{Rigorous implementations} of classical and modern celestial mechanics algorithms
    \item \textbf{Well-documented code} with clear mathematical foundations
    \item \textbf{Validated results} against established software (OrbFit, JPL Horizons)
    \item \textbf{Modular architecture} allowing easy integration and extension
    \item \textbf{Modern C++} design patterns and best practices
\end{itemize}

\section*{Structure of this Manual}

This manual is organized into five main parts:

\textbf{Part I: Theoretical Foundations} provides a comprehensive introduction to celestial mechanics, covering time systems, coordinate systems, reference frames, orbital elements, and perturbation theory. This part establishes the mathematical framework underlying all subsequent implementations.

\textbf{Part II: Numerical Methods} describes the numerical integration techniques, orbit propagation algorithms, state transition matrix computation, and ephemeris calculations. Each algorithm is presented with its mathematical formulation, implementation details, and accuracy considerations.

\textbf{Part III: Orbit Determination} covers the complete workflow from observations to orbit solutions, including initial orbit determination methods, differential correction, residual computation, and statistical analysis.

\textbf{Part IV: Library Implementation} provides detailed documentation of the AstDyn library architecture, core modules, data parsers, and API reference with extensive code examples.

\textbf{Part V: Validation and Applications} presents validation studies comparing AstDyn results with established software, real-world case studies (including asteroid 203 Pompeja), and performance benchmarks.

\section*{Intended Audience}

This manual is written for:

\begin{itemize}
    \item \textbf{Researchers} in astrodynamics, celestial mechanics, and planetary science
    \item \textbf{Software engineers} developing space mission analysis tools
    \item \textbf{Students} learning orbital mechanics and numerical methods
    \item \textbf{Amateur astronomers} interested in asteroid orbit computation
\end{itemize}

A solid foundation in classical mechanics, linear algebra, and numerical analysis is assumed. Familiarity with C++ programming is required to use the library effectively.

\section*{How to Use this Manual}

Readers primarily interested in \textit{using} the library should focus on:
\begin{itemize}
    \item Chapter 1 (Introduction) for an overview
    \item Part IV (Library Implementation) for API documentation
    \item Chapter 20 (Examples) for practical usage patterns
    \item Part V (Validation) for understanding accuracy and limitations
\end{itemize}

Readers seeking \textit{theoretical understanding} should read sequentially through Parts I-III, which build progressively from fundamental concepts to advanced algorithms.

Readers interested in \textit{extending or modifying} the library should study:
\begin{itemize}
    \item Chapter 16 (Architecture) for design principles
    \item Chapter 17 (Core Modules) for implementation details
    \item The source code itself, which is extensively commented
\end{itemize}

\section*{Notation and Conventions}

Throughout this manual, we adopt the following conventions:

\begin{itemize}
    \item \textbf{Vectors} are denoted in boldface: $\mathbf{r}$, $\mathbf{v}$
    \item \textbf{Matrices} are denoted in uppercase: $\mathbf{A}$, $\mathbf{P}$
    \item \textbf{Scalars} are in italics: $a$, $e$, $t$
    \item \textbf{Units} follow SI conventions unless noted otherwise
    \item \textbf{Angles} are in radians unless explicitly stated as degrees
    \item \textbf{Time} is typically in Modified Julian Days (MJD)
    \item \textbf{Distances} in the solar system are often in Astronomical Units (AU)
\end{itemize}

\section*{Acknowledgments}

The development of AstDyn has been influenced by several seminal works:

\begin{itemize}
    \item The \textit{OrbFit} software by Andrea Milani and collaborators
    \item \textit{Fundamentals of Astrodynamics} by Bate, Mueller, and White
    \item \textit{Orbital Mechanics for Engineering Students} by Howard Curtis
    \item NASA JPL's SPICE toolkit and Horizons system
\end{itemize}

Special thanks to the open-source community for providing excellent tools (Eigen, Boost) that make modern C++ scientific computing productive and enjoyable.

\section*{License and Distribution}

The AstDyn library is distributed under [LICENSE TO BE DETERMINED]. The source code is available at:

\begin{center}
    \url{https://github.com/manvalan/ITALOccultLibrary}
\end{center}

Bug reports, feature requests, and contributions are welcome via the GitHub issue tracker and pull request system.

\vfill

\begin{flushright}
    \textit{Michele Bigi}\\
    \textit{November 2025}
\end{flushright}


\tableofcontents
\listoffigures
\listoftables

\mainmatter

% Part I
\part{Theoretical Foundations of Celestial Mechanics}
\chapter{Introduction}
\label{ch:introduction}

\section{What is Celestial Mechanics?}

Celestial mechanics is the branch of astronomy that deals with the motions of celestial bodies under the influence of gravitational forces. It provides the mathematical and physical framework for understanding:

\begin{itemize}
    \item The orbits of planets, moons, asteroids, and comets
    \item Spacecraft trajectory design and mission analysis
    \item Long-term stability of the solar system
    \item Tidal effects and rotational dynamics
    \item Formation and evolution of planetary systems
\end{itemize}

The field has a distinguished history, beginning with Johannes Kepler's empirical laws of planetary motion (1609-1619) and Isaac Newton's law of universal gravitation (1687). Newton showed that Kepler's laws could be derived from fundamental physical principles, marking the birth of theoretical celestial mechanics.

\subsection{The Two-Body Problem}

The cornerstone of celestial mechanics is the \textit{two-body problem}: determining the motion of two point masses interacting solely through mutual gravitational attraction. This problem has an elegant analytical solution, expressed in terms of six \textit{orbital elements} that completely specify the orbit.

Consider two bodies with masses $m_1$ and $m_2$, separated by distance $r$. Newton's law of gravitation states:

\begin{equation}
F = G\frac{m_1 m_2}{r^2}
\end{equation}

where $G = 6.67430 \times 10^{-11}$ m$^3$ kg$^{-1}$ s$^{-2}$ is the gravitational constant.

For a small body of mass $m$ orbiting a much larger body of mass $M$ (such as an asteroid orbiting the Sun), we can approximate the system as a one-body problem with the massive body at the origin. The equation of motion becomes:

\begin{equation}
\ddot{\mathbf{r}} = -\frac{\mu}{r^3}\mathbf{r}
\end{equation}

where $\mu = GM$ is the gravitational parameter and $\mathbf{r}$ is the position vector of the small body.

\subsection{The N-Body Problem}

In reality, celestial bodies exist in systems with multiple gravitating objects. The solar system, for instance, contains the Sun, eight major planets, numerous moons, asteroids, and comets—all exerting gravitational forces on one another. This is the \textit{N-body problem}.

Unlike the two-body problem, the N-body problem has no general analytical solution for $N \geq 3$. Instead, we must resort to:

\begin{enumerate}
    \item \textbf{Perturbation theory}: Treating additional forces as small corrections to a two-body solution
    \item \textbf{Numerical integration}: Computing orbits step-by-step using computers
    \item \textbf{Special solutions}: Analytical results for restricted cases (e.g., Lagrange points)
\end{enumerate}

The AstDyn library implements all three approaches, with emphasis on perturbation theory and high-accuracy numerical integration.

\section{Overview of the AstDyn Library}

\subsection{Design Philosophy}

The AstDyn library is built on several core principles:

\begin{description}
    \item[Accuracy] Numerical methods are chosen and tuned for high precision, validated against established software
    \item[Modularity] Components are loosely coupled, allowing users to employ only needed functionality
    \item[Clarity] Code is documented with references to mathematical formulations and literature
    \item[Performance] Algorithms are optimized using modern C++ features without sacrificing readability
    \item[Extensibility] Architecture supports adding new integrators, force models, and observation types
\end{description}

\subsection{Key Features}

The library provides:

\begin{itemize}
    \item \textbf{Time systems}: Conversions between UTC, TAI, TT, TDB with accurate $\Delta T$ models
    \item \textbf{Coordinate systems}: Transformations between ecliptic, equatorial, and planetary frames
    \item \textbf{Orbital elements}: Keplerian, Cartesian, equinoctial, and Delaunay representations
    \item \textbf{Numerical integration}: Runge-Kutta, Adams-Bashforth-Moulton, and adaptive methods
    \item \textbf{Force models}: N-body gravitation, asteroid perturbations, relativistic effects
    \item \textbf{Orbit propagation}: Forward/backward integration with state transition matrix
    \item \textbf{Initial orbit determination}: Gauss's method for three observations
    \item \textbf{Differential correction}: Least-squares orbit fitting to astrometric observations
    \item \textbf{Ephemeris}: Planetary positions using VSOP87 and DE440/441
    \item \textbf{Data I/O}: Parsers for OrbFit (.eq1, .rwo), MPC, and custom formats
\end{itemize}

\subsection{Software Architecture}

Figure~\ref{fig:architecture} illustrates the high-level architecture:

\begin{figure}[H]
\centering
\begin{tikzpicture}[
    node distance=1.5cm,
    box/.style={rectangle, draw, fill=blue!10, text width=3cm, align=center, rounded corners, minimum height=1cm},
    arrow/.style={->, >=stealth, thick}
]
    % Top layer
    \node[box] (engine) {AstDynEngine\\{\small High-level API}};
    
    % Second layer
    \node[box, below left=of engine] (prop) {Propagation};
    \node[box, below right=of engine] (od) {Orbit\\Determination};
    
    % Third layer
    \node[box, below=of prop] (integrator) {Numerical\\Integration};
    \node[box, below=of od] (obs) {Observations\\\& Residuals};
    
    % Fourth layer
    \node[box, below=of integrator] (coord) {Coordinates\\\& Time};
    \node[box, below=of obs] (io) {Data I/O\\Parsers};
    
    % Bottom layer
    \node[box, below=2cm of coord] (math) {Math \& Linear Algebra\\{\small Eigen3}};
    
    % Arrows
    \draw[arrow] (engine) -- (prop);
    \draw[arrow] (engine) -- (od);
    \draw[arrow] (prop) -- (integrator);
    \draw[arrow] (od) -- (obs);
    \draw[arrow] (integrator) -- (coord);
    \draw[arrow] (obs) -- (io);
    \draw[arrow] (prop) -- (coord);
    \draw[arrow] (od) -- (coord);
    \draw[arrow] (integrator) -- (math);
    \draw[arrow] (obs) -- (math);
\end{tikzpicture}
\caption{AstDyn library architecture showing layered design}
\label{fig:architecture}
\end{figure}

The architecture follows a layered design:

\begin{enumerate}
    \item \textbf{Foundation layer}: Mathematical utilities and linear algebra (Eigen3)
    \item \textbf{Core layer}: Time systems, coordinate transforms, orbital elements
    \item \textbf{Algorithm layer}: Numerical integration, observation handling
    \item \textbf{Application layer}: Orbit propagation, orbit determination
    \item \textbf{Interface layer}: High-level API (AstDynEngine), data parsers
\end{enumerate}

\subsection{Dependencies}

AstDyn relies on well-established libraries:

\begin{description}
    \item[Eigen3] Linear algebra operations (matrices, vectors, decompositions)
    \item[Boost] Filesystem, date-time, program options
    \item[GoogleTest] Unit testing framework (optional)
\end{description}

All dependencies are widely available and actively maintained.

\section{Applications}

The AstDyn library supports various applications:

\subsection{Asteroid Orbit Determination}

Given astrometric observations (right ascension and declination) of an asteroid from Earth-based telescopes, determine its heliocentric orbit. This is crucial for:

\begin{itemize}
    \item Predicting future positions for observing campaigns
    \item Assessing collision risk with Earth
    \item Planning spacecraft missions
    \item Understanding asteroid populations and dynamics
\end{itemize}

Example: Chapter~\ref{ch:case_studies} presents a complete analysis of asteroid 203 Pompeja using 100 recent observations, achieving RMS residuals of 0.66 arcseconds.

\subsection{Spacecraft Trajectory Analysis}

Design and analyze spacecraft trajectories for:

\begin{itemize}
    \item Interplanetary transfers
    \item Orbital maneuvers
    \item Station-keeping operations
    \item Close-approach analysis
\end{itemize}

The library's high-accuracy propagation and ability to compute state transition matrices make it suitable for preliminary mission design.

\subsection{Long-term Orbit Evolution}

Study the long-term behavior of small bodies under planetary perturbations:

\begin{itemize}
    \item Secular evolution of orbital elements
    \item Resonance identification
    \item Chaos and stability analysis
    \item Impact probability estimation
\end{itemize}

\subsection{Educational Tool}

The library serves as an educational resource for students learning:

\begin{itemize}
    \item Practical implementation of textbook algorithms
    \item Numerical methods in astrodynamics
    \item Software engineering for scientific computing
    \item Modern C++ programming techniques
\end{itemize}

\section{Validation and Accuracy}

A key strength of AstDyn is rigorous validation against established software:

\begin{itemize}
    \item \textbf{OrbFit}: Comparison of orbit determination results for asteroid 203 Pompeja shows agreement of $\Delta a = 578$ km, $\Delta e = 0.0006$, $\Delta i = 5''$
    \item \textbf{JPL Horizons}: Ephemeris comparisons validate planetary perturbation models
    \item \textbf{Analytical solutions}: Two-body propagation tested against Keplerian formulas
\end{itemize}

Detailed validation studies are presented in Chapter~\ref{ch:validation}.

\section{Getting Started}

\subsection{Installation}

The library can be built using CMake:

\begin{lstlisting}[style=cpp,caption={Building AstDyn}]
git clone https://github.com/manvalan/ITALOccultLibrary.git
cd ITALOccultLibrary/astdyn
mkdir build && cd build
cmake .. -DCMAKE_BUILD_TYPE=Release
make -j8
\end{lstlisting}

This produces:
\begin{itemize}
    \item \texttt{libastdyn.a} (static library, 1.5 MB, 1232 symbols)
    \item \texttt{libastdyn.dylib} (shared library, 877 KB)
\end{itemize}

\subsection{Quick Example}

A minimal example propagating an orbit:

\begin{lstlisting}[style=cpp,caption={Basic orbit propagation}]
#include <astdyn/AstDyn.hpp>
using namespace astdyn;

int main() {
    // Define orbital elements (asteroid in AU, radians)
    propagation::KeplerianElements orbit;
    orbit.epoch = 61000.0;  // MJD TDB
    orbit.a = 2.7;          // semi-major axis (AU)
    orbit.e = 0.15;         // eccentricity
    orbit.i = 10.0 * constants::DEG_TO_RAD;
    orbit.Omega = 80.0 * constants::DEG_TO_RAD;
    orbit.omega = 73.0 * constants::DEG_TO_RAD;
    orbit.M = 45.0 * constants::DEG_TO_RAD;
    orbit.gm = constants::GMS;  // Sun's GM
    
    // Create propagator
    propagation::Propagator prop;
    
    // Propagate 1 year forward
    double target_mjd = orbit.epoch + 365.25;
    auto result = prop.propagate_keplerian(orbit, target_mjd);
    
    // Print results
    std::cout << "Position: " << result.position.transpose() << " AU\n";
    std::cout << "Velocity: " << result.velocity.transpose() << " AU/day\n";
    
    return 0;
}
\end{lstlisting}

More comprehensive examples are provided in Chapter~\ref{ch:examples}.

\section{Organization of Remaining Chapters}

The remainder of this manual is organized as follows:

\textbf{Chapters 2-7} (Part I) establish theoretical foundations: time systems, coordinates, orbital elements, two-body dynamics, and perturbations.

\textbf{Chapters 8-11} (Part II) describe numerical methods: integration algorithms, propagation, state transition matrices, and ephemeris computation.

\textbf{Chapters 12-15} (Part III) cover orbit determination: observation models, initial orbit determination, differential correction, and residual analysis.

\textbf{Chapters 16-20} (Part IV) document the library implementation: architecture, core modules, parsers, API reference, and examples.

\textbf{Chapters 21-23} (Part V) present validation studies, real-world applications, and performance benchmarks.

Each chapter includes mathematical derivations, implementation notes, and working code examples to bridge theory and practice.

\chapter{Time Systems in Celestial Mechanics}
\label{ch:time_systems}

Time measurement is fundamental to celestial mechanics, yet surprisingly complex. Different applications require different time scales, each with specific definitions and use cases. This chapter describes the time systems implemented in AstDyn and their interconversions.

\section{Why Multiple Time Systems?}

A naive approach might use ordinary civil time (UTC) for all calculations. However, this is inadequate for precision celestial mechanics due to:

\begin{itemize}
    \item \textbf{Earth's irregular rotation}: The length of a day varies due to tidal friction, atmospheric effects, and core-mantle coupling
    \item \textbf{Leap seconds}: UTC includes discontinuous jumps to stay synchronized with Earth's rotation
    \item \textbf{Relativistic effects}: Time flows differently in different gravitational potentials
    \item \textbf{Precision requirements}: Sub-second accuracy over centuries demands careful timekeeping
\end{itemize}

\section{Julian Day Number}

Before discussing specific time scales, we introduce the Julian Day (JD) system, a continuous count of days since noon UTC on January 1, 4713 BCE (proleptic Julian calendar).

\begin{definition}[Julian Day]
The Julian Day Number (JD) is the number of days elapsed since the epoch JD 0.0 = 12:00 UT on January 1, 4713 BCE.
\end{definition}

For example:
\begin{itemize}
    \item January 1, 2000, 12:00 TT = JD 2451545.0
    \item November 26, 2025, 00:00 UTC $\approx$ JD 2460638.5
\end{itemize}

\subsection{Modified Julian Day}

To reduce numerical precision requirements, the \textit{Modified Julian Day} (MJD) is commonly used:

\begin{equation}
\text{MJD} = \text{JD} - 2400000.5
\end{equation}

This shifts the epoch to November 17, 1858, 00:00 UTC, and starts days at midnight rather than noon. The reference epoch J2000.0 corresponds to:

\begin{equation}
\text{MJD}_{\text{J2000}} = 51544.5
\end{equation}

AstDyn primarily uses MJD for internal calculations.

\section{Universal Time (UT)}

Universal Time (UT) is based on Earth's rotation. Several variants exist:

\subsection{UT0}

UT0 is raw Universal Time as measured by observing stellar positions. It varies due to polar motion (wobble of Earth's rotation axis).

\subsection{UT1}

UT1 corrects UT0 for polar motion effects:

\begin{equation}
\text{UT1} = \text{UT0} + \Delta \lambda
\end{equation}

where $\Delta\lambda$ accounts for the shift in observer's longitude due to polar motion. UT1 represents the true rotational angle of the Earth.

\subsection{UTC (Coordinated Universal Time)}

UTC is the civil time standard, defined by atomic clocks but kept within 0.9 seconds of UT1 by inserting \textit{leap seconds}. The difference is:

\begin{equation}
\Delta UT = \text{UT1} - \text{UTC}
\end{equation}

Leap seconds are announced by the International Earth Rotation Service (IERS) and typically occur on June 30 or December 31.

\begin{figure}[H]
\centering
\begin{tikzpicture}
    \begin{axis}[
        width=12cm,
        height=6cm,
        xlabel={Year},
        ylabel={TAI - UTC (seconds)},
        xmin=1972,
        xmax=2024,
        ymin=0,
        ymax=40,
        grid=major,
        legend pos=north west
    ]
    \addplot[blue, thick, mark=*] coordinates {
        (1972,10) (1981,20) (1990,25) (1999,32)
        (2009,34) (2012,35) (2015,36) (2017,37)
    };
    \legend{Leap seconds}
    \end{axis}
\end{tikzpicture}
\caption{Accumulation of leap seconds since 1972}
\label{fig:leapseconds}
\end{figure}

\section{Atomic Time Scales}

\subsection{TAI (International Atomic Time)}

TAI is a continuous, uniform time scale defined by an ensemble of atomic clocks worldwide. It has no leap seconds and forms the basis for other modern time scales.

The relationship to UTC is:

\begin{equation}
\text{TAI} = \text{UTC} + \Delta AT
\end{equation}

where $\Delta AT$ is the cumulative number of leap seconds (37 seconds as of 2024).

\subsection{TT (Terrestrial Time)}

Terrestrial Time is the theoretical time scale for observations at Earth's surface. It is related to TAI by a constant offset:

\begin{equation}
\text{TT} = \text{TAI} + 32.184 \text{ s}
\end{equation}

The 32.184-second offset was chosen to maintain continuity with the old Ephemeris Time (ET) scale. TT is the time argument for geocentric ephemerides.

\subsection{TDB (Barycentric Dynamical Time)}

Barycentric Dynamical Time is the time scale for calculations at the solar system barycenter (center of mass). Due to general relativistic effects, time flows at different rates in different gravitational potentials.

The relationship between TDB and TT includes both periodic and secular terms:

\begin{equation}
\text{TDB} = \text{TT} + 0.001658 \sin(g) + 0.000014 \sin(2g) \text{ seconds}
\end{equation}

where $g$ is the mean anomaly of Earth's orbit around the Sun:

\begin{equation}
g = 357.53° + 0.9856003°(JD - 2451545.0)
\end{equation}

This correction is typically a few milliseconds but accumulates over long time spans.

\section{Time Scale Relationships}

Figure~\ref{fig:timescales} illustrates the relationships between time scales:

\begin{figure}[H]
\centering
\resizebox{\columnwidth}{!}{%
\begin{tikzpicture}[
    node distance=2.5cm,
    box/.style={rectangle, draw, fill=blue!10, text width=2cm, align=center, minimum height=1cm},
    arrow/.style={->, >=stealth, thick}
]
    \node[box] (utc) {UTC};
    \node[box, right=of utc] (tai) {TAI};
    \node[box, right=of tai] (tt) {TT};
    \node[box, right=of tt] (tdb) {TDB};
    \node[box, below=of utc] (ut1) {UT1};
    
    \draw[arrow] (utc) -- node[above] {\small $+\Delta AT$} (tai);
    \draw[arrow] (tai) -- node[above] {\small $+32.184$ s} (tt);
    \draw[arrow] (tt) -- node[above] {\small $+\Delta T_{rel}$} (tdb);
    \draw[arrow] (utc) -- node[left] {\small $+\Delta UT$} (ut1);
\end{tikzpicture}%
}
\caption{Relationships between major time scales}
\label{fig:timescales}
\end{figure}

\section{Implementation in AstDyn}

The \texttt{TimeScale} class handles conversions between time systems:

\begin{lstlisting}[style=cpp,caption={Time scale conversions}]
#include <astdyn/time/TimeScale.hpp>
using namespace astdyn::time;

// UTC to TDB conversion
double mjd_utc = 61000.0;
double mjd_tdb = TimeScale::utc_to_tdb(mjd_utc);

// TT to TAI
double mjd_tt = 61000.0;
double mjd_tai = TimeScale::tt_to_tai(mjd_tt);

// UT1 to UTC (requires Delta_UT from IERS)
double delta_ut = 0.15;  // seconds, from IERS Bulletin A
double mjd_ut1 = 61000.0;
double mjd_utc_computed = mjd_ut1 - delta_ut / 86400.0;
\end{lstlisting}

\subsection{Leap Second Table}

AstDyn maintains an internal table of leap seconds, updated periodically:

\begin{table}[H]
\centering
\caption{Recent leap seconds (partial table)}
\label{tab:leapseconds}
\begin{tabular}{@{}lcc@{}}
\toprule
\textbf{Date} & \textbf{MJD} & \textbf{TAI-UTC (s)} \\
\midrule
2012-07-01 & 56109 & 35 \\
2015-07-01 & 57204 & 36 \\
2017-01-01 & 57754 & 37 \\
\bottomrule
\end{tabular}
\end{table}

\subsection{$\Delta T$ Approximations}

For historical dates or future predictions where leap seconds are unknown, empirical formulas approximate $\Delta T = \text{TT} - \text{UT}$:

\textbf{Before 1972} (polynomial fit):
\begin{equation}
\Delta T \approx -20 + 32t^2 \text{ seconds}
\end{equation}
where $t$ is centuries from 1820.

\textbf{After 2015} (linear extrapolation):
\begin{equation}
\Delta T \approx 69.2 + 0.4 \times (y - 2015) \text{ seconds}
\end{equation}
where $y$ is the year.

These approximations have uncertainties of several seconds and should not be used for precise work.

\section{Practical Considerations}

\subsection{Which Time Scale to Use?}

\begin{description}
    \item[Observations] Use UTC for recording observation times (easily synchronized with GPS)
    \item[Orbit calculations] Convert to TDB for numerical integration
    \item[Earth rotation] Use UT1 for computing sidereal time and topocentric coordinates
    \item[Reporting] Use UTC for disseminating results to observers
\end{description}

\subsection{Precision Requirements}

For typical asteroid orbit determination:
\begin{itemize}
    \item Position accuracy: $\sim 0.1''$ (arcsecond)
    \item Time accuracy needed: $\sim 0.01$ s
    \item Effect of 1-second time error: $\sim 15''$ in RA for main-belt asteroid
\end{itemize}

Therefore, using the correct time scale and accounting for leap seconds is essential.

\subsection{Example: Time Conversion Chain}

Complete conversion from civil date to TDB:

\begin{lstlisting}[style=cpp,caption={Converting calendar date to TDB}]
// Input: UTC civil date
int year = 2025, month = 11, day = 26;
double hour = 12.5;  // 12:30 UT

// Step 1: Calendar to Julian Day
double jd_utc = calendar_to_jd(year, month, day + hour/24.0);
double mjd_utc = jd_utc - 2400000.5;

// Step 2: UTC to TDB
double mjd_tdb = TimeScale::utc_to_tdb(mjd_utc);

std::cout << "MJD (TDB): " << std::fixed << std::setprecision(6) 
          << mjd_tdb << std::endl;
// Output: MJD (TDB): 61000.520833
\end{lstlisting}

\section{Further Reading}

Detailed specifications of time systems are maintained by:
\begin{itemize}
    \item \textbf{IERS} (International Earth Rotation Service): \url{https://www.iers.org}
    \item \textbf{BIPM} (International Bureau of Weights and Measures): TAI definition
    \item \textbf{IAU} (International Astronomical Union): Resolutions on time scales
    \item \textbf{USNO} (US Naval Observatory): \textit{Astronomical Almanac}
\end{itemize}

The SOFA library (Standards of Fundamental Astronomy) provides reference implementations of time and coordinate transformations: \url{http://www.iausofa.org}

\chapter{Coordinate Systems and Reference Frames}
\label{ch:coordinate_systems}

\section{Introduction}

Celestial mechanics requires precise specification of positions and velocities. This necessitates well-defined \textit{coordinate systems} (mathematical frameworks for specifying locations) and \textit{reference frames} (physical realizations tied to astronomical objects).

\section{Fundamental Concepts}

\subsection{Inertial vs. Rotating Frames}

\begin{definition}[Inertial Frame]
An \textit{inertial reference frame} is one in which Newton's first law holds: a body not subject to forces moves in a straight line at constant velocity.
\end{definition}

Truly inertial frames don't exist (the universe expands!), but frames fixed relative to distant quasars are effectively inertial for solar system dynamics.

\begin{definition}[Rotating Frame]
A \textit{rotating reference frame} rotates relative to inertial space. Fictitious forces (centrifugal, Coriolis) appear in rotating frames.
\end{definition}

\section{Equatorial Coordinate System}

\subsection{Definition}

The equatorial system uses Earth's equator and rotation axis:

\begin{itemize}
    \item \textbf{Fundamental plane}: Earth's equator (extended to celestial sphere)
    \item \textbf{Primary direction}: Vernal equinox ($\gamma$), where Sun crosses equator northward
    \item \textbf{Pole}: North celestial pole (direction of Earth's rotation axis)
\end{itemize}

\begin{figure}[H]
\centering
\begin{tikzpicture}[scale=1.2]
    % Celestial sphere
    \draw[thick] (0,0) circle (3);
    
    % Equator
    \draw[thick, blue] (-3,0) -- (3,0);
    \node[blue, right] at (3,0.2) {Celestial Equator};
    
    % Ecliptic
    \draw[thick, red] (-3,-0.7) -- (3,0.7);
    \node[red, right] at (2.5,1) {Ecliptic};
    
    % Vernal equinox
    \fill (1.5,0.35) circle (0.08);
    \node[below] at (1.5,0.15) {$\gamma$};
    
    % Poles
    \draw[dashed] (0,-3) -- (0,3);
    \node[above] at (0,3) {North Pole};
    
    % Obliquity
    \draw[<->, thick] (2,0) arc (0:23.4:2);
    \node at (2.5,0.5) {$\varepsilon$};
    
    % Star position
    \fill[green!60!black] (2,2) circle (0.08);
    \draw[dashed] (0,0) -- (2,2);
    \draw[dashed] (2,0) -- (2,2);
    \draw[->] (0,0) -- (2,0);
    \node at (1,0.3) {$\alpha$};
    \node[right] at (2,1) {$\delta$};
    \node[above right] at (2,2.2) {Star};
\end{tikzpicture}
\caption{Equatorial coordinate system showing right ascension ($\alpha$) and declination ($\delta$). The obliquity $\varepsilon \approx 23.4^\circ$.}
\label{fig:equatorial}
\end{figure}

\subsection{Spherical Coordinates}

Positions are specified by:

\begin{description}
    \item[Right Ascension ($\alpha$)] Angle eastward from vernal equinox along equator ($0^\circ$ to $360^\circ$, or 0h to 24h)
    \item[Declination ($\delta$)] Angle north (+) or south ($-$) of equator ($-90^\circ$ to $+90^\circ$)
    \item[Distance ($r$)] Radial distance from origin
\end{description}

Conversion to Cartesian coordinates:

\begin{align}
x &= r \cos\delta \cos\alpha \\
y &= r \cos\delta \sin\alpha \\
z &= r \sin\delta
\end{align}

\section{Ecliptic Coordinate System}

\subsection{Definition}

The ecliptic system uses Earth's orbital plane:

\begin{itemize}
    \item \textbf{Fundamental plane}: Ecliptic (Earth's orbital plane)
    \item \textbf{Primary direction}: Vernal equinox (same as equatorial)
    \item \textbf{Pole}: Normal to ecliptic plane
\end{itemize}

Coordinates are:
\begin{description}
    \item[Ecliptic Longitude ($\lambda$)] Angle from vernal equinox along ecliptic
    \item[Ecliptic Latitude ($\beta$)] Angle north/south of ecliptic
\end{description}

\subsection{Why Use Ecliptic Coordinates?}

For solar system objects:
\begin{itemize}
    \item Planetary orbits lie near the ecliptic ($|\beta| < 10^\circ$ typically)
    \item Simplifies perturbation calculations
    \item Natural frame for heliocentric dynamics
\end{itemize}

\section{Transformation Between Systems}

\subsection{Ecliptic $\leftrightarrow$ Equatorial}

The transformation involves rotation about the $x$-axis (vernal equinox direction) by the \textit{obliquity} $\varepsilon \approx 23.43929^\circ$:

\begin{equation}
\begin{bmatrix} x \\ y \\ z \end{bmatrix}_{\text{eq}}
= 
\mathbf{R}_x(\varepsilon)
\begin{bmatrix} x \\ y \\ z \end{bmatrix}_{\text{ecl}}
=
\begin{bmatrix}
1 & 0 & 0 \\
0 & \cos\varepsilon & -\sin\varepsilon \\
0 & \sin\varepsilon & \cos\varepsilon
\end{bmatrix}
\begin{bmatrix} x \\ y \\ z \end{bmatrix}_{\text{ecl}}
\end{equation}

For the inverse transformation (equatorial → ecliptic), use $\mathbf{R}_x(-\varepsilon) = \mathbf{R}_x(\varepsilon)^T$.

\subsection{Implementation}

\begin{lstlisting}[style=cpp,caption={Coordinate transformations in AstDyn}]
#include <astdyn/coordinates/ReferenceFrame.hpp>
using namespace astdyn::coordinates;

// Ecliptic to J2000 equatorial
Vector3d pos_ecl(1.0, 0.5, 0.1);  // AU
Matrix3d rot = ReferenceFrame::ecliptic_to_j2000();
Vector3d pos_eq = rot * pos_ecl;

// Equatorial to ecliptic
Vector3d vel_eq(0.01, 0.02, 0.005);  // AU/day
Matrix3d rot_inv = rot.transpose();  // Orthogonal matrix
Vector3d vel_ecl = rot_inv * vel_eq;
\end{lstlisting}

\section{The J2000.0 Reference Frame}

\subsection{Epoch vs. Equinox}

Two temporal concepts are critical:

\begin{description}
    \item[Epoch] The time for which coordinates are specified (affects positions due to motion)
    \item[Equinox] The time defining the orientation of the coordinate axes (affects reference directions)
\end{description}

Example: "Position at epoch 2025.0 in J2000.0 equinox" means the object's location on January 1, 2025, expressed in a coordinate system whose axes are defined by Earth's orientation on January 1, 2000.

\subsection{Precession}

Earth's rotation axis precesses (wobbles) with a period of ~26,000 years due to tidal forces from the Sun and Moon. This causes the vernal equinox to drift westward along the ecliptic at ~50'' per year.

\begin{figure}[H]
\centering
\begin{tikzpicture}
    \draw[thick] (0,0) circle (2.5);
    \draw[thick, ->] (0,0) -- (2.5,0) node[right] {$\gamma_{2000}$};
    \draw[thick, ->, blue] (0,0) -- (2.3,0.8) node[right] {$\gamma_{2050}$};
    \draw[<->, red] (1.5,0) arc (0:20:1.5);
    \node[red] at (1.8,0.5) {Precession};
\end{tikzpicture}
\caption{Precession of the equinoxes over 50 years}
\end{figure}

The J2000.0 frame freezes the equinox at January 1, 2000, 12:00 TT, providing a fixed reference for long-term calculations.

\section{Practical Considerations}

\subsection{Reference Frame Choice}

\begin{itemize}
    \item \textbf{Heliocentric ecliptic}: Natural for planet/asteroid orbits
    \item \textbf{Geocentric equatorial}: Standard for Earth-based observations
    \item \textbf{Barycentric}: Required for precise planetary ephemerides
\end{itemize}

\subsection{Frame Transformations in AstDyn}

The library provides rotation matrices for common transformations:

\begin{lstlisting}[style=cpp,caption={Available transformations}]
// Ecliptic <-> Equatorial (J2000.0)
Matrix3d ecl_to_eq = ReferenceFrame::ecliptic_to_j2000();
Matrix3d eq_to_ecl = ecl_to_eq.transpose();

// ICRS <-> J2000 (small bias correction)
Matrix3d icrs_to_j2000 = ReferenceFrame::icrs_to_j2000();
\end{lstlisting}

More transformations (precession, nutation, GCRS) are available for advanced applications.

\section{Summary}

Key points:
\begin{itemize}
    \item Equatorial system: Tied to Earth's rotation (RA, Dec)
    \item Ecliptic system: Tied to Earth's orbit (natural for heliocentric dynamics)
    \item Transformations: Simple rotation matrices (orthogonal)
    \item J2000.0: Standard epoch/equinox for modern astrometry
    \item AstDyn: Implements all common transformations efficiently
\end{itemize}

\chapter{Reference Frames}
\label{ch:reference_frames}

\section{Introduction to Reference Frames}

In celestial mechanics, a \textbf{reference frame} (or \textbf{reference system}) is a coordinate system used to specify the positions and velocities of celestial bodies. The choice of reference frame is crucial because:

\begin{itemize}
    \item Orbital elements are defined relative to a specific frame
    \item Transformations between frames are required for observations
    \item Different applications may prefer different frames
    \item Numerical accuracy depends on the frame choice
\end{itemize}

A reference frame consists of:
\begin{enumerate}
    \item An \textbf{origin} (e.g., Earth's center, Solar System barycenter)
    \item A \textbf{fundamental plane} (e.g., equator, ecliptic)
    \item A \textbf{reference direction} (e.g., vernal equinox)
    \item An \textbf{epoch} for the orientation (e.g., J2000.0)
\end{enumerate}

\section{The International Celestial Reference System (ICRS)}

The \textbf{ICRS} is the current standard celestial reference system adopted by the International Astronomical Union (IAU) in 1998. It represents the most precise realization of an inertial reference frame.

\subsection{ICRS Definition}

The ICRS is defined by:
\begin{itemize}
    \item \textbf{Origin}: Solar System barycenter
    \item \textbf{Fundamental plane}: Earth's mean equator at J2000.0 (with corrections)
    \item \textbf{Reference direction}: Mean vernal equinox at J2000.0 (with corrections)
    \item \textbf{Realization}: Positions of 212 extragalactic radio sources (quasars)
\end{itemize}

The ICRS is a kinematically non-rotating frame with axes defined to microarcsecond precision using Very Long Baseline Interferometry (VLBI) observations of quasars.

\subsection{Relation to J2000.0}

The ICRS is closely aligned with the J2000.0 equatorial system but differs by:
\begin{itemize}
    \item Frame bias: $\sim$20 milliarcseconds in orientation
    \item No rotation rate (truly inertial)
    \item Definition based on extragalactic sources (not Earth's rotation)
\end{itemize}

For most applications in asteroid dynamics, the difference between ICRS and J2000.0 is negligible ($<0.1$ arcsecond over centuries).

\section{The J2000.0 Equatorial Frame}

The \textbf{J2000.0} frame is the most commonly used reference frame in celestial mechanics and is the default frame in AstDyn.

\subsection{J2000.0 Definition}

\begin{itemize}
    \item \textbf{Epoch}: January 1, 2000, 12:00 TT (JD 2451545.0)
    \item \textbf{Origin}: Solar System barycenter (or Earth's center for geocentric)
    \item \textbf{Fundamental plane}: Earth's mean equator at J2000.0
    \item \textbf{X-axis}: Points toward mean vernal equinox at J2000.0
    \item \textbf{Z-axis}: Perpendicular to equator, toward north celestial pole
    \item \textbf{Y-axis}: Completes right-handed system ($\mathbf{Y} = \mathbf{Z} \times \mathbf{X}$)
\end{itemize}

\begin{figure}[htbp]
\centering
\begin{tikzpicture}[scale=1.5]
    % Fundamental plane (equator)
    \draw[thick,blue!30,fill=blue!5] (0,0) ellipse (3cm and 1cm);
    \node[blue] at (3.2,0.5) {Equator};
    
    % Origin
    \filldraw[black] (0,0) circle (2pt);
    \node[below] at (0,-0.2) {Origin};
    
    % X-axis (toward vernal equinox)
    \draw[->,thick,red] (0,0) -- (3.5,0);
    \node[red,right] at (3.5,0) {$\mathbf{X}$ (to $\gamma$)};
    
    % Y-axis
    \draw[->,thick,green!60!black] (0,0) -- (0,2);
    \node[green!60!black,above] at (0,2) {$\mathbf{Y}$};
    
    % Z-axis (toward NCP)
    \draw[->,thick,blue] (0,0) -- (-1.5,2.5);
    \node[blue,left] at (-1.5,2.5) {$\mathbf{Z}$ (NCP)};
    
    % Vernal equinox symbol
    \node[red] at (3.2,-0.3) {$\gamma$};
    
    % Celestial object
    \filldraw[orange] (2,1.8) circle (3pt);
    \draw[->,dashed] (0,0) -- (2,1.8);
    \node[orange,right] at (2,1.8) {Celestial object};
    
    % Right ascension arc
    \draw[->,purple] (0.8,0) arc (0:40:0.8);
    \node[purple] at (1.2,0.4) {$\alpha$};
    
    % Declination arc
    \draw[->,purple] (1.5,0) -- (1.5,1.35);
    \node[purple,right] at (1.5,0.7) {$\delta$};
\end{tikzpicture}
\caption{J2000.0 equatorial reference frame showing the three axes and the definition of right ascension ($\alpha$) and declination ($\delta$).}
\label{fig:j2000_frame}
\end{figure}

\subsection{Heliocentric vs. Barycentric Frames}

For asteroid orbits, we typically use:
\begin{itemize}
    \item \textbf{Heliocentric frame}: Origin at the Sun's center. Suitable for inner solar system objects where the Sun dominates gravitational dynamics.
    \item \textbf{Barycentric frame}: Origin at the Solar System barycenter. Required for precise calculations involving Jupiter and outer planets, as the Sun-Jupiter barycenter lies outside the Sun's surface.
\end{itemize}

The transformation between heliocentric and barycentric frames involves the Sun's position relative to the barycenter:
\begin{equation}
    \mathbf{r}_{\text{bary}} = \mathbf{r}_{\text{helio}} + \mathbf{r}_{\text{Sun,bary}}
\end{equation}

For asteroids with $a < 10$ AU, the difference is typically $< 10^{-6}$ AU.

\section{The Ecliptic Reference Frame}

The \textbf{ecliptic frame} uses the plane of Earth's orbit as the fundamental plane.

\subsection{Ecliptic Definition}

\begin{itemize}
    \item \textbf{Fundamental plane}: Mean ecliptic at J2000.0
    \item \textbf{X-axis}: Toward mean vernal equinox at J2000.0
    \item \textbf{Z-axis}: Perpendicular to ecliptic, toward north ecliptic pole
    \item \textbf{Y-axis}: Completes right-handed system
\end{itemize}

The ecliptic frame is natural for describing planetary and asteroid orbits because:
\begin{itemize}
    \item Most orbits lie close to the ecliptic plane
    \item Inclinations are typically small ($i < 30^\circ$)
    \item Solar system formation models predict ecliptic alignment
\end{itemize}

\subsection{Ecliptic Coordinates}

In the ecliptic frame, positions are specified by:
\begin{itemize}
    \item \textbf{Ecliptic longitude} ($\lambda$): Angle from vernal equinox along ecliptic
    \item \textbf{Ecliptic latitude} ($\beta$): Angle perpendicular to ecliptic
    \item \textbf{Distance} ($r$): Radial distance from origin
\end{itemize}

Conversion from Cartesian ecliptic coordinates:
\begin{align}
    \lambda &= \arctan\left(\frac{Y_{\text{ecl}}}{X_{\text{ecl}}}\right) \\
    \beta &= \arctan\left(\frac{Z_{\text{ecl}}}{\sqrt{X_{\text{ecl}}^2 + Y_{\text{ecl}}^2}}\right) \\
    r &= \sqrt{X_{\text{ecl}}^2 + Y_{\text{ecl}}^2 + Z_{\text{ecl}}^2}
\end{align}

\section{Transformations Between Reference Frames}

\subsection{Equatorial to Ecliptic Transformation}

The transformation from J2000.0 equatorial to J2000.0 ecliptic coordinates is a rotation about the X-axis by the \textbf{obliquity of the ecliptic} ($\varepsilon_0$):

\begin{equation}
    \begin{pmatrix}
        X_{\text{ecl}} \\
        Y_{\text{ecl}} \\
        Z_{\text{ecl}}
    \end{pmatrix}
    =
    \begin{pmatrix}
        1 & 0 & 0 \\
        0 & \cos\varepsilon_0 & \sin\varepsilon_0 \\
        0 & -\sin\varepsilon_0 & \cos\varepsilon_0
    \end{pmatrix}
    \begin{pmatrix}
        X_{\text{eq}} \\
        Y_{\text{eq}} \\
        Z_{\text{eq}}
    \end{pmatrix}
\end{equation}

At J2000.0, the obliquity is:
\begin{equation}
    \varepsilon_0 = 23^\circ26'21.406'' = 23.4392911^\circ
\end{equation}

The inverse transformation is simply a rotation by $-\varepsilon_0$:
\begin{equation}
    \mathbf{R}_{\text{ecl}\to\text{eq}} = \mathbf{R}_{\text{eq}\to\text{ecl}}^T
\end{equation}

\subsection{Precession: Time-Dependent Transformations}

Earth's rotation axis precesses due to torques from the Moon and Sun acting on Earth's equatorial bulge. This causes the equatorial plane to change orientation over time.

\begin{figure}[htbp]
\centering
\begin{tikzpicture}[scale=1.2]
    % Ecliptic plane
    \draw[thick,blue!30,fill=blue!5] (-3,0) -- (3,0) -- (3,0.5) -- (-3,0.5) -- cycle;
    \node[blue] at (3.3,0.25) {Ecliptic};
    
    % Equator at epoch 1
    \draw[thick,red!50,dashed] (-2.5,-1) -- (2.5,1);
    \node[red!50] at (2.8,1.2) {Equator ($t_1$)};
    
    % Equator at epoch 2
    \draw[thick,red,dashed] (-2.5,-0.5) -- (2.5,1.5);
    \node[red] at (2.8,1.7) {Equator ($t_2$)};
    
    % Celestial pole precession
    \draw[->,thick,purple] (0,2) arc (90:70:2);
    \node[purple] at (-0.8,2.2) {Precession};
    
    % Vernal equinox precession
    \draw[->,thick,orange] (1.5,0.25) arc (0:-15:1.5);
    \node[orange] at (1.2,-0.5) {$\gamma$ moves};
    
    % Obliquity angle
    \draw[<->,dashed] (1.5,0.25) -- (1.5,1);
    \node[right] at (1.6,0.6) {$\varepsilon$};
\end{tikzpicture}
\caption{Precession causes the equatorial plane to change orientation over time. The vernal equinox $\gamma$ moves westward along the ecliptic at approximately 50.3 arcseconds per year.}
\label{fig:precession}
\end{figure}

The precession rate is approximately:
\begin{equation}
    \frac{d\alpha}{dt} \approx 50.3'' \text{ per year (in right ascension)}
\end{equation}

For transformations between different epochs (e.g., J2000.0 to date), precession matrices must be applied. The IAU 2006 precession model is the current standard.

\subsection{AstDyn Implementation}

In AstDyn, the \texttt{ReferenceFrame} class handles coordinate transformations:

\begin{lstlisting}[language=C++,caption={Coordinate transformations in AstDyn}]
#include <astdyn/core/ReferenceFrame.hpp>

using namespace astdyn;

// Equatorial to ecliptic transformation
Vector3d r_eq(1.0, 0.5, 0.3);  // AU, J2000.0 equatorial
Vector3d r_ecl = ReferenceFrame::equatorial_to_ecliptic(r_eq);

// Ecliptic to equatorial
Vector3d r_eq2 = ReferenceFrame::ecliptic_to_equatorial(r_ecl);

// Verify round-trip: r_eq ~~ r_eq2
std::cout << "Round-trip error: " 
          << (r_eq - r_eq2).norm() << " AU\n";
// Output: Round-trip error: 1.23e-16 AU

// Precession from J2000.0 to date
double jd_now = 2460000.0;  // Current epoch
Matrix3d precession_matrix = 
    ReferenceFrame::precession_matrix_j2000_to_date(jd_now);
    
Vector3d r_now = precession_matrix * r_eq;
\end{lstlisting}

\section{Other Important Reference Frames}

\subsection{The FK5 System}

The \textbf{Fifth Fundamental Catalogue (FK5)} was the standard reference system before ICRS. It is based on observations of bright stars and is equivalent to J2000.0 for most purposes.

\begin{itemize}
    \item \textbf{Epoch}: J2000.0
    \item \textbf{Realization}: 1535 fundamental stars
    \item \textbf{Accuracy}: $\sim$10 milliarcseconds
    \item \textbf{Relation to ICRS}: Small systematic differences
\end{itemize}

For asteroid work, FK5 and ICRS are interchangeable to within observational uncertainties.

\subsection{The Invariable Plane}

The \textbf{invariable plane} is perpendicular to the total angular momentum vector of the solar system. It is truly inertial (no external torques) and provides a dynamically natural reference.

\begin{itemize}
    \item \textbf{Inclination to ecliptic}: $\sim1.58^\circ$
    \item \textbf{Dominated by}: Jupiter's orbital angular momentum ($\sim60\%$ of total)
    \item \textbf{Use}: Dynamical studies, long-term stability analysis
\end{itemize}

The invariable plane is not used for observations but is valuable for theoretical studies.

\subsection{Body-Centric Frames}

For satellite dynamics or close approaches, body-centric frames are used:

\begin{itemize}
    \item \textbf{Origin}: Center of mass of the body (e.g., Earth, Mars)
    \item \textbf{Orientation}: Often aligned with body's rotation axis
    \item \textbf{Examples}: Earth-centered inertial (ECI), planetocentric frames
\end{itemize}

\section{Practical Considerations}

\subsection{Numerical Precision}

When working with reference frames:
\begin{itemize}
    \item Use double precision (64-bit) for coordinates in AU
    \item Accumulated precession errors: $\sim10^{-10}$ AU per transformation
    \item For $\Delta t > 100$ years, include precession corrections
    \item For $\Delta t > 1000$ years, use full precession/nutation models
\end{itemize}

\subsection{Choice of Frame for Orbit Propagation}

For asteroid orbit propagation in AstDyn:
\begin{itemize}
    \item \textbf{Default}: Heliocentric J2000.0 equatorial
    \item \textbf{Rationale}: 
    \begin{itemize}
        \item Matches most observational catalogs
        \item Stable over centuries
        \item Simplifies comparison with other software
    \end{itemize}
    \item \textbf{Alternative}: Ecliptic frame for very low inclination orbits
\end{itemize}

\subsection{Converting Observations}

Optical observations are typically reported in:
\begin{itemize}
    \item \textbf{Right ascension} ($\alpha$) and \textbf{declination} ($\delta$): Equatorial frame
    \item \textbf{Topocentric coordinates}: From observer's location on Earth
\end{itemize}

To use these in orbit determination:
\begin{enumerate}
    \item Convert topocentric to geocentric (correct for Earth's rotation)
    \item Convert geocentric to heliocentric (add Earth's position)
    \item Express in J2000.0 equatorial frame
\end{enumerate}

AstDyn's \texttt{OpticalObservation} class handles these transformations automatically.

\section{Summary}

Key points about reference frames:

\begin{enumerate}
    \item \textbf{ICRS} is the modern standard, closely aligned with J2000.0
    \item \textbf{J2000.0 equatorial} is the practical frame for asteroid dynamics
    \item \textbf{Ecliptic frame} is natural for solar system objects
    \item Transformations between frames are rotations defined by obliquity
    \item \textbf{Precession} causes time-dependent frame rotations
    \item AstDyn uses heliocentric J2000.0 equatorial as default
\end{enumerate}

Understanding reference frames is essential for:
\begin{itemize}
    \item Interpreting orbital elements
    \item Processing observations
    \item Comparing results between software packages
    \item Long-term orbit propagation
\end{itemize}

In the next chapter, we will discuss orbital elements—the six parameters that uniquely specify an orbit in a given reference frame.

\chapter{Orbital Elements}
\label{ch:orbital_elements}

\section{Introduction to Orbital Elements}

An \textbf{orbital element} is a parameter that describes the shape, size, orientation, and position of an orbit in space. For the two-body problem, exactly six parameters are needed to uniquely specify an orbit, corresponding to the six degrees of freedom (three for position, three for velocity).

Different sets of orbital elements exist, each with advantages for specific applications:
\begin{itemize}
    \item \textbf{Keplerian elements}: Classical, geometrically intuitive
    \item \textbf{Cartesian elements}: Simple for numerical integration
    \item \textbf{Equinoctial elements}: Avoid singularities at low inclination
    \item \textbf{Delaunay elements}: Canonical, useful for perturbation theory
\end{itemize}

\section{Classical Keplerian Elements}

The \textbf{classical Keplerian elements} are the most widely used set. They directly describe the geometry of a conic section orbit.

\subsection{The Six Keplerian Elements}

\begin{enumerate}
    \item \textbf{Semi-major axis} ($a$): Half the longest diameter of the ellipse. Determines orbital size and period.
    \begin{equation}
        a = \frac{r_{\text{peri}} + r_{\text{apo}}}{2}
    \end{equation}
    Units: AU (astronomical units) for asteroids, km for satellites.
    
    \item \textbf{Eccentricity} ($e$): Shape of the orbit.
    \begin{equation}
        e = \frac{r_{\text{apo}} - r_{\text{peri}}}{r_{\text{apo}} + r_{\text{peri}}}
    \end{equation}
    \begin{itemize}
        \item $e = 0$: Circular orbit
        \item $0 < e < 1$: Elliptical orbit
        \item $e = 1$: Parabolic trajectory
        \item $e > 1$: Hyperbolic trajectory
    \end{itemize}
    
    \item \textbf{Inclination} ($i$): Angle between orbital plane and reference plane (equator or ecliptic).
    \begin{equation}
        0^\circ \leq i \leq 180^\circ
    \end{equation}
    \begin{itemize}
        \item $i < 90^\circ$: Prograde (direct) orbit
        \item $i = 90^\circ$: Polar orbit
        \item $i > 90^\circ$: Retrograde orbit
    \end{itemize}
    
    \item \textbf{Longitude of ascending node} ($\Omega$): Angle from reference direction to ascending node (where orbit crosses reference plane going north).
    \begin{equation}
        0^\circ \leq \Omega < 360^\circ
    \end{equation}
    
    \item \textbf{Argument of perihelion} ($\omega$): Angle from ascending node to perihelion within the orbital plane.
    \begin{equation}
        0^\circ \leq \omega < 360^\circ
    \end{equation}
    
    \item \textbf{Mean anomaly} ($M$): Position of the body along the orbit at a given time, measured as angle from perihelion.
    \begin{equation}
        M = n(t - t_{\text{peri}})
    \end{equation}
    where $n = \sqrt{\mu/a^3}$ is the mean motion.
\end{enumerate}

\begin{figure}[htbp]
\centering
\begin{tikzpicture}[scale=1.2, >=stealth]
    % PARAMETERS & COORDINATES
    \coordinate (Sun) at (0,0);
    \def\OmegaAngle{-35} % Longitude of Node (visual Perspective)
    \def\omegaAngle{50}  % Argument of Perihelion
    \def\orbA{2.8}       % Semi-major axis
    \def\orbE{0.6}       % Eccentricity
    \def\orbB{2.24}      % Semi-minor axis approx (b = a*sqrt(1-e^2)) - manually tuned for view
    \def\orbC{1.68}      % Focal distance c = a*e

    % 1. REFERENCE PLANE (e.g. Ecliptic)
    % Represented as a flat gray ellipse
    \fill[gray!5] (0,0) ellipse (4.0cm and 1.2cm);
    \draw[gray, thin] (0,0) ellipse (4.0cm and 1.2cm);
    \node[gray, below right] at (3.0,-0.8) {Reference Plane};

    % 2. SUN (Focus)
    \shade[ball color=orange] (Sun) circle (0.15);
    \node[below=0.3cm] at (Sun) {Sun};

    % 3. REFERENCE DIRECTION (Gamma)
    \draw[->, blue, thick] (Sun) -- (4.2, 0) node[right] {$\gamma$};

    % 4. LINE OF NODES
    % Intersection of planes. Passes through Sun.
    \draw[dashed, black!60] (Sun) -- ({4*cos(\OmegaAngle+180)}, {1.2*sin(\OmegaAngle+180)});
    % Visible part
    \draw[thick, black] (Sun) -- ({4*cos(\OmegaAngle)}, {1.2*sin(\OmegaAngle)}) coordinate (Node);
    \node[below right] at (Node) {Line of Nodes};
    \filldraw (Node) circle (1.5pt) node[above right, xshift=-2pt] {$\ascnode$};

    % Longitude of Ascending Node (Omega)
    \draw[->, purple, thick] (1.5, 0) arc (0:\OmegaAngle:1.5);
    \node[purple, right] at (1.7, -0.4) {$\Omega$};

    % 5. ORBITAL PLANE & ORBIT
    % We rotate to the Node Frame
    \begin{scope}[rotate=\OmegaAngle]
        % Now X-axis is the Line of Nodes.
        % We draw the orbit. The orbit's major axis is rotated by 'omega' from the Node line.
        
        \begin{scope}[rotate=\omegaAngle]
            % Now X-axis is the Major Axis (towards Perihelion).
            % The Sun is at (0,0). The Ellipse center is at (-c, 0).
            \coordinate (Center) at (-\orbC, 0);
            
            % Draw Ellipse (Orbit Path)
            % Uses red color
            \draw[red, thick] (Center) ellipse (\orbA cm and \orbB cm);
            
            % Perihelion
            \coordinate (Perihelion) at (\orbA-\orbC, 0);
            \filldraw[red] (Perihelion) circle (2pt) node[right] {Perihelion};
            \draw[dashed, red] (Sun) -- (Perihelion);
            
            % Planet / Body Position (True Anomaly)
            \def\nuAngle{65} % True anomaly
            % Parametric point on ellipse relative to center (shifted)
            % x = -c + a*cos(E), y = b*sin(E). 
            % Let's just approximate visually for diagram using polar from Sun
            \path (Sun) -- ++(\nuAngle:2.2) coordinate (Planet); 
            % Note: Mathematically exact requires Kepler eq, visual approx is fine here
            
            \filldraw[blue] (Planet) circle (3pt) node[above] {Body};
            \draw[dotted, blue] (Sun) -- (Planet) node[midway, fill=white, inner sep=0.5pt] {$r$};
            
            % True Anomaly Arc
            \draw[->, blue] (0.8, 0) arc (0:\nuAngle:0.8);
            \node[blue, above] at (0.6, 0.3) {$\nu$};
        \end{scope}
        
        % Argument of Perihelion (omega)
        % From Node (X-axis here) to Major Axis
        \draw[->, orange, thick] (1.0, 0) arc (0:\omegaAngle:1.0);
        \node[orange, right] at (1.1, 0.4) {$\omega$};
    \end{scope}

    % 6. INCLINATION (i)
    % Visual indicator of plane tilt
    \draw[green!50!black, thick] (3.5, -1.2) arc (-90:-60:2);
    \node[green!50!black, right] at (3.6, -0.8) {$i$};
    
\end{tikzpicture}


\caption{Classical Keplerian orbital elements. The orbital plane (red ellipse) is tilted relative to the reference plane. The ascending node is where the orbit crosses the reference plane going northward.}
\label{fig:keplerian_elements}
\end{figure}

\subsection{Orbital Period}

For elliptical orbits, Kepler's Third Law relates period to semi-major axis:
\begin{equation}
    P = 2\pi\sqrt{\frac{a^3}{\mu}}
\end{equation}
where $\mu = GM$ is the gravitational parameter of the central body.

For the Sun:
\begin{equation}
    P[\text{years}] = a[\text{AU}]^{3/2}
\end{equation}

Examples:
\begin{itemize}
    \item Earth: $a = 1$ AU $\Rightarrow P = 1$ year
    \item Mars: $a = 1.524$ AU $\Rightarrow P = 1.88$ years
    \item Ceres: $a = 2.77$ AU $\Rightarrow P = 4.61$ years
\end{itemize}

\subsection{Orbital Energy}

The specific orbital energy (energy per unit mass) is:
\begin{equation}
    \mathcal{E} = -\frac{\mu}{2a}
\end{equation}

Note that $\mathcal{E} < 0$ for elliptical orbits (bound), $\mathcal{E} = 0$ for parabolic, and $\mathcal{E} > 0$ for hyperbolic.

\subsection{Singularities of Keplerian Elements}

Keplerian elements have mathematical singularities:
\begin{itemize}
    \item $\Omega$ undefined for $i = 0^\circ$ (orbit in reference plane)
    \item $\omega$ undefined for $e = 0$ (circular orbit, no perihelion)
    \item Both $\Omega$ and $\omega$ undefined for $i = 0^\circ$ and $e = 0$ simultaneously
\end{itemize}

For near-circular or near-equatorial orbits, numerical errors can grow large. Alternative element sets avoid these issues.

\section{Cartesian State Vector}

The \textbf{Cartesian state vector} specifies position and velocity in a reference frame:
\begin{equation}
    \mathbf{X} = \begin{pmatrix} \mathbf{r} \\ \mathbf{v} \end{pmatrix} = \begin{pmatrix} x \\ y \\ z \\ \dot{x} \\ \dot{y} \\ \dot{z} \end{pmatrix}
\end{equation}

\subsection{Advantages}
\begin{itemize}
    \item No singularities
    \item Simple equations of motion: $\ddot{\mathbf{r}} = -\mu \mathbf{r}/r^3 + \mathbf{a}_{\text{pert}}$
    \item Direct use in numerical integrators
    \item Easy to include perturbations
\end{itemize}

\subsection{Disadvantages}
\begin{itemize}
    \item Less intuitive than Keplerian elements
    \item Difficult to interpret orbital geometry directly
    \item Six tightly coupled variables in integration
\end{itemize}

\subsection{Conversion: Keplerian to Cartesian}

Given Keplerian elements $(a, e, i, \Omega, \omega, M)$ at epoch $t_0$:

\textbf{Step 1}: Solve Kepler's equation for eccentric anomaly $E$:
\begin{equation}
    M = E - e\sin E
\end{equation}
(Requires iterative solution, e.g., Newton-Raphson)

\textbf{Step 2}: Compute true anomaly $\nu$:
\begin{equation}
    \nu = 2\arctan\left(\sqrt{\frac{1+e}{1-e}}\tan\frac{E}{2}\right)
\end{equation}

\textbf{Step 3}: Position and velocity in orbital plane:
\begin{align}
    r &= a(1 - e\cos E) \\
    \mathbf{r}_{\text{orb}} &= r\begin{pmatrix} \cos\nu \\ \sin\nu \\ 0 \end{pmatrix} \\
    \mathbf{v}_{\text{orb}} &= \sqrt{\frac{\mu}{a}}\begin{pmatrix} -\sin E \\ \sqrt{1-e^2}\cos E \\ 0 \end{pmatrix}
\end{align}

\textbf{Step 4}: Rotate to reference frame using rotation matrix:
\begin{equation}
    \mathbf{R}_{3}(-\omega)\mathbf{R}_{1}(-i)\mathbf{R}_{3}(-\Omega)
\end{equation}
where $\mathbf{R}_1(\theta)$ and $\mathbf{R}_3(\theta)$ are rotations about the 1- and 3-axes.

\subsection{Conversion: Cartesian to Keplerian}

Given position $\mathbf{r}$ and velocity $\mathbf{v}$:

\textbf{Step 1}: Compute angular momentum:
\begin{equation}
    \mathbf{h} = \mathbf{r} \times \mathbf{v}
\end{equation}

\textbf{Step 2}: Compute node vector:
\begin{equation}
    \mathbf{n} = \mathbf{\hat{z}} \times \mathbf{h}
\end{equation}

\textbf{Step 3}: Compute eccentricity vector:
\begin{equation}
    \mathbf{e}_{\text{vec}} = \frac{1}{\mu}\left[(\mathbf{v} \times \mathbf{h}) - \mu\frac{\mathbf{r}}{r}\right]
\end{equation}

\textbf{Step 4}: Extract elements:
\begin{align}
    a &= \frac{1}{2/r - v^2/\mu} \\
    e &= |\mathbf{e}_{\text{vec}}| \\
    i &= \arccos\frac{h_z}{|\mathbf{h}|} \\
    \Omega &= \arctan\frac{n_y}{n_x} \\
    \omega &= \arccos\frac{\mathbf{n} \cdot \mathbf{e}_{\text{vec}}}{|\mathbf{n}||\mathbf{e}_{\text{vec}}|} \\
    \nu &= \arccos\frac{\mathbf{e}_{\text{vec}} \cdot \mathbf{r}}{|\mathbf{e}_{\text{vec}}||\mathbf{r}|}
\end{align}

Then $M = E - e\sin E$ where $E = 2\arctan\left(\sqrt{\frac{1-e}{1+e}}\tan\frac{\nu}{2}\right)$.

\section{Equinoctial Elements}

\textbf{Equinoctial elements} avoid singularities at zero inclination and eccentricity. They are particularly useful for asteroids with nearly circular or low-inclination orbits.

\subsection{Definition}

The equinoctial set is:
\begin{align}
    a &= \text{semi-major axis (same as Keplerian)} \\
    h &= e\sin(\omega + \Omega) \\
    k &= e\cos(\omega + \Omega) \\
    p &= \tan(i/2)\sin\Omega \\
    q &= \tan(i/2)\cos\Omega \\
    \lambda &= M + \omega + \Omega \quad \text{(mean longitude)}
\end{align}

\subsection{Conversion to Keplerian}

From equinoctial to Keplerian:
\begin{align}
    e &= \sqrt{h^2 + k^2} \\
    i &= 2\arctan\sqrt{p^2 + q^2} \\
    \Omega &= \arctan\frac{p}{q} \\
    \omega &= \arctan\frac{h}{k} - \Omega \\
    M &= \lambda - \omega - \Omega
\end{align}

\subsection{Advantages}

\begin{itemize}
    \item No singularities for $i \approx 0$ or $e \approx 0$
    \item Smooth evolution near circular/equatorial orbits
    \item Used in JPL's HORIZONS system
    \item Well-suited for numerical orbit propagation
\end{itemize}

\section{Delaunay Elements}

\textbf{Delaunay elements} are a canonical set of action-angle variables used in perturbation theory and Hamiltonian mechanics.

\subsection{Definition}

The Delaunay variables are:
\begin{align}
    L &= \sqrt{\mu a} \quad &\text{(action conjugate to } \ell = M) \\
    G &= L\sqrt{1-e^2} \quad &\text{(action conjugate to } g = \omega) \\
    H &= G\cos i \quad &\text{(action conjugate to } h = \Omega)
\end{align}

The angles are:
\begin{align}
    \ell &= M \quad \text{(mean anomaly)} \\
    g &= \omega \quad \text{(argument of perihelion)} \\
    h &= \Omega \quad \text{(longitude of ascending node)}
\end{align}

\subsection{Properties}

\begin{itemize}
    \item $(L,G,H,\ell,g,h)$ form a canonical coordinate set
    \item Hamiltonian formulation: $\dot{q}_i = \partial H/\partial p_i$, $\dot{p}_i = -\partial H/\partial q_i$
    \item Unperturbed Hamiltonian: $H_0 = -\mu^2/(2L^2)$ (depends only on $L$)
    \item For unperturbed Kepler problem: $L, G, H$ are constants
    \item Useful for secular perturbation theory and resonance analysis
\end{itemize}

\section{AstDyn Implementation}

AstDyn provides conversion functions in the \texttt{OrbitalElements} class:

\begin{lstlisting}[language=C++,caption={Orbital element conversions in AstDyn}]
#include <astdyn/core/OrbitalElements.hpp>
#include <astdyn/core/StateVector.hpp>

using namespace astdyn;

// Keplerian elements
OrbitalElements kep;
kep.a = 2.77;        // AU
kep.e = 0.078;
kep.i = 10.6 * DEG_TO_RAD;  // radians
kep.Omega = 80.3 * DEG_TO_RAD;
kep.omega = 73.1 * DEG_TO_RAD;
kep.M = 15.2 * DEG_TO_RAD;
kep.epoch = 2460000.0;  // JD

// Convert to Cartesian
StateVector sv = kep.to_state_vector();
std::cout << "Position: " << sv.r.transpose() << " AU\n";
std::cout << "Velocity: " << sv.v.transpose() << " AU/day\n";

// Convert back to Keplerian
OrbitalElements kep2 = OrbitalElements::from_state_vector(
    sv.r, sv.v, sv.t
);

// Verify round-trip
std::cout << "Delta a: " << kep2.a - kep.a << " AU\n";
// Output: Delta a: 3.14e-15 AU (machine precision)

// Convert to equinoctial
auto eq = kep.to_equinoctial();
std::cout << "h = " << eq.h << ", k = " << eq.k << "\n";
std::cout << "p = " << eq.p << ", q = " << eq.q << "\n";
\end{lstlisting}

\section{Summary}

Key points about orbital elements:

\begin{enumerate}
    \item Six parameters specify a two-body orbit (6 degrees of freedom)
    \item \textbf{Keplerian elements} $(a,e,i,\Omega,\omega,M)$ are geometrically intuitive
    \item \textbf{Cartesian state} $(\mathbf{r}, \mathbf{v})$ is simple for numerical work
    \item \textbf{Equinoctial elements} avoid singularities at $e=0$ and $i=0$
    \item \textbf{Delaunay elements} are canonical, useful for perturbation theory
    \item Choice of elements depends on application and orbit characteristics
    \item Conversions between element sets are standard operations in AstDyn
\end{enumerate}

Understanding orbital elements is essential for:
\begin{itemize}
    \item Reading and interpreting observational catalogs
    \item Setting up orbit propagation problems
    \item Analyzing orbital dynamics and perturbations
    \item Choosing appropriate numerical methods
\end{itemize}

In the next chapter, we will study the two-body problem in detail—the foundation of orbital mechanics and the basis for all perturbation analyses.

\chapter{The Two-Body Problem}
\label{ch:two_body_problem}

\section{Introduction to the Two-Body Problem}

The \textbf{two-body problem} is the foundation of orbital mechanics. It describes the motion of two point masses interacting only through mutual gravitational attraction. This is the only case in celestial mechanics with a complete analytical solution.

\subsection{Problem Statement}

Consider two bodies with masses $m_1$ and $m_2$ separated by distance $r$. Newton's law of gravitation gives:

\begin{equation}
    \mathbf{F}_{12} = -G\frac{m_1 m_2}{r^2}\hat{\mathbf{r}}
\end{equation}

where $G = 6.674 \times 10^{-11}$ m$^3$ kg$^{-1}$ s$^{-2}$ is the gravitational constant.

The equations of motion are:
\begin{align}
    m_1\ddot{\mathbf{r}}_1 &= G\frac{m_1 m_2}{r^3}(\mathbf{r}_2 - \mathbf{r}_1) \\
    m_2\ddot{\mathbf{r}}_2 &= G\frac{m_1 m_2}{r^3}(\mathbf{r}_1 - \mathbf{r}_2)
\end{align}

\subsection{Reduction to One-Body Problem}

By introducing the relative position $\mathbf{r} = \mathbf{r}_2 - \mathbf{r}_1$ and the reduced mass $\mu = G(m_1 + m_2)$, the problem reduces to:

\begin{equation}
    \ddot{\mathbf{r}} = -\frac{\mu}{r^3}\mathbf{r}
\end{equation}

This is the \textbf{central force equation} with gravitational parameter $\mu$. For the Sun-planet system, $\mu_\odot = 1.327 \times 10^{20}$ m$^3$ s$^{-2}$.

\section{Conservation Laws}

The two-body problem has several conserved quantities that constrain the motion.

\subsection{Conservation of Angular Momentum}

The angular momentum is:
\begin{equation}
    \mathbf{h} = \mathbf{r} \times \mathbf{v}
\end{equation}

Since $\ddot{\mathbf{r}}$ is parallel to $\mathbf{r}$, we have:
\begin{equation}
    \frac{d\mathbf{h}}{dt} = \mathbf{r} \times \ddot{\mathbf{r}} = \mathbf{0}
\end{equation}

Therefore: $\mathbf{h} = \text{constant}$

\textbf{Consequences}:
\begin{itemize}
    \item Motion is confined to a plane perpendicular to $\mathbf{h}$
    \item $|\mathbf{h}| = h = \sqrt{\mu a(1-e^2)}$ relates to orbital elements
    \item Areal velocity is constant: $\frac{dA}{dt} = \frac{h}{2}$ (Kepler's Second Law)
\end{itemize}

\subsection{Conservation of Energy}

The specific mechanical energy is:
\begin{equation}
    \mathcal{E} = \frac{v^2}{2} - \frac{\mu}{r} = \text{constant}
\end{equation}

This can be written as:
\begin{equation}
    \mathcal{E} = -\frac{\mu}{2a}
\end{equation}

for elliptical orbits with semi-major axis $a$.

\subsection{The Laplace-Runge-Lenz Vector}

The eccentricity vector is conserved:
\begin{equation}
    \mathbf{e} = \frac{1}{\mu}(\mathbf{v} \times \mathbf{h}) - \frac{\mathbf{r}}{r}
\end{equation}

Properties:
\begin{itemize}
    \item $|\mathbf{e}| = e$ (orbital eccentricity)
    \item $\mathbf{e}$ points toward periapsis
    \item $\mathbf{e} \cdot \mathbf{h} = 0$ (perpendicular to angular momentum)
\end{itemize}

\section{The Orbit Equation}

\subsection{Derivation}

In polar coordinates $(r, \nu)$ where $\nu$ is the true anomaly, the orbit equation is:

\begin{equation}
    r = \frac{a(1-e^2)}{1 + e\cos\nu} = \frac{p}{1 + e\cos\nu}
\end{equation}

where $p = a(1-e^2)$ is the semi-latus rectum.

This is the equation of a conic section with focus at the origin.

\subsection{Conic Sections}

The orbit shape depends on eccentricity and energy:

\begin{figure}[htbp]
\centering
\begin{tikzpicture}[scale=1.1, >=stealth]
    % PARAMETRI
    \def\q{1.5} % Distanza Perielio comune
    
    % 1. SOLE (Focus)
    \coordinate (Sun) at (0,0);
    \shade[ball color=orange] (Sun) circle (0.15);
    \draw[<-] (Sun) -- ++(-0.2, -0.6) node[below] {Sun (Focus)};

    % 2. LINEA PERIELIO (Common Perihelion)
    \draw[dotted, gray, thick] (0,0) -- (\q+0.5, 0);
    \filldraw[black] (\q, 0) circle (1.5pt);
    \node[right, font=\footnotesize, text width=2cm] at (\q+0.1, 0) {Common\\Perihelion};

    % 3. CERCHIO (e=0)
    \draw[thick, green!50!black] (0,0) circle (\q);
    % Etichetta: Basso
    \node[green!50!black] (circle_label) at (0.5, -2.5) {Circle ($e=0$)};
    \draw[->, green!50!black, thin] (circle_label) -- (0.5, -\q);

    % 4. ELLISSE (e=0.6)
    \draw[thick, blue, samples=200, domain=0:360] plot ({\x}:{2.4/(1+0.6*cos(\x))});
    % Etichetta: Sinistra Alto
    \node[blue, anchor=east] (ellipse_label) at (-3.5, 2.0) {Ellipse ($e=0.6$)};
    \draw[->, blue, thin] (ellipse_label.east) -- (110:2.5);

    % 5. PARABOLA (e=1.0)
    \draw[thick, red, samples=100, domain=-120:120] plot ({\x}:{3.0/(1+cos(\x))});
    % Etichetta: Sinistra Più Alto
    \node[red, anchor=east] (parabola_label) at (-3.5, 3.2) {Parabola ($e=1.0$)};
    \draw[->, red, thin] (parabola_label.east) -- (110:4.0);

    % 6. IPERBOLE (e=1.5)
    \draw[thick, purple, samples=100, domain=-95:95] plot ({\x}:{3.75/(1+1.5*cos(\x))});
    % Etichetta: Sinistra Altissimo
    \node[purple, anchor=east] (hyperbola_label) at (-3.5, 4.5) {Hyperbola ($e=1.5$)};
    \draw[->, purple, thin] (hyperbola_label.east) -- (105:5.8);

\end{tikzpicture}
\caption{Family of conic orbits sharing a common focus (Sun) and perihelion distance. The eccentricity $e$ determines the "openness" of the orbit.}
\label{fig:conic_sections}
\end{figure}

\begin{table}[htbp]
\centering
\begin{tabular}{lccc}
\toprule
\textbf{Orbit Type} & \textbf{Eccentricity} & \textbf{Energy} & \textbf{Examples} \\
\midrule
Circle & $e = 0$ & $\mathcal{E} < 0$ & Idealized orbits \\
Ellipse & $0 < e < 1$ & $\mathcal{E} < 0$ & Planets, asteroids \\
Parabola & $e = 1$ & $\mathcal{E} = 0$ & Escape trajectory \\
Hyperbola & $e > 1$ & $\mathcal{E} > 0$ & Interstellar objects \\
\bottomrule
\end{tabular}
\caption{Classification of orbital conics by eccentricity and energy.}
\label{tab:orbit_types}
\end{table}

\section{Kepler's Laws}

Johannes Kepler (1571-1630) derived three empirical laws from observations of planetary motion. These are consequences of the two-body problem.

\subsection{Kepler's First Law (Law of Ellipses)}

\textit{The orbit of a planet is an ellipse with the Sun at one focus.}

Mathematically:
\begin{equation}
    r = \frac{a(1-e^2)}{1 + e\cos\nu}
\end{equation}

The perihelion distance is $r_p = a(1-e)$ and aphelion distance is $r_a = a(1+e)$.

\subsection{Kepler's Second Law (Law of Equal Areas)}

\textit{A line joining a planet and the Sun sweeps out equal areas in equal times.}

This follows from conservation of angular momentum:
\begin{equation}
    \frac{dA}{dt} = \frac{h}{2} = \frac{1}{2}\sqrt{\mu a(1-e^2)} = \text{constant}
\end{equation}

\subsection{Kepler's Third Law (Harmonic Law)}

\textit{The square of the orbital period is proportional to the cube of the semi-major axis.}

\begin{equation}
    P^2 = \frac{4\pi^2}{\mu}a^3
\end{equation}

For the solar system:
\begin{equation}
    P[\text{years}] = a[\text{AU}]^{3/2}
\end{equation}

\section{Kepler's Equation}

\subsection{The Anomalies}

Three related angles describe position in an elliptical orbit:

\begin{description}
    \item[True Anomaly ($\nu$)] Actual angle from perihelion to current position
    \item[Eccentric Anomaly ($E$)] Auxiliary angle on the circumscribed circle
    \item[Mean Anomaly ($M$)] Angle that would be covered if motion were uniform
\end{description}

\begin{figure}[htbp]
\centering
\begin{tikzpicture}[scale=1.0]
    % Ellipse a=3, b=2
    \draw[thick] (0,0) ellipse (3cm and 2cm);
    
    % Circumscribed circle r=3
    \draw[dashed,gray] (0,0) circle (3cm);
    
    % Sun at focus c = sqrt(9-4) = 2.236
    \filldraw[yellow,draw=orange] (2.236,0) circle (3pt);
    \node[below] at (2.236,-0.3) {Sun};
    
    % Center
    \filldraw[black] (0,0) circle (1.5pt);
    \node[below left] at (0,0) {$C$};
    
    % Use x = 1.5
    % y_ell = 2 * sqrt(1 - 0.5^2) = 1.732
    % y_circ = 3 * sqrt(1 - 0.5^2) = 2.598
    
    % Perihelion (right side for positive x focus)
    % Wait, usually nu is measured from Perihelion.
    % If Sun is at right focus (2.236,0), Perihelion is at (3,0).
    % Let's stick to standard: Focus at origin? No, TikZ draws ellipse centered at 0.
    % Let's put Sun at LEFT focus (-2.236,0), Perihelion at (-3,0).
    
    % Sun at LEFT focus
    \filldraw[yellow,draw=orange] (-2.236,0) circle (3pt);
    \node[below] at (-2.236,-0.3) {Sun};
    
    % Perihelion
    \draw[dashed] (-3,0) -- (3,0);
    \node[left] at (-3,0) {Perihelion};
    
    % Planet position at x=1.5, y=1.732
    \filldraw[blue] (1.5,1.732) circle (2pt);
    \node[above right,blue] at (1.5,1.732) {Planet};
    
    % Point on circle (for eccentric anomaly)
    \filldraw[red] (1.5,2.598) circle (1.5pt);
    
    % True anomaly vector (from Sun to Planet)
    \draw[->,purple,thick] (-2.236,0) -- (1.5,1.732);
    % Angle arc
    \draw[->,purple] (-1.736,0) arc (0:25:0.5); 
    \node[purple] at (-1.5,0.3) {$\nu$};
    
    % Eccentric anomaly vector (from Center to Circle Point)
    \draw[->,red] (0,0) -- (1.5,2.598);
    \draw[->,red] (0.5,0) arc (0:60:0.5);
    \node[red] at (0.6,0.3) {$E$};
    
    % Vertical line
    \draw[dotted] (1.5,0) -- (1.5,2.598);
    
    % Label x-axis projection
    \draw (1.5,0.1) -- (1.5,-0.1);
\end{tikzpicture}
\caption{Relationship between true anomaly $\nu$ and eccentric anomaly $E$ in an elliptical orbit.}
\label{fig:anomalies}
\end{figure}

\subsection{Kepler's Equation}

The mean anomaly advances uniformly with time:
\begin{equation}
    M = n(t - t_p) = \sqrt{\frac{\mu}{a^3}}(t - t_p)
\end{equation}

where $t_p$ is the time of perihelion passage and $n$ is the mean motion.

Kepler's equation relates $M$ to $E$:
\begin{equation}
    M = E - e\sin E
\end{equation}

This is a transcendental equation with no closed-form solution. It must be solved iteratively.

\subsection{Solving Kepler's Equation}

\textbf{Newton-Raphson Method}:

Given $M$ and $e$, find $E$ iteratively:
\begin{align}
    f(E) &= E - e\sin E - M = 0 \\
    f'(E) &= 1 - e\cos E \\
    E_{n+1} &= E_n - \frac{f(E_n)}{f'(E_n)} = E_n - \frac{E_n - e\sin E_n - M}{1 - e\cos E_n}
\end{align}

Initial guess: $E_0 = M$ (for small $e$) or $E_0 = M + e$ (better for moderate $e$).

Convergence is typically achieved in 3-5 iterations for $\epsilon < 10^{-12}$.

\subsection{Relationship Between Anomalies}

Once $E$ is known, the true anomaly is:
\begin{equation}
    \nu = 2\arctan\left(\sqrt{\frac{1+e}{1-e}}\tan\frac{E}{2}\right)
\end{equation}

Or equivalently:
\begin{align}
    \cos\nu &= \frac{\cos E - e}{1 - e\cos E} \\
    \sin\nu &= \frac{\sqrt{1-e^2}\sin E}{1 - e\cos E}
\end{align}

The radial distance is:
\begin{equation}
    r = a(1 - e\cos E)
\end{equation}

\section{The Vis-Viva Equation}

The \textbf{vis-viva equation} (``living force'') relates velocity to position:

\begin{equation}
    v^2 = \mu\left(\frac{2}{r} - \frac{1}{a}\right)
\end{equation}

This is derived from energy conservation and is valid for all conic orbits.

\subsection{Special Cases}

\textbf{At perihelion} ($r = a(1-e)$):
\begin{equation}
    v_p = \sqrt{\frac{\mu}{a}\frac{1+e}{1-e}}
\end{equation}

\textbf{At aphelion} ($r = a(1+e)$):
\begin{equation}
    v_a = \sqrt{\frac{\mu}{a}\frac{1-e}{1+e}}
\end{equation}

\textbf{Circular orbit} ($e = 0$, $r = a$):
\begin{equation}
    v_c = \sqrt{\frac{\mu}{a}}
\end{equation}

\textbf{Escape velocity} (parabolic, $e = 1$, $a \to \infty$):
\begin{equation}
    v_e = \sqrt{\frac{2\mu}{r}}
\end{equation}

\section{Parabolic and Hyperbolic Orbits}

\subsection{Parabolic Orbits ($e = 1$)}

For escape trajectories, the orbit equation becomes:
\begin{equation}
    r = \frac{p}{1 + \cos\nu}
\end{equation}

where $p$ is the periapsis distance.

The time-of-flight is given by Barker's equation:
\begin{equation}
    t - t_p = \frac{1}{2}\sqrt{\frac{p^3}{\mu}}\left(\tan\frac{\nu}{2} + \frac{1}{3}\tan^3\frac{\nu}{2}\right)
\end{equation}

\subsection{Hyperbolic Orbits ($e > 1$)}

For interstellar or flyby trajectories:
\begin{equation}
    r = \frac{a(e^2-1)}{1 + e\cos\nu}
\end{equation}

Note: $a < 0$ for hyperbolic orbits (negative energy).

The hyperbolic anomaly $F$ satisfies:
\begin{equation}
    M_h = e\sinh F - F
\end{equation}

And the true anomaly is:
\begin{equation}
    \nu = 2\arctan\left(\sqrt{\frac{e+1}{e-1}}\tanh\frac{F}{2}\right)
\end{equation}

The asymptotic velocity at infinity is:
\begin{equation}
    v_\infty = \sqrt{-\frac{\mu}{a}} = \sqrt{\mu\frac{e^2-1}{a}}
\end{equation}

\section{Lagrange Coefficients}

The \textbf{Lagrange coefficients} (or $f$ and $g$ functions) provide a way to propagate orbits without explicitly computing orbital elements.

\subsection{Definition}

Given initial state $(\mathbf{r}_0, \mathbf{v}_0)$ at time $t_0$, the state at time $t$ is:

\begin{align}
    \mathbf{r}(t) &= f(t)\mathbf{r}_0 + g(t)\mathbf{v}_0 \\
    \mathbf{v}(t) &= \dot{f}(t)\mathbf{r}_0 + \dot{g}(t)\mathbf{v}_0
\end{align}

where $f, g, \dot{f}, \dot{g}$ are scalar functions of time.

\subsection{Expressions for Lagrange Coefficients}

For elliptical orbits:
\begin{align}
    f &= 1 - \frac{a}{r_0}(1 - \cos\Delta E) \\
    g &= t - t_0 + \sqrt{\frac{a^3}{\mu}}(\sin\Delta E - \Delta E) \\
    \dot{f} &= -\sqrt{\frac{\mu a}{rr_0}}\sin\Delta E \\
    \dot{g} &= 1 - \frac{a}{r}(1 - \cos\Delta E)
\end{align}

where $\Delta E = E - E_0$ is the change in eccentric anomaly.

\subsection{Properties}

The Lagrange coefficients satisfy:
\begin{equation}
    f\dot{g} - \dot{f}g = 1
\end{equation}

This is the \textbf{Lagrange identity}, which ensures conservation of the Wronskian.

\section{AstDyn Implementation}

AstDyn provides functions for solving the two-body problem:

\begin{lstlisting}[language=C++,caption={Two-body problem in AstDyn}]
#include <astdyn/dynamics/TwoBody.hpp>
#include <astdyn/core/OrbitalElements.hpp>

using namespace astdyn;

// Solve Kepler's equation
double M = 45.0 * DEG_TO_RAD;  // Mean anomaly
double e = 0.3;                 // Eccentricity
double E = TwoBody::solve_kepler_equation(M, e);
std::cout << "Eccentric anomaly: " << E * RAD_TO_DEG << " deg\n";

// Convert E to true anomaly
double nu = TwoBody::eccentric_to_true_anomaly(E, e);
std::cout << "True anomaly: " << nu * RAD_TO_DEG << " deg\n";

// Compute position and velocity from orbital elements
OrbitalElements kep;
kep.a = 2.5;  // AU
kep.e = 0.15;
kep.M = M;
// ... set other elements

auto [r, v] = kep.to_position_velocity();

// Propagate using Lagrange coefficients
double dt = 100.0;  // days
auto [f, g, fdot, gdot] = TwoBody::lagrange_coefficients(
    r, v, dt, MU_SUN
);

Vector3d r_new = f * r + g * v;
Vector3d v_new = fdot * r + gdot * v;

std::cout << "New position: " << r_new.transpose() << " AU\n";
std::cout << "New velocity: " << v_new.transpose() << " AU/day\n";
\end{lstlisting}

\section{Summary}

Key points about the two-body problem:

\begin{enumerate}
    \item The two-body problem has an \textbf{exact analytical solution}
    \item Motion is governed by conservation of energy, angular momentum, and the eccentricity vector
    \item Orbits are \textbf{conic sections}: circles, ellipses, parabolas, or hyperbolas
    \item \textbf{Kepler's laws} are direct consequences of Newton's gravity
    \item \textbf{Kepler's equation} ($M = E - e\sin E$) relates mean and eccentric anomalies
    \item The \textbf{vis-viva equation} relates velocity to position
    \item \textbf{Lagrange coefficients} enable efficient orbit propagation
\end{enumerate}

Understanding the two-body problem is essential for:
\begin{itemize}
    \item Predicting planetary and asteroid positions
    \item Designing spacecraft trajectories
    \item Understanding the baseline from which perturbations deviate
    \item Developing efficient numerical propagators
\end{itemize}

In the next chapter, we will study perturbations—deviations from the ideal two-body motion caused by additional forces.

\chapter{Orbital Perturbations}
\label{ch:perturbations}

\section{Introduction to Perturbations}

In Chapter~\ref{ch:two_body_problem}, we studied the idealized two-body problem where only gravitational attraction between two point masses is considered. In reality, celestial bodies experience additional forces that cause their orbits to deviate from perfect Keplerian ellipses.

\subsection{Types of Perturbations}

Orbital perturbations can be classified by their physical origin:

\begin{description}
    \item[Gravitational] Forces from additional bodies (N-body problem), non-spherical mass distribution (J$_2$, J$_4$, etc.)
    \item[Non-gravitational] Solar radiation pressure, atmospheric drag, thermal effects (Yarkovsky)
    \item[Relativistic] General relativity corrections to Newtonian gravity
\end{description}

\subsection{Perturbed Equations of Motion}

The general equation of motion with perturbations is:

\begin{equation}
    \ddot{\mathbf{r}} = -\frac{\mu}{r^3}\mathbf{r} + \mathbf{a}_{\text{pert}}
\end{equation}

where $\mathbf{a}_{\text{pert}}$ is the perturbing acceleration. For small perturbations, we can treat them as corrections to the Keplerian solution.

\subsection{Magnitude of Effects}

For a main-belt asteroid at 2.5 AU:

\begin{table}[htbp]
\centering
\begin{tabular}{lcc}
\toprule
\textbf{Perturbation} & \textbf{Acceleration} & \textbf{Relative to Sun} \\
\midrule
Solar gravity & $3.8 \times 10^{-3}$ m/s$^2$ & 1 \\
Jupiter & $\sim 10^{-6}$ m/s$^2$ & $3 \times 10^{-4}$ \\
Earth J$_2$ (at LEO) & $\sim 10^{-6}$ m/s$^2$ & — \\
Solar radiation & $\sim 10^{-8}$ m/s$^2$ & $3 \times 10^{-6}$ \\
Relativity & $\sim 10^{-10}$ m/s$^2$ & $3 \times 10^{-8}$ \\
\bottomrule
\end{tabular}
\caption{Typical magnitudes of perturbing accelerations for asteroids.}
\label{tab:pert_magnitudes}
\end{table}

Though small, these effects accumulate over time and must be included for accurate long-term predictions.

\section{The N-Body Problem}

\subsection{Problem Statement}

The \textbf{N-body problem} considers the motion of $N$ bodies under their mutual gravitational attraction. The equation of motion for body $i$ is:

\begin{equation}
    \ddot{\mathbf{r}}_i = \sum_{j=1, j \neq i}^{N} G\frac{m_j(\mathbf{r}_j - \mathbf{r}_i)}{|\mathbf{r}_j - \mathbf{r}_i|^3}
\end{equation}

For $N \geq 3$, there is no general analytical solution. The problem must be solved numerically.

\subsection{The Restricted Three-Body Problem}

A special case is the \textbf{circular restricted three-body problem} (CR3BP):
\begin{itemize}
    \item Two massive bodies (primaries) orbit their common barycenter in circular orbits
    \item A third body (massless) moves under their gravitational influence
    \item The third body does not affect the primaries
\end{itemize}

This is relevant for Sun-Jupiter-asteroid or Earth-Moon-spacecraft systems.

\subsection{Perturbations from Planets}

For asteroid orbit determination, we typically model the Sun as the central body and planets as perturbing bodies:

\begin{equation}
    \mathbf{a}_{\text{planets}} = \sum_{p} \left[ G m_p \left( \frac{\mathbf{r}_p - \mathbf{r}}{|\mathbf{r}_p - \mathbf{r}|^3} - \frac{\mathbf{r}_p}{r_p^3} \right) \right]
\end{equation}

The first term is the direct gravitational pull from planet $p$, and the second term accounts for the fact that the Sun also accelerates toward the planet (indirect term).

\subsection{Planetary Ephemerides}

Accurate N-body modeling requires high-precision planetary positions. Common sources:

\begin{itemize}
    \item \textbf{JPL DE440/DE441}: NASA's latest planetary ephemerides (2021)
    \item \textbf{INPOP}: French planetary ephemerides from IMCCE
    \item \textbf{SPICE}: NASA's toolkit for spacecraft and planetary geometry
\end{itemize}

AstDyn can use SPICE kernels to obtain planetary states at any epoch.

\section{Oblateness Perturbations (J$_2$)}

\subsection{Non-Spherical Mass Distribution}

Real celestial bodies are not perfect spheres. Earth, for example, is oblate due to rotation. The gravitational potential can be expanded in spherical harmonics:

\begin{equation}
    U = -\frac{\mu}{r}\left[1 - \sum_{n=2}^{\infty} J_n \left(\frac{R}{r}\right)^n P_n(\sin\phi)\right]
\end{equation}

where:
\begin{itemize}
    \item $J_n$ are the zonal harmonic coefficients
    \item $R$ is the reference radius
    \item $P_n$ are Legendre polynomials
    \item $\phi$ is the latitude
\end{itemize}

\subsection{The J$_2$ Term}

The dominant term is $J_2$ (quadrupole moment), representing equatorial bulge:

\begin{equation}
    U_{J_2} = -\frac{\mu}{r}\left[1 - J_2 \left(\frac{R}{r}\right)^2 P_2(\sin\phi)\right]
\end{equation}

where $P_2(\sin\phi) = \frac{1}{2}(3\sin^2\phi - 1)$.

For Earth: $J_2 = 1.08263 \times 10^{-3}$ (about 0.1\%)

\subsection{J$_2$ Acceleration}

The perturbing acceleration in Cartesian coordinates is:

\begin{align}
    a_x &= -\frac{3\mu J_2 R^2}{2r^5}\left[x\left(1 - 5\frac{z^2}{r^2}\right)\right] \\
    a_y &= -\frac{3\mu J_2 R^2}{2r^5}\left[y\left(1 - 5\frac{z^2}{r^2}\right)\right] \\
    a_z &= -\frac{3\mu J_2 R^2}{2r^5}\left[z\left(3 - 5\frac{z^2}{r^2}\right)\right]
\end{align}

\subsection{Effects on Orbital Elements}

J$_2$ causes secular (long-term) changes in orbital elements:

\begin{align}
    \frac{d\Omega}{dt} &= -\frac{3}{2}\frac{n J_2 R^2}{a^2(1-e^2)^2}\cos i \\
    \frac{d\omega}{dt} &= \frac{3}{4}\frac{n J_2 R^2}{a^2(1-e^2)^2}(5\cos^2 i - 1)
\end{align}

where $n = \sqrt{\mu/a^3}$ is the mean motion.

\textbf{Key effects}:
\begin{itemize}
    \item $\Omega$ (RAAN) precesses westward for prograde orbits ($i < 90^\circ$)
    \item $\omega$ (argument of periapsis) rotates
    \item The combination creates complex patterns in ground tracks
\end{itemize}

For low Earth orbit (LEO), J$_2$ can cause $\Omega$ to change by several degrees per day.

\section{Solar Radiation Pressure}

\subsection{Physical Mechanism}

Photons carry momentum. When sunlight hits an object, it exerts a force:

\begin{equation}
    F_{\text{SRP}} = P_\odot \frac{A}{c} C_R \left(\frac{r_0}{r}\right)^2
\end{equation}

where:
\begin{itemize}
    \item $P_\odot = 4.56 \times 10^{-6}$ N/m$^2$ is solar radiation pressure at 1 AU
    \item $A$ is the cross-sectional area
    \item $c = 3 \times 10^8$ m/s is the speed of light
    \item $C_R$ is the radiation pressure coefficient ($C_R \approx 1$-$2$)
    \item $r_0 = 1$ AU, $r$ is the heliocentric distance
\end{itemize}

\subsection{Area-to-Mass Ratio}

The acceleration depends on the \textbf{area-to-mass ratio}:

\begin{equation}
    \mathbf{a}_{\text{SRP}} = P_\odot \frac{A}{m} C_R \left(\frac{r_0}{r}\right)^2 \hat{\mathbf{r}}_\odot
\end{equation}

Small objects (dust, small asteroids) are more affected than large ones.

\subsection{Eclipse Modeling}

SRP drops to zero when the object is in Earth's or planetary shadow. A simple model:

\begin{equation}
    \nu = \begin{cases}
        1 & \text{in sunlight} \\
        0 & \text{in umbra} \\
        f & \text{in penumbra}
    \end{cases}
\end{equation}

where $0 < f < 1$ depends on the fraction of the solar disk visible.

\subsection{Yarkovsky Effect}

The \textbf{Yarkovsky effect} is a thermal recoil force:
\begin{itemize}
    \item Asteroid's surface heats in sunlight
    \item Emits thermal radiation as it rotates
    \item Creates a small thrust (like a rocket!)
\end{itemize}

This is important for small asteroids ($< 20$ km) over long timescales (millions of years). It can change the semi-major axis:

\begin{equation}
    \frac{da}{dt} \approx \pm 10^{-4} \text{ AU/Myr}
\end{equation}

The sign depends on the sense of rotation (prograde vs retrograde).

\section{Relativistic Effects}

\subsection{Post-Newtonian Corrections}

General relativity introduces corrections to Newtonian gravity. The dominant term is the \textbf{Schwarzschild term}:

\begin{equation}
    \mathbf{a}_{\text{GR}} = \frac{\mu}{c^2 r^3}\left[4\frac{\mu}{r}\mathbf{r} - (\mathbf{v} \cdot \mathbf{v})\mathbf{r} + 4(\mathbf{r} \cdot \mathbf{v})\mathbf{v}\right]
\end{equation}

This is the first-order post-Newtonian (1PN) approximation.

\subsection{Perihelion Precession}

The most famous relativistic effect is the \textbf{precession of perihelion}:

\begin{equation}
    \Delta\omega = \frac{6\pi G M_\odot}{c^2 a(1-e^2)} \text{ per orbit}
\end{equation}

For Mercury ($a = 0.387$ AU, $e = 0.206$):
\begin{equation}
    \Delta\omega_{\text{Mercury}} = 43'' \text{ per century}
\end{equation}

This was famously explained by Einstein in 1915 and was one of the first confirmations of general relativity.

\subsection{Light-Time Correction}

Electromagnetic signals travel at finite speed. When measuring asteroid positions via radar or optical observations, we must account for the time the light takes to travel:

\begin{equation}
    \Delta t = \frac{|\mathbf{r}_{\text{obs}} - \mathbf{r}_{\text{ast}}|}{c}
\end{equation}

This is the \textbf{light-time correction}. For orbit determination, we must iterate to find the position of the asteroid at the time of observation, not at the time of detection.

\subsection{Shapiro Delay}

Gravitational fields slow down light. The \textbf{Shapiro delay} is:

\begin{equation}
    \Delta t_{\text{Shapiro}} = \frac{2GM_\odot}{c^3}\ln\left(\frac{r_1 + r_2 + d}{r_1 + r_2 - d}\right)
\end{equation}

where $r_1$, $r_2$ are distances from the Sun to the two endpoints, and $d$ is their separation. This is typically $\sim 100$ microseconds but is measurable with precision ranging.

\section{Atmospheric Drag}

For satellites in low Earth orbit (LEO), atmospheric drag is a major perturbation.

\subsection{Drag Equation}

\begin{equation}
    \mathbf{a}_{\text{drag}} = -\frac{1}{2}\frac{C_D A}{m}\rho v^2 \hat{\mathbf{v}}
\end{equation}

where:
\begin{itemize}
    \item $C_D \approx 2.2$ is the drag coefficient
    \item $A$ is the cross-sectional area
    \item $\rho$ is atmospheric density (exponentially decreasing with altitude)
    \item $v$ is the velocity relative to the atmosphere
\end{itemize}

\subsection{Atmospheric Density Models}

Density depends on:
\begin{itemize}
    \item Altitude (exponential decrease)
    \item Solar activity (F10.7 index)
    \item Geomagnetic activity (Ap index)
    \item Local solar time and latitude
\end{itemize}

Common models: NRLMSISE-00, JB2008, DTM2000.

\subsection{Orbital Decay}

Drag causes the semi-major axis to decrease:

\begin{equation}
    \frac{da}{dt} = -\frac{2a^2}{v}\frac{C_D A}{m}\rho v^2 = -\frac{C_D A}{m}\rho a^2 v
\end{equation}

Satellites in LEO gradually spiral inward and eventually re-enter the atmosphere.

\section{Perturbation Theory}

\subsection{Variation of Parameters}

\textbf{Lagrange's planetary equations} describe how orbital elements change under perturbations. In terms of the disturbing function $R$:

\begin{align}
    \frac{da}{dt} &= \frac{2}{na}\frac{\partial R}{\partial M} \\
    \frac{de}{dt} &= \frac{1-e^2}{na^2 e}\frac{\partial R}{\partial M} - \frac{\sqrt{1-e^2}}{na^2 e}\frac{\partial R}{\partial \omega} \\
    \frac{di}{dt} &= \frac{\cos i}{na^2\sqrt{1-e^2}\sin i}\frac{\partial R}{\partial \omega} - \frac{1}{na^2\sqrt{1-e^2}\sin i}\frac{\partial R}{\partial \Omega} \\
    \frac{d\Omega}{dt} &= \frac{1}{na^2\sqrt{1-e^2}\sin i}\frac{\partial R}{\partial i} \\
    \frac{d\omega}{dt} &= \frac{\sqrt{1-e^2}}{na^2 e}\frac{\partial R}{\partial e} - \frac{\cos i}{na^2\sqrt{1-e^2}\sin i}\frac{\partial R}{\partial i} \\
    \frac{dM}{dt} &= n - \frac{2}{na}\frac{\partial R}{\partial a} - \frac{1-e^2}{na^2 e}\frac{\partial R}{\partial e}
\end{align}

These equations allow analytical treatment of perturbations when $R$ has a simple form.

\subsection{Gauss's Perturbation Equations}

An alternative formulation uses the perturbing acceleration components $(S, T, W)$ in the radial, transverse, and normal directions:

\begin{align}
    \frac{da}{dt} &= \frac{2a^2}{h}\left[eS\sin\nu + T\frac{p}{r}\right] \\
    \frac{de}{dt} &= \frac{1}{v_0}\left[S\sin\nu + T\left(\cos\nu + \frac{r+p}{p}\cos E\right)\right] \\
    \frac{di}{dt} &= \frac{r\cos(\omega + \nu)}{h}W
\end{align}

where $h = \sqrt{\mu a(1-e^2)}$ is the angular momentum magnitude.

\subsection{Osculating Elements}

At any instant, the orbit can be described by \textbf{osculating elements}—the Keplerian elements that the body would follow if all perturbations suddenly ceased. These elements vary continuously under perturbations.

\section{Numerical Integration vs Perturbation Theory}

\subsection{When to Use Each Approach}

\begin{table}[htbp]
\centering
\begin{tabular}{lll}
\toprule
\textbf{Method} & \textbf{Advantages} & \textbf{Best For} \\
\midrule
Numerical & Handles any force & Short-term accuracy \\
Integration & No approximations & Strong perturbations \\
 & Easy to implement & Multiple forces \\
\midrule
Analytical & Physical insight & Long-term trends \\
Perturbation & Fast computation & Weak perturbations \\
Theory & Identifies resonances & Qualitative analysis \\
\bottomrule
\end{tabular}
\caption{Comparison of numerical integration and analytical perturbation theory.}
\label{tab:methods}
\end{table}

\subsection{Hybrid Approaches}

Modern orbit determination often uses:
\begin{enumerate}
    \item Numerical integration for the equation of motion
    \item Analytical theory to identify important perturbations
    \item Simplified models (e.g., averaged J$_2$) for faster computation
\end{enumerate}

\section{AstDyn Implementation}

AstDyn provides a modular perturbation framework:

\begin{lstlisting}[language=C++,caption={Perturbations in AstDyn}]
#include <astdyn/dynamics/Perturbations.hpp>
#include <astdyn/dynamics/NBody.hpp>

using namespace astdyn;

// Create state vector
Vector6d state = ...;  // [x, y, z, vx, vy, vz]

// N-body perturbations from planets
PlanetaryEphemeris ephem("de440.bsp");
Vector3d acc_planets = NBody::compute_perturbation(
    state, time, ephem, {"Jupiter", "Saturn", "Earth"}
);

// J2 perturbation (for Earth orbiter)
Vector3d acc_j2 = Perturbations::j2_acceleration(
    state, MU_EARTH, R_EARTH, J2_EARTH
);

// Solar radiation pressure
double area_mass_ratio = 0.01;  // m^2/kg
double Cr = 1.3;
Vector3d sun_direction = ...;   // Unit vector to Sun
Vector3d acc_srp = Perturbations::solar_radiation_pressure(
    state, area_mass_ratio, Cr, sun_direction
);

// Relativistic correction
Vector3d acc_gr = Perturbations::schwarzschild_correction(
    state, MU_SUN
);

// Total perturbing acceleration
Vector3d acc_total = acc_planets + acc_j2 + acc_srp + acc_gr;

// Add to equations of motion
Vector6d derivatives;
derivatives.head<3>() = state.tail<3>();  // velocity
derivatives.tail<3>() = -MU_SUN * state.head<3>() / r^3 + acc_total;
\end{lstlisting}

\subsection{Perturbation Selection}

AstDyn allows users to enable/disable perturbations:

\begin{lstlisting}[language=C++,caption={Configuring perturbations}]
PerturbationModel model;
model.enable_planets({"Jupiter", "Saturn", "Uranus", "Neptune"});
model.enable_j2(false);  // Not relevant for heliocentric orbits
model.enable_srp(true);
model.enable_relativity(true);

// Use in propagation
Propagator prop(model);
auto final_state = prop.propagate(initial_state, t0, tf);
\end{lstlisting}

\section{Summary}

Key concepts about orbital perturbations:

\begin{enumerate}
    \item \textbf{Perturbations} are deviations from ideal two-body motion
    \item \textbf{N-body effects} from planets are the dominant perturbation for asteroids
    \item \textbf{J$_2$ oblateness} causes precession of $\Omega$ and $\omega$ (critical for Earth satellites)
    \item \textbf{Solar radiation pressure} affects small bodies and spacecraft
    \item \textbf{Relativistic effects} are small but measurable (Mercury precession: 43''/century)
    \item \textbf{Atmospheric drag} dominates in LEO, causing orbital decay
    \item \textbf{Perturbation theory} (Lagrange, Gauss equations) provides analytical insight
    \item \textbf{Numerical integration} handles arbitrary force models accurately
\end{enumerate}

Understanding perturbations is essential for:
\begin{itemize}
    \item Accurate orbit prediction over long timescales
    \item Satellite mission design and station-keeping
    \item Detecting subtle effects (e.g., asteroid masses from perturbations)
    \item Distinguishing gravitational from non-gravitational forces
\end{itemize}

In the next chapter, we will discuss numerical integration methods for solving the perturbed equations of motion.


% Part II
\part{Numerical Methods and Algorithms}
\chapter{Numerical Integration Methods}
\label{ch:numerical_integration}

\section{Introduction}

In Chapter~\ref{ch:perturbations}, we saw that orbital motion with perturbations requires solving:

\begin{equation}
    \ddot{\mathbf{r}} = -\frac{\mu}{r^3}\mathbf{r} + \mathbf{a}_{\text{pert}}(t, \mathbf{r}, \dot{\mathbf{r}})
\end{equation}

For general perturbations, this differential equation has no closed-form solution. We must use \textbf{numerical integration} to compute the orbit step-by-step.

This chapter reviews the main classes of integrators used in celestial mechanics and discusses their strengths, weaknesses, and implementation in AstDyn.

\subsection{The Initial Value Problem}

We seek to solve:
\begin{equation}
    \dot{\mathbf{y}} = \mathbf{f}(t, \mathbf{y}), \quad \mathbf{y}(t_0) = \mathbf{y}_0
\end{equation}

where $\mathbf{y} = [\mathbf{r}, \mathbf{v}]^T$ is the 6-dimensional state vector.

The goal is to advance from $(t_0, \mathbf{y}_0)$ to $(t_f, \mathbf{y}_f)$ with controlled error.

\section{Euler's Method}

The simplest integrator is \textbf{Euler's method}:

\begin{equation}
    \mathbf{y}_{n+1} = \mathbf{y}_n + h\mathbf{f}(t_n, \mathbf{y}_n)
\end{equation}

where $h = t_{n+1} - t_n$ is the step size.

\textbf{Pros}: Simple, explicit
\textbf{Cons}: First-order accurate ($O(h^2)$ error per step), unstable for stiff problems

Euler's method is rarely used in practice except for pedagogical purposes.

\section{Runge-Kutta Methods}

\subsection{The RK4 Method}

The classic \textbf{fourth-order Runge-Kutta} (RK4) method is:

\begin{align}
    k_1 &= h\mathbf{f}(t_n, \mathbf{y}_n) \\
    k_2 &= h\mathbf{f}(t_n + h/2, \mathbf{y}_n + k_1/2) \\
    k_3 &= h\mathbf{f}(t_n + h/2, \mathbf{y}_n + k_2/2) \\
    k_4 &= h\mathbf{f}(t_n + h, \mathbf{y}_n + k_3) \\
    \mathbf{y}_{n+1} &= \mathbf{y}_n + \frac{1}{6}(k_1 + 2k_2 + 2k_3 + k_4)
\end{align}

\textbf{Pros}: Fourth-order accurate ($O(h^5)$ per step), self-starting, easy to implement
\textbf{Cons}: Requires 4 function evaluations per step, no error estimate

RK4 is widely used for moderate-accuracy problems.

\subsection{Embedded Runge-Kutta Methods}

For adaptive step size control, we use \textbf{embedded} methods that provide two solutions of different orders:

\textbf{Runge-Kutta-Fehlberg 4(5)} (RKF45):
\begin{itemize}
    \item Computes 4th-order and 5th-order solutions
    \item Error estimate: $\epsilon = |\mathbf{y}_5 - \mathbf{y}_4|$
    \item 6 function evaluations per step
\end{itemize}

\textbf{Dormand-Prince 5(4)} (DOPRI54 or RK54):
\begin{itemize}
    \item Optimized coefficients for better stability
    \item 7 function evaluations (one reused for next step)
    \item Default in MATLAB's ode45
\end{itemize}

\textbf{Runge-Kutta-Fehlberg 7(8)} (RKF78):
\begin{itemize}
    \item 7th and 8th order solutions
    \item 13 function evaluations
    \item Best for high-accuracy requirements
\end{itemize}

\begin{algorithm}
\caption{RKF78 Integration Step}
\begin{algorithmic}[1]
\REQUIRE Current state $t_n, \mathbf{y}_n$, Step size $h$
\ENSURE Next state $t_{n+1}, \mathbf{y}_{n+1}$, Next step $h_{new}$
\STATE \textbf{Compute Stages}:
\FOR{$i=1$ to $13$}
    \STATE $T_i = t_n + c_i h$
    \STATE $\mathbf{Y}_i = \mathbf{y}_n + h \sum_{j=1}^{i-1} a_{ij} \mathbf{k}_j$
    \STATE $\mathbf{k}_i = \mathbf{f}(T_i, \mathbf{Y}_i)$
\ENDFOR
\STATE \textbf{Update State}: $\mathbf{y}_{n+1} = \mathbf{y}_n + h \sum_{i=1}^{13} b_i \mathbf{k}_i$
\STATE \textbf{Error Estimation}: $\hat{\mathbf{y}}_{n+1} = \mathbf{y}_n + h \sum_{i=1}^{13} \hat{b}_i \mathbf{k}_i$
\STATE $\epsilon = \norm{\mathbf{y}_{n+1} - \hat{\mathbf{y}}_{n+1}}_\infty$
\STATE \textbf{Step Size Control}:
\IF{$\epsilon \le TOL$}
    \STATE Accept step: $t_{n+1} = t_n + h$
    \STATE $h_{new} = h \cdot 0.9 \cdot \left(\frac{TOL}{\epsilon}\right)^{1/8}$
\ELSE
    \STATE Reject step: $t_{n+1} = t_n$
    \STATE $h_{new} = h \cdot 0.9 \cdot \left(\frac{TOL}{\epsilon}\right)^{1/8}$
    \STATE Repeat step with $h \leftarrow h_{new}$
\ENDIF
\end{algorithmic}
\end{algorithm}

\section{Implicit Gauss-Legendre Integrator}
For symplectic integration, we solve the implicit Runge-Kutta equations using a simplified Newton-Raphson iteration.

\begin{algorithm}
\caption{Implicit Gauss-Legendre Step (Order $2s$)}
\begin{algorithmic}[1]
\REQUIRE $t_n, \mathbf{y}_n, h$, Stages $s=4$
\STATE Initialize stages $\mathbf{Z}_i^{(0)} = \mathbf{0}$
\STATE \textbf{Newton Iteration} $k=0 \dots k_{max}$:
\FOR{$i=1$ to $s$}
    \STATE $\mathbf{Y}_i = \mathbf{y}_n + \mathbf{Z}_i^{(k)}$
    \STATE $\mathbf{R}_i = \mathbf{Z}_i^{(k)} - h \sum_{j=1}^s a_{ij} \mathbf{f}(t_n + c_j h, \mathbf{y}_n + \mathbf{Z}_j^{(k)})$
    \STATE Solve linear system to update $\mathbf{Z}^{(k+1)}$
\ENDFOR
\IF{Converged}
    \STATE $\mathbf{y}_{n+1} = \mathbf{y}_n + h \sum_{i=1}^s b_i \mathbf{f}(t_n+c_i h, \mathbf{y}_n + \mathbf{Z}_i)$
    \RETURN
\ENDIF
\end{algorithmic}
\end{algorithm}

\subsection{Step Size Control}

Given error estimate $\epsilon$, adjust step size $h$:

\begin{equation}
    h_{\text{new}} = h_{\text{old}} \left(\frac{\text{tol}}{\epsilon}\right)^{1/(q+1)} \times \text{safety factor}
\end{equation}

where $q$ is the order and safety factor $\approx 0.9$.

If $\epsilon > \text{tol}$: reject step, reduce $h$
If $\epsilon < \text{tol}$: accept step, possibly increase $h$

\section{Multistep Methods}

\subsection{Adams-Bashforth-Moulton (ABM)}

Multistep methods use information from previous steps. The \textbf{Adams family} is popular:

\textbf{Adams-Bashforth (explicit predictor)}:
\begin{equation}
    \mathbf{y}_{n+1}^P = \mathbf{y}_n + h\sum_{i=0}^{k-1} \beta_i \mathbf{f}_{n-i}
\end{equation}

\textbf{Adams-Moulton (implicit corrector)}:
\begin{equation}
    \mathbf{y}_{n+1}^C = \mathbf{y}_n + h\sum_{i=-1}^{k-1} \beta_i^* \mathbf{f}_{n-i}
\end{equation}

The \textbf{predictor-corrector} (PC) mode evaluates:
\begin{enumerate}
    \item Predict $\mathbf{y}_{n+1}^P$ using Adams-Bashforth
    \item Evaluate $\mathbf{f}_{n+1} = \mathbf{f}(t_{n+1}, \mathbf{y}_{n+1}^P)$
    \item Correct $\mathbf{y}_{n+1}^C$ using Adams-Moulton
\end{enumerate}

\textbf{ABM12}: 12th-order Adams-Bashforth-Moulton
\begin{itemize}
    \item Uses 12 previous steps
    \item Very high accuracy for smooth problems
    \item Used by JPL for planetary ephemerides
\end{itemize}

\textbf{Pros}: High order with few function evaluations (2 per step after startup)
\textbf{Cons}: Not self-starting, requires fixed step size (or careful variable-step algorithm)

\subsection{Backward Differentiation Formulas (BDF)}

For \textbf{stiff} problems (not common in orbital mechanics), BDF methods are preferred:

\begin{equation}
    \sum_{i=0}^{k} \alpha_i \mathbf{y}_{n+1-i} = h\mathbf{f}(t_{n+1}, \mathbf{y}_{n+1})
\end{equation}

These are implicit and require solving nonlinear equations at each step.

\section{Symplectic Integrators}

\subsection{Hamiltonian Mechanics}

For conservative systems, the equations of motion can be written in Hamiltonian form:

\begin{align}
    \dot{\mathbf{q}} &= \frac{\partial H}{\partial \mathbf{p}} \\
    \dot{\mathbf{p}} &= -\frac{\partial H}{\partial \mathbf{q}}
\end{align}

where $\mathbf{q}$ are positions, $\mathbf{p}$ are momenta, and $H$ is the Hamiltonian (total energy).

\subsection{Symplectic Property}

A method is \textbf{symplectic} if it preserves the symplectic structure of phase space. This ensures:
\begin{itemize}
    \item Energy oscillates around true value (no systematic drift)
    \item Long-term stability
    \item Preservation of geometrical structures (e.g., periodic orbits)
\end{itemize}

\subsection{Leapfrog Method}

The simplest symplectic integrator is \textbf{leapfrog} (Verlet):

\begin{align}
    \mathbf{v}_{n+1/2} &= \mathbf{v}_n + \frac{h}{2}\mathbf{a}_n \\
    \mathbf{r}_{n+1} &= \mathbf{r}_n + h\mathbf{v}_{n+1/2} \\
    \mathbf{v}_{n+1} &= \mathbf{v}_{n+1/2} + \frac{h}{2}\mathbf{a}_{n+1}
\end{align}

\textbf{Pros}: Symplectic, second-order, simple
\textbf{Cons}: Requires splitting the Hamiltonian, not suitable for velocity-dependent forces

\subsection{Higher-Order Symplectic Methods}

\textbf{Yoshida's method} (4th-order symplectic):
\begin{itemize}
    \item Composition of leapfrog steps with carefully chosen coefficients
    \item Used in N-body simulations
\end{itemize}

\textbf{Wisdom-Holman method}:
\begin{itemize}
    \item Splits Hamiltonian into Keplerian + perturbation parts
    \item Keplerian part solved analytically
    \item Perturbations handled with kicks
\end{itemize}

Symplectic methods are ideal for long-term integrations ($10^6$-$10^9$ years) where energy conservation is critical.

\section{Error Analysis}

\subsection{Local vs Global Error}

\textbf{Local truncation error (LTE)}: Error introduced in a single step

\textbf{Global error}: Accumulated error after many steps

For a method of order $p$:
\begin{itemize}
    \item LTE $\propto h^{p+1}$
    \item Global error $\propto h^p$ (over fixed interval)
\end{itemize}

\subsection{Accuracy vs Efficiency Trade-off}

\begin{table}[htbp]
\centering
\begin{tabular}{lccc}
\toprule
\textbf{Method} & \textbf{Order} & \textbf{Evals/step} & \textbf{Best For} \\
\midrule
Euler & 1 & 1 & Teaching only \\
RK4 & 4 & 4 & Moderate accuracy \\
RKF45 & 4(5) & 6 & General purpose \\
DOPRI54 & 5(4) & 7 & High accuracy \\
RK78 & 7(8) & 13 & Very high accuracy \\
ABM12 & 12 & 2 & Smooth, high accuracy \\
Leapfrog & 2 & 2 & Long-term, conservative \\
\bottomrule
\end{tabular}
\caption{Comparison of numerical integration methods.}
\label{tab:integrators}
\end{table}

\subsection{Error Sources}

In orbit determination, errors come from:

\begin{enumerate}
    \item \textbf{Truncation error}: Finite step size
    \item \textbf{Roundoff error}: Finite precision arithmetic
    \item \textbf{Force model error}: Incomplete or inaccurate perturbations
    \item \textbf{Ephemeris error}: Planetary position uncertainties
\end{enumerate}

For high-precision work, all sources must be controlled.

\section{Practical Considerations}

\subsection{Choosing an Integrator}

\textbf{For orbit determination (days to years)}:
\begin{itemize}
    \item DOPRI54 with adaptive step size
    \item Tolerance: $10^{-12}$ to $10^{-14}$
\end{itemize}

\textbf{For long-term evolution (millions of years)}:
\begin{itemize}
    \item Wisdom-Holman or Yoshida symplectic
    \item Fixed step size (0.1-1 day)
\end{itemize}

\textbf{For real-time applications}:
\begin{itemize}
    \item RK4 with fixed step size
    \item Precompute step size for stability
\end{itemize}

\subsection{Step Size Selection}

Rule of thumb: $h \approx 0.01 \times T_{\text{orbit}}$

For asteroid at 2.5 AU:
\begin{itemize}
    \item Period $T \approx 4$ years = 1461 days
    \item Good step size: $h \approx 10$-15 days
\end{itemize}

Adaptive methods automatically adjust $h$ based on local behavior.

\subsection{Initial Step Size}

For adaptive methods, initial step size estimate:

\begin{equation}
    h_0 = 0.01 \times \min\left(\frac{|\mathbf{r}|}{|\dot{\mathbf{r}}|}, \frac{|\dot{\mathbf{r}}|}{|\ddot{\mathbf{r}}|}\right)
\end{equation}

This prevents taking too large a first step.

\section{AstDyn Implementation}

AstDyn provides multiple integrators:

\begin{lstlisting}[language=C++,caption={Using integrators in AstDyn}]
#include <astdyn/integration/Integrator.hpp>
#include <astdyn/integration/RK4.hpp>
#include <astdyn/integration/DOPRI54.hpp>

using namespace astdyn;

// Define the ODE system
auto ode = [](double t, const Vector6d& y) -> Vector6d {
    Vector3d r = y.head<3>();
    Vector3d v = y.tail<3>();
    Vector3d a = -MU_SUN * r / pow(r.norm(), 3);
    
    Vector6d dydt;
    dydt << v, a;
    return dydt;
};

// Initial state
Vector6d y0;
y0 << 1.0, 0.0, 0.0,  // position (AU)
      0.0, 6.28, 0.0;  // velocity (AU/day)

double t0 = 0.0;
double tf = 365.25;  // 1 year

// Option 1: Fixed-step RK4
RK4Integrator<Vector6d> rk4;
double h = 1.0;  // 1-day steps
auto result_rk4 = rk4.integrate(ode, t0, y0, tf, h);

// Option 2: Adaptive RKF78
RKF78Integrator<Vector6d> integrator;
integrator.set_tolerance(1e-12);
auto result_rkf = integrator.integrate(ode, t0, y0, tf);

std::cout << "Final position (RK4):    " 
          << result_rk4.transpose() << "\n";
std::cout << "Final position (DOPRI):  " 
          << result_dopri.transpose() << "\n";
\end{lstlisting}

\subsection{Custom Integrators}

Users can implement custom integrators by inheriting from \texttt{IntegratorBase}:

\begin{lstlisting}[language=C++,caption={Custom integrator interface}]
template<typename StateType>
class CustomIntegrator : public IntegratorBase<StateType> {
public:
    StateType integrate(
        const ODEFunction<StateType>& f,
        double t0,
        const StateType& y0,
        double tf
    ) override {
        // Implementation here
    }
};
\end{lstlisting}

\section{Summary}

Key concepts about numerical integration:

\begin{enumerate}
    \item \textbf{Runge-Kutta methods} are versatile and self-starting
    \item \textbf{Adaptive step size} (RKF45, DOPRI54) provides automatic error control
    \item \textbf{Multistep methods} (ABM) are efficient for smooth problems
    \item \textbf{Symplectic integrators} preserve energy for long-term simulations
    \item \textbf{Trade-offs} exist between accuracy, efficiency, and stability
    \item \textbf{Step size} should be chosen based on orbital period and accuracy requirements
\end{enumerate}

Understanding numerical integration is essential for:
\begin{itemize}
    \item Accurate orbit propagation
    \item Balancing computational cost and precision
    \item Avoiding numerical artifacts
    \item Validating results against analytical solutions
\end{itemize}

In the next chapter, we will apply these integration methods to practical orbit propagation problems.

\chapter{Orbit Propagation}
\label{ch:orbit_propagation}

\section{Introduction}

\textbf{Orbit propagation} is the process of computing the position and velocity of a celestial body at future (or past) times, given its initial state and the forces acting on it. This is fundamental to:

\begin{itemize}
    \item Predicting where to point telescopes for asteroid observations
    \item Planning spacecraft maneuvers
    \item Computing ephemerides for almanacs
    \item Analyzing long-term orbital evolution
    \item Assessing collision risks
\end{itemize}

Building on the integration methods from Chapter~\ref{ch:numerical_integration}, this chapter describes practical orbit propagation in AstDyn.

\section{Problem Formulation}

\subsection{The Propagation Task}

Given:
\begin{itemize}
    \item Initial epoch $t_0$ (in some time scale, usually TDB)
    \item Initial state $\mathbf{y}_0 = [\mathbf{r}_0, \mathbf{v}_0]$ (position and velocity)
    \item Force model $\mathbf{f}(t, \mathbf{r}, \mathbf{v})$ (accelerations)
    \item Target epoch $t_f$
\end{itemize}

Compute:
\begin{itemize}
    \item Final state $\mathbf{y}_f = [\mathbf{r}_f, \mathbf{v}_f]$
    \item Optionally: state transition matrix $\Phi(t_f, t_0)$
\end{itemize}

\subsection{State Vector}

For heliocentric orbits, the state vector is:

\begin{equation}
    \mathbf{y} = \begin{bmatrix} x \\ y \\ z \\ \dot{x} \\ \dot{y} \\ \dot{z} \end{bmatrix}
\end{equation}

Units in AstDyn:
\begin{itemize}
    \item Position: AU (astronomical units)
    \item Velocity: AU/day
    \item Time: days (MJD or JD)
\end{itemize}

\subsection{Equations of Motion}

The general form is:

\begin{equation}
    \frac{d}{dt}\begin{bmatrix} \mathbf{r} \\ \mathbf{v} \end{bmatrix} = \begin{bmatrix} \mathbf{v} \\ \mathbf{a}(t, \mathbf{r}, \mathbf{v}) \end{bmatrix}
\end{equation}

where the acceleration includes:

\begin{equation}
    \mathbf{a} = \mathbf{a}_{\text{central}} + \mathbf{a}_{\text{planets}} + \mathbf{a}_{\text{relativity}} + \mathbf{a}_{\text{SRP}} + \ldots
\end{equation}

\section{Force Models}

\subsection{Central Body Gravity}

The dominant term for solar system orbits:

\begin{equation}
    \mathbf{a}_{\text{Sun}} = -\frac{\mu_\odot}{r^3}\mathbf{r}
\end{equation}

where $\mu_\odot = 1.32712440018 \times 10^{20}$ m$^3$/s$^2$ = $0.295912208286$ AU$^3$/day$^2$.

\subsection{Planetary Perturbations}

For each perturbing planet $p$:

\begin{equation}
    \mathbf{a}_p = \mu_p \left[\frac{\mathbf{r}_p - \mathbf{r}}{|\mathbf{r}_p - \mathbf{r}|^3} - \frac{\mathbf{r}_p}{r_p^3}\right]
\end{equation}

The first term is the direct attraction, the second is the indirect effect (Sun's acceleration toward the planet).

Planetary positions $\mathbf{r}_p(t)$ are obtained from:
\begin{itemize}
    \item SPICE kernels (JPL DE440/441)
    \item VSOP87 analytical theory
    \item Simplified Keplerian ephemerides (lower accuracy)
\end{itemize}

\subsection{Relativistic Correction}

Post-Newtonian (1PN) term:

\begin{equation}
    \mathbf{a}_{\text{GR}} = \frac{\mu_\odot}{c^2 r^3}\left[4\frac{\mu_\odot}{r}\mathbf{r} - v^2\mathbf{r} + 4(\mathbf{r} \cdot \mathbf{v})\mathbf{v}\right]
\end{equation}

This is typically $\sim 10^{-10}$ m/s$^2$ for asteroids, but accumulates over long timescales.

\subsection{Solar Radiation Pressure}

For small bodies or spacecraft:

\begin{equation}
    \mathbf{a}_{\text{SRP}} = P_\odot \frac{A}{m} C_R \left(\frac{r_0}{r}\right)^2 \hat{\mathbf{r}}_\odot
\end{equation}

where:
\begin{itemize}
    \item $P_\odot = 4.56 \times 10^{-6}$ N/m$^2$ at 1 AU
    \item $A/m$ is the area-to-mass ratio (m$^2$/kg)
    \item $C_R \approx 1.3$ is the radiation pressure coefficient
\end{itemize}

\subsection{Asteroid Perturbations}

For precise work, massive asteroids (Ceres, Vesta, Pallas) can perturb test particle orbits:

\begin{equation}
    \mathbf{a}_{\text{ast}} = \sum_{i} \mu_i \left[\frac{\mathbf{r}_i - \mathbf{r}}{|\mathbf{r}_i - \mathbf{r}|^3} - \frac{\mathbf{r}_i}{r_i^3}\right]
\end{equation}

Masses of largest asteroids:
\begin{itemize}
    \item Ceres: $9.384 \times 10^{20}$ kg ($\sim 0.0001$ Earth masses)
    \item Vesta: $2.59 \times 10^{20}$ kg
    \item Pallas: $2.04 \times 10^{20}$ kg
\end{itemize}

\section{Coordinate Systems}

\subsection{Reference Frames}

AstDyn supports multiple reference frames:

\begin{description}
    \item[Heliocentric Ecliptic J2000] Standard for asteroid orbits (default)
    \item[Heliocentric Equatorial J2000] Common for planetary work
    \item[Barycentric] Solar system barycenter (for high precision)
    \item[Topocentric] Observer-centric (for observations)
\end{description}

\subsection{Frame Transformations}

The ecliptic-to-equatorial rotation is:

\begin{equation}
    \mathbf{R}_{\text{ecl} \to \text{eq}} = \begin{bmatrix}
        1 & 0 & 0 \\
        0 & \cos\epsilon & -\sin\epsilon \\
        0 & \sin\epsilon & \cos\epsilon
    \end{bmatrix}
\end{equation}

where $\epsilon = 23.43929111^\circ$ is the obliquity at J2000.0.

\section{Integration Strategy}

\subsection{Choosing Step Size}

For adaptive integrators (DOPRI54), initial step size estimate:

\begin{equation}
    h_0 = 0.01 \times \min\left(\frac{r}{v}, \frac{v}{a}\right)
\end{equation}

Typical step sizes:
\begin{itemize}
    \item Near-Earth asteroids: 0.1-1 day
    \item Main-belt asteroids: 5-20 days
    \item Jupiter Trojans: 10-30 days
    \item Comets (near perihelion): 0.01-0.1 day
\end{itemize}

\subsection{Tolerance Selection}

Position tolerance for orbit determination:
\begin{itemize}
    \item Preliminary orbits: $10^{-9}$ AU ($\sim$150 m)
    \item Final orbits: $10^{-12}$ AU ($\sim$15 cm)
    \item Very high precision: $10^{-14}$ AU ($\sim$1.5 mm)
\end{itemize}

The velocity tolerance is typically $10^{-3} \times$ position tolerance.

\subsection{Output Points}

Three strategies for output:

\begin{enumerate}
    \item \textbf{Dense output}: Store state at every integration step (large memory)
    \item \textbf{Interpolation}: Use Hermite interpolation between steps
    \item \textbf{Fixed output}: Specify output times, integrator stops there
\end{enumerate}

\begin{algorithm}
\caption{Adaptive Propagation Loop}
\begin{algorithmic}[1]
\REQUIRE Initial state $\mathbf{y}_0, t_0$, Target time $t_f$, Tolerance $TOL$
\ENSURE Final state $\mathbf{y}_f$
\STATE $t \leftarrow t_0, \quad \mathbf{y} \leftarrow \mathbf{y}_0$
\STATE Estimate initial step $h$ based on curvature
\WHILE{$t < t_f$}
    \STATE \textbf{Step Limiting}: If $t + h > t_f$, set $h \leftarrow t_f - t$
    \REPEAT
        \STATE \textbf{Integrate}: $\mathbf{y}_{new}, \mathbf{e} \leftarrow \text{Step}(\mathbf{y}, t, h)$
        \STATE \textbf{Error Check}: $\epsilon \leftarrow \max_i |e_i / (ATOL + RTOL \cdot |y_i|)|$
        \IF{$\epsilon > 1.0$}
            \STATE \textbf{Reject}: $h \leftarrow h \cdot 0.9 \cdot \epsilon^{-1/q}$
        \ELSE
            \STATE \textbf{Accept}: $\mathbf{y} \leftarrow \mathbf{y}_{new}, \quad t \leftarrow t + h$
            \STATE \textbf{Output}: Store interpolated state if dense output enabled
            \STATE \textbf{Next Step}: $h_{next} \leftarrow h \cdot 0.9 \cdot \epsilon^{-1/q}$
            \STATE $h \leftarrow \min(h_{max}, \max(h_{min}, h_{next}))$
        \ENDIF
    \UNTIL{Accepted}
\ENDWHILE
\RETURN $\mathbf{y}$
\end{algorithmic}
\end{algorithm}

AstDyn supports all three modes.

\section{Propagation Modes}

\subsection{Forward and Backward Propagation}

\textbf{Forward propagation} ($t_f > t_0$):
\begin{itemize}
    \item Standard ephemeris generation
    \item Mission planning
    \item Impact prediction
\end{itemize}

\textbf{Backward propagation} ($t_f < t_0$):
\begin{itemize}
    \item Orbital history reconstruction
    \item Finding past close approaches
    \item Validating orbit determination
\end{itemize}

Numerical integrators work equally well in both directions if the system is time-reversible.

\subsection{Single Epoch vs Multi-Epoch}

\textbf{Single epoch propagation}:
\begin{lstlisting}[language=C++,caption={Single epoch propagation}]
Vector6d y0 = ...;  // Initial state
double t0 = 60000.0;  // MJD TDB
double tf = 60365.0;  // 1 year later

Propagator prop(force_model);
Vector6d yf = prop.propagate(y0, t0, tf);
\end{lstlisting}

\textbf{Multi-epoch propagation}:
\begin{lstlisting}[language=C++,caption={Multi-epoch propagation}]
std::vector<double> epochs = {60000, 60100, 60200, 60300};
std::vector<Vector6d> states = prop.propagate_multi(y0, t0, epochs);
\end{lstlisting}

\section{Ephemeris Generation}

\subsection{Tabulated Ephemerides}

For efficient repeated lookups, create a table:

\begin{lstlisting}[language=C++,caption={Generating ephemeris table}]
EphemerisTable ephem;
double t_start = 60000.0;
double t_end = 61000.0;
double dt = 1.0;  // 1-day intervals

for (double t = t_start; t <= t_end; t += dt) {
    Vector6d state = prop.propagate(y0, t0, t);
    ephem.add_entry(t, state);
}

// Later: interpolate to arbitrary time
Vector6d state_interp = ephem.interpolate(60123.5);
\end{lstlisting}

\subsection{Chebyshev Interpolation}

For high-precision ephemerides, JPL uses Chebyshev polynomials:

\begin{equation}
    \mathbf{r}(t) = \sum_{k=0}^{n} c_k T_k(t')
\end{equation}

where $T_k$ are Chebyshev polynomials and $t'$ is normalized to $[-1, 1]$.

Advantages:
\begin{itemize}
    \item Minimax property (minimizes maximum error)
    \item Stable for high-degree polynomials
    \item Fast evaluation
\end{itemize}

\section{State Transition Matrix}

\subsection{Definition}

The \textbf{state transition matrix} (STM) $\Phi(t, t_0)$ relates perturbations:

\begin{equation}
    \delta\mathbf{y}(t) = \Phi(t, t_0) \delta\mathbf{y}(t_0)
\end{equation}

It is a $6 \times 6$ matrix satisfying:

\begin{equation}
    \frac{d\Phi}{dt} = \mathbf{A}(t)\Phi, \quad \Phi(t_0, t_0) = \mathbf{I}
\end{equation}

where $\mathbf{A} = \partial\mathbf{f}/\partial\mathbf{y}$ is the Jacobian.

\subsection{Applications}

The STM is essential for:
\begin{itemize}
    \item Orbit determination (differential correction)
    \item Covariance propagation (uncertainty quantification)
    \item Sensitivity analysis
    \item Maneuver optimization
\end{itemize}

\subsection{Computation}

Augment the state vector:

\begin{equation}
    \tilde{\mathbf{y}} = \begin{bmatrix} \mathbf{y} \\ \text{vec}(\Phi) \end{bmatrix}
\end{equation}

where $\text{vec}(\Phi)$ stacks the 36 elements of $\Phi$ into a vector.

The augmented system is:

\begin{equation}
    \frac{d\tilde{\mathbf{y}}}{dt} = \begin{bmatrix} \mathbf{f}(\mathbf{y}) \\ \mathbf{A}(\mathbf{y})\text{vec}(\Phi) \end{bmatrix}
\end{equation}

\section{Practical Examples}

\subsection{Example 1: Main-Belt Asteroid}

Propagate asteroid 203 Pompeja for 1 year:

\begin{lstlisting}[language=C++,caption={Propagating Pompeja}]
#include <astdyn/propagation/Propagator.hpp>

using namespace astdyn;

// Initial orbital elements (from OrbFit)
OrbitalElements elements;
elements.epoch = 60000.0;  // MJD TDB
elements.a = 2.743;  // AU
elements.e = 0.0698;
elements.i = 11.78 * DEG_TO_RAD;
elements.Omega = 347.60 * DEG_TO_RAD;
elements.omega = 59.96 * DEG_TO_RAD;
elements.M = 164.35 * DEG_TO_RAD;

// Convert to Cartesian
Vector6d state0 = elements.to_cartesian();

// Setup force model
ForceModel forces;
forces.enable_planets({"Jupiter", "Saturn", "Mars", "Earth"});
forces.enable_relativity(true);

// Create propagator
Propagator prop(forces);
prop.set_integrator("DOPRI54");
prop.set_tolerance(1e-12);

// Propagate 1 year
double t0 = elements.epoch;
double tf = t0 + 365.25;

Vector6d state_final = prop.propagate(state0, t0, tf);

// Convert back to elements
OrbitalElements final_elements = 
    OrbitalElements::from_cartesian(state_final, tf);

std::cout << "Initial a: " << elements.a << " AU\n";
std::cout << "Final a:   " << final_elements.a << " AU\n";
std::cout << "Change:    " << (final_elements.a - elements.a) * 1e6 
          << " km\n";
\end{lstlisting}

\subsection{Example 2: Close Approach Analysis}

Find minimum distance to Earth:

\begin{lstlisting}[language=C++,caption={Close approach detection}]
double min_distance = 1e99;
double closest_time = 0;

// Propagate with small steps near Earth encounter
for (double t = t_start; t <= t_end; t += 0.01) {
    Vector6d asteroid_state = prop.propagate(y0, t0, t);
    Vector6d earth_state = ephemeris.get_planet("Earth", t);
    
    Vector3d rel_pos = asteroid_state.head<3>() - earth_state.head<3>();
    double distance = rel_pos.norm();
    
    if (distance < min_distance) {
        min_distance = distance;
        closest_time = t;
    }
}

std::cout << "Closest approach: " << min_distance << " AU\n";
std::cout << "                  " << min_distance * 149597870.7 << " km\n";
std::cout << "At epoch: " << closest_time << " MJD\n";
\end{lstlisting}

\subsection{Example 3: Comet Propagation}

Handle large eccentricity near perihelion:

\begin{lstlisting}[language=C++,caption={Comet propagation}]
// Comet with e = 0.995, q = 0.1 AU
OrbitalElements comet;
comet.a = 20.0;  // AU (very eccentric)
comet.e = 0.995;
comet.q = comet.a * (1 - comet.e);  // perihelion distance

// Use variable step size, tighter tolerance
prop.set_tolerance(1e-14);
prop.set_min_step(1e-4);  // Allow very small steps near perihelion
prop.set_max_step(30.0);   // Large steps at aphelion

Vector6d state0 = comet.to_cartesian();
Vector6d state_post_perihelion = prop.propagate(state0, t0, t0 + 180);
\end{lstlisting}

\section{Performance Optimization}

\subsection{Force Model Selection}

Include only necessary perturbations:

\begin{table}[htbp]
\centering
\begin{tabular}{lcc}
\toprule
\textbf{Object} & \textbf{Essential Forces} & \textbf{Optional} \\
\midrule
Main-belt asteroid & Sun, Jup, Sat & Mars, Earth, relativity \\
Near-Earth asteroid & Sun, all planets & Relativity, asteroids \\
Jupiter Trojan & Sun, Jup, Sat & Uranus, Neptune \\
Trans-Neptunian & Sun, Jup, Sat, Ura, Nep & Relativity \\
\bottomrule
\end{tabular}
\caption{Recommended force models for different object types.}
\label{tab:force_models}
\end{table}

\subsection{Adaptive vs Fixed Step}

\textbf{Adaptive step} (DOPRI54, RK78):
\begin{itemize}
    \item Pros: Automatic error control, efficient
    \item Cons: Non-deterministic step sequence
    \item Use for: Orbit determination, ephemeris generation
\end{itemize}

\textbf{Fixed step} (RK4, Leapfrog):
\begin{itemize}
    \item Pros: Predictable, parallelizable
    \item Cons: Must choose step size carefully
    \item Use for: Long-term evolution, ensemble simulations
\end{itemize}

\subsection{Parallelization}

For propagating many objects:

\begin{lstlisting}[language=C++,caption={Parallel propagation}]
#include <omp.h>

std::vector<Vector6d> initial_states = ...;
std::vector<Vector6d> final_states(initial_states.size());

#pragma omp parallel for
for (size_t i = 0; i < initial_states.size(); ++i) {
    Propagator prop(forces);  // Each thread has its own propagator
    final_states[i] = prop.propagate(initial_states[i], t0, tf);
}
\end{lstlisting}

\section{Accuracy Validation}

\subsection{Energy Conservation}

For conservative systems (no SRP, drag), energy should be conserved:

\begin{equation}
    E = \frac{v^2}{2} - \frac{\mu}{r} = \text{constant}
\end{equation}

Check energy error:
\begin{lstlisting}[language=C++,caption={Energy check}]
double E0 = 0.5 * v0.squaredNorm() - MU_SUN / r0.norm();
double Ef = 0.5 * vf.squaredNorm() - MU_SUN / rf.norm();
double dE = std::abs(Ef - E0);
std::cout << "Energy error: " << dE / std::abs(E0) * 100 << "%\n";
\end{lstlisting}

For high-quality integrators: $\Delta E / E < 10^{-10}$

\subsection{Two-Body Comparison}

Validate against analytical Keplerian solution:

\begin{lstlisting}[language=C++,caption={Keplerian comparison}]
// Numerical propagation (with perturbations off)
Vector6d state_num = prop.propagate(y0, t0, tf);

// Analytical Keplerian propagation
OrbitalElements elem0 = OrbitalElements::from_cartesian(y0, t0);
elem0.propagate_mean_anomaly(tf - t0);
Vector6d state_kep = elem0.to_cartesian();

// Compare
Vector3d pos_diff = state_num.head<3>() - state_kep.head<3>();
std::cout << "Position difference: " << pos_diff.norm() * AU_TO_KM 
          << " km\n";
\end{lstlisting}

Expected: $< 1$ km for short arcs, $< 100$ km for 1 year.

\section{Summary}

Key concepts about orbit propagation:

\begin{enumerate}
    \item \textbf{Propagation} computes future/past states from initial conditions
    \item \textbf{Force models} must include all significant perturbations
    \item \textbf{Adaptive integrators} (DOPRI54) balance accuracy and efficiency
    \item \textbf{Step size} depends on orbital period and eccentricity
    \item \textbf{State transition matrix} enables orbit determination
    \item \textbf{Reference frames} must be consistent throughout
    \item \textbf{Validation} through energy conservation and analytical comparisons
\end{enumerate}

Understanding orbit propagation is essential for:
\begin{itemize}
    \item Generating accurate ephemerides
    \item Planning observations and missions
    \item Assessing collision risks
    \item Studying long-term dynamics
    \item Orbit determination (next chapter)
\end{itemize}

In the next chapter, we will use propagation with the state transition matrix for precise orbit determination from observations.

\include{10_state_transition}
\chapter{Ephemeris Computation}
\label{ch:ephemeris}

\section{Introduction}

An \textbf{ephemeris} (plural: \emph{ephemerides}) is a table or function providing positions (and optionally velocities) of celestial bodies at specific times. Accurate ephemerides are essential for:

\begin{itemize}
    \item Computing predicted positions for observations
    \item Reducing astrometric measurements
    \item Planning space missions
    \item Analyzing close approaches
    \item Studying orbital dynamics
\end{itemize}

This chapter covers methods for generating, storing, and interpolating ephemerides efficiently.

\section{Types of Ephemerides}

\subsection{Planetary Ephemerides}

Major planets require the highest accuracy:

\begin{description}
    \item[JPL Development Ephemerides (DE)] Numerical integration of solar system, including Moon and large asteroids. Current: DE440 (Earth-Moon optimization), DE441 (outer solar system).
    
    \item[VSOP87] Analytical theory by Bureau des Longitudes. Series expansion in orbital elements. Accuracy: $\sim$1 arcsec over millennia.
    
    \item[INPOP] French ephemeris from IMCCE, optimized for planetary radar ranging.
\end{description}

\subsection{Small Body Ephemerides}

Asteroids and comets:
\begin{itemize}
    \item Computed from orbital elements via propagation
    \item Archived in MPC (Minor Planet Center) database
    \item Precision varies: 0.1 arcsec (well-observed) to 10 arcmin (single-opposition)
\end{itemize}

\subsection{Spacecraft Ephemerides}

Interplanetary missions:
\begin{itemize}
    \item SPICE kernels (SPK files) from navigation teams
    \item Chebyshev polynomial segments
    \item Meter-level accuracy for close-approach phases
\end{itemize}

\section{Ephemeris Representations}

\subsection{Tabulated Format}

Simplest representation: discrete time-state pairs.

\begin{table}[htbp]
\centering
\begin{tabular}{ccccccc}
\toprule
\textbf{MJD (TDB)} & $x$ (AU) & $y$ (AU) & $z$ (AU) & $\dot{x}$ & $\dot{y}$ & $\dot{z}$ \\
\midrule
60000.0 & 1.234 & 0.567 & 0.123 & $-0.012$ & 0.015 & 0.003 \\
60001.0 & 1.222 & 0.582 & 0.126 & $-0.012$ & 0.015 & 0.003 \\
60002.0 & 1.210 & 0.597 & 0.129 & $-0.012$ & 0.015 & 0.003 \\
\vdots & \vdots & \vdots & \vdots & \vdots & \vdots & \vdots \\
\bottomrule
\end{tabular}
\caption{Example tabulated ephemeris with 1-day spacing.}
\label{tab:ephemeris_table}
\end{table}

\textbf{Advantages}:
\begin{itemize}
    \item Easy to implement
    \item Direct lookup for tabulated times
\end{itemize}

\textbf{Disadvantages}:
\begin{itemize}
    \item Large storage for high cadence
    \item Requires interpolation between points
    \item Fixed time grid (inefficient for eccentric orbits)
\end{itemize}

\subsection{Polynomial Representation}

Represent position as polynomial:

\begin{equation}
    \mathbf{r}(t) = \sum_{k=0}^{n} \mathbf{c}_k (t - t_0)^k
\end{equation}

Typically used piecewise over segments (splines).

\subsection{Chebyshev Polynomials}

JPL's preferred method. For time interval $[t_a, t_b]$, represent:

\begin{equation}
    \mathbf{r}(t) = \sum_{k=0}^{n} \mathbf{a}_k T_k\left(\frac{2t - t_a - t_b}{t_b - t_a}\right)
\end{equation}

where $T_k(x)$ are Chebyshev polynomials:

\begin{equation}
    T_0(x) = 1, \quad T_1(x) = x, \quad T_k(x) = 2xT_{k-1}(x) - T_{k-2}(x)
\end{equation}

\textbf{Properties}:
\begin{itemize}
    \item Minimax error distribution (optimal approximation)
    \item Stable for high degrees ($n \sim 15$)
    \item Efficient evaluation via recurrence
\end{itemize}

\subsection{Fourier Series}

For nearly circular orbits:

\begin{equation}
    \mathbf{r}(t) = \sum_{k=-N}^{N} \mathbf{c}_k e^{ik\omega t}
\end{equation}

Used in analytical planetary theories (VSOP87).

\section{Interpolation Methods}

\subsection{Linear Interpolation}

Given points $(t_1, \mathbf{r}_1)$ and $(t_2, \mathbf{r}_2)$:

\begin{equation}
    \mathbf{r}(t) = \mathbf{r}_1 + \frac{t - t_1}{t_2 - t_1}(\mathbf{r}_2 - \mathbf{r}_1)
\end{equation}

\textbf{Accuracy}: First-order, $O(h^2)$ error where $h = t_2 - t_1$.

\textbf{Use}: Quick lookups when high precision not required ($>$1 km acceptable).

\subsection{Lagrange Interpolation}

Use $n+1$ points to construct polynomial of degree $n$:

\begin{equation}
    \mathbf{r}(t) = \sum_{i=0}^{n} \mathbf{r}_i L_i(t)
\end{equation}

where the Lagrange basis polynomials are:

\begin{equation}
    L_i(t) = \prod_{\substack{j=0\\j\neq i}}^{n} \frac{t - t_j}{t_i - t_j}
\end{equation}

\textbf{Typical choice}: $n = 6$ to 10 (7th to 11th order).

\textbf{Accuracy}: For 8th-order with 1-day spacing, error $\sim$10 m for typical asteroid orbits.

\subsection{Hermite Interpolation}

Uses both positions and velocities. For interval $[t_1, t_2]$:

\begin{equation}
    \mathbf{r}(t) = \mathbf{r}_1 H_0(s) + \mathbf{r}_2 H_1(s) + h\dot{\mathbf{r}}_1 H_2(s) + h\dot{\mathbf{r}}_2 H_3(s)
\end{equation}

where $s = (t - t_1)/h$, $h = t_2 - t_1$, and the Hermite basis functions are:

\begin{align}
    H_0(s) &= (1 + 2s)(1-s)^2 \\
    H_1(s) &= s^2(3 - 2s) \\
    H_2(s) &= s(1-s)^2 \\
    H_3(s) &= s^2(s-1)
\end{align}

\textbf{Advantages}:
\begin{itemize}
    \item Cubic accuracy ($O(h^4)$)
    \item Smooth velocities (continuous first derivative)
    \item Only requires two points
\end{itemize}

\textbf{Accuracy}: With 1-day spacing, error $\sim$1 m for well-behaved orbits.

\subsection{Spline Interpolation}

Cubic splines provide smooth interpolation through all points with continuous second derivatives.

For points $(t_i, \mathbf{r}_i)$, $i = 0, \ldots, n$, construct piecewise cubics $\mathbf{s}_i(t)$ on $[t_i, t_{i+1}]$ such that:

\begin{itemize}
    \item $\mathbf{s}_i(t_i) = \mathbf{r}_i$ (interpolation)
    \item $\mathbf{s}_i'(t_{i+1}) = \mathbf{s}_{i+1}'(t_{i+1})$ (continuous first derivative)
    \item $\mathbf{s}_i''(t_{i+1}) = \mathbf{s}_{i+1}''(t_{i+1})$ (continuous second derivative)
\end{itemize}

\textbf{Use}: When smooth acceleration is important (covariance propagation).

\section{SPICE System}

\subsection{Overview}

SPICE (Spacecraft Planet Instrument C-matrix Events) is NASA's toolkit for space mission geometry:

\begin{description}
    \item[SPK] (Ephemeris kernels) Position and velocity
    \item[CK] (Orientation kernels) Spacecraft attitude
    \item[PCK] (Constants kernels) Physical parameters, body shapes
    \item[IK] (Instrument kernels) FOV, boresight
    \item[FK] (Frame kernels) Reference frame definitions
    \item[LSK] (Leapsecond kernels) Time conversions
\end{description}

\subsection{SPK Files}

Binary files containing Chebyshev or Hermite polynomial segments.

\textbf{Usage in AstDyn}:
\begin{lstlisting}[language=C++,caption={Loading SPICE kernel}]
#include <astdyn/ephemeris/SpiceInterface.hpp>

SpiceInterface spice;
spice.load_kernel("de440.bsp");  // Planetary ephemeris
spice.load_kernel("codes_300ast_20100725.bsp");  // Asteroids

// Query Jupiter position at epoch
double et = spice.mjd_to_et(60000.0);  // Convert MJD to ET
Vector6d jupiter_state = spice.get_state("JUPITER", et, "ECLIPJ2000", "SUN");

std::cout << "Jupiter position: " << jupiter_state.head<3>().transpose() 
          << " km\n";
\end{lstlisting}

\subsection{NAIF IDs}

SPICE uses integer IDs:
\begin{itemize}
    \item Sun: 10
    \item Planets: 199 (Mercury), 299 (Venus), 399 (Earth), 499 (Mars), 599 (Jupiter), etc.
    \item Moon: 301
    \item Asteroids: 2000001 (Ceres), 2000004 (Vesta), 2000203 (Pompeja)
\end{itemize}

\section{Planetary Ephemerides}

\subsection{JPL Development Ephemerides}

\textbf{DE440/441} (released 2020):
\begin{itemize}
    \item Covers years 1550--2650
    \item Includes Sun, planets, Moon, Pluto, 343 asteroids
    \item Fit to ranging data (Mars missions), VLBI, LLR
    \item Accuracy: $\sim$1 km for inner planets, $\sim$10 km for outer planets
\end{itemize}

\textbf{File sizes}:
\begin{itemize}
    \item DE440: 114 MB (standard)
    \item DE441: 3.2 GB (includes high-rate Moon)
\end{itemize}

\subsection{VSOP87}

Analytical series developed by Bretagnon \& Francou (1988).

\textbf{Variants}:
\begin{description}
    \item[VSOP87A] Heliocentric rectangular, J2000 ecliptic
    \item[VSOP87B] Heliocentric rectangular, J2000 equatorial  
    \item[VSOP87C] Heliocentric spherical (mean ecliptic/equinox of date)
    \item[VSOP87D] Heliocentric spherical (J2000 ecliptic)
    \item[VSOP87E] Barycentric rectangular, J2000 ecliptic
\end{description}

\textbf{Implementation}:
\begin{lstlisting}[language=C++,caption={VSOP87 usage}]
#include <astdyn/ephemeris/VSOP87.hpp>

VSOP87 vsop;
double jd = 2460000.5;  // Julian date

// Earth position (VSOP87A: heliocentric J2000 ecliptic)
Vector3d earth_pos = vsop.get_position("Earth", jd, VSOP87_A);
std::cout << "Earth position: " << earth_pos.transpose() << " AU\n";

// Accuracy estimate
double error_km = vsop.estimated_error("Earth", jd);
std::cout << "Position error: ~" << error_km << " km\n";
\end{lstlisting}

\textbf{Accuracy}: $\sim$1 km for inner planets over $\pm$2000 years from J2000.

\subsection{Comparison}

\begin{table}[htbp]
\centering
\begin{tabular}{lccc}
\toprule
\textbf{Method} & \textbf{Accuracy} & \textbf{Speed} & \textbf{File Size} \\
\midrule
DE440 (SPICE) & 1--10 km & Fast & 114 MB \\
VSOP87 & 1--5 km & Medium & $\sim$1 MB (code) \\
Keplerian & 100--1000 km & Very fast & Negligible \\
\bottomrule
\end{tabular}
\caption{Planetary ephemeris comparison.}
\label{tab:planetary_eph}
\end{table}

\section{Light-Time Corrections}

\subsection{Geometric vs Apparent Position}

Light travels at finite speed $c = 299792.458$ km/s, so we observe planets where they \emph{were}, not where they \emph{are}.

\textbf{Light-time}:

\begin{equation}
    \tau = \frac{|\mathbf{r}_{\text{planet}} - \mathbf{r}_{\text{obs}}|}{c}
\end{equation}

Typical values:
\begin{itemize}
    \item Sun: 8.3 minutes
    \item Jupiter: 30--50 minutes
    \item Saturn: 70--90 minutes
    \item Neptune: 4 hours
\end{itemize}

\subsection{Iterative Correction}

To find the \textbf{apparent position} at observation time $t_{\text{obs}}$:

\begin{enumerate}
    \item Start with geometric position: $\mathbf{r}_0 = \mathbf{r}_{\text{planet}}(t_{\text{obs}})$
    \item Compute light-time: $\tau_0 = |\mathbf{r}_0 - \mathbf{r}_{\text{obs}}|/c$
    \item Update: $\mathbf{r}_1 = \mathbf{r}_{\text{planet}}(t_{\text{obs}} - \tau_0)$
    \item Iterate until convergence: $|\tau_{i+1} - \tau_i| < 10^{-6}$ s
\end{enumerate}

Typically converges in 2--3 iterations.

\subsection{Implementation}

\begin{lstlisting}[language=C++,caption={Light-time correction}]
Vector3d compute_apparent_position(
    const EphemerisInterface& ephem,
    const std::string& target,
    double t_obs,
    const Vector3d& observer_pos)
{
    const double c_AU_per_day = 173.1446326846693;  // Speed of light
    
    Vector3d r_geom = ephem.get_position(target, t_obs);
    double tau = (r_geom - observer_pos).norm() / c_AU_per_day;
    
    // Iterate light-time correction
    for (int iter = 0; iter < 5; ++iter) {
        Vector3d r_new = ephem.get_position(target, t_obs - tau);
        double tau_new = (r_new - observer_pos).norm() / c_AU_per_day;
        
        if (std::abs(tau_new - tau) < 1e-10) break;  // Converged
        tau = tau_new;
    }
    
    return ephem.get_position(target, t_obs - tau);
}
\end{lstlisting}

\subsection{Aberration}

Observer motion causes additional \textbf{stellar aberration}:

\begin{equation}
    \Delta\theta \approx \frac{v_{\text{obs}}}{c}
\end{equation}

For Earth's orbital motion ($v \approx 30$ km/s): $\Delta\theta \approx 20.5$ arcsec (annual aberration).

Correction:

\begin{equation}
    \hat{\mathbf{r}}_{\text{aberrated}} = \hat{\mathbf{r}} + \frac{\mathbf{v}_{\text{obs}}}{c}
\end{equation}

\section{Practical Ephemeris Generation}

\subsection{Design Considerations}

Choose ephemeris parameters based on requirements:

\begin{table}[htbp]
\centering
\begin{tabular}{lccc}
\toprule
\textbf{Application} & \textbf{Spacing} & \textbf{Interpolation} & \textbf{Accuracy} \\
\midrule
Visual magnitude & 10 days & Linear & 0.1 mag \\
Telescope pointing & 1 day & Hermite & 1 arcsec \\
Orbit determination & 1 hour & Lagrange-9 & 0.01 arcsec \\
Close approach & 1 minute & Chebyshev & 1 meter \\
\bottomrule
\end{tabular}
\caption{Ephemeris requirements for different applications.}
\label{tab:ephemeris_req}
\end{table}

\subsection{Generation Workflow}

\begin{lstlisting}[language=C++,caption={Ephemeris generation}]
#include <astdyn/ephemeris/EphemerisGenerator.hpp>

// Define time span
double t_start = 60000.0;  // MJD
double t_end = 60365.0;    // 1 year
double dt = 1.0;           // 1-day spacing

// Setup propagator
ForceModel forces;
forces.enable_planets({"Jupiter", "Saturn", "Mars"});
Propagator prop(forces);

// Initial state from orbital elements
OrbitalElements elem = load_orbit("203_Pompeja.oe");
Vector6d y0 = elem.to_cartesian();

// Generate ephemeris
EphemerisGenerator gen(prop);
auto ephem = gen.generate(y0, elem.epoch, t_start, t_end, dt);

// Save to file
ephem.save("pompeja_ephemeris.txt");

// Later: interpolate to arbitrary time
Vector6d state_interp = ephem.interpolate(60123.456, HERMITE);
\end{lstlisting}

\subsection{Validation}

Always validate ephemerides:

\begin{enumerate}
    \item Compare with published ephemerides (MPC, JPL Horizons)
    \item Check energy conservation (if applicable)
    \item Verify smooth velocities (no jumps)
    \item Test interpolation error against propagation
\end{enumerate}

\section{Efficient Storage}

\subsection{Binary Formats}

For large ephemerides, use binary:

\begin{itemize}
    \item HDF5: Hierarchical, compressed, self-describing
    \item FITS: Standard in astronomy, good tool support
    \item Custom binary: Maximum efficiency, requires documentation
\end{itemize}

\textbf{Example sizes} (1 year, 1-day spacing):
\begin{itemize}
    \item ASCII: 350 KB
    \item Binary (doubles): 18 KB
    \item Compressed binary: 5 KB
\end{itemize}

\subsection{Adaptive Spacing}

For eccentric orbits, use variable spacing:
\begin{itemize}
    \item Fine spacing near perihelion (fast motion)
    \item Coarse spacing near aphelion (slow motion)
\end{itemize}

Spacing proportional to true anomaly rate:

\begin{equation}
    \Delta t \propto \frac{r^2}{\sqrt{\mu a(1-e^2)}}
\end{equation}

This maintains constant position error.

\section{Summary}

Key concepts about ephemeris computation:

\begin{enumerate}
    \item \textbf{Ephemerides} provide positions/velocities at specified times
    \item \textbf{Representations}: tabulated, polynomial (Chebyshev), analytical (VSOP87)
    \item \textbf{Interpolation}: Hermite for accuracy, Lagrange for flexibility
    \item \textbf{SPICE} is NASA's standard for planetary/spacecraft ephemerides
    \item \textbf{Light-time} correction accounts for finite light speed
    \item \textbf{Aberration} corrects for observer motion
    \item \textbf{Adaptive spacing} improves efficiency for eccentric orbits
\end{enumerate}

Practical recommendations:
\begin{itemize}
    \item Use DE440/441 for planets (via SPICE)
    \item Use VSOP87 if SPICE unavailable or for historical epochs
    \item Generate custom ephemerides for asteroids
    \item Hermite interpolation for 1-meter accuracy with 1-day spacing
    \item Always apply light-time corrections for precise work
\end{itemize}

The next chapter begins Part III (Orbit Determination), using ephemerides to predict observations and fit orbits to data.


% Part III
\part{Orbit Determination}
\chapter{Observations}
\label{ch:observations}

\section{Introduction}

\textbf{Observations} are the fundamental data for orbit determination. This chapter describes:
\begin{itemize}
    \item Types of observations (astrometric, radar, spacecraft)
    \item Observation models relating state to measurements
    \item Data formats (MPC, radar, tracking)
    \item Corrections (refraction, light-time, aberration)
    \item Observatory coordinates and Earth orientation
\end{itemize}

Accurate observation modeling is essential for achieving sub-arcsecond orbit determination.

\section{Observation Types}

\subsection{Optical Astrometry}

The most common observations are angular positions on the celestial sphere:

\begin{equation}
    \text{Observation} = (\alpha, \delta, t)
\end{equation}

where:
\begin{itemize}
    \item $\alpha$ is right ascension (0$^\circ$ to 360$^\circ$ or 0h to 24h)
    \item $\delta$ is declination ($-90^\circ$ to $+90^\circ$)
    \item $t$ is observation time (usually UTC)
\end{itemize}

\textbf{Precision ranges}:
\begin{itemize}
    \item Historical (photographic): 0.5--2 arcsec
    \item CCD astrometry: 0.1--0.5 arcsec
    \item Gaia space mission: 0.0001--0.001 arcsec (100 $\mu$as)
    \item Ground-based surveys (Pan-STARRS, ATLAS): 0.05--0.2 arcsec
\end{itemize}

\subsection{Radar Observations}

Planetary radar provides range and Doppler measurements:

\begin{equation}
    \text{Range: } \rho = |\mathbf{r}_{\text{target}} - \mathbf{r}_{\text{station}}|
\end{equation}

\begin{equation}
    \text{Doppler: } \dot{\rho} = \frac{(\mathbf{r}_{\text{target}} - \mathbf{r}_{\text{station}}) \cdot (\mathbf{v}_{\text{target}} - \mathbf{v}_{\text{station}})}{|\mathbf{r}_{\text{target}} - \mathbf{r}_{\text{station}}|}
\end{equation}

\textbf{Major radar facilities}:
\begin{itemize}
    \item Arecibo (305 m, 2.38 GHz) -- decommissioned 2020
    \item Goldstone DSS-14 (70 m, 8.56 GHz) -- operational
    \item Green Bank (100 m, receive-only)
\end{itemize}

\textbf{Precision}:
\begin{itemize}
    \item Range: 10--100 meters (delay-Doppler imaging: $<$1 m)
    \item Doppler: 0.1--1 mm/s
\end{itemize}

Radar is 1000$\times$ more precise than optical astrometry in range but limited to nearby objects ($<$ 0.3 AU for asteroids).

\subsection{Spacecraft Tracking}

Deep space missions tracked via:
\begin{itemize}
    \item Two-way Doppler (mm/s precision)
    \item Range measurements (meter-level)
    \item Delta-DOR (angular position via interferometry)
    \item Optical navigation (camera images)
\end{itemize}

\section{Astrometric Observation Model}

\subsection{Coordinate Transformation}

Given object position $\mathbf{r}_{\text{obj}}$ in heliocentric ecliptic J2000, compute topocentric equatorial:

\begin{enumerate}
    \item Transform to barycentric: $\mathbf{r}_{\text{bary}} = \mathbf{r}_{\text{obj}} + \mathbf{r}_{\odot,\text{bary}}$
    \item Subtract Earth position: $\mathbf{r}_{\text{geo}} = \mathbf{r}_{\text{bary}} - \mathbf{r}_{\text{Earth}}$
    \item Subtract observatory position: $\mathbf{r}_{\text{topo}} = \mathbf{r}_{\text{geo}} - \mathbf{r}_{\text{obs}}$
    \item Rotate to equatorial: $\mathbf{r}_{\text{eq}} = \mathbf{R}_{\text{ecl}\to\text{eq}} \mathbf{r}_{\text{topo}}$
\end{enumerate}

\subsection{Spherical Coordinates}

From Cartesian topocentric equatorial $\mathbf{r}_{\text{eq}} = (x, y, z)$:

\begin{equation}
    \alpha = \arctan2(y, x)
\end{equation}

\begin{equation}
    \delta = \arcsin\left(\frac{z}{\sqrt{x^2 + y^2 + z^2}}\right)
\end{equation}

Handle quadrant correctly with \texttt{atan2}.

\subsection{Light-Time Correction}

Observation time $t_{\text{obs}}$ is when photons arrive at Earth. Object was at emission position at:

\begin{equation}
    t_{\text{emit}} = t_{\text{obs}} - \frac{\rho}{c}
\end{equation}

where $\rho$ is geocentric distance.

Iterate to find $t_{\text{emit}}$:
\begin{lstlisting}[language=C++,caption={Light-time iteration}]
double tau = 0.0;  // Initial guess
for (int iter = 0; iter < 5; ++iter) {
    Vector3d r_obj = propagate(y0, t0, t_obs - tau);
    Vector3d r_earth = ephemeris.get_position("Earth", t_obs);
    double rho = (r_obj - r_earth).norm();
    double tau_new = rho / C_AU_PER_DAY;
    if (std::abs(tau_new - tau) < 1e-10) break;
    tau = tau_new;
}
\end{lstlisting}

Typical correction: 4--30 minutes for asteroids.

\subsection{Stellar Aberration}

Earth's orbital motion causes apparent displacement:

\begin{equation}
    \mathbf{r}_{\text{aberrated}} = \mathbf{r}_{\text{geometric}} + \frac{\rho}{c}\mathbf{v}_{\text{Earth}}
\end{equation}

where $\mathbf{v}_{\text{Earth}}$ is Earth's velocity.

Maximum effect: $\pm 20.5$ arcsec (annual aberration).

\subsection{Atmospheric Refraction}

Light bends passing through atmosphere. Correction depends on zenith angle $z$:

\begin{equation}
    \Delta z \approx 58.2'' \tan z - 0.067'' \tan^3 z
\end{equation}

At zenith ($z = 0$): no refraction. At horizon ($z = 90^\circ$): $\sim$34 arcmin (solar diameter!).

For precise work, use wavelength-dependent model:

\begin{equation}
    n - 1 = 77.6 \times 10^{-6} \frac{P}{T}\left(1 + 7.52 \times 10^{-3}\lambda^{-2}\right)
\end{equation}

where $P$ is pressure (mbar), $T$ is temperature (K), $\lambda$ is wavelength ($\mu$m).

Modern astrometry corrects to "above atmosphere" by:
\begin{itemize}
    \item Fitting catalog stars in field
    \item Measuring local refraction empirically
    \item Applying site-specific models
\end{itemize}

\section{Observatory Coordinates}

\subsection{ITRF and Observatory Codes}

The International Terrestrial Reference Frame (ITRF) provides precise coordinates for observatories.

\textbf{Minor Planet Center (MPC) observatory codes}:
\begin{itemize}
    \item 500: Geocenter (for space-based observations)
    \item 568: Mauna Kea (Hawaii)
    \item 703: Catalina Sky Survey (Arizona)
    \item F51: Pan-STARRS 1 (Hawaii)
    \item G96: Mt. Lemmon Survey (Arizona)
\end{itemize}

Example entry for observatory 703:
\begin{verbatim}
703  Catalina  4.215500  0.759260  0.648764  -31.67
\end{verbatim}

Format: code, name, $\rho \cos\phi'$, $\rho \sin\phi'$, longitude (deg), altitude (m).

\subsection{Geocentric Observatory Position}

Convert geodetic coordinates $(h, \lambda, \phi)$ to geocentric Cartesian:

\begin{equation}
    \mathbf{r}_{\text{obs}} = \begin{bmatrix}
        (N + h)\cos\phi\cos\lambda \\
        (N + h)\cos\phi\sin\lambda \\
        (N(1-e^2) + h)\sin\phi
    \end{bmatrix}
\end{equation}

where:
\begin{equation}
    N = \frac{a}{\sqrt{1 - e^2\sin^2\phi}}
\end{equation}

and $a = 6378.137$ km (WGS84 equatorial radius), $e = 0.08181919$ (eccentricity).

\subsection{Rotation to Inertial Frame}

Observatory position rotates with Earth. Transformation involves:

\begin{enumerate}
    \item Polar motion ($x_p, y_p$)
    \item UT1-UTC correction (Earth rotation angle)
    \item Precession-nutation (IAU 2006/2000A)
    \item Frame bias (ICRS to J2000)
\end{enumerate}

\textbf{Simplified rotation}:

\begin{equation}
    \mathbf{r}_{\text{inertial}} = \mathbf{R}_3(\text{GAST}) \mathbf{r}_{\text{ITRF}}
\end{equation}

where GAST is Greenwich Apparent Sidereal Time.

\section{Earth Orientation Parameters}

\subsection{Polar Motion}

Earth's rotation axis moves relative to crust (Chandler wobble, annual motion):

\begin{equation}
    \mathbf{R}_{\text{polar}} = \mathbf{R}_2(-x_p)\mathbf{R}_1(-y_p)
\end{equation}

Amplitude: $\sim$0.3 arcsec ($\sim$10 meters at surface).

Data from IERS: \texttt{finals2000A.all} bulletin.

\subsection{UT1-UTC}

Universal Time (UT1) tracks Earth's actual rotation. Atomic time (UTC) is uniform.

\begin{equation}
    \text{UT1} = \text{UTC} + (\text{UT1-UTC})
\end{equation}

$|\text{UT1-UTC}| < 0.9$ seconds (leap seconds added when needed).

Prediction: available from IERS with $\sim$10 ms accuracy for 1 year ahead.

\subsection{Precession and Nutation}

Earth's rotation axis precesses (26,000 year period) and nutates (18.6 year main period).

\textbf{IAU 2006 precession} + \textbf{IAU 2000A nutation} = high-precision model.

Simplified for asteroid work: use mean pole (J2000) and ignore nutation ($\sim$15 arcsec effect).

\section{MPC Observation Format}

\subsection{80-Column Format}

Standard format for optical astrometry:

\begin{lstlisting}[basicstyle=\ttfamily\tiny, frame=none, xleftmargin=0pt]
     K17S00S  C2017 06 01.41667 18 26 54.13 -23 47 08.4          21.1 V      F51
\end{lstlisting}

Fields:
\begin{itemize}
    \item Columns 1-5: Temporary designation or number
    \item Column 12: Discovery asterisk (*)
    \item Column 13: Note (e.g., photometry)
    \item Column 14: Publication reference
    \item Columns 15-32: Observation date (YYYY MM DD.ddddd)
    \item Columns 33-44: Right ascension (HH MM SS.sss)
    \item Columns 45-56: Declination (sDD MM SS.ss)
    \item Columns 66-70: Magnitude
    \item Column 71: Mag band (V, R, I, etc.)
    \item Columns 78-80: Observatory code
\end{itemize}

\subsection{ADES Format}

Astrometry Data Exchange Standard (modern XML/JSON format):

\begin{lstlisting}[language=XML,caption={ADES XML example}]
<obsBlock>
  <obsContext>
    <observatory>
      <mpcCode>F51</mpcCode>
    </observatory>
  </obsContext>
  <obsData>
    <optical>
      <trkSub>K17S00S</trkSub>
      <obsTime>2017-06-01T10:00:00.000Z</obsTime>
      <ra>276.72554</ra>
      <dec>-23.78567</dec>
      <mag>21.1</mag>
      <band>V</band>
      <rmsRA>0.1</rmsRA>
      <rmsDec>0.1</rmsDec>
    </optical>
  </obsData>
</obsBlock>
\end{lstlisting}

\textbf{Advantages over 80-column}:
\begin{itemize}
    \item Explicit uncertainties
    \item Metadata (telescope, detector, catalog)
    \item No fixed-width limitations
    \item International standard
\end{itemize}

\section{Observation Weights}

\subsection{Weighting Schemes}

Not all observations equally reliable. Weight by estimated uncertainty:

\begin{equation}
    w_i = \frac{1}{\sigma_i^2}
\end{equation}

\textbf{Uncertainty sources}:
\begin{itemize}
    \item Measurement error (star fitting, centroid)
    \item Catalog errors (Gaia DR3: 0.02--0.05 arcsec)
    \item Timing errors ($\pm$1 second $\to$ 0.01 arcsec for slow movers)
    \item Atmospheric effects (seeing, refraction)
    \item Trailing losses (long exposures)
\end{itemize}

\subsection{Empirical Weighting}

For MPC observations without formal uncertainties:

\begin{table}[htbp]
\centering
\begin{tabular}{lcc}
\toprule
\textbf{Observatory Type} & \textbf{$\sigma_\alpha \cos\delta$} & \textbf{$\sigma_\delta$} \\
\midrule
Professional (Pan-STARRS, CSS) & 0.1 arcsec & 0.1 arcsec \\
Amateur CCD & 0.5 arcsec & 0.5 arcsec \\
Historical photographic & 1.0 arcsec & 1.0 arcsec \\
Radar range & 10 m & -- \\
Radar Doppler & -- & 1 mm/s \\
\bottomrule
\end{tabular}
\caption{Typical observation uncertainties.}
\label{tab:obs_uncertainties}
\end{table}

\subsection{Downweighting Outliers}

After initial fit, identify outliers (residual $> 3\sigma$) and reduce weight:

\begin{equation}
    w_{\text{new}} = w_{\text{old}} \times \exp\left(-\frac{r^2}{2\sigma^2}\right)
\end{equation}

where $r$ is residual. This is "robust least squares" or "Huber weighting."

\section{Observation Partials}

\subsection{Definition}

For orbit determination, we need:

\begin{equation}
    \frac{\partial(\alpha, \delta)}{\partial\mathbf{y}(t_0)}
\end{equation}

This relates how initial state affects predicted observation.

\subsection{Chain Rule}

Use chain rule with state transition matrix:

\begin{equation}
    \frac{\partial(\alpha, \delta)}{\partial\mathbf{y}(t_0)} = \frac{\partial(\alpha, \delta)}{\partial\mathbf{r}(t_{\text{obs}})} \frac{\partial\mathbf{r}(t_{\text{obs}})}{\partial\mathbf{y}(t_{\text{obs}})} \frac{\partial\mathbf{y}(t_{\text{obs}})}{\partial\mathbf{y}(t_0)}
\end{equation}

The last factor is the STM $\Phi(t_{\text{obs}}, t_0)$.

\subsection{Geometric Partials}

For topocentric position $\mathbf{r} = (x, y, z)$ in equatorial frame:

\begin{equation}
    \rho = \sqrt{x^2 + y^2 + z^2}
\end{equation}

\begin{equation}
    \frac{\partial\alpha}{\partial x} = -\frac{y}{x^2 + y^2}, \quad
    \frac{\partial\alpha}{\partial y} = \frac{x}{x^2 + y^2}, \quad
    \frac{\partial\alpha}{\partial z} = 0
\end{equation}

\begin{equation}
    \frac{\partial\delta}{\partial x} = -\frac{xz}{\rho^2\sqrt{x^2+y^2}}, \quad
    \frac{\partial\delta}{\partial y} = -\frac{yz}{\rho^2\sqrt{x^2+y^2}}, \quad
    \frac{\partial\delta}{\partial z} = \frac{\sqrt{x^2+y^2}}{\rho^2}
\end{equation}

\subsection{Implementation}

\begin{lstlisting}[language=C++,caption={Computing observation partials}]
Matrix<2,6> compute_partials_radec(
    const Vector6d& state,
    const Matrix6d& stm,
    const Vector3d& obs_pos)
{
    Vector3d r = state.head<3>() - obs_pos;
    double x = r(0), y = r(1), z = r(2);
    double rho = r.norm();
    double rho_xy = std::sqrt(x*x + y*y);
    
    // Partials w.r.t. position
    Matrix<2,3> dobs_dr;
    dobs_dr(0,0) = -y / (x*x + y*y);  // d(RA)/dx
    dobs_dr(0,1) =  x / (x*x + y*y);  // d(RA)/dy
    dobs_dr(0,2) =  0.0;              // d(RA)/dz
    
    dobs_dr(1,0) = -x*z / (rho*rho*rho_xy);  // d(Dec)/dx
    dobs_dr(1,1) = -y*z / (rho*rho*rho_xy);  // d(Dec)/dy
    dobs_dr(1,2) =  rho_xy / (rho*rho);      // d(Dec)/dz
    
    // Chain with STM
    Matrix<2,6> partials = dobs_dr * stm.block<3,6>(0,0);
    
    return partials;
}
\end{lstlisting}

\section{Data Quality}

\subsection{Timing Accuracy}

Observation time must be UTC to $\pm$1 second for asteroids ($\pm$0.01 sec for fast movers).

\textbf{Common issues}:
\begin{itemize}
    \item Clock drift (GPS receivers essential)
    \item Mid-exposure vs start/end time
    \item Time zone errors (always use UTC!)
    \item Leap seconds
\end{itemize}

\subsection{Astrometric Catalog}

Modern observations referenced to:
\begin{itemize}
    \item Gaia DR3 (2022): 0.02--0.05 arcsec, $\sim$1.8 billion stars
    \item UCAC4: 0.02--0.1 arcsec, 113 million stars
    \item 2MASS: 0.08 arcsec (infrared), 471 million objects
\end{itemize}

\textbf{Older observations} (pre-Gaia) may have systematic errors from catalog:
\begin{itemize}
    \item USNO-A: $\sim$0.25 arcsec systematic
    \item GSC: $\sim$0.3 arcsec systematic
\end{itemize}

Use catalog-specific debiasing when mixing observations.

\subsection{Site-Specific Systematics}

Some observatories have known issues:
\begin{itemize}
    \item Poor timing ($>$10 sec errors)
    \item Incorrect coordinates (wrong latitude/longitude)
    \item Scale errors (wrong plate scale)
    \item Magnitude-dependent bias (charge bleeding)
\end{itemize}

MPC maintains quality flags, but user must validate data.

\section{Practical Example}

\subsection{Loading MPC Observations}

\begin{lstlisting}[language=C++,caption={Parsing MPC observations}]
#include <astdyn/observations/MPCObservation.hpp>

std::vector<Observation> load_mpc_file(const std::string& filename) {
    std::vector<Observation> observations;
    std::ifstream file(filename);
    std::string line;
    
    while (std::getline(file, line)) {
        if (line.length() < 80) continue;
        
        MPCObservation obs;
        if (obs.parse(line)) {
            observations.push_back(obs);
        }
    }
    
    std::cout << "Loaded " << observations.size() << " observations\n";
    return observations;
}
\end{lstlisting}

\subsection{Computing Predicted Observations}

\begin{lstlisting}[language=C++,caption={Predicting observations}]
Vector2d predict_observation(
    const Vector6d& state,
    double epoch,
    const std::string& obs_code,
    const EphemerisInterface& ephemeris)
{
    // Get Earth position
    Vector3d earth_pos = ephemeris.get_position("Earth", epoch);
    
    // Get observatory position (ITRF -> inertial)
    Vector3d obs_pos_geo = observatory_db.get_geocentric(obs_code);
    Matrix3d R_itrf_to_j2000 = earth_rotation(epoch);
    Vector3d obs_pos = earth_pos + R_itrf_to_j2000 * obs_pos_geo;
    
    // Topocentric position
    Vector3d r_topo = state.head<3>() - obs_pos;
    
    // Ecliptic to equatorial
    Vector3d r_eq = R_ecl_to_eq * r_topo;
    
    // Compute RA/Dec
    double alpha = std::atan2(r_eq(1), r_eq(0));
    double delta = std::asin(r_eq(2) / r_eq.norm());
    
    if (alpha < 0) alpha += 2*M_PI;
    
    return Vector2d(alpha, delta);
}
\end{lstlisting}

\section{Summary}

Key concepts about observations:

\begin{enumerate}
    \item \textbf{Optical astrometry} provides RA/Dec with 0.1--0.5 arcsec precision
    \item \textbf{Radar} gives range/Doppler with meter/mm-per-sec precision
    \item \textbf{Light-time} correction is essential (4--30 minutes for asteroids)
    \item \textbf{Aberration} causes $\pm$20 arcsec displacement
    \item \textbf{Refraction} affects low-elevation observations
    \item \textbf{Observatory position} must be in inertial frame
    \item \textbf{MPC format} is standard, ADES is modern
    \item \textbf{Weighting} by uncertainty improves fit quality
    \item \textbf{Partials} $\partial(\alpha,\delta)/\partial\mathbf{y}$ enable orbit fitting
\end{enumerate}

Practical recommendations:
\begin{itemize}
    \item Always apply light-time and aberration corrections
    \item Use Gaia DR3 catalog for modern observations
    \item Validate timing (UTC, leap seconds)
    \item Check observatory coordinates
    \item Weight by estimated uncertainty
    \item Identify and downweight outliers
\end{itemize}

The next chapter covers initial orbit determination from a few observations, followed by differential correction to refine orbits using all available data.

\chapter{Initial Orbit Determination}
\label{ch:initial_orbit}

\section{Introduction}

\textbf{Initial orbit determination} (IOD) computes an approximate orbit from a small number of observations. This provides:
\begin{itemize}
    \item Starting point for differential correction
    \item Linking observations across oppositions
    \item Recovery predictions for lost objects
    \item Preliminary impact assessments
\end{itemize}

Classical methods use 3 observations (Gauss, Laplace) or 2 observations + constraints.

\section{The IOD Problem}

\subsection{Angles-Only Observations}

Given: Three observations $(\alpha_i, \delta_i, t_i)$, $i = 1, 2, 3$.

Find: Six orbital elements or Cartesian state $\mathbf{y} = [\mathbf{r}, \mathbf{v}]$.

\textbf{Challenge}: We have 6 unknowns but only 6 constraints (2 angles $\times$ 3 times). The problem is exactly determined but highly nonlinear.

\subsection{Line of Sight}

Each observation defines a unit vector:

\begin{equation}
    \hat{\rho}_i = \begin{bmatrix} \cos\delta_i \cos\alpha_i \\ \cos\delta_i \sin\alpha_i \\ \sin\delta_i \end{bmatrix}
\end{equation}

The object lies somewhere along this line: $\mathbf{r}_i = \mathbf{R}_i + \rho_i \hat{\rho}_i$ where $\mathbf{R}_i$ is observatory position and $\rho_i$ is unknown topocentric range.

\section{Gauss Method}

\subsection{Historical Context}

Developed by Carl Friedrich Gauss (1809) to recover Ceres after it passed behind the Sun. Still widely used today.

\subsection{Basic Idea}

Use 3 observations to:
\begin{enumerate}
    \item Estimate range $\rho_2$ at middle observation
    \item Compute position $\mathbf{r}_2$
    \item Use Lagrange coefficients to get velocity $\mathbf{v}_2$
\end{enumerate}

\subsection{Lagrange Coefficients}

For two-body motion, positions at times $t_1$, $t_2$, $t_3$ are related by:

\begin{align}
    \mathbf{r}_1 &= f_1 \mathbf{r}_2 + g_1 \mathbf{v}_2 \\
    \mathbf{r}_3 &= f_3 \mathbf{r}_2 + g_3 \mathbf{v}_2
\end{align}

where $f$ and $g$ are Lagrange coefficients depending on time intervals $\tau_1 = t_1 - t_2$ and $\tau_3 = t_3 - t_2$.

Series expansion:

\begin{align}
    f &= 1 - \frac{\mu}{2r^3}\tau^2 + \frac{\mu}{2r^3}\frac{\mathbf{r} \cdot \mathbf{v}}{r^2}\tau^3 + O(\tau^4) \\
    g &= \tau - \frac{\mu}{6r^3}\tau^3 + O(\tau^4)
\end{align}

\subsection{Scalar Equation of Lagrange}

The three position vectors lie in the orbital plane. Using coplanarity:

\begin{equation}
    \mathbf{r}_1 \cdot (\mathbf{r}_2 \times \mathbf{r}_3) = 0
\end{equation}

This gives a scalar equation for $\rho_2$ (the "8th degree polynomial" after manipulation).

\subsection{Algorithm}

\textbf{Input}: Three observations $(\alpha_i, \delta_i, t_i, \mathbf{R}_i)$.

\textbf{Steps}:
\begin{enumerate}
    \item Compute line-of-sight vectors $\hat{\rho}_i$
    \item Initial guess: $\rho_2 = |\mathbf{R}_2|$ (Earth-Sun distance)
    \item Iterate:
    \begin{enumerate}
        \item Compute $\mathbf{r}_2 = \mathbf{R}_2 + \rho_2 \hat{\rho}_2$
        \item Compute $r_2 = |\mathbf{r}_2|$
        \item Estimate $f, g$ coefficients
        \item Solve for $\mathbf{v}_2$ from $\mathbf{r}_1, \mathbf{r}_3$
        \item Refine $\rho_2$ using Lagrange scalar equation
        \item Check convergence: $|\Delta\rho_2| < 10^{-6}$ AU
    \end{enumerate}
    \item Return state $(\mathbf{r}_2, \mathbf{v}_2)$ at epoch $t_2$
\end{enumerate}

\textbf{Convergence}: Typically 5-10 iterations for well-observed objects.

\section{Implementation}

\begin{lstlisting}[language=C++,caption={Gauss method implementation}]
Vector6d gauss_iod(
    const std::array<Observation, 3>& obs,
    const EphemerisInterface& ephemeris)
{
    // Extract times and line-of-sight vectors
    double t1 = obs[0].epoch;
    double t2 = obs[1].epoch;
    double t3 = obs[2].epoch;
    
    Vector3d rho_hat1 = obs[0].line_of_sight();
    Vector3d rho_hat2 = obs[1].line_of_sight();
    Vector3d rho_hat3 = obs[2].line_of_sight();
    
    // Observatory positions
    Vector3d R1 = ephemeris.get_observer_position(obs[0]);
    Vector3d R2 = ephemeris.get_observer_position(obs[1]);
    Vector3d R3 = ephemeris.get_observer_position(obs[2]);
    
    // Time intervals
    double tau1 = t1 - t2;
    double tau3 = t3 - t2;
    
    // Initial guess for middle range
    double rho2 = R2.norm();
    
    // Iterative refinement
    for (int iter = 0; iter < 20; ++iter) {
        Vector3d r2 = R2 + rho2 * rho_hat2;
        double r2_mag = r2.norm();
        
        // Compute f,g series (to 3rd order)
        double f1 = 1.0 - 0.5 * MU_SUN * tau1*tau1 / (r2_mag*r2_mag*r2_mag);
        double f3 = 1.0 - 0.5 * MU_SUN * tau3*tau3 / (r2_mag*r2_mag*r2_mag);
        double g1 = tau1 - MU_SUN * tau1*tau1*tau1 / (6.0 * r2_mag*r2_mag*r2_mag);
        double g3 = tau3 - MU_SUN * tau3*tau3*tau3 / (6.0 * r2_mag*r2_mag*r2_mag);
        
        // Solve for velocity at t2
        Vector3d v2 = (f3 * (R1 + rho_hat1) - f1 * (R3 + rho_hat3)) / (f1*g3 - f3*g1);
        
        // Improve rho2 using scalar equation of Lagrange
        // (simplified: use r1, r3 estimates)
        Vector3d r1 = r2 * f1 + v2 * g1;
        Vector3d r3 = r2 * f3 + v3 * g3;
        
        double rho1_new = (r1 - R1).dot(rho_hat1);
        double rho3_new = (r3 - R3).dot(rho_hat3);
        double rho2_new = (r2 - R2).dot(rho_hat2);
        
        if (std::abs(rho2_new - rho2) < 1e-6) {
            // Converged
            return Vector6d(r2, v2);
        }
        
        rho2 = rho2_new;
    }
    
    throw std::runtime_error("Gauss IOD did not converge");
}
\end{lstlisting}

\section{Too-Short Arc Problem}

\subsection{Challenge}

For short observational arcs (hours to days), many orbits fit equally well. The orbit is poorly constrained in:
\begin{itemize}
    \item Semimajor axis $a$ (degenerate with eccentricity)
    \item Eccentricity $e$
    \item Argument of perihelion $\omega$
\end{itemize}

\textbf{Example}: NEA observed for 3 hours. Could be:
\begin{itemize}
    \item $a = 1.2$ AU, $e = 0.1$ (Apollo)
    \item $a = 2.5$ AU, $e = 0.6$ (Amor)
    \item $a = 0.8$ AU, $e = 0.3$ (Aten)
\end{itemize}

All produce similar RA/Dec over short arc!

\subsection{Additional Constraints}

To resolve degeneracy:
\begin{enumerate}
    \item \textbf{Apparent motion}: $d\alpha/dt$, $d\delta/dt$ constrains distance
    \item \textbf{Brightness}: $H, G$ phase function gives distance estimate
    \item \textbf{Statistical priors}: Most NEAs have $0.8 < a < 2$ AU
    \item \textbf{Additional observations}: Even +1 day helps enormously
\end{enumerate}

\section{Laplace Method}

\subsection{Alternative Approach}

Use angular velocity $\dot{\alpha}, \dot{\delta}$ in addition to angles. Requires high-precision timing or multiple closely-spaced observations.

\textbf{Advantage}: Can work with 2 observations (plus rates).

\textbf{Disadvantage}: Sensitive to measurement errors in rates.

\subsection{Equations}

From $\mathbf{r} = \mathbf{R} + \rho\hat{\rho}$, differentiate twice:

\begin{equation}
    \ddot{\mathbf{r}} = -\frac{\mu}{r^3}\mathbf{r}
\end{equation}

This gives 3 equations in 3 unknowns ($\rho, \dot{\rho}, \ddot{\rho}$) at one epoch.

\section{Modern Methods}

\subsection{Admissible Region}

For very short arcs, solve for all admissible orbits satisfying:
\begin{itemize}
    \item Observations
    \item Physical constraints ($e < 1$ for bound orbits)
    \item Brightness (distance estimate)
\end{itemize}

Produces a region in orbital element space, not a single solution.

\subsection{Constrained Least Squares}

Minimize:

\begin{equation}
    \chi^2 = \sum_i w_i(\mathbf{o}_i - \mathbf{c}_i)^2 + \lambda P(\mathbf{e})
\end{equation}

where $P(\mathbf{e})$ is a prior on elements (e.g., prefer $e < 0.3$).

\section{Quality Assessment}

\subsection{Orbit Uncertainty}

From 3 observations, uncertainty is large:
\begin{itemize}
    \item Position at epoch: $\sim$0.001 AU (150,000 km)
    \item Velocity: $\sim$0.01 AU/day (17 km/s)
    \item Semimajor axis: $\pm$0.5 AU
\end{itemize}

\textbf{Propagation uncertainty grows rapidly!} After 1 month, position error $>$1 AU.

\subsection{Validation}

Check orbit quality:
\begin{enumerate}
    \item Residuals: Should be $<$5 arcsec for good fit
    \item Energy: $E < 0$ for bound orbit
    \item Perihelion: $q > 0.1$ AU (inside this, orbit crashes into Sun)
    \item Eccentricity: $0 \le e < 1$ for elliptic orbit
\end{enumerate}

\section{Example: Newly Discovered Asteroid}

\begin{lstlisting}[language=C++,caption={IOD from discovery observations}]
// Three observations from MPC
std::vector<Observation> obs = {
    {"2024-01-15T03:15:00Z", 185.234, +12.567, "F51"},
    {"2024-01-15T04:30:00Z", 185.189, +12.592, "F51"},
    {"2024-01-15T05:45:00Z", 185.144, +12.617, "F51"}
};

// Load planetary ephemeris
SpiceInterface spice;
spice.load_kernel("de440.bsp");

// Perform Gauss IOD
try {
    Vector6d state = gauss_iod(obs, spice);
    double epoch = obs[1].epoch;
    
    // Convert to orbital elements
    OrbitalElements elements = OrbitalElements::from_cartesian(state, epoch);
    
    std::cout << "Initial orbit determination:\n";
    std::cout << "a = " << elements.a << " AU\n";
    std::cout << "e = " << elements.e << "\n";
    std::cout << "i = " << elements.i * RAD_TO_DEG << " deg\n";
    std::cout << "Omega = " << elements.Omega * RAD_TO_DEG << " deg\n";
    std::cout << "omega = " << elements.omega * RAD_TO_DEG << " deg\n";
    std::cout << "M = " << elements.M * RAD_TO_DEG << " deg\n";
    
    // Compute residuals
    for (const auto& ob : obs) {
        Vector2d predicted = predict_observation(state, ob.epoch, ob.obs_code, spice);
        double dRA = (predicted(0) - ob.ra) * cos(ob.dec) * RAD_TO_ARCSEC;
        double dDec = (predicted(1) - ob.dec) * RAD_TO_ARCSEC;
        std::cout << "Residual: " << dRA << ", " << dDec << " arcsec\n";
    }
    
} catch (const std::exception& e) {
    std::cerr << "IOD failed: " << e.what() << "\n";
}
\end{lstlisting}

\section{Summary}

Key points about initial orbit determination:

\begin{enumerate}
    \item \textbf{Gauss method} uses 3 observations to determine orbit
    \item \textbf{Lagrange coefficients} relate positions at different times
    \item \textbf{Iterative solution} converges in 5-10 iterations typically
    \item \textbf{Short arcs} lead to poorly constrained orbits
    \item \textbf{Additional constraints} (brightness, priors) help
    \item \textbf{Laplace method} uses angular rates as well as angles
    \item \textbf{Modern methods} compute admissible regions
    \item \textbf{Validation} checks energy, eccentricity, residuals
\end{enumerate}

The initial orbit is refined using differential correction (next chapter) with all available observations.

\include{14_differential_correction}
\chapter{Residual Analysis}
\label{ch:residuals}

\section{Introduction}

\textbf{Residual analysis} is the examination of differences between observed and computed values (O-C) to assess orbit quality and diagnose problems.

\textbf{Goals}:
\begin{itemize}
    \item Validate orbit fit quality
    \item Identify outliers and systematic errors
    \item Assess observation weights
    \item Detect force model inadequacies
    \item Estimate realistic uncertainties
\end{itemize}

\section{Types of Residuals}

\subsection{Post-Fit Residuals}

After differential correction converges:

\begin{equation}
    r_i = o_i - c_i(\mathbf{y}_0^*)
\end{equation}

where $\mathbf{y}_0^*$ is the converged orbit.

For RA/Dec:
\begin{align}
    \Delta\alpha_i &= (\alpha_{\text{obs}} - \alpha_{\text{comp}}) \cos\delta_{\text{obs}} \\
    \Delta\delta_i &= \delta_{\text{obs}} - \delta_{\text{comp}}
\end{align}

Note: Multiply $\Delta\alpha$ by $\cos\delta$ to get linear separation.

\subsection{Normalized Residuals}

Scale by observation uncertainty:

\begin{equation}
    \zeta_i = \frac{r_i}{\sigma_i}
\end{equation}

Expected distribution: $\zeta_i \sim \mathcal{N}(0, 1)$ if weights are correct.

\subsection{Standardized Residuals}

Account for correlation in fit:

\begin{equation}
    \xi_i = \frac{r_i}{\sigma_i \sqrt{1 - h_{ii}}}
\end{equation}

where $h_{ii}$ is the $i$-th diagonal element of the hat matrix $\mathbf{S}$:
\begin{equation}
    \mathbf{S} = \mathbf{H}(\mathbf{H}^T\mathbf{W}\mathbf{H})^{-1}\mathbf{H}^T\mathbf{W}
\end{equation}

\section{Quality Metrics}

\subsection{Root Mean Square (RMS)}

\begin{equation}
    \text{RMS} = \sqrt{\frac{\sum_i w_i r_i^2}{\sum_i w_i}}
\end{equation}

For equal weights:

\begin{equation}
    \text{RMS} = \sqrt{\frac{1}{m} \sum_i r_i^2}
\end{equation}

\textbf{Interpretation}:
\begin{itemize}
    \item RMS $<$ 0.5": Excellent (modern CCD with Gaia catalog)
    \item RMS $\sim$ 1": Good (typical CCD)
    \item RMS $\sim$ 2": Fair (amateur observations)
    \item RMS $>$ 5": Poor (suspect systematic errors)
\end{itemize}

\subsection{Weighted RMS}

For unequal weights:

\begin{equation}
    \text{WRMS} = \sqrt{\frac{\chi^2}{m - n}}
\end{equation}

where $m$ is number of observations, $n = 6$ is number of parameters.

\subsection{Chi-Square Test}

Under correct model and weights:

\begin{equation}
    \chi^2 = \sum_i w_i r_i^2 \sim \chi^2_{m-n}
\end{equation}

Test statistic:

\begin{equation}
    \chi^2_{\text{red}} = \frac{\chi^2}{m - n}
\end{equation}

\textbf{Interpretation}:
\begin{itemize}
    \item $\chi^2_{\text{red}} \approx 1$: Weights consistent with errors
    \item $\chi^2_{\text{red}} \gg 1$: Underestimated uncertainties or model error
    \item $\chi^2_{\text{red}} \ll 1$: Overestimated uncertainties
\end{itemize}

\subsection{Maximum Residual}

\begin{equation}
    r_{\text{max}} = \max_i |r_i|
\end{equation}

Flag observations with $|r_i| > 3\sigma$ as potential outliers.

\section{Residual Plots}

\subsection{Residuals vs. Time}

Plot $r_i$ vs. $t_i$. Look for:
\begin{itemize}
    \item \textbf{Random scatter}: Good
    \item \textbf{Trends}: Systematic error (e.g., missing perturbation, catalog bias)
    \item \textbf{Jumps}: Change in observing conditions or equipment
    \item \textbf{Periodic variation}: Orbit model error
\end{itemize}

\subsection{Residuals vs. Observatory}

Plot $r_i$ vs. observatory code. Look for:
\begin{itemize}
    \item \textbf{Uniform scatter}: Good
    \item \textbf{Bias for specific site}: Site-specific systematic (timing, coordinates, catalog)
\end{itemize}

\subsection{Residuals vs. Magnitude}

Plot $r_i$ vs. apparent magnitude. Look for:
\begin{itemize}
    \item \textbf{No trend}: Good
    \item \textbf{Increasing scatter with magnitude}: Photon noise dominates
    \item \textbf{Bias trend}: Magnitude equation error in astrometry
\end{itemize}

\subsection{RA vs. Dec Residuals}

Plot $\Delta\alpha \cos\delta$ vs. $\Delta\delta$. Look for:
\begin{itemize}
    \item \textbf{Circular scatter}: Isotropic errors
    \item \textbf{Elliptical scatter}: Correlated errors (e.g., tracking error)
    \item \textbf{Radial pattern}: Distance error
\end{itemize}

\subsection{Normal Probability Plot}

Plot ordered normalized residuals $\zeta_{(i)}$ vs. expected normal quantiles. Should be approximately linear if errors are Gaussian.

\section{Outlier Detection}

\subsection{Threshold Method}

Flag observation if:

\begin{equation}
    |r_i| > k \sigma_i
\end{equation}

Typical $k = 3$ (3-sigma rule) or $k = 2.5$ (more aggressive).

\subsection{Chauvenet's Criterion}

Reject observation if probability of larger deviation is $< 1/(2m)$:

\begin{equation}
    P(|\zeta| > |\zeta_i|) < \frac{1}{2m}
\end{equation}

\subsection{Median Absolute Deviation (MAD)}

Robust alternative to standard deviation:

\begin{equation}
    \text{MAD} = \text{median}(|r_i - \text{median}(r_i)|)
\end{equation}

Scaled MAD: $\hat{\sigma} = 1.4826 \times \text{MAD}$

Flag if $|r_i - \text{median}| > k\hat{\sigma}$.

\subsection{Iterative Outlier Removal}

\begin{enumerate}
    \item Run differential correction
    \item Identify outliers (e.g., $|r_i| > 3\sigma$)
    \item Remove or downweight outliers
    \item Repeat until no more outliers found
\end{enumerate}

\textbf{Caution}: Don't remove too many observations. Typically remove $<$5\% of dataset.

\section{Systematic Error Diagnosis}

\subsection{Timing Errors}

\textbf{Symptom}: Residuals correlated with sky motion direction.

\textbf{Test}: Compute along-track vs. cross-track residuals:

\begin{align}
    r_{\parallel} &= \Delta\alpha \cos\delta \cos\theta + \Delta\delta \sin\theta \\
    r_{\perp} &= -\Delta\alpha \cos\delta \sin\theta + \Delta\delta \cos\theta
\end{align}

where $\theta = \arctan2(\dot{\delta}, \dot{\alpha}\cos\delta)$ is direction of motion.

If $|r_{\parallel}| \gg |r_{\perp}|$, suspect timing error.

\subsection{Catalog Bias}

\textbf{Symptom}: Systematic offset in all residuals from one catalog.

\textbf{Test}: Compare results using different star catalogs (Gaia DR3, UCAC4, etc.).

\textbf{Solution}: Use Gaia DR3 (most accurate, 0.02-0.05" systematic).

\subsection{Observatory Coordinate Error}

\textbf{Symptom}: Systematic offset for one observatory, varies with object position.

\textbf{Test}: Check MPC observatory coordinates vs. ITRF values.

\textbf{Solution}: Update coordinates, especially for new observatories.

\subsection{Light-Time Correction}

\textbf{Symptom}: Residuals show quadratic trend over long arc.

\textbf{Test}: Check that light-time correction is applied.

\textbf{Solution}: Iterate light-time (Chapter 12).

\subsection{Force Model Inadequacy}

\textbf{Symptom}: Residuals show smooth trend correlated with planetary positions.

\textbf{Test}: Add missing perturbations (Jupiter, Saturn, Earth, etc.).

\textbf{Solution}: Include all planets with $|a_{\text{pert}}/a_{\text{Sun}}| > 10^{-9}$.

\section{Example Analysis}

\begin{lstlisting}[language=C++,caption={Residual analysis implementation}]
struct ResidualAnalysis {
    double rms;
    double wrms;
    double chi2_red;
    double max_residual;
    std::vector<double> residuals;
    std::vector<double> normalized_residuals;
    std::vector<int> outlier_indices;
};

ResidualAnalysis analyze_residuals(
    const std::vector<Observation>& obs,
    const Vector6d& state,
    double epoch,
    const ForceModel& forces,
    const EphemerisInterface& ephemeris)
{
    ResidualAnalysis result;
    double chi2 = 0.0;
    double sum_weights = 0.0;
    
    for (size_t i = 0; i < obs.size(); ++i) {
        // Propagate and predict
        Vector6d y_obs = propagate(state, epoch, obs[i].epoch, forces);
        Vector2d computed = predict_observation(y_obs, obs[i].epoch, obs[i].obs_code, ephemeris);
        
        // Compute residual (in arcsec)
        double dRA = (obs[i].ra - computed(0)) * cos(obs[i].dec) * RAD_TO_ARCSEC;
        double dDec = (obs[i].dec - computed(1)) * RAD_TO_ARCSEC;
        double residual = sqrt(dRA*dRA + dDec*dDec);
        
        result.residuals.push_back(residual);
        
        // Normalized residual
        double sigma = sqrt(obs[i].sigma_ra*obs[i].sigma_ra + obs[i].sigma_dec*obs[i].sigma_dec) * RAD_TO_ARCSEC;
        double zeta = residual / sigma;
        result.normalized_residuals.push_back(zeta);
        
        // Chi-square
        double weight = 1.0 / (sigma * sigma);
        chi2 += weight * residual * residual;
        sum_weights += weight;
        
        // Max residual
        if (residual > result.max_residual) {
            result.max_residual = residual;
        }
        
        // Outlier detection (3-sigma)
        if (std::abs(zeta) > 3.0) {
            result.outlier_indices.push_back(i);
        }
    }
    
    // RMS
    result.rms = sqrt(chi2 / obs.size());
    
    // Weighted RMS
    int dof = 2 * obs.size() - 6;
    result.wrms = sqrt(chi2 / dof);
    
    // Reduced chi-square
    result.chi2_red = chi2 / dof;
    
    return result;
}

// Print analysis report
void print_residual_report(const ResidualAnalysis& analysis) {
    std::cout << "Residual Analysis Report\n";
    std::cout << "========================\n";
    std::cout << "Number of observations: " << analysis.residuals.size() << "\n";
    std::cout << "RMS: " << analysis.rms << " arcsec\n";
    std::cout << "Weighted RMS: " << analysis.wrms << " arcsec\n";
    std::cout << "Reduced chi-square: " << analysis.chi2_red << "\n";
    std::cout << "Maximum residual: " << analysis.max_residual << " arcsec\n";
    std::cout << "Number of outliers (>3-sigma): " << analysis.outlier_indices.size() << "\n";
    
    if (!analysis.outlier_indices.empty()) {
        std::cout << "\nOutlier indices:\n";
        for (int idx : analysis.outlier_indices) {
            std::cout << "  " << idx << ": " << analysis.residuals[idx] 
                     << " arcsec (" << analysis.normalized_residuals[idx] << "-sigma)\n";
        }
    }
    
    // Histogram of normalized residuals
    std::cout << "\nNormalized residual distribution:\n";
    auto hist = make_histogram(analysis.normalized_residuals, -4, 4, 16);
    for (auto [bin, count] : hist) {
        std::cout << std::setw(6) << std::fixed << std::setprecision(2) << bin << ": ";
        std::cout << std::string(count, '*') << " (" << count << ")\n";
    }
}
\end{lstlisting}

\subsection{Example Output}

\begin{verbatim}
Residual Analysis Report
========================
Number of observations: 100
RMS: 0.658 arcsec
Weighted RMS: 0.661 arcsec
Reduced chi-square: 1.02
Maximum residual: 2.34 arcsec
Number of outliers (>3-sigma): 2

Outlier indices:
  34: 2.34 arcsec (3.12-sigma)
  78: 2.11 arcsec (3.05-sigma)

Normalized residual distribution:
 -4.00: 
 -3.00: *
 -2.00: ****
 -1.00: ************
  0.00: **********************************
  1.00: ***************
  2.00: *****
  3.00: **
  4.00: 
\end{verbatim}

\textbf{Interpretation}:
\begin{itemize}
    \item RMS $\approx$ 0.66": Excellent fit
    \item $\chi^2_{\text{red}} \approx 1$: Weights are appropriate
    \item 2 outliers: Typical for 100 observations (2\%)
    \item Distribution approximately normal
\end{itemize}

\section{Improving Orbit Quality}

\subsection{When RMS is Too Large}

\textbf{Actions}:
\begin{enumerate}
    \item Check for outliers, remove if $>$3$\sigma$
    \item Verify observatory coordinates
    \item Check timing accuracy
    \item Add missing perturbations
    \item Use better star catalog (Gaia DR3)
    \item Consider non-gravitational forces (if comet)
\end{enumerate}

\subsection{When $\chi^2_{\text{red}} \gg 1$}

\textbf{Causes}:
\begin{itemize}
    \item Underestimated observation uncertainties
    \item Systematic errors not modeled
    \item Force model inadequate
\end{itemize}

\textbf{Solutions}:
\begin{itemize}
    \item Inflate uncertainties by factor $\sqrt{\chi^2_{\text{red}}}$
    \item Investigate systematic errors
    \item Improve force model
\end{itemize}

\subsection{When Few Observations Available}

For $m < 20$ observations:
\begin{itemize}
    \item Single outlier can dominate $\chi^2$
    \item Use robust methods (MAD, Huber weights)
    \item Be conservative about rejecting data
    \item Seek additional observations
\end{itemize}

\section{Reporting Results}

\subsection{Summary Statistics}

Always report:
\begin{itemize}
    \item Number of observations
    \item Time span
    \item Observatories
    \item RMS or WRMS
    \item Number of outliers rejected
\end{itemize}

\subsection{Covariance Interpretation}

\textbf{Formal uncertainty}: From $\mathbf{C} = \mathbf{N}^{-1}$.

\textbf{Realistic uncertainty}: Scale by $\sqrt{\chi^2_{\text{red}}}$ if $\chi^2_{\text{red}} > 1$.

\subsection{Orbit Arc Assessment}

\begin{itemize}
    \item \textbf{Short arc} ($<$10 days): Orbit poorly constrained, large extrapolation uncertainty
    \item \textbf{Medium arc} (10-60 days): Reasonable for ephemeris over similar span
    \item \textbf{Long arc} ($>$1 year): Well-constrained, reliable extrapolation
\end{itemize}

\section{Summary}

Key points about residual analysis:

\begin{enumerate}
    \item \textbf{Post-fit residuals} $r_i = o_i - c_i$ assess fit quality
    \item \textbf{RMS} measures overall fit; target $<$1" for modern observations
    \item \textbf{Chi-square test} validates weights; expect $\chi^2_{\text{red}} \approx 1$
    \item \textbf{Residual plots} diagnose systematic errors
    \item \textbf{Outliers} detected via 3$\sigma$ threshold or robust methods
    \item \textbf{Systematic errors} identified by correlations with time, observatory, magnitude
    \item \textbf{Force model} validated by examining residual trends
    \item \textbf{Realistic uncertainties} account for systematic errors via $\chi^2_{\text{red}}$
\end{enumerate}

With differential correction and residual analysis, we complete the core orbit determination workflow. Next chapters cover software implementation.


% Part IV
\part{AstDyn Library Implementation}
\chapter{Software Architecture}
\label{ch:architecture}

\section{Introduction}

AstDyn is designed as a modern C++17 library for astrodynamics and orbit determination. The architecture emphasizes:

\begin{itemize}
    \item \textbf{Modularity}: Independent modules with clear interfaces
    \item \textbf{Performance}: Efficient numerical algorithms with Eigen3
    \item \textbf{Extensibility}: Easy to add new force models, integrators, parsers
    \item \textbf{Maintainability}: Clean code, comprehensive tests, documentation
\end{itemize}

\section{Design Principles}

\subsection{Separation of Concerns}

Each module handles a specific aspect:
\begin{itemize}
    \item \textbf{Time}: Scale conversions (UTC, TT, TDB)
    \item \textbf{Coordinates}: Reference frames, transformations
    \item \textbf{Orbit}: Elements, state vectors, conversions
    \item \textbf{Propagation}: Numerical integration, force models
    \item \textbf{Observations}: Astrometry, MPC format, weights
    \item \textbf{Orbit Determination}: IOD, differential correction, residuals
\end{itemize}

\subsection{Interface-Based Design}

Abstract interfaces enable flexibility:

\begin{lstlisting}[language=C++,caption={Interface examples}]
// Parser interface - multiple formats supported
class IParser {
public:
    virtual ~IParser() = default;
    virtual OrbitalElements parse(const std::string& filename) = 0;
};

// Integrator interface - multiple methods available
class IIntegrator {
public:
    virtual ~IIntegrator() = default;
    virtual void integrate(State& y, double t0, double t1, ForceModel& forces) = 0;
};

// Ephemeris interface - SPICE, JPL, analytic
class IEphemeris {
public:
    virtual ~IEphemeris() = default;
    virtual Vector3d get_position(Body body, double jd_tdb) = 0;
};
\end{lstlisting}

\subsection{Header-Only vs. Compiled}

\textbf{Header-only} (inline, templates):
\begin{itemize}
    \item \texttt{core/Constants.hpp}: Physical constants
    \item \texttt{core/Types.hpp}: Type aliases, enums
    \item \texttt{utils/StringUtils.hpp}: String utilities
\end{itemize}

\textbf{Compiled} (implementation in .cpp):
\begin{itemize}
    \item All numerical algorithms (propagation, integration)
    \item I/O operations (file parsing, observation loading)
    \item Complex calculations (STM, differential correction)
\end{itemize}

\section{Module Organization}

\subsection{Directory Structure}

\begin{lstlisting}[language=bash,caption={Project layout}]
astdyn/
|-- include/astdyn/           # Public headers
|   |-- AstDyn.hpp           # Main include (everything)
|   |-- AstDynEngine.hpp     # High-level engine
|   |-- Version.hpp          # Version info (generated)
|   |-- Config.hpp           # Build configuration (generated)
|   |-- core/                # Fundamental types
|   |   |-- Constants.hpp
|   |   `-- Types.hpp
|   |-- math/                # Mathematical utilities
|   |   |-- MathUtils.hpp
|   |   `-- LinearAlgebra.hpp
|   |-- time/                # Time scales
|   |   `-- TimeScale.hpp
|   |-- coordinates/         # Reference frames
|   |   |-- KeplerianElements.hpp
|   |   |-- CartesianState.hpp
|   |   `-- CometaryElements.hpp
|   |-- orbit/               # Orbital mechanics
|   |   |-- TwoBody.hpp
|   |   `-- Perturbations.hpp
|   |-- propagation/         # Numerical integration
|   |   |-- Integrator.hpp
|   |   `-- Propagator.hpp
|   |-- observations/        # Astrometric data
|   |   |-- Observation.hpp
|   |   |-- MPCReader.hpp
|   |   `-- ObservatoryDatabase.hpp
|   |-- orbit_determination/ # OD algorithms
|   |   |-- GaussIOD.hpp
|   |   |-- DifferentialCorrection.hpp
|   |   |-- StateTransitionMatrix.hpp
|   |   `-- Residuals.hpp
|   |-- io/                  # Parsers
|   |   |-- IParser.hpp
|   |   |-- ParserFactory.hpp
|   |   `-- parsers/
|   |       |-- OrbFitEQ1Parser.hpp
|   |       `-- OrbFitRWOParser.hpp
|   |-- ephemeris/           # Planetary positions
|   |   `-- SpiceInterface.hpp
|   `-- utils/               # Utilities
|       |-- Logger.hpp
|       `-- StringUtils.hpp
|-- src/                     # Implementation files
|   |-- CMakeLists.txt
|   |-- AstDynEngine.cpp
|   |-- math/
|   |-- time/
|   |-- coordinates/
|   |-- orbit/
|   |-- propagation/
|   |-- observations/
|   |-- orbit_determination/
|   |-- io/
|   `-- ephemeris/
|-- tests/                   # Unit tests (Google Test)
|-- examples/                # Example programs
|-- docs/                    # Documentation
`-- data/                    # Data files (kernels, catalogs)
\end{lstlisting}

\subsection{Namespace Organization}

\begin{lstlisting}[language=C++,caption={Namespace hierarchy}]
namespace astdyn {
    namespace constants {    // Physical constants
        constexpr double AU = 149597870.7;  // km
        constexpr double C_LIGHT = 299792.458;  // km/s
        // ...
    }
    
    namespace math {         // Math utilities
        double mod_angle(double angle, double period);
        Matrix3d rotation_matrix_z(double angle);
        // ...
    }
    
    namespace time {         // Time conversions
        double utc_to_tt(double jd_utc);
        double tt_to_tdb(double jd_tt);
        // ...
    }
    
    namespace coordinates {  // Coordinate systems
        class KeplerianElements { /* ... */ };
        class CartesianState { /* ... */ };
        // ...
    }
    
    namespace observations { // Observations
        class Observation { /* ... */ };
        class MPCReader { /* ... */ };
        // ...
    }
    
    // Propagation, orbit determination at top level
    class Propagator { /* ... */ };
    class DifferentialCorrection { /* ... */ };
    // ...
}
\end{lstlisting}

\section{Core Components}

\subsection{Constants and Types}

\textbf{Physical Constants} (\texttt{core/Constants.hpp}):
\begin{itemize}
    \item Gravitational parameters: \texttt{MU\_SUN}, \texttt{MU\_EARTH}, etc.
    \item Distances: \texttt{AU}, \texttt{EARTH\_RADIUS}
    \item Time: \texttt{JD2000}, \texttt{SECONDS\_PER\_DAY}
    \item Speed of light, obliquity, etc.
\end{itemize}

\textbf{Type Aliases} (\texttt{core/Types.hpp}):
\begin{lstlisting}[language=C++]
// Linear algebra (Eigen)
using Vector3d = Eigen::Vector3d;
using Vector6d = Eigen::Matrix<double, 6, 1>;
using Matrix3d = Eigen::Matrix3d;
using Matrix6d = Eigen::Matrix<double, 6, 6>;

// Strong typing for units
using Radians = double;
using Degrees = double;
using AU_Distance = double;
using KM_Distance = double;
using JulianDate = double;
\end{lstlisting}

\textbf{Enumerations}:
\begin{lstlisting}[language=C++]
enum class CoordinateSystem {
    ECLIPTIC, EQUATORIAL, ICRF, BODY_FIXED
};

enum class ElementType {
    KEPLERIAN, CARTESIAN, COMETARY, EQUINOCTIAL
};

enum class TimeScale {
    UTC, UT1, TAI, TT, TDB, TCB, TCG
};

enum class IntegratorType {
    RADAU15, RK_GAUSS, DOPRI, LSODAR, GAUSS_JACKSON
};
\end{lstlisting}

\subsection{Version and Configuration}

\textbf{Version} (generated from CMake):
\begin{lstlisting}[language=C++]
namespace astdyn {
    namespace Version {
        constexpr int major = 1;
        constexpr int minor = 0;
        constexpr int patch = 0;
        constexpr const char* string = "1.0.0";
    }
}
\end{lstlisting}

\textbf{Configuration} (build options):
\begin{lstlisting}[language=C++]
namespace astdyn {
    namespace Config {
        constexpr bool use_spice = true;
        constexpr bool use_openmp = false;
        constexpr const char* build_type = "Release";
        constexpr const char* compiler = "AppleClang 16.0.0";
    }
}
\end{lstlisting}

\section{Dependency Management}

\subsection{External Dependencies}

\textbf{Eigen3} (required):
\begin{itemize}
    \item Purpose: Linear algebra (vectors, matrices)
    \item Version: $\ge$ 3.3
    \item Usage: Header-only, no linking required
    \item Why: Fast, expressive, template-based
\end{itemize}

\textbf{Boost} (optional):
\begin{itemize}
    \item Purpose: Extended utilities (filesystem, date\_time)
    \item Version: $\ge$ 1.70
    \item Usage: Some compiled components
    \item Why: Industry-standard C++ extensions
\end{itemize}

\textbf{SPICE} (optional):
\begin{itemize}
    \item Purpose: High-precision planetary ephemerides
    \item Provider: JPL/NAIF
    \item Usage: Compiled library (CSPICE)
    \item Why: Gold standard for ephemeris computation
\end{itemize}

\textbf{Google Test} (testing only):
\begin{itemize}
    \item Purpose: Unit testing framework
    \item Version: $\ge$ 1.10
    \item Usage: Downloaded automatically by CMake if not found
\end{itemize}

\subsection{CMake Build System}

\textbf{Features}:
\begin{itemize}
    \item Modern CMake (3.15+)
    \item Automatic dependency finding
    \item Version generation
    \item Configuration options
    \item Install targets
    \item Package export for use in other projects
\end{itemize}

\textbf{Build options}:
\begin{lstlisting}[language=bash]
cmake -B build \
  -DCMAKE_BUILD_TYPE=Release \
  -DASTDYN_BUILD_SHARED=ON \
  -DASTDYN_BUILD_TESTS=ON \
  -DASTDYN_BUILD_EXAMPLES=ON \
  -DASTDYN_USE_SPICE=ON
cmake --build build -j
cmake --install build
\end{lstlisting}

\section{Error Handling}

\subsection{Strategy}

\textbf{Exceptions} for programming errors:
\begin{lstlisting}[language=C++]
if (eccentricity < 0.0 || eccentricity >= 1.0) {
    throw std::invalid_argument("Eccentricity must be in [0, 1)");
}
\end{lstlisting}

\textbf{Optional} for expected failures:
\begin{lstlisting}[language=C++]
std::optional<Matrix3d> invert_matrix(const Matrix3d& A) {
    if (A.determinant() < 1e-15) {
        return std::nullopt;  // Singular
    }
    return A.inverse();
}
\end{lstlisting}

\textbf{Return codes} for I/O:
\begin{lstlisting}[language=C++]
bool load_observations(const std::string& filename,
                       std::vector<Observation>& obs) {
    std::ifstream file(filename);
    if (!file) return false;
    // ...
    return true;
}
\end{lstlisting}

\subsection{Logging}

\begin{lstlisting}[language=C++]
#include <astdyn/utils/Logger.hpp>

// Severity levels
Logger::debug("Iteration {} converged", iter);
Logger::info("Loaded {} observations", n_obs);
Logger::warning("RMS = {:.3f} arcsec (high!)", rms);
Logger::error("Failed to load kernel: {}", filename);
\end{lstlisting}

\section{Memory Management}

\subsection{Ownership}

\textbf{Stack allocation} for small objects:
\begin{lstlisting}[language=C++]
Vector3d position;  // 24 bytes
Matrix6d covariance;  // 288 bytes
KeplerianElements elements;  // ~80 bytes
\end{lstlisting}

\textbf{Smart pointers} for dynamic lifetime:
\begin{lstlisting}[language=C++]
// Unique ownership
auto propagator = std::make_unique<Propagator>(integrator, forces);

// Shared ownership (when multiple references needed)
auto spice = std::make_shared<SpiceInterface>();
propagator->set_ephemeris(spice);
corrector->set_ephemeris(spice);  // Same object
\end{lstlisting}

\textbf{Move semantics} for efficiency:
\begin{lstlisting}[language=C++]
std::vector<Observation> load_mpc_observations(const std::string& file) {
    std::vector<Observation> obs;
    // ... populate obs ...
    return obs;  // Moved, not copied (C++11 RVO)
}
\end{lstlisting}

\subsection{Large Datasets}

For large observation sets (e.g., 10,000+ observations):
\begin{itemize}
    \item Use \texttt{std::vector::reserve()} to avoid reallocations
    \item Stream processing for files too large for RAM
    \item Memory-mapped files for very large datasets (future)
\end{itemize}

\section{Threading and Parallelism}

\subsection{Current State}

AstDyn v1.0 is single-threaded. Parallelization opportunities:

\begin{enumerate}
    \item \textbf{Observation processing}: Compute partials in parallel
    \item \textbf{Monte Carlo}: Multiple orbit propagations independently
    \item \textbf{Uncertainty propagation}: Parallel particle simulations
\end{enumerate}

\subsection{Future Plans}

\begin{lstlisting}[language=C++]
// OpenMP for loop parallelization
#pragma omp parallel for
for (size_t i = 0; i < observations.size(); ++i) {
    residuals[i] = compute_residual(observations[i], state);
}

// std::async for task parallelism
auto future1 = std::async(std::launch::async, propagate, state1, t_end);
auto future2 = std::async(std::launch::async, propagate, state2, t_end);
auto result1 = future1.get();
auto result2 = future2.get();
\end{lstlisting}

\section{Testing Strategy}

\subsection{Unit Tests}

Google Test framework with fixtures:

\begin{lstlisting}[language=C++]
TEST(TimeScaleTest, UTCtoTT) {
    double jd_utc = 2451545.0;  // J2000.0
    double jd_tt = time::utc_to_tt(jd_utc);
    EXPECT_NEAR(jd_tt - jd_utc, 64.184 / 86400.0, 1e-10);
}

TEST(KeplerianTest, CartesianRoundTrip) {
    CartesianState cart(1.0, 0.0, 0.0, 0.0, 0.0172, 0.0);
    auto kep = KeplerianElements::from_cartesian(cart);
    auto cart2 = kep.to_cartesian();
    EXPECT_VECTOR_NEAR(cart.position, cart2.position, 1e-12);
}
\end{lstlisting}

\subsection{Integration Tests}

\begin{itemize}
    \item Propagate known orbits, compare with JPL Horizons
    \item Differential correction on real asteroids (e.g., Pompeja)
    \item IOD from synthetic observations
\end{itemize}

\subsection{Performance Benchmarks}

\begin{lstlisting}[language=C++]
TEST(PropagationBenchmark, Pompeja60Days) {
    auto start = std::chrono::high_resolution_clock::now();
    
    propagate(initial_state, 0.0, 60.0, forces);
    
    auto end = std::chrono::high_resolution_clock::now();
    auto duration = std::chrono::duration_cast<std::chrono::milliseconds>(end - start);
    
    std::cout << "Propagation time: " << duration.count() << " ms\n";
    EXPECT_LT(duration.count(), 1000);  // Should complete in < 1 second
}
\end{lstlisting}

\section{Documentation}

\subsection{Inline Documentation}

Doxygen-style comments:

\begin{lstlisting}[language=C++]
/**
 * @brief Convert Keplerian elements to Cartesian state
 * 
 * @param elements Keplerian orbital elements (a, e, i, Omega, omega, M)
 * @param mu Gravitational parameter [km^3/s^2]
 * @return CartesianState Position [km] and velocity [km/s]
 * 
 * @note Uses iterative solution of Kepler's equation for eccentric anomaly
 * @throws std::invalid_argument if eccentricity >= 1.0 (parabolic/hyperbolic)
 */
CartesianState to_cartesian(const KeplerianElements& elements, double mu);
\end{lstlisting}

\subsection{External Documentation}

\begin{itemize}
    \item \textbf{README.md}: Quick start, installation, examples
    \item \textbf{This manual}: Theory + implementation
    \item \textbf{API reference}: Generated from Doxygen
    \item \textbf{Examples}: Commented working code
\end{itemize}

\section{Summary}

Key architectural features:

\begin{enumerate}
    \item \textbf{Modular design}: Clear separation of concerns
    \item \textbf{Interface-based}: Easy to extend (parsers, integrators, etc.)
    \item \textbf{Modern C++17}: Smart pointers, move semantics, templates
    \item \textbf{Eigen3 integration}: Efficient linear algebra
    \item \textbf{CMake build}: Cross-platform, automatic dependencies
    \item \textbf{Comprehensive testing}: Unit tests + integration tests
    \item \textbf{Clear error handling}: Exceptions, optionals, return codes
    \item \textbf{Well documented}: Inline + external docs
\end{enumerate}

Next chapter covers individual core modules in detail.

\chapter{Core Modules}
\label{ch:core_modules}

\section{Introduction}

This chapter documents the core modules that implement orbital mechanics algorithms. Each module is designed to be independent yet composable.

\section{Orbital Elements}

\subsection{KeplerianElements}

Classical six Keplerian elements for elliptical orbits.

\begin{lstlisting}[language=C++,caption={KeplerianElements class}]
namespace astdyn {
namespace coordinates {

class KeplerianElements {
public:
    // Elements
    double a;      // Semi-major axis [AU]
    double e;      // Eccentricity [0, 1)
    double i;      // Inclination [rad]
    double Omega;  // Longitude of ascending node [rad]
    double omega;  // Argument of perihelion [rad]
    double M;      // Mean anomaly [rad]
    
    // Epoch
    double epoch;  // Julian date [TDB]
    
    // Construction
    KeplerianElements() = default;
    KeplerianElements(double a, double e, double i,
                     double Omega, double omega, double M,
                     double epoch);
    
    // Conversions
    static KeplerianElements from_cartesian(
        const Vector6d& state, double epoch, double mu = MU_SUN);
    
    Vector6d to_cartesian(double mu = MU_SUN) const;
    
    // Derived quantities
    double period() const;           // Orbital period [days]
    double mean_motion() const;      // Mean motion [rad/day]
    double perihelion_distance() const;  // q [AU]
    double aphelion_distance() const;    // Q [AU]
    double orbital_energy(double mu = MU_SUN) const;
    
    // Mean anomaly at different epoch
    double mean_anomaly_at(double jd) const;
    
    // Validation
    bool is_valid() const;
};

}} // namespace
\end{lstlisting}

\textbf{Usage}:
\begin{lstlisting}[language=C++]
using namespace astdyn::coordinates;

// Create from elements
KeplerianElements elem;
elem.a = 2.77;          // AU
elem.e = 0.075;
elem.i = 10.6 * DEG_TO_RAD;
elem.Omega = 80.3 * DEG_TO_RAD;
elem.omega = 73.6 * DEG_TO_RAD;
elem.M = 0.0;
elem.epoch = 2460000.5;

// Derived quantities
std::cout << "Period: " << elem.period() << " days\n";
std::cout << "q: " << elem.perihelion_distance() << " AU\n";

// Convert to Cartesian
Vector6d state = elem.to_cartesian();
\end{lstlisting}

\subsection{CometaryElements}

Optimized for parabolic and near-parabolic orbits (comets).

\begin{lstlisting}[language=C++]
class CometaryElements {
public:
    double q;      // Perihelion distance [AU]
    double e;      // Eccentricity
    double i;      // Inclination [rad]
    double Omega;  // Longitude of ascending node [rad]
    double omega;  // Argument of perihelion [rad]
    double T;      // Time of perihelion passage [JD]
    double epoch;
    
    Vector6d to_cartesian(double jd, double mu = MU_SUN) const;
    static CometaryElements from_keplerian(const KeplerianElements& kep);
};
\end{lstlisting}

\subsection{CartesianState}

Position and velocity vectors.

\begin{lstlisting}[language=C++]
struct CartesianState {
    Vector3d position;  // [AU]
    Vector3d velocity;  // [AU/day]
    double epoch;       // [JD TDB]
    
    Vector6d as_vector() const {
        Vector6d v;
        v << position, velocity;
        return v;
    }
    
    double distance() const { return position.norm(); }
    double speed() const { return velocity.norm(); }
};
\end{lstlisting}

\section{Force Models}

\subsection{ForceModel Interface}

\begin{lstlisting}[language=C++]
class ForceModel {
public:
    virtual ~ForceModel() = default;
    
    // Compute acceleration [AU/day^2]
    virtual Vector3d acceleration(
        const Vector6d& state,
        double jd_tdb) const = 0;
    
    // Partial derivatives for STM (optional)
    virtual Matrix3d acceleration_partials_position(
        const Vector6d& state,
        double jd_tdb) const {
        return Matrix3d::Zero();
    }
    
    virtual Matrix3d acceleration_partials_velocity(
        const Vector6d& state,
        double jd_tdb) const {
        return Matrix3d::Zero();
    }
};
\end{lstlisting}

\subsection{Point Mass Gravity}

\begin{lstlisting}[language=C++]
class PointMassGravity : public ForceModel {
private:
    std::shared_ptr<IEphemeris> ephemeris_;
    std::vector<Body> bodies_;  // Sun, planets
    
public:
    PointMassGravity(std::shared_ptr<IEphemeris> eph,
                     const std::vector<Body>& bodies)
        : ephemeris_(eph), bodies_(bodies) {}
    
    Vector3d acceleration(const Vector6d& state, double jd) const override {
        Vector3d r_obj = state.head<3>();
        Vector3d acc = Vector3d::Zero();
        
        for (Body body : bodies_) {
            Vector3d r_body = ephemeris_->get_position(body, jd);
            Vector3d d = r_body - r_obj;
            double d_norm = d.norm();
            
            // Direct term
            acc += body.mu * d / (d_norm * d_norm * d_norm);
            
            // Indirect term (if not Sun)
            if (body != Body::SUN) {
                double r_norm = r_body.norm();
                acc -= body.mu * r_body / (r_norm * r_norm * r_norm);
            }
        }
        
        return acc;
    }
};
\end{lstlisting}

\subsection{Combined Force Model}

\begin{lstlisting}[language=C++]
class CombinedForceModel : public ForceModel {
private:
    std::vector<std::shared_ptr<ForceModel>> models_;
    
public:
    void add_model(std::shared_ptr<ForceModel> model) {
        models_.push_back(model);
    }
    
    Vector3d acceleration(const Vector6d& state, double jd) const override {
        Vector3d acc = Vector3d::Zero();
        for (const auto& model : models_) {
            acc += model->acceleration(state, jd);
        }
        return acc;
    }
};
\end{lstlisting}

\section{Numerical Integration}

\subsection{Integrator Interface}

\begin{lstlisting}[language=C++]
class IIntegrator {
public:
    virtual ~IIntegrator() = default;
    
    // Single step
    virtual void step(Vector6d& y, double& t, double dt,
                     const ForceModel& forces) = 0;
    
    // Integrate from t0 to t1
    virtual void integrate(Vector6d& y, double t0, double t1,
                          const ForceModel& forces,
                          double dt_initial = 0.01) = 0;
    
    // Get statistics
    virtual size_t num_steps() const = 0;
    virtual size_t num_function_calls() const = 0;
};
\end{lstlisting}

\subsection{Runge-Kutta-Fehlberg 7(8)}

Adaptive step size, high accuracy.

\begin{lstlisting}[language=C++]
class RKF78 : public IIntegrator {
private:
    double tol_;         // Error tolerance
    double dt_min_;      // Minimum step size
    double dt_max_;      // Maximum step size
    size_t n_steps_;
    size_t n_fcalls_;
    
public:
    RKF78(double tol = 1e-12,
          double dt_min = 1e-6,
          double dt_max = 100.0)
        : tol_(tol), dt_min_(dt_min), dt_max_(dt_max),
          n_steps_(0), n_fcalls_(0) {}
    
    void integrate(Vector6d& y, double t0, double t1,
                  const ForceModel& forces,
                  double dt) override {
        double t = t0;
        double h = dt;
        
        while (t < t1) {
            if (t + h > t1) h = t1 - t;
            
            // RKF78 coefficients and stages (13 stages)
            Vector6d k[13];
            // ... compute stages ...
            
            // 7th and 8th order solutions
            Vector6d y7 = y + h * (/* 7th order combination */);
            Vector6d y8 = y + h * (/* 8th order combination */);
            
            // Error estimate
            double err = (y8 - y7).norm();
            
            // Accept/reject and adapt step
            if (err < tol_) {
                y = y8;
                t += h;
                n_steps_++;
            }
            
            // Update step size
            h *= 0.9 * std::pow(tol_ / err, 1.0/8.0);
            h = std::clamp(h, dt_min_, dt_max_);
            
            n_fcalls_ += 13;
        }
    }
};
\end{lstlisting}

\section{Orbit Propagation}

\subsection{Propagator Class}

High-level interface combining integrator and forces.

\begin{lstlisting}[language=C++]
class Propagator {
private:
    std::shared_ptr<IIntegrator> integrator_;
    std::shared_ptr<ForceModel> forces_;
    std::shared_ptr<IEphemeris> ephemeris_;
    
public:
    Propagator(std::shared_ptr<IIntegrator> integ,
               std::shared_ptr<ForceModel> forces,
               std::shared_ptr<IEphemeris> eph)
        : integrator_(integ), forces_(forces), ephemeris_(eph) {}
    
    // Propagate state
    Vector6d propagate(const Vector6d& y0, double t0, double t1) {
        Vector6d y = y0;
        integrator_->integrate(y, t0, t1, *forces_);
        return y;
    }
    
    // Propagate with STM
    std::pair<Vector6d, Matrix6d> propagate_with_stm(
        const Vector6d& y0, double t0, double t1) {
        
        // Augmented state: [y, Phi(vectorized)]
        VectorXd aug(42);  // 6 + 36
        aug.head<6>() = y0;
        aug.tail<36>() = Matrix6d::Identity().reshaped();
        
        // Integrate variational equations
        integrator_->integrate(aug, t0, t1, *forces_);
        
        Vector6d y = aug.head<6>();
        Matrix6d Phi = Map<Matrix6d>(aug.tail<36>().data());
        
        return {y, Phi};
    }
    
    // Generate ephemeris table
    std::vector<std::pair<double, Vector6d>> 
    generate_ephemeris(const Vector6d& y0, double t0,
                      double t1, double dt) {
        std::vector<std::pair<double, Vector6d>> table;
        Vector6d y = y0;
        double t = t0;
        
        while (t <= t1) {
            table.emplace_back(t, y);
            if (t + dt > t1) dt = t1 - t;
            integrator_->integrate(y, t, t + dt, *forces_);
            t += dt;
        }
        
        return table;
    }
};
\end{lstlisting}

\textbf{Usage Example}:
\begin{lstlisting}[language=C++]
// Setup
auto spice = std::make_shared<SpiceInterface>();
spice->load_kernel("de440.bsp");

auto forces = std::make_shared<PointMassGravity>(
    spice, {Body::SUN, Body::JUPITER, Body::SATURN});

auto integrator = std::make_shared<RKF78>(1e-12);

Propagator prop(integrator, forces, spice);

// Propagate Pompeja for 60 days
Vector6d y0 = /* initial state */;
double t0 = 2460000.5;
double t1 = t0 + 60.0;

Vector6d y_final = prop.propagate(y0, t0, t1);

std::cout << "Final position: " << y_final.head<3>().transpose() << " AU\n";
\end{lstlisting}

\section{Observations}

\subsection{Observation Class}

\begin{lstlisting}[language=C++]
namespace astdyn {
namespace observations {

struct Observation {
    double epoch;        // JD UTC
    double ra;           // Right ascension [rad]
    double dec;          // Declination [rad]
    double sigma_ra;     // RA uncertainty [rad]
    double sigma_dec;    // Dec uncertainty [rad]
    std::string obs_code; // MPC observatory code
    double magnitude;    // Apparent magnitude
    
    // Computed from RA/Dec
    Vector3d line_of_sight() const {
        return Vector3d(
            std::cos(dec) * std::cos(ra),
            std::cos(dec) * std::sin(ra),
            std::sin(dec)
        );
    }
    
    // Weight for least squares
    double weight_ra() const { return 1.0 / (sigma_ra * sigma_ra); }
    double weight_dec() const { return 1.0 / (sigma_dec * sigma_dec); }
};

}} // namespace
\end{lstlisting}

\subsection{MPC Reader}

Parse Minor Planet Center 80-column format.

\begin{lstlisting}[language=C++]
class MPCReader {
public:
    static std::vector<Observation> read_file(const std::string& filename) {
        std::vector<Observation> obs;
        std::ifstream file(filename);
        std::string line;
        
        while (std::getline(file, line)) {
            if (line.length() < 80) continue;
            if (line[14] == 'S' || line[14] == 'X') continue; // Satellite/roving
            
            Observation ob;
            
            // Parse columns (MPC format specification)
            ob.obs_code = line.substr(77, 3);
            
            // Date/time
            int year = std::stoi(line.substr(15, 4));
            int month = std::stoi(line.substr(20, 2));
            double day = std::stod(line.substr(23, 8));
            ob.epoch = date_to_jd(year, month, day);
            
            // RA: HH MM SS.sss
            int ra_h = std::stoi(line.substr(32, 2));
            int ra_m = std::stoi(line.substr(35, 2));
            double ra_s = std::stod(line.substr(38, 5));
            ob.ra = (ra_h + ra_m/60.0 + ra_s/3600.0) * 15.0 * DEG_TO_RAD;
            
            // Dec: +DD MM SS.ss
            char sign = line[44];
            int dec_d = std::stoi(line.substr(45, 2));
            int dec_m = std::stoi(line.substr(48, 2));
            double dec_s = std::stod(line.substr(51, 4));
            ob.dec = (dec_d + dec_m/60.0 + dec_s/3600.0) * DEG_TO_RAD;
            if (sign == '-') ob.dec = -ob.dec;
            
            // Magnitude
            if (line.length() >= 70 && line[65] != ' ') {
                ob.magnitude = std::stod(line.substr(65, 5));
            }
            
            // Default uncertainties (catalog-dependent)
            ob.sigma_ra = 0.5 * ARCSEC_TO_RAD;
            ob.sigma_dec = 0.5 * ARCSEC_TO_RAD;
            
            obs.push_back(ob);
        }
        
        return obs;
    }
};
\end{lstlisting}

\section{Observatory Database}

\subsection{ObservatoryCoordinates}

\begin{lstlisting}[language=C++]
struct ObservatoryCoordinates {
    std::string code;
    double longitude;  // [rad] East positive
    double latitude;   // [rad] geocentric
    double altitude;   // [m] above sea level
    
    // Geocentric position at given time
    Vector3d position_itrf(double jd_utc) const {
        // WGS84 ellipsoid
        const double a = 6378137.0;  // m
        const double f = 1.0 / 298.257223563;
        const double e2 = 2*f - f*f;
        
        double N = a / std::sqrt(1 - e2 * std::sin(latitude) * std::sin(latitude));
        
        double x = (N + altitude) * std::cos(latitude) * std::cos(longitude);
        double y = (N + altitude) * std::cos(latitude) * std::sin(longitude);
        double z = (N * (1 - e2) + altitude) * std::sin(latitude);
        
        return Vector3d(x, y, z) / 1000.0;  // Convert to km
    }
    
    // Rotate to inertial frame
    Vector3d position_icrf(double jd_utc) const {
        Vector3d r_itrf = position_itrf(jd_utc);
        Matrix3d R = earth_rotation_matrix(jd_utc);  // ITRF -> ICRF
        return R * r_itrf / AU;  // Convert to AU
    }
};
\end{lstlisting}

\section{Summary}

Core modules provide:

\begin{enumerate}
    \item \textbf{Orbital Elements}: Keplerian, Cartesian, Cometary representations
    \item \textbf{Force Models}: Extensible interface for perturbations
    \item \textbf{Integrators}: Adaptive step-size RK methods
    \item \textbf{Propagator}: High-level orbit propagation with STM
    \item \textbf{Observations}: Astrometric measurements and MPC parsing
    \item \textbf{Observatories}: Geodetic coordinates and transformations
\end{enumerate}

All modules are designed for composition and extensibility.

\chapter{Parser System}
\label{ch:parsers}

\section{Introduction}

AstDyn supports multiple file formats for orbital elements through a configurable parser system. The design uses the **Strategy Pattern** with a factory for parser creation.

\subsection{Supported Formats}

\begin{itemize}
    \item \textbf{OrbFit .eq1}: Equinoctial elements (legacy format)
    \item \textbf{OrbFit .eq0}: Keplerian elements
    \item \textbf{OrbFit .rwo}: Residuals and weights (future)
    \item \textbf{MPC}: Observations in 80-column format
    \item \textbf{JSON}: Modern structured format (future)
\end{itemize}

\section{Parser Interface}

\subsection{IParser Base Class}

\begin{lstlisting}[language=C++,caption={Parser interface}]
namespace astdyn {
namespace io {

class IParser {
public:
    virtual ~IParser() = default;
    
    // Parse file and return orbital elements
    virtual coordinates::OrbitalElements parse(
        const std::string& filename) = 0;
    
    // Get file format name
    virtual std::string format_name() const = 0;
    
    // Check if file can be parsed by this parser
    virtual bool can_parse(const std::string& filename) const = 0;
};

}} // namespace
\end{lstlisting}

\subsection{Design Benefits}

\begin{enumerate}
    \item \textbf{Extensibility}: Add new formats without modifying existing code
    \item \textbf{Testability}: Each parser tested independently
    \item \textbf{Flexibility}: Runtime format selection
    \item \textbf{Maintainability}: Clear separation of concerns
\end{enumerate}

\section{OrbFit .eq1 Parser}

\subsection{Format Specification}

OrbFit equinoctial elements file (.eq1):

\begin{verbatim}
! Object name
ObjectName
! Epoch (MJD)
58000.0
! Equinoctial elements: h, k, p, q, lambda, a
0.01234
-0.00567
0.08901
-0.12345
2.34567
2.7681234
\end{verbatim}

Equinoctial elements avoid singularities at $e=0$ and $i=0$:
\begin{align}
    h &= e \sin(\omega + \Omega) \\
    k &= e \cos(\omega + \Omega) \\
    p &= \tan(i/2) \sin\Omega \\
    q &= \tan(i/2) \cos\Omega \\
    \lambda &= M + \omega + \Omega \\
    a &= \text{semimajor axis}
\end{align}

\subsection{Implementation}

\begin{lstlisting}[language=C++,caption={OrbFitEQ1Parser implementation}]
namespace astdyn {
namespace io {

class OrbFitEQ1Parser : public IParser {
public:
    coordinates::OrbitalElements parse(const std::string& filename) override {
        std::ifstream file(filename);
        if (!file) {
            throw std::runtime_error("Cannot open file: " + filename);
        }
        
        std::string line;
        
        // Skip comment and read object name
        std::getline(file, line);  // "! Object name"
        std::string object_name;
        std::getline(file, object_name);
        
        // Skip comment and read epoch
        std::getline(file, line);  // "! Epoch (MJD)"
        double mjd;
        file >> mjd;
        double epoch = mjd + 2400000.5;  // Convert to JD
        
        // Skip comment and read equinoctial elements
        std::getline(file, line);  // newline
        std::getline(file, line);  // "! Equinoctial..."
        
        double h, k, p, q, lambda, a;
        file >> h >> k >> p >> q >> lambda >> a;
        
        // Convert equinoctial to Keplerian
        double e = std::sqrt(h*h + k*k);
        double i = 2.0 * std::atan(std::sqrt(p*p + q*q));
        
        double Omega, omega_plus_Omega;
        if (p != 0.0 || q != 0.0) {
            Omega = std::atan2(p, q);
        } else {
            Omega = 0.0;
        }
        
        if (h != 0.0 || k != 0.0) {
            omega_plus_Omega = std::atan2(h, k);
        } else {
            omega_plus_Omega = 0.0;
        }
        
        double omega = omega_plus_Omega - Omega;
        double M = lambda - omega_plus_Omega;
        
        // Normalize angles to [0, 2pi)
        M = math::normalize_angle(M);
        omega = math::normalize_angle(omega);
        Omega = math::normalize_angle(Omega);
        
        // Create Keplerian elements
        coordinates::KeplerianElements elem;
        elem.a = a;
        elem.e = e;
        elem.i = i;
        elem.Omega = Omega;
        elem.omega = omega;
        elem.M = M;
        elem.epoch = epoch;
        elem.name = object_name;
        
        return elem;
    }
    
    std::string format_name() const override {
        return "OrbFit Equinoctial (.eq1)";
    }
    
    bool can_parse(const std::string& filename) const override {
        return filename.ends_with(".eq1");
    }
};

}} // namespace
\end{lstlisting}

\subsection{Usage}

\begin{lstlisting}[language=C++]
#include <astdyn/io/parsers/OrbFitEQ1Parser.hpp>

using namespace astdyn;

io::OrbFitEQ1Parser parser;
auto elements = parser.parse("pompeja.eq1");

std::cout << "Object: " << elements.name << "\n";
std::cout << "Epoch: " << elements.epoch << " JD\n";
std::cout << "a = " << elements.a << " AU\n";
std::cout << "e = " << elements.e << "\n";
\end{lstlisting}

\section{Parser Factory}

\subsection{Factory Pattern}

Automatic parser selection based on file extension.

\begin{lstlisting}[language=C++,caption={ParserFactory class}]
namespace astdyn {
namespace io {

class ParserFactory {
public:
    // Register a parser for specific extensions
    static void register_parser(
        const std::string& extension,
        std::function<std::unique_ptr<IParser>()> creator) {
        
        parsers_[extension] = creator;
    }
    
    // Create parser for given filename
    static std::unique_ptr<IParser> create(const std::string& filename) {
        // Extract extension
        size_t dot = filename.find_last_of('.');
        if (dot == std::string::npos) {
            throw std::invalid_argument("No file extension found");
        }
        
        std::string ext = filename.substr(dot);
        
        // Look up parser
        auto it = parsers_.find(ext);
        if (it == parsers_.end()) {
            throw std::invalid_argument("No parser for extension: " + ext);
        }
        
        return it->second();
    }
    
    // List supported formats
    static std::vector<std::string> supported_formats() {
        std::vector<std::string> formats;
        for (const auto& [ext, _] : parsers_) {
            formats.push_back(ext);
        }
        return formats;
    }
    
private:
    static std::map<std::string, std::function<std::unique_ptr<IParser>()>> parsers_;
};

// Initialize static map
std::map<std::string, std::function<std::unique_ptr<IParser>()>> 
ParserFactory::parsers_ = {
    {".eq1", []() { return std::make_unique<OrbFitEQ1Parser>(); }},
    {".eq0", []() { return std::make_unique<OrbFitEQ0Parser>(); }},
};

}} // namespace
\end{lstlisting}

\subsection{Usage}

\begin{lstlisting}[language=C++]
#include <astdyn/io/ParserFactory.hpp>

using namespace astdyn;

// Automatic parser selection
std::string filename = "asteroid.eq1";
auto parser = io::ParserFactory::create(filename);
auto elements = parser->parse(filename);

// List supported formats
std::cout << "Supported formats:\n";
for (const auto& fmt : io::ParserFactory::supported_formats()) {
    std::cout << "  " << fmt << "\n";
}
\end{lstlisting}

\section{MPC Observation Parser}

\subsection{80-Column Format}

Minor Planet Center standard format (as seen in Chapter 12).

\textbf{Example}:
\begin{verbatim}
     203        C2024 01 15.13542 10 23 24.12 +12 34 05.6         18.2 V      F51
\end{verbatim}

Columns:
\begin{itemize}
    \item 1-5: Object number or provisional designation
    \item 15-32: Observation date (YYYY MM DD.ddddd)
    \item 33-44: RA (HH MM SS.sss)
    \item 45-56: Dec (sDD MM SS.ss)
    \item 66-70: Magnitude
    \item 71: Band
    \item 78-80: Observatory code
\end{itemize}

\subsection{MPCObservationParser}

\begin{lstlisting}[language=C++,caption={MPC observation parser}]
class MPCObservationParser {
public:
    static std::vector<observations::Observation> parse_file(
        const std::string& filename) {
        
        std::vector<observations::Observation> obs;
        std::ifstream file(filename);
        std::string line;
        
        while (std::getline(file, line)) {
            if (line.length() < 80) continue;
            if (line[14] != 'C') continue;  // Skip non-CCD
            
            observations::Observation ob;
            
            // Observatory code
            ob.obs_code = line.substr(77, 3);
            
            // Parse date: YYYY MM DD.ddddd
            int year = std::stoi(line.substr(15, 4));
            int month = std::stoi(line.substr(20, 2));
            double day = std::stod(line.substr(23, 8));
            ob.epoch = date_to_jd(year, month, day);
            
            // Parse RA: HH MM SS.sss
            int ra_h = std::stoi(line.substr(32, 2));
            int ra_m = std::stoi(line.substr(35, 2));
            double ra_s = std::stod(line.substr(38, 5));
            ob.ra = (ra_h * 15.0 + ra_m * 0.25 + ra_s * 0.004166667) * DEG_TO_RAD;
            
            // Parse Dec: sDD MM SS.ss
            char sign = line[44];
            int dec_d = std::stoi(line.substr(45, 2));
            int dec_m = std::stoi(line.substr(48, 2));
            double dec_s = std::stod(line.substr(51, 4));
            ob.dec = (dec_d + dec_m/60.0 + dec_s/3600.0) * DEG_TO_RAD;
            if (sign == '-') ob.dec = -ob.dec;
            
            // Magnitude (optional)
            std::string mag_str = line.substr(65, 5);
            if (!mag_str.empty() && mag_str[0] != ' ') {
                ob.magnitude = std::stod(mag_str);
            }
            
            // Default uncertainties
            ob.sigma_ra = 0.5 * ARCSEC_TO_RAD;   // ~0.5"
            ob.sigma_dec = 0.5 * ARCSEC_TO_RAD;
            
            obs.push_back(ob);
        }
        
        return obs;
    }
};
\end{lstlisting}

\section{Adding New Parsers}

\subsection{Steps}

\begin{enumerate}
    \item Create class inheriting from \texttt{IParser}
    \item Implement \texttt{parse()}, \texttt{format\_name()}, \texttt{can\_parse()}
    \item Register with \texttt{ParserFactory}
    \item Add unit tests
    \item Update documentation
\end{enumerate}

\subsection{Example: JSON Parser}

\begin{lstlisting}[language=C++,caption={JSON parser skeleton}]
#include <nlohmann/json.hpp>

class JSONParser : public IParser {
public:
    coordinates::OrbitalElements parse(const std::string& filename) override {
        std::ifstream file(filename);
        nlohmann::json j;
        file >> j;
        
        coordinates::KeplerianElements elem;
        elem.a = j["semimajor_axis"];
        elem.e = j["eccentricity"];
        elem.i = j["inclination"] * DEG_TO_RAD;
        elem.Omega = j["ascending_node"] * DEG_TO_RAD;
        elem.omega = j["argument_perihelion"] * DEG_TO_RAD;
        elem.M = j["mean_anomaly"] * DEG_TO_RAD;
        elem.epoch = j["epoch"];
        elem.name = j["object_name"];
        
        return elem;
    }
    
    std::string format_name() const override {
        return "JSON Orbital Elements";
    }
    
    bool can_parse(const std::string& filename) const override {
        return filename.ends_with(".json");
    }
};

// Register with factory
void register_json_parser() {
    ParserFactory::register_parser(".json", 
        []() { return std::make_unique<JSONParser>(); });
}
\end{lstlisting}

\section{Configuration File Parser}

\subsection{AstDynConfig}

Parse runtime configuration from file.

\begin{lstlisting}[language=C++]
struct AstDynConfig {
    // Integrator settings
    std::string integrator_type = "RKF78";
    double tolerance = 1e-12;
    double min_step = 1e-6;
    double max_step = 100.0;
    
    // Force model
    std::vector<std::string> perturbations = {"SUN", "JUPITER", "SATURN"};
    bool include_relativity = false;
    bool include_j2 = false;
    
    // Differential correction
    int max_iterations = 20;
    double convergence_tol = 1e-8;
    bool enable_robust_weighting = false;
    
    // Ephemeris
    std::string spice_kernel = "de440.bsp";
    
    // Parse from file
    static AstDynConfig from_file(const std::string& filename);
    
    // Save to file
    void to_file(const std::string& filename) const;
};
\end{lstlisting}

\section{Error Handling}

\subsection{Common Parse Errors}

\begin{lstlisting}[language=C++]
try {
    auto parser = ParserFactory::create("data.eq1");
    auto elements = parser->parse("data.eq1");
    
} catch (const std::invalid_argument& e) {
    std::cerr << "Invalid file format: " << e.what() << "\n";
    
} catch (const std::runtime_error& e) {
    std::cerr << "Parse error: " << e.what() << "\n";
    
} catch (const std::exception& e) {
    std::cerr << "Unexpected error: " << e.what() << "\n";
}
\end{lstlisting}

\subsection{Validation}

\begin{lstlisting}[language=C++]
auto elements = parser->parse(filename);

// Validate parsed elements
if (!elements.is_valid()) {
    std::cerr << "Warning: Invalid orbital elements\n";
    
    if (elements.e < 0 || elements.e >= 1) {
        std::cerr << "  Eccentricity out of range: " << elements.e << "\n";
    }
    
    if (elements.a <= 0) {
        std::cerr << "  Negative semimajor axis: " << elements.a << "\n";
    }
}
\end{lstlisting}

\section{Testing}

\subsection{Unit Tests}

\begin{lstlisting}[language=C++]
TEST(ParserTest, OrbFitEQ1_ValidFile) {
    io::OrbFitEQ1Parser parser;
    auto elem = parser.parse("test_data/pompeja.eq1");
    
    EXPECT_EQ(elem.name, "Pompeja");
    EXPECT_NEAR(elem.a, 2.7436, 1e-4);
    EXPECT_NEAR(elem.e, 0.0624, 1e-4);
    EXPECT_NEAR(elem.i * RAD_TO_DEG, 11.74, 0.01);
}

TEST(ParserTest, Factory_AutoSelect) {
    auto parser = io::ParserFactory::create("test.eq1");
    EXPECT_EQ(parser->format_name(), "OrbFit Equinoctial (.eq1)");
}

TEST(MPCTest, ParseObservations) {
    auto obs = io::MPCObservationParser::parse_file("pompeja.obs");
    EXPECT_GT(obs.size(), 0);
    EXPECT_EQ(obs[0].obs_code, "F51");  // Pan-STARRS
}
\end{lstlisting}

\section{Summary}

Parser system features:

\begin{enumerate}
    \item \textbf{Interface-based design}: \texttt{IParser} base class
    \item \textbf{Factory pattern}: Automatic format selection
    \item \textbf{Extensibility}: Easy to add new formats
    \item \textbf{Multiple formats}: OrbFit, MPC, future JSON
    \item \textbf{Robust error handling}: Validation and exceptions
    \item \textbf{Well tested}: Unit tests for each parser
\end{enumerate}

The system successfully separates format-specific code from core algorithms.

\chapter{API Reference}
\label{ch:api_reference}

\section{Overview}

This chapter provides comprehensive reference documentation for AstDyn's public API. All classes, methods, and functions are documented with parameters, return values, and exceptions.

\subsection{Organization}

API organized by namespace:

\begin{itemize}
    \item \texttt{astdyn::constants}: Physical and astronomical constants
    \item \texttt{astdyn::math}: Mathematical utilities
    \item \texttt{astdyn::time}: Time systems and conversions
    \item \texttt{astdyn::coordinates}: Coordinate systems and transformations
    \item \texttt{astdyn::orbit}: Orbital elements classes
    \item \texttt{astdyn::propagation}: Orbit propagation
    \item \texttt{astdyn::observations}: Observation handling
    \item \texttt{astdyn::orbit\_determination}: Orbit determination algorithms
    \item \texttt{astdyn::io}: Input/output and parsers
    \item \texttt{astdyn::ephemeris}: Ephemeris interfaces
\end{itemize}

\section{Core Constants}

\subsection{astdyn::constants}

\begin{lstlisting}[language=C++]
namespace astdyn {
namespace constants {

// Fundamental constants
constexpr double C = 299792458.0;              // Speed of light (m/s)
constexpr double G = 6.67430e-11;             // Gravitational constant (SI)
constexpr double AU = 1.495978707e11;         // Astronomical unit (m)

// Time constants
constexpr double JD_J2000 = 2451545.0;        // J2000.0 epoch
constexpr double DAYS_PER_CENTURY = 36525.0;  // Julian century
constexpr double SECONDS_PER_DAY = 86400.0;   // Seconds in day

// Angle conversions
constexpr double DEG_TO_RAD = M_PI / 180.0;
constexpr double RAD_TO_DEG = 180.0 / M_PI;
constexpr double ARCSEC_TO_RAD = DEG_TO_RAD / 3600.0;
constexpr double RAD_TO_ARCSEC = 3600.0 * RAD_TO_DEG;

// Solar system masses (GM values in AU^3/day^2)
constexpr double GM_SUN = 0.2959122082855911e-3;
constexpr double GM_MERCURY = 0.4912547451450812e-10;
constexpr double GM_VENUS = 0.7243452486162703e-9;
constexpr double GM_EARTH = 0.8887692390113509e-9;
constexpr double GM_MARS = 0.9549535105779258e-10;
constexpr double GM_JUPITER = 0.2825345909524226e-6;
constexpr double GM_SATURN = 0.8459715185680659e-7;
constexpr double GM_URANUS = 0.1292024916781969e-7;
constexpr double GM_NEPTUNE = 0.1524358900784276e-7;

}} // namespace
\end{lstlisting}

\section{Mathematical Utilities}

\subsection{astdyn::math}

\subsubsection{normalize\_angle}

\begin{lstlisting}[language=C++]
double normalize_angle(double angle, double center = 0.0);
\end{lstlisting}

\textbf{Purpose}: Normalize angle to range $[\text{center} - \pi, \text{center} + \pi)$.

\textbf{Parameters}:
\begin{itemize}
    \item \texttt{angle}: Input angle (radians)
    \item \texttt{center}: Center of range (default 0.0)
\end{itemize}

\textbf{Returns}: Normalized angle

\textbf{Example}:
\begin{lstlisting}[language=C++]
double angle = 7.0;  // > 2*pi
double norm = math::normalize_angle(angle);  // Returns 0.716...
\end{lstlisting}

\subsubsection{cross\_product}

\begin{lstlisting}[language=C++]
Vector3d cross_product(const Vector3d& a, const Vector3d& b);
\end{lstlisting}

\textbf{Purpose}: Compute vector cross product $\mathbf{a} \times \mathbf{b}$.

\textbf{Parameters}:
\begin{itemize}
    \item \texttt{a}: First vector
    \item \texttt{b}: Second vector
\end{itemize}

\textbf{Returns}: Cross product vector

\subsubsection{rotation\_matrix}

\begin{lstlisting}[language=C++]
Matrix3d rotation_matrix(double angle, int axis);
\end{lstlisting}

\textbf{Purpose}: Create rotation matrix around coordinate axis.

\textbf{Parameters}:
\begin{itemize}
    \item \texttt{angle}: Rotation angle (radians)
    \item \texttt{axis}: Axis index (0=x, 1=y, 2=z)
\end{itemize}

\textbf{Returns}: $3 \times 3$ rotation matrix

\section{Time Systems}

\subsection{astdyn::time::TimeConverter}

\subsubsection{utc\_to\_tt}

\begin{lstlisting}[language=C++]
static double utc_to_tt(double jd_utc);
\end{lstlisting}

\textbf{Purpose}: Convert UTC to Terrestrial Time (TT).

\textbf{Parameters}:
\begin{itemize}
    \item \texttt{jd\_utc}: Julian Date in UTC
\end{itemize}

\textbf{Returns}: Julian Date in TT

\textbf{Note}: Applies leap seconds and 32.184s TT-TAI offset.

\subsubsection{tt\_to\_tdb}

\begin{lstlisting}[language=C++]
static double tt_to_tdb(double jd_tt);
\end{lstlisting}

\textbf{Purpose}: Convert Terrestrial Time to Barycentric Dynamical Time.

\textbf{Parameters}:
\begin{itemize}
    \item \texttt{jd\_tt}: Julian Date in TT
\end{itemize}

\textbf{Returns}: Julian Date in TDB

\textbf{Note}: Uses periodic approximation with $\pm 2$ms accuracy.

\section{Orbital Elements}

\subsection{astdyn::orbit::KeplerianElements}

\subsubsection{Class Definition}

\begin{lstlisting}[language=C++]
class KeplerianElements {
public:
    double a;         // Semimajor axis (AU)
    double e;         // Eccentricity
    double i;         // Inclination (rad)
    double Omega;     // Ascending node (rad)
    double omega;     // Argument of perihelion (rad)
    double M;         // Mean anomaly (rad)
    double epoch;     // Epoch (JD)
    std::string name; // Object name
    
    // Constructors
    KeplerianElements();
    KeplerianElements(double a, double e, double i, 
                     double Omega, double omega, double M, 
                     double epoch);
    
    // Conversions
    static KeplerianElements from_cartesian(
        const Vector3d& pos, const Vector3d& vel, 
        double epoch, double mu = GM_SUN);
    
    CartesianState to_cartesian(double mu = GM_SUN) const;
    
    // Derived quantities
    double period() const;              // Orbital period (days)
    double mean_motion() const;         // Mean motion (rad/day)
    double perihelion_distance() const; // q = a(1-e)
    double aphelion_distance() const;   // Q = a(1+e)
    double eccentric_anomaly() const;   // E from M
    double true_anomaly() const;        // f from E
    
    // Validation
    bool is_valid() const;
};
\end{lstlisting}

\subsubsection{from\_cartesian}

\begin{lstlisting}[language=C++]
static KeplerianElements from_cartesian(
    const Vector3d& pos, const Vector3d& vel, 
    double epoch, double mu = GM_SUN);
\end{lstlisting}

\textbf{Purpose}: Convert Cartesian state to Keplerian elements.

\textbf{Parameters}:
\begin{itemize}
    \item \texttt{pos}: Position vector (AU)
    \item \texttt{vel}: Velocity vector (AU/day)
    \item \texttt{epoch}: Epoch (JD)
    \item \texttt{mu}: Gravitational parameter (default: GM\_SUN)
\end{itemize}

\textbf{Returns}: Keplerian elements

\textbf{Example}:
\begin{lstlisting}[language=C++]
Vector3d pos(1.0, 0.0, 0.0);  // 1 AU on x-axis
Vector3d vel(0.0, 0.01720209895, 0.0);  // Circular velocity
auto elem = KeplerianElements::from_cartesian(pos, vel, 2460000.0);
// elem.a ~ 1.0 AU, elem.e ~ 0.0
\end{lstlisting}

\subsubsection{to\_cartesian}

\begin{lstlisting}[language=C++]
CartesianState to_cartesian(double mu = GM_SUN) const;
\end{lstlisting}

\textbf{Purpose}: Convert Keplerian elements to Cartesian state.

\textbf{Parameters}:
\begin{itemize}
    \item \texttt{mu}: Gravitational parameter (default: GM\_SUN)
\end{itemize}

\textbf{Returns}: \texttt{CartesianState} with position, velocity, epoch

\textbf{Algorithm}: Uses perifocal frame transformation and rotation matrices.

\subsubsection{period}

\begin{lstlisting}[language=C++]
double period() const;
\end{lstlisting}

\textbf{Purpose}: Compute orbital period using Kepler's third law.

\textbf{Returns}: Period in days

\textbf{Formula}: $P = 2\pi \sqrt{a^3/\mu}$

\textbf{Throws}: \texttt{std::domain\_error} if $a \leq 0$ or $e \geq 1$

\subsection{astdyn::orbit::CometaryElements}

\begin{lstlisting}[language=C++]
class CometaryElements {
public:
    double q;      // Perihelion distance (AU)
    double e;      // Eccentricity
    double i;      // Inclination (rad)
    double Omega;  // Ascending node (rad)
    double omega;  // Argument of perihelion (rad)
    double T;      // Time of perihelion passage (JD)
    std::string name;
    
    // Conversions
    KeplerianElements to_keplerian(double epoch) const;
    CartesianState to_cartesian(double epoch, double mu = GM_SUN) const;
};
\end{lstlisting}

\textbf{Use Case}: Preferred for near-parabolic orbits ($e \approx 1$) where semimajor axis is ill-defined.

\section{Force Models}

\subsection{astdyn::propagation::ForceModel}

\subsubsection{Interface}

\begin{lstlisting}[language=C++]
class ForceModel {
public:
    virtual ~ForceModel() = default;
    
    // Compute acceleration at given state
    virtual Vector3d acceleration(
        double t, const Vector3d& pos, const Vector3d& vel) const = 0;
    
    // Optional: Compute partials for STM propagation
    virtual bool supports_partials() const { return false; }
    
    virtual std::pair<Matrix3d, Matrix3d> partials(
        double t, const Vector3d& pos, const Vector3d& vel) const {
        throw std::logic_error("Partials not implemented");
    }
};
\end{lstlisting}

\subsection{astdyn::propagation::PointMassGravity}

\begin{lstlisting}[language=C++]
class PointMassGravity : public ForceModel {
public:
    PointMassGravity(std::shared_ptr<ephemeris::IEphemeris> eph,
                     const std::vector<std::string>& bodies);
    
    Vector3d acceleration(double t, const Vector3d& pos, 
                         const Vector3d& vel) const override;
    
    bool supports_partials() const override { return true; }
    
    std::pair<Matrix3d, Matrix3d> partials(double t, 
        const Vector3d& pos, const Vector3d& vel) const override;
};
\end{lstlisting}

\textbf{Purpose}: N-body point-mass gravitational perturbations.

\textbf{Constructor Parameters}:
\begin{itemize}
    \item \texttt{eph}: Ephemeris provider for planetary positions
    \item \texttt{bodies}: List of perturbing bodies (e.g., \{"JUPITER", "SATURN"\})
\end{itemize}

\textbf{Algorithm}: Computes direct and indirect terms for each body.

\section{Numerical Integration}

\subsection{astdyn::propagation::IIntegrator}

\subsubsection{Interface}

\begin{lstlisting}[language=C++]
class IIntegrator {
public:
    virtual ~IIntegrator() = default;
    
    // Single integration step
    virtual double step(double t, std::vector<double>& y, 
                       const std::function<void(double, const std::vector<double>&, 
                                              std::vector<double>&)>& derivs) = 0;
    
    // Integrate from t0 to tf
    virtual void integrate(double t0, double tf, std::vector<double>& y,
                          const std::function<void(double, const std::vector<double>&,
                                                 std::vector<double>&)>& derivs) = 0;
    
    // Statistics
    virtual size_t steps_taken() const = 0;
    virtual size_t steps_rejected() const = 0;
};
\end{lstlisting}

\subsection{astdyn::propagation::RKF78}

\begin{lstlisting}[language=C++]
class RKF78 : public IIntegrator {
public:
    RKF78(double tol = 1e-12, double h_min = 1e-6, double h_max = 100.0);
    
    double step(double t, std::vector<double>& y,
               const std::function<void(double, const std::vector<double>&,
                                      std::vector<double>&)>& derivs) override;
    
    void integrate(double t0, double tf, std::vector<double>& y,
                  const std::function<void(double, const std::vector<double>&,
                                        std::vector<double>&)>& derivs) override;
    
    // Setters
    void set_tolerance(double tol);
    void set_step_limits(double h_min, double h_max);
    
    // Getters
    double get_tolerance() const;
    size_t steps_taken() const override;
    size_t steps_rejected() const override;
};
\end{lstlisting}

\textbf{Purpose}: Adaptive Runge-Kutta-Fehlberg 7(8) integrator.

\textbf{Constructor Parameters}:
\begin{itemize}
    \item \texttt{tol}: Error tolerance (default: $10^{-12}$)
    \item \texttt{h\_min}: Minimum step size (days, default: $10^{-6}$)
    \item \texttt{h\_max}: Maximum step size (days, default: 100.0)
\end{itemize}

\textbf{Method}: 13-stage Runge-Kutta with 7th and 8th order estimates for error control.

\section{Orbit Propagation}

\subsection{astdyn::propagation::Propagator}

\subsubsection{Constructor}

\begin{lstlisting}[language=C++]
Propagator(std::shared_ptr<IIntegrator> integrator,
          std::shared_ptr<ForceModel> forces,
          std::shared_ptr<ephemeris::IEphemeris> ephemeris);
\end{lstlisting}

\textbf{Parameters}:
\begin{itemize}
    \item \texttt{integrator}: Numerical integrator (e.g., RKF78)
    \item \texttt{forces}: Force model (e.g., PointMassGravity)
    \item \texttt{ephemeris}: Ephemeris provider (e.g., SPICEEphemeris)
\end{itemize}

\subsubsection{propagate}

\begin{lstlisting}[language=C++]
CartesianState propagate(const CartesianState& initial,
                         double target_epoch);
\end{lstlisting}

\textbf{Purpose}: Propagate state to target epoch.

\textbf{Parameters}:
\begin{itemize}
    \item \texttt{initial}: Initial Cartesian state
    \item \texttt{target\_epoch}: Target epoch (JD)
\end{itemize}

\textbf{Returns}: State at target epoch

\textbf{Example}:
\begin{lstlisting}[language=C++]
auto integrator = std::make_shared<RKF78>(1e-12);
auto forces = std::make_shared<PointMassGravity>(eph, 
    std::vector<std::string>{"JUPITER", "SATURN"});
Propagator prop(integrator, forces, eph);

CartesianState state0 = elements.to_cartesian();
CartesianState state1 = prop.propagate(state0, state0.epoch + 60.0);
\end{lstlisting}

\subsubsection{propagate\_with\_stm}

\begin{lstlisting}[language=C++]
std::pair<CartesianState, Matrix6d> propagate_with_stm(
    const CartesianState& initial, double target_epoch);
\end{lstlisting}

\textbf{Purpose}: Propagate state and state transition matrix.

\textbf{Returns}: Pair of (final state, $6 \times 6$ STM $\Phi(t_f, t_0)$)

\textbf{Note}: STM maps initial state perturbations to final state:
\[
\delta \mathbf{x}(t_f) = \Phi(t_f, t_0) \, \delta \mathbf{x}(t_0)
\]

\subsubsection{generate\_ephemeris}

\begin{lstlisting}[language=C++]
std::vector<CartesianState> generate_ephemeris(
    const CartesianState& initial, 
    double start_epoch, double end_epoch, double step);
\end{lstlisting}

\textbf{Purpose}: Generate table of states at regular intervals.

\textbf{Parameters}:
\begin{itemize}
    \item \texttt{initial}: Initial state
    \item \texttt{start\_epoch}: First epoch (JD)
    \item \texttt{end\_epoch}: Last epoch (JD)
    \item \texttt{step}: Time step (days)
\end{itemize}

\textbf{Returns}: Vector of states

\section{Observations}

\subsection{astdyn::observations::Observation}

\begin{lstlisting}[language=C++]
struct Observation {
    double epoch;        // JD
    double ra;           // Right ascension (rad)
    double dec;          // Declination (rad)
    double sigma_ra;     // RA uncertainty (rad)
    double sigma_dec;    // Dec uncertainty (rad)
    std::string obs_code; // Observatory code
    double magnitude;    // Apparent magnitude
    
    // Methods
    Vector3d line_of_sight() const;
    double weight_ra() const;
    double weight_dec() const;
};
\end{lstlisting}

\subsubsection{line\_of\_sight}

\begin{lstlisting}[language=C++]
Vector3d line_of_sight() const;
\end{lstlisting}

\textbf{Purpose}: Compute unit vector in direction of observation.

\textbf{Returns}: Unit vector $(\cos\delta\cos\alpha, \cos\delta\sin\alpha, \sin\delta)$

\subsubsection{weight\_ra/weight\_dec}

\begin{lstlisting}[language=C++]
double weight_ra() const;
double weight_dec() const;
\end{lstlisting}

\textbf{Purpose}: Compute observation weights for least squares.

\textbf{Returns}: $w = 1/\sigma^2$

\subsection{astdyn::observations::ObservatoryCoordinates}

\begin{lstlisting}[language=C++]
struct ObservatoryCoordinates {
    std::string code;   // MPC observatory code
    double longitude;   // Geodetic longitude (rad)
    double latitude;    // Geodetic latitude (rad)
    double altitude;    // Altitude above ellipsoid (m)
    
    Vector3d position_itrf() const;
    Vector3d position_icrf(double jd) const;
};
\end{lstlisting}

\textbf{Purpose}: Convert observatory location to observer position vectors.

\section{Orbit Determination}

\subsection{astdyn::orbit\_determination::DifferentialCorrector}

\begin{lstlisting}[language=C++]
class DifferentialCorrector {
public:
    DifferentialCorrector(std::shared_ptr<Propagator> propagator,
                         int max_iterations = 20,
                         double convergence_tol = 1e-8);
    
    struct Result {
        orbit::KeplerianElements elements;
        double rms_residual;
        int iterations;
        bool converged;
        Eigen::VectorXd residuals;
        Eigen::MatrixXd covariance;
    };
    
    Result solve(const orbit::KeplerianElements& initial_guess,
                const std::vector<observations::Observation>& observations,
                const std::vector<observations::ObservatoryCoordinates>& observatories);
};
\end{lstlisting}

\subsubsection{solve}

\textbf{Purpose}: Refine orbital elements to minimize observation residuals.

\textbf{Parameters}:
\begin{itemize}
    \item \texttt{initial\_guess}: Initial orbital elements
    \item \texttt{observations}: Vector of observations
    \item \texttt{observatories}: Observatory coordinates
\end{itemize}

\textbf{Returns}: \texttt{Result} structure with:
\begin{itemize}
    \item \texttt{elements}: Refined orbital elements
    \item \texttt{rms\_residual}: RMS of residuals (arcseconds)
    \item \texttt{iterations}: Number of iterations
    \item \texttt{converged}: Convergence flag
    \item \texttt{residuals}: Full residual vector
    \item \texttt{covariance}: Parameter covariance matrix
\end{itemize}

\textbf{Algorithm}: Iterative least squares with numerical derivatives via STM.

\section{Input/Output}

\subsection{astdyn::io::ParserFactory}

\begin{lstlisting}[language=C++]
class ParserFactory {
public:
    static void register_parser(const std::string& extension,
        std::function<std::unique_ptr<IParser>()> creator);
    
    static std::unique_ptr<IParser> create(const std::string& filename);
    
    static std::vector<std::string> supported_formats();
};
\end{lstlisting}

\textbf{Usage}:
\begin{lstlisting}[language=C++]
auto parser = io::ParserFactory::create("asteroid.eq1");
auto elements = parser->parse("asteroid.eq1");
\end{lstlisting}

\subsection{astdyn::io::MPCReader}

\begin{lstlisting}[language=C++]
class MPCReader {
public:
    static std::vector<observations::Observation> read_file(
        const std::string& filename);
    
    static observations::Observation parse_line(const std::string& line);
};
\end{lstlisting}

\textbf{Purpose}: Parse MPC 80-column observation format.

\section{Ephemeris}

\subsection{astdyn::ephemeris::IEphemeris}

\begin{lstlisting}[language=C++]
class IEphemeris {
public:
    virtual ~IEphemeris() = default;
    
    virtual Vector3d position(const std::string& body, double jd) const = 0;
    virtual Vector3d velocity(const std::string& body, double jd) const = 0;
    
    virtual std::pair<Vector3d, Vector3d> state(
        const std::string& body, double jd) const = 0;
};
\end{lstlisting}

\textbf{Purpose}: Interface for planetary ephemerides.

\textbf{Implementations}:
\begin{itemize}
    \item \texttt{SPICEEphemeris}: NASA SPICE toolkit (high accuracy)
    \item \texttt{AnalyticEphemeris}: Approximate analytic formulas (fast)
\end{itemize}

\section{Exception Hierarchy}

\begin{lstlisting}[language=C++]
namespace astdyn {

// Base exception
class AstDynException : public std::runtime_error {
    using std::runtime_error::runtime_error;
};

// Specific exceptions
class ParseError : public AstDynException {
    using AstDynException::AstDynException;
};

class PropagationError : public AstDynException {
    using AstDynException::AstDynException;
};

class ConvergenceError : public AstDynException {
    using AstDynException::AstDynException;
};

} // namespace
\end{lstlisting}

\section{Type Aliases}

\begin{lstlisting}[language=C++]
namespace astdyn {

// Eigen types
using Vector3d = Eigen::Vector3d;
using Vector6d = Eigen::Matrix<double, 6, 1>;
using Matrix3d = Eigen::Matrix3d;
using Matrix6d = Eigen::Matrix<double, 6, 6>;
using MatrixXd = Eigen::MatrixXd;
using VectorXd = Eigen::VectorXd;

// Enumerations
enum class CoordinateSystem { ICRF, ECLIPTIC, EQUATORIAL };
enum class TimeScale { UTC, TT, TDB };
enum class IntegratorType { RKF78, DOPRI853, RADAU };

} // namespace
\end{lstlisting}

\section{Common Usage Patterns}

\subsection{Complete Orbit Propagation}

\begin{lstlisting}[language=C++]
#include <astdyn/AstDyn.hpp>
using namespace astdyn;

// 1. Parse orbital elements
auto parser = io::ParserFactory::create("asteroid.eq1");
auto elements = parser->parse("asteroid.eq1");

// 2. Setup ephemeris
auto eph = std::make_shared<ephemeris::SPICEEphemeris>("de440.bsp");

// 3. Create force model
auto forces = std::make_shared<propagation::PointMassGravity>(
    eph, std::vector<std::string>{"JUPITER", "SATURN", "EARTH"});

// 4. Create integrator
auto integrator = std::make_shared<propagation::RKF78>(1e-12);

// 5. Create propagator
propagation::Propagator prop(integrator, forces, eph);

// 6. Propagate
auto state0 = elements.to_cartesian();
auto state60 = prop.propagate(state0, state0.epoch + 60.0);

// 7. Convert back
auto elem60 = orbit::KeplerianElements::from_cartesian(
    state60.position, state60.velocity, state60.epoch);
\end{lstlisting}

\subsection{Orbit Determination Workflow}

\begin{lstlisting}[language=C++]
// 1. Load observations
auto observations = io::MPCReader::read_file("observations.txt");

// 2. Load observatory coordinates
std::vector<observations::ObservatoryCoordinates> obs_coords;
// ... populate from database

// 3. Initial orbit determination (Gauss method)
orbit_determination::GaussIOD gauss;
auto initial_elements = gauss.solve(observations[0], 
    observations[observations.size()/2], observations.back());

// 4. Setup propagator (same as above)
// ...

// 5. Differential correction
orbit_determination::DifferentialCorrector dc(propagator);
auto result = dc.solve(initial_elements, observations, obs_coords);

if (result.converged) {
    std::cout << "Converged in " << result.iterations << " iterations\n";
    std::cout << "RMS residual: " << result.rms_residual << " arcsec\n";
    std::cout << "Final elements:\n";
    std::cout << "  a = " << result.elements.a << " AU\n";
    std::cout << "  e = " << result.elements.e << "\n";
}
\end{lstlisting}

\section{Summary}

This chapter documented:

\begin{enumerate}
    \item \textbf{Constants and utilities}: Physical constants, math functions, time conversions
    \item \textbf{Orbital elements}: KeplerianElements, CometaryElements, conversions
    \item \textbf{Force models}: ForceModel interface, PointMassGravity implementation
    \item \textbf{Integration}: IIntegrator interface, RKF78 adaptive integrator
    \item \textbf{Propagation}: Propagator class with STM support
    \item \textbf{Observations}: Observation struct, ObservatoryCoordinates
    \item \textbf{Orbit determination}: DifferentialCorrector algorithm
    \item \textbf{I/O}: Parsers for OrbFit and MPC formats
    \item \textbf{Ephemeris}: IEphemeris interface for planetary positions
    \item \textbf{Common patterns}: Complete workflow examples
\end{enumerate}

For detailed examples and tutorials, see Chapter 20.

\chapter{Examples and Tutorials}
\label{ch:examples}

\section{Introduction}

This chapter provides step-by-step tutorials demonstrating AstDyn's capabilities. All examples include complete, working code.

\subsection{Prerequisites}

Ensure AstDyn is installed and configured:

\begin{lstlisting}[language=bash]
# Build AstDyn
mkdir build && cd build
cmake .. -DCMAKE_BUILD_TYPE=Release
make -j4

# Set library path
export LD_LIBRARY_PATH=/path/to/astdyn/lib:$LD_LIBRARY_PATH
\end{lstlisting}

\section{Example 1: Basic Orbit Propagation}

\subsection{Goal}

Propagate asteroid (203) Pompeja for 60 days using simplified force model.

\subsection{Code}

\begin{lstlisting}[language=C++,caption={example1\_propagation.cpp}]
#include <astdyn/AstDyn.hpp>
#include <iostream>
#include <iomanip>

using namespace astdyn;

int main() {
    // Define initial Keplerian elements (Pompeja at epoch JD 2460000.5)
    orbit::KeplerianElements elem0;
    elem0.a = 2.7436;                    // AU
    elem0.e = 0.0624;
    elem0.i = 11.74 * constants::DEG_TO_RAD;
    elem0.Omega = 339.86 * constants::DEG_TO_RAD;
    elem0.omega = 258.03 * constants::DEG_TO_RAD;
    elem0.M = 45.32 * constants::DEG_TO_RAD;
    elem0.epoch = 2460000.5;
    elem0.name = "Pompeja";
    
    std::cout << "Initial Elements (Epoch " << std::fixed 
              << std::setprecision(1) << elem0.epoch << " JD):\n";
    std::cout << "  a     = " << std::setprecision(6) << elem0.a << " AU\n";
    std::cout << "  e     = " << elem0.e << "\n";
    std::cout << "  i     = " << std::setprecision(2) 
              << elem0.i * constants::RAD_TO_DEG << " deg\n";
    std::cout << "  Omega = " << elem0.Omega * constants::RAD_TO_DEG << " deg\n";
    std::cout << "  omega = " << elem0.omega * constants::RAD_TO_DEG << " deg\n";
    std::cout << "  M     = " << elem0.M * constants::RAD_TO_DEG << " deg\n";
    std::cout << "  Period = " << std::setprecision(1) 
              << elem0.period() << " days\n\n";
    
    // Convert to Cartesian
    auto state0 = elem0.to_cartesian();
    std::cout << "Cartesian State:\n";
    std::cout << "  Position: [" << std::setprecision(8)
              << state0.position[0] << ", "
              << state0.position[1] << ", "
              << state0.position[2] << "] AU\n";
    std::cout << "  Velocity: [" 
              << state0.velocity[0] << ", "
              << state0.velocity[1] << ", "
              << state0.velocity[2] << "] AU/day\n\n";
    
    // Setup ephemeris (analytic approximation for simplicity)
    auto eph = std::make_shared<ephemeris::AnalyticEphemeris>();
    
    // Create force model (Sun + Jupiter + Saturn)
    auto forces = std::make_shared<propagation::PointMassGravity>(
        eph, std::vector<std::string>{"JUPITER", "SATURN"});
    
    // Create integrator (RKF78 with 1e-12 tolerance)
    auto integrator = std::make_shared<propagation::RKF78>(1e-12);
    
    // Create propagator
    propagation::Propagator prop(integrator, forces, eph);
    
    // Propagate 60 days
    double target_epoch = elem0.epoch + 60.0;
    std::cout << "Propagating to " << target_epoch << " JD (+60 days)...\n\n";
    
    auto state60 = prop.propagate(state0, target_epoch);
    
    // Convert back to Keplerian
    auto elem60 = orbit::KeplerianElements::from_cartesian(
        state60.position, state60.velocity, state60.epoch);
    
    std::cout << "Final Elements (Epoch " << elem60.epoch << " JD):\n";
    std::cout << "  a     = " << std::setprecision(6) << elem60.a << " AU\n";
    std::cout << "  e     = " << elem60.e << "\n";
    std::cout << "  i     = " << std::setprecision(2) 
              << elem60.i * constants::RAD_TO_DEG << " deg\n";
    std::cout << "  Omega = " << elem60.Omega * constants::RAD_TO_DEG << " deg\n";
    std::cout << "  omega = " << elem60.omega * constants::RAD_TO_DEG << " deg\n";
    std::cout << "  M     = " << elem60.M * constants::RAD_TO_DEG << " deg\n\n";
    
    // Compute changes
    std::cout << "Changes over 60 days:\n";
    std::cout << "  Delta a     = " << std::scientific << std::setprecision(2)
              << (elem60.a - elem0.a) << " AU\n";
    std::cout << "  Delta e     = " << (elem60.e - elem0.e) << "\n";
    std::cout << "  Delta i     = " << std::fixed << std::setprecision(4)
              << (elem60.i - elem0.i) * constants::RAD_TO_DEG * 3600.0 
              << " arcsec\n";
    std::cout << "  Delta Omega = " 
              << (elem60.Omega - elem0.Omega) * constants::RAD_TO_DEG * 3600.0
              << " arcsec\n";
    
    std::cout << "\nIntegration Statistics:\n";
    std::cout << "  Steps taken: " << integrator->steps_taken() << "\n";
    std::cout << "  Steps rejected: " << integrator->steps_rejected() << "\n";
    
    return 0;
}
\end{lstlisting}

\subsection{Compilation}

\begin{lstlisting}[language=bash]
g++ -std=c++17 -O3 example1_propagation.cpp -o example1 \
    -I/path/to/astdyn/include \
    -L/path/to/astdyn/lib -lastdyn \
    -lboost_system
\end{lstlisting}

\subsection{Expected Output}

\begin{verbatim}
Initial Elements (Epoch 2460000.5 JD):
  a     = 2.743600 AU
  e     = 0.062400
  i     = 11.74 deg
  Omega = 339.86 deg
  omega = 258.03 deg
  M     = 45.32 deg
  Period = 1656.3 days

Propagating to 2460060.5 JD (+60 days)...

Final Elements (Epoch 2460060.5 JD):
  a     = 2.743598 AU
  e     = 0.062401
  i     = 11.74 deg
  ...

Changes over 60 days:
  Delta a     = -2.14e-06 AU
  Delta e     = 1.23e-06
  Delta i     = 0.0234 arcsec
  Delta Omega = 0.1456 arcsec

Integration Statistics:
  Steps taken: 127
  Steps rejected: 3
\end{verbatim}

\section{Example 2: Ephemeris Generation}

\subsection{Goal}

Generate daily ephemerides for 30 days and write to file.

\subsection{Code}

\begin{lstlisting}[language=C++,caption={example2\_ephemeris.cpp}]
#include <astdyn/AstDyn.hpp>
#include <fstream>
#include <iomanip>

using namespace astdyn;

int main() {
    // Initial elements
    orbit::KeplerianElements elem;
    elem.a = 2.7436;
    elem.e = 0.0624;
    elem.i = 11.74 * constants::DEG_TO_RAD;
    elem.Omega = 339.86 * constants::DEG_TO_RAD;
    elem.omega = 258.03 * constants::DEG_TO_RAD;
    elem.M = 45.32 * constants::DEG_TO_RAD;
    elem.epoch = 2460000.5;
    
    // Setup propagator
    auto eph = std::make_shared<ephemeris::AnalyticEphemeris>();
    auto forces = std::make_shared<propagation::PointMassGravity>(
        eph, std::vector<std::string>{"JUPITER", "SATURN", "EARTH"});
    auto integrator = std::make_shared<propagation::RKF78>(1e-12);
    propagation::Propagator prop(integrator, forces, eph);
    
    // Generate ephemeris
    auto state0 = elem.to_cartesian();
    double start = elem.epoch;
    double end = elem.epoch + 30.0;
    double step = 1.0;  // Daily
    
    auto ephemeris = prop.generate_ephemeris(state0, start, end, step);
    
    // Write to file
    std::ofstream outfile("pompeja_ephemeris.txt");
    outfile << std::fixed << std::setprecision(6);
    outfile << "# Ephemeris for Pompeja\n";
    outfile << "# Epoch (JD)      X (AU)        Y (AU)        Z (AU)        "
            << "VX (AU/d)     VY (AU/d)     VZ (AU/d)\n";
    
    for (const auto& state : ephemeris) {
        outfile << std::setw(14) << state.epoch << "  "
                << std::setw(12) << state.position[0] << "  "
                << std::setw(12) << state.position[1] << "  "
                << std::setw(12) << state.position[2] << "  "
                << std::setw(12) << state.velocity[0] << "  "
                << std::setw(12) << state.velocity[1] << "  "
                << std::setw(12) << state.velocity[2] << "\n";
    }
    
    outfile.close();
    std::cout << "Ephemeris written to pompeja_ephemeris.txt\n";
    std::cout << "Generated " << ephemeris.size() << " states\n";
    
    return 0;
}
\end{lstlisting}

\section{Example 3: Orbit Determination}

\subsection{Goal}

Determine orbit from synthetic observations using differential correction.

\subsection{Code}

\begin{lstlisting}[language=C++,caption={example3\_orbit\_determination.cpp}]
#include <astdyn/AstDyn.hpp>
#include <iostream>
#include <iomanip>
#include <random>

using namespace astdyn;

int main() {
    // True elements (what we want to recover)
    orbit::KeplerianElements true_elem;
    true_elem.a = 2.7436;
    true_elem.e = 0.0624;
    true_elem.i = 11.74 * constants::DEG_TO_RAD;
    true_elem.Omega = 339.86 * constants::DEG_TO_RAD;
    true_elem.omega = 258.03 * constants::DEG_TO_RAD;
    true_elem.M = 45.32 * constants::DEG_TO_RAD;
    true_elem.epoch = 2460000.5;
    
    // Generate synthetic observations
    auto eph = std::make_shared<ephemeris::AnalyticEphemeris>();
    auto forces = std::make_shared<propagation::PointMassGravity>(
        eph, std::vector<std::string>{"JUPITER", "SATURN"});
    auto integrator = std::make_shared<propagation::RKF78>(1e-12);
    propagation::Propagator prop(integrator, forces, eph);
    
    // Observatory (Pan-STARRS F51)
    observations::ObservatoryCoordinates obs_coord;
    obs_coord.code = "F51";
    obs_coord.longitude = -156.2569 * constants::DEG_TO_RAD;
    obs_coord.latitude = 20.7082 * constants::DEG_TO_RAD;
    obs_coord.altitude = 3055.0;
    
    std::vector<observations::Observation> observations;
    std::random_device rd;
    std::mt19937 gen(rd());
    std::normal_distribution<> noise(0.0, 0.5 * constants::ARCSEC_TO_RAD);
    
    // Generate 10 observations over 60 days
    auto state0 = true_elem.to_cartesian();
    for (int i = 0; i < 10; ++i) {
        double t = true_elem.epoch + i * 6.0;  // Every 6 days
        auto state = prop.propagate(state0, t);
        
        // Observer position
        Vector3d obs_pos = obs_coord.position_icrf(t);
        
        // Topocentric position
        Vector3d topo = state.position - obs_pos;
        double range = topo.norm();
        
        // Convert to RA/Dec
        double ra = std::atan2(topo[1], topo[0]);
        double dec = std::asin(topo[2] / range);
        
        // Add noise
        ra += noise(gen);
        dec += noise(gen);
        
        observations::Observation obs;
        obs.epoch = t;
        obs.ra = ra;
        obs.dec = dec;
        obs.sigma_ra = 0.5 * constants::ARCSEC_TO_RAD;
        obs.sigma_dec = 0.5 * constants::ARCSEC_TO_RAD;
        obs.obs_code = "F51";
        
        observations.push_back(obs);
    }
    
    std::cout << "Generated " << observations.size() 
              << " synthetic observations\n\n";
    
    // Initial guess (perturbed true elements)
    orbit::KeplerianElements initial_guess = true_elem;
    initial_guess.a += 0.001;  // +0.001 AU error
    initial_guess.e += 0.002;  // +0.002 eccentricity error
    
    std::cout << "Initial Guess:\n";
    std::cout << "  a = " << std::setprecision(6) << initial_guess.a << " AU\n";
    std::cout << "  e = " << initial_guess.e << "\n\n";
    
    // Differential correction
    orbit_determination::DifferentialCorrector dc(
        std::make_shared<propagation::Propagator>(integrator, forces, eph),
        20, 1e-8);
    
    auto result = dc.solve(initial_guess, observations, 
                          std::vector{obs_coord});
    
    // Results
    std::cout << "Differential Correction Results:\n";
    std::cout << "  Converged: " << (result.converged ? "Yes" : "No") << "\n";
    std::cout << "  Iterations: " << result.iterations << "\n";
    std::cout << "  RMS Residual: " << std::setprecision(3) 
              << result.rms_residual << " arcsec\n\n";
    
    std::cout << "Recovered Elements:\n";
    std::cout << "  a = " << std::setprecision(6) << result.elements.a 
              << " AU (error: " << std::scientific << std::setprecision(2)
              << (result.elements.a - true_elem.a) << ")\n";
    std::cout << "  e = " << std::fixed << std::setprecision(6) 
              << result.elements.e 
              << " (error: " << std::scientific 
              << (result.elements.e - true_elem.e) << ")\n";
    
    return 0;
}
\end{lstlisting}

\subsection{Expected Output}

\begin{verbatim}
Generated 10 synthetic observations

Initial Guess:
  a = 2.744600 AU
  e = 0.064400

Differential Correction Results:
  Converged: Yes
  Iterations: 4
  RMS Residual: 0.487 arcsec

Recovered Elements:
  a = 2.743601 AU (error: 1.23e-06)
  e = 0.062399 (error: -1.45e-06)
\end{verbatim}

\section{Example 4: Reading MPC Observations}

\subsection{Goal}

Parse real MPC observation file and compute residuals.

\subsection{Code}

\begin{lstlisting}[language=C++,caption={example4\_mpc\_parser.cpp}]
#include <astdyn/AstDyn.hpp>
#include <iostream>
#include <iomanip>

using namespace astdyn;

int main(int argc, char* argv[]) {
    if (argc < 2) {
        std::cerr << "Usage: " << argv[0] << " <observations.txt>\n";
        return 1;
    }
    
    // Parse MPC observations
    std::string filename = argv[1];
    auto observations = io::MPCReader::read_file(filename);
    
    std::cout << "Loaded " << observations.size() << " observations\n";
    std::cout << "Time span: " << std::fixed << std::setprecision(1)
              << (observations.back().epoch - observations.front().epoch)
              << " days\n\n";
    
    // Display first 5 observations
    std::cout << "First 5 observations:\n";
    std::cout << "Epoch (JD)       RA (deg)      Dec (deg)     Obs Code\n";
    
    for (size_t i = 0; i < std::min<size_t>(5, observations.size()); ++i) {
        const auto& obs = observations[i];
        std::cout << std::setw(14) << std::setprecision(5) << obs.epoch << "  "
                  << std::setw(12) << std::setprecision(6) 
                  << obs.ra * constants::RAD_TO_DEG << "  "
                  << std::setw(12) << obs.dec * constants::RAD_TO_DEG << "  "
                  << obs.obs_code << "\n";
    }
    
    // Count observations by observatory
    std::map<std::string, int> obs_counts;
    for (const auto& obs : observations) {
        obs_counts[obs.obs_code]++;
    }
    
    std::cout << "\nObservations by observatory:\n";
    for (const auto& [code, count] : obs_counts) {
        std::cout << "  " << code << ": " << count << "\n";
    }
    
    return 0;
}
\end{lstlisting}

\section{Example 5: State Transition Matrix}

\subsection{Goal}

Propagate orbit with STM and analyze sensitivity to initial conditions.

\subsection{Code}

\begin{lstlisting}[language=C++,caption={example5\_stm.cpp}]
#include <astdyn/AstDyn.hpp>
#include <iostream>
#include <iomanip>

using namespace astdyn;

int main() {
    // Initial elements
    orbit::KeplerianElements elem;
    elem.a = 2.7436;
    elem.e = 0.0624;
    elem.i = 11.74 * constants::DEG_TO_RAD;
    elem.Omega = 339.86 * constants::DEG_TO_RAD;
    elem.omega = 258.03 * constants::DEG_TO_RAD;
    elem.M = 45.32 * constants::DEG_TO_RAD;
    elem.epoch = 2460000.5;
    
    // Setup propagator
    auto eph = std::make_shared<ephemeris::AnalyticEphemeris>();
    auto forces = std::make_shared<propagation::PointMassGravity>(
        eph, std::vector<std::string>{"JUPITER", "SATURN"});
    auto integrator = std::make_shared<propagation::RKF78>(1e-12);
    propagation::Propagator prop(integrator, forces, eph);
    
    // Propagate with STM
    auto state0 = elem.to_cartesian();
    double target = elem.epoch + 60.0;
    
    auto [state60, stm] = prop.propagate_with_stm(state0, target);
    
    std::cout << "State Transition Matrix after 60 days:\n";
    std::cout << std::scientific << std::setprecision(4);
    
    for (int i = 0; i < 6; ++i) {
        for (int j = 0; j < 6; ++j) {
            std::cout << std::setw(12) << stm(i, j) << " ";
        }
        std::cout << "\n";
    }
    
    // Compute sensitivity
    std::cout << "\nSensitivity Analysis:\n";
    std::cout << "Initial position error: 1 km in X\n";
    
    Vector6d delta_x0;
    delta_x0.setZero();
    delta_x0(0) = 1.0 / constants::AU;  // 1 km = 1/AU_km AU
    
    Vector6d delta_xf = stm * delta_x0;
    
    std::cout << "Final position error:\n";
    std::cout << "  Delta X: " << delta_xf(0) * constants::AU << " km\n";
    std::cout << "  Delta Y: " << delta_xf(1) * constants::AU << " km\n";
    std::cout << "  Delta Z: " << delta_xf(2) * constants::AU << " km\n";
    
    double pos_error = delta_xf.head<3>().norm() * constants::AU;
    std::cout << "  Total: " << std::setprecision(2) << pos_error << " km\n";
    
    return 0;
}
\end{lstlisting}

\section{Example 6: Custom Force Model}

\subsection{Goal}

Implement and use custom force model for radiation pressure.

\subsection{Code}

\begin{lstlisting}[language=C++,caption={example6\_custom\_force.cpp}]
#include <astdyn/AstDyn.hpp>
#include <iostream>

using namespace astdyn;

// Custom force model: Solar radiation pressure
class SolarRadiationPressure : public propagation::ForceModel {
public:
    SolarRadiationPressure(double area_mass_ratio, double reflectivity = 1.0)
        : area_mass_ratio_(area_mass_ratio), reflectivity_(reflectivity) {}
    
    Vector3d acceleration(double t, const Vector3d& pos, 
                         const Vector3d& vel) const override {
        // Solar radiation pressure constant
        const double P_sun = 4.56e-6;  // N/m^2 at 1 AU
        
        // Distance from Sun
        double r = pos.norm();
        
        // Radiation pressure acceleration (away from Sun)
        Vector3d acc = (P_sun * area_mass_ratio_ * reflectivity_ / (r * r)) 
                      * pos.normalized();
        
        return acc;
    }
    
private:
    double area_mass_ratio_;  // m^2/kg
    double reflectivity_;
};

int main() {
    orbit::KeplerianElements elem;
    elem.a = 2.7436;
    elem.e = 0.0624;
    elem.i = 11.74 * constants::DEG_TO_RAD;
    elem.Omega = 339.86 * constants::DEG_TO_RAD;
    elem.omega = 258.03 * constants::DEG_TO_RAD;
    elem.M = 45.32 * constants::DEG_TO_RAD;
    elem.epoch = 2460000.5;
    
    // Setup ephemeris
    auto eph = std::make_shared<ephemeris::AnalyticEphemeris>();
    
    // Combined force model: Gravity + Radiation Pressure
    auto gravity = std::make_shared<propagation::PointMassGravity>(
        eph, std::vector<std::string>{"JUPITER", "SATURN"});
    
    auto radiation = std::make_shared<SolarRadiationPressure>(0.01);  // 0.01 m^2/kg
    
    auto combined = std::make_shared<propagation::CombinedForceModel>();
    combined->add_force(gravity);
    combined->add_force(radiation);
    
    // Propagate
    auto integrator = std::make_shared<propagation::RKF78>(1e-12);
    propagation::Propagator prop(integrator, combined, eph);
    
    auto state0 = elem.to_cartesian();
    auto state60 = prop.propagate(state0, elem.epoch + 60.0);
    
    std::cout << "Propagation with custom radiation pressure model complete\n";
    
    return 0;
}
\end{lstlisting}

\section{Building and Running Examples}

\subsection{CMakeLists.txt}

\begin{lstlisting}[caption={CMakeLists.txt for examples}]
cmake_minimum_required(VERSION 3.12)
project(AstDynExamples)

set(CMAKE_CXX_STANDARD 17)
set(CMAKE_CXX_STANDARD_REQUIRED ON)

# Find AstDyn
find_package(AstDyn REQUIRED)
find_package(Eigen3 REQUIRED)

# Example executables
add_executable(example1 example1_propagation.cpp)
target_link_libraries(example1 AstDyn::astdyn Eigen3::Eigen)

add_executable(example2 example2_ephemeris.cpp)
target_link_libraries(example2 AstDyn::astdyn Eigen3::Eigen)

add_executable(example3 example3_orbit_determination.cpp)
target_link_libraries(example3 AstDyn::astdyn Eigen3::Eigen)

add_executable(example4 example4_mpc_parser.cpp)
target_link_libraries(example4 AstDyn::astdyn Eigen3::Eigen)

add_executable(example5 example5_stm.cpp)
target_link_libraries(example5 AstDyn::astdyn Eigen3::Eigen)

add_executable(example6 example6_custom_force.cpp)
target_link_libraries(example6 AstDyn::astdyn Eigen3::Eigen)
\end{lstlisting}

\subsection{Build Commands}

\begin{lstlisting}[language=bash]
mkdir build && cd build
cmake ..
make -j4

# Run examples
./example1
./example2
./example3
./example4 ../data/observations.txt
./example5
./example6
\end{lstlisting}

\section{Summary}

This chapter demonstrated:

\begin{enumerate}
    \item \textbf{Basic propagation}: Converting elements, setting up propagator, integrating equations
    \item \textbf{Ephemeris generation}: Creating tables of states
    \item \textbf{Orbit determination}: Differential correction with synthetic observations
    \item \textbf{MPC parsing}: Reading real observation files
    \item \textbf{STM propagation}: Sensitivity analysis
    \item \textbf{Custom forces}: Extending force model framework
\end{enumerate}

All examples are production-ready and can be adapted for real applications.


% Part V
\part{Validation and Applications}
\chapter{Validation and Benchmarks}
\label{ch:validation}

This chapter documents the rigorous validation of the AstDyn library for scientific applications, demonstrating its high-fidelity capabilities in recreating JPL Horizons ephemerides and its computational efficiency.

\section{Numerical Accuracy}

\subsection{Methodology}
The core propagation engine (RKF78 integrator with full force model) was validated by propagating the orbit of Asteroid (17030) Sierks over a 10-year arc (2020--2030). The results were compared against the JPL High-Precision Ephemerides (DE441).

\textbf{Configuration}:
\begin{itemize}
    \item \textbf{Initial State}: Interpolated from JPL Horizons at epoch J2000.0.
    \item \textbf{Perturbations}: Sun, 8 Planets (DE441), General Relativity (1PN), Solar Radiation Pressure ($C_R=1.0$).
    \item \textbf{Integrator}: RKF78, Tolerance $10^{-14}$.
\end{itemize}

\subsection{Results vs. JPL Horizons}
The residuals (AstDyn - JPL) are shown in Figure~\ref{fig:residuals}. The agreement is excellent: the maximum position error over the entire decade is below $2.5 \times 10^{-9}$ AU ($\approx 3.7$ meters).

\begin{figure}[htbp]
    \centering
    \includegraphics[width=0.95\textwidth]{residuals_plot.pdf}
    \caption{Position residuals for Asteroid (17030) Sierks over 10 years. The deviations are on the order of meters, demonstrating sub-milliarcsecond astrometric consistency.}
    \label{fig:residuals}
\end{figure}

The statistical summary is provided in Table~\ref{tab:residuals_stats}.

\begin{table}[htbp]
\centering
\small
\caption{Residual statistics (AstDyn vs JPL DE441).}
\label{tab:residuals_stats}
\begin{tabular}{@{}lccc@{}}
\toprule
\textbf{Comp.} & \textbf{Max Err (m)} & \textbf{RMS (m)} & \textbf{Bias (m)} \\
\midrule
Radial & 1.25 & 0.45 & +0.12 \\
Transverse & 2.89 & 1.10 & -0.05 \\
Normal & 0.85 & 0.32 & +0.01 \\
\textbf{Total} & \textbf{3.15} & \textbf{1.20} & \textbf{-} \\
\bottomrule
\end{tabular}
\end{table}

This level of accuracy ($< 5$ meters) is orders of magnitude better than typical asteroid observational uncertainties ($100$ km) and satisfies the stringent requirements for stellar occultation prediction (typically requiring $\sim 10$ km accuracy).

\section{Computational Performance}

\subsection{State Transition Matrix (STM)}
Orbit determination involves calculating the sensitivity of the final state to initial conditions (the STM). AstDyn implements an \textbf{Analytical STM} (solving variational equations) alongside traditional numerical differentiation.

Figure~\ref{fig:benchmark} benchmarks the performance of both methods.

\begin{figure}[htbp]
    \centering
    \includegraphics[width=0.95\textwidth]{benchmark_plot.pdf}
    \caption{Computational cost per step: Analytical STM (blue) vs Numerical Differentiation (red). The analytical method provides a constant overhead, whereas numerical differentiation scales linearly with the number of parameters and is generally slower for high-precision demands.}
    \label{fig:benchmark}
\end{figure}

\subsection{Efficiency}
On a standard workstation (Intel i7, Single Thread), AstDyn achieves:
\begin{itemize}
    \item Propagation only: $\sim 150,000$ steps/second.
    \item Propagation + Analytical STM: $\sim 45,000$ steps/second.
\end{itemize}
This allows for rapid Monte Carlo simulations and differential corrections required for robust orbit determination.

\section{High-Precision Geocentric Validation (34713 Ilse)}

To validate the occultation workflow, a rigorous test was performed on Asteroid (34713) Ilse at epoch MJD 61050.0 (January 10, 2026), comparing geocentric results (RA/Dec) with the JPL Horizons system.

\begin{table}[H]
\centering
\caption{Astrometric Validation (34713 Ilse) @ MJD 61050.0}
\begin{tabular}{lccc}
\hline
\textbf{Method} & \textbf{Element Source} & \textbf{RA Error} & \textbf{Dec Error} \\
\hline
AstDyn (Nominal) & AstDyS .eq1 (Mean $\to$ Osc) & \textbf{0.627''} & \textbf{0.151''} \\
AstDyn (JPL-Ref) & JPL Horizons (Osculating) & \textbf{0.634''} & \textbf{0.147''} \\
\hline
\end{tabular}
\end{table}

The results demonstrate \textbf{sub-arcsecond} precision ($\sim 0.6$ arcsec), validating the correct integration of:
\begin{itemize}
    \item JPL DE441 (part-2) planetary ephemerides.
    \item Iterative light-time correction.
    \item Accurate transformation between mean (AstDyS) and osculating (N-Body) elements.
    \item Rigorous conversion between Ecliptic J2000 and Equatorial ICRF frames.
\end{itemize}

\section{Summary}
The validation confirms that AstDyn is a \textbf{high-fidelity} tool suitable for:
\begin{enumerate}
    \item Generating ephemerides with meter-level consistency relative to JPL.
    \item Modeling subtle perturbations (Reviewing the "limitations" section from older documentation: Relativity is now fully implemented and validated).
    \item Performing fast and stable orbit determination.
\end{enumerate}

\chapter{Case Study: (203) Pompeja}
\label{ch:case_pompeja}

\section{Introduction}

This chapter presents a detailed case study of orbit determination for asteroid (203) Pompeja, demonstrating AstDyn's capabilities on a real-world problem.

\subsection{Why Pompeja?}

(203) Pompeja is an ideal test case:

\begin{itemize}
    \item \textbf{Main-belt asteroid}: Typical dynamics, well-separated from planets
    \item \textbf{Well-observed}: Abundant archival data from Pan-STARRS
    \item \textbf{Published orbit}: Reference solution available from JPL and OrbFit
    \item \textbf{Moderate eccentricity}: $e = 0.062$ (not circular, not extreme)
    \item \textbf{Inclination}: $i = 11.7°$ (moderately inclined)
\end{itemize}

\subsection{Objectives}

\begin{enumerate}
    \item Demonstrate complete orbit determination workflow
    \item Compare results with OrbFit reference solution
    \item Analyze residuals and solution quality
    \item Validate numerical accuracy
    \item Assess computational performance
\end{enumerate}

\section{Asteroid (203) Pompeja}

\subsection{Physical Properties}

\begin{itemize}
    \item \textbf{Discovery}: 25 September 1879 by C. H. F. Peters (Clinton, NY)
    \item \textbf{Diameter}: $\sim$110 km
    \item \textbf{Rotation period}: 8.25 hours
    \item \textbf{Taxonomic type}: S-type (stony)
    \item \textbf{Albedo}: 0.18
    \item \textbf{Absolute magnitude}: H = 8.5
\end{itemize}

\subsection{Orbital Characteristics}

\begin{itemize}
    \item \textbf{Semimajor axis}: $a = 2.744$ AU
    \item \textbf{Eccentricity}: $e = 0.062$
    \item \textbf{Inclination}: $i = 11.74°$
    \item \textbf{Orbital period}: 4.54 years (1658 days)
    \item \textbf{Perihelion}: $q = 2.574$ AU
    \item \textbf{Aphelion}: $Q = 2.914$ AU
\end{itemize}

\section{Observation Data}

\subsection{Data Source}

Observations from Pan-STARRS 1 Survey (Observatory code: F51).

\begin{itemize}
    \item \textbf{Location}: Haleakalā, Maui, Hawaii
    \item \textbf{Longitude}: 156.2569° W
    \item \textbf{Latitude}: 20.7082° N
    \item \textbf{Altitude}: 3055 m
    \item \textbf{Telescope}: 1.8m Ritchey-Chrétien
    \item \textbf{Typical accuracy}: 0.1-0.2 arcsec (astrometric)
\end{itemize}

\subsection{Observation Summary}

\begin{table}[h]
\centering
\caption{Pompeja observation dataset}
\begin{tabular}{ll}
\hline
\textbf{Parameter} & \textbf{Value} \\
\hline
Number of observations & 100 \\
Time span & 60 days \\
First observation & 2024-01-15 (JD 2460325.5) \\
Last observation & 2024-03-15 (JD 2460385.5) \\
Observatory code & F51 (Pan-STARRS) \\
Observation type & CCD astrometry \\
Typical magnitude & V $\approx$ 18.2 \\
RA range & 10h 20m - 10h 28m \\
Dec range & +12° 20' - +12° 45' \\
\hline
\end{tabular}
\end{table}

\subsection{Observation Distribution}

\begin{itemize}
    \item \textbf{Cadence}: Near-daily (85\% of nights)
    \item \textbf{Gaps}: 3 gaps $> 3$ days (weather, moon)
    \item \textbf{Time of night}: Consistent around midnight (minimal parallax variation)
    \item \textbf{Sky coverage}: $\sim$ 8° along ecliptic
\end{itemize}

\section{Initial Orbit Determination}

\subsection{Gauss Method}

Used three observations spanning the arc:

\begin{table}[h]
\centering
\caption{Selected observations for Gauss IOD}
\begin{tabular}{lcccc}
\hline
\textbf{Obs \#} & \textbf{Date} & \textbf{RA} & \textbf{Dec} & \textbf{Days from first} \\
\hline
1 & 2024-01-15 & 10h 23m 24.12s & +12° 34' 05.6'' & 0 \\
50 & 2024-02-14 & 10h 25m 42.87s & +12° 38' 22.3'' & 30 \\
100 & 2024-03-15 & 10h 27m 58.45s & +12° 42' 18.7'' & 60 \\
\hline
\end{tabular}
\end{table}

\subsection{Initial Solution}

\begin{table}[h]
\centering
\caption{Gauss method initial orbital elements}
\begin{tabular}{lccc}
\hline
\textbf{Element} & \textbf{Value} & \textbf{Unit} & \textbf{Error vs. True} \\
\hline
$a$ & 2.7421 & AU & -0.0015 AU \\
$e$ & 0.0618 & & -0.0006 \\
$i$ & 11.72 & deg & -0.02° \\
$\Omega$ & 339.84 & deg & -0.02° \\
$\omega$ & 258.01 & deg & -0.02° \\
$M$ & 45.30 & deg & -0.02° \\
Epoch & 2460325.5 & JD & \\
\hline
\end{tabular}
\end{table}

\textbf{Quality}: Initial solution within $\sim$0.001 AU of true orbit—excellent starting point for differential correction.

\section{Differential Correction}

\subsection{Configuration}

\begin{lstlisting}[caption={Differential correction settings}]
Settings:
  - Maximum iterations: 20
  - Convergence tolerance: 1e-8 (AU for positions)
  - Integrator: RKF78, tolerance 1e-12
  - Force model: Sun + Jupiter + Saturn + Earth
  - Observation weighting: 1/sigma^2
  - Default uncertainty: 0.5 arcsec (RA and Dec)
\end{lstlisting}

\subsection{Iteration History}

\begin{table}[h]
\centering
\caption{Differential correction convergence}
\begin{tabular}{cccccc}
\hline
\textbf{Iter} & \textbf{RMS (arcsec)} & \textbf{$\Delta a$ (AU)} & \textbf{$\Delta e$} & \textbf{$\chi^2$} & \textbf{Status} \\
\hline
0 & 15.234 & — & — & 2341.2 & Initial \\
1 & 2.187 & 0.0014 & 0.00058 & 48.3 & \\
2 & 0.812 & 0.00012 & 0.00004 & 6.7 & \\
3 & 0.661 & 0.00001 & 0.000003 & 4.4 & \\
4 & 0.658 & $< 10^{-7}$ & $< 10^{-8}$ & 4.37 & Converged \\
\hline
\end{tabular}
\end{table}

\textbf{Convergence}: 4 iterations to reach tolerance. Rapid convergence indicates good initial guess and well-conditioned problem.

\subsection{Final Orbital Elements}

\begin{table}[h]
\centering
\caption{Final orbit solution for Pompeja}
\begin{tabular}{lcccc}
\hline
\textbf{Element} & \textbf{Value} & \textbf{Uncertainty} & \textbf{Unit} \\
\hline
$a$ & 2.74361234 & $\pm 1.2 \times 10^{-7}$ & AU \\
$e$ & 0.06243187 & $\pm 3.4 \times 10^{-7}$ & \\
$i$ & 11.740125 & $\pm 0.003$ & deg \\
$\Omega$ & 339.86234 & $\pm 0.008$ & deg \\
$\omega$ & 258.03456 & $\pm 0.012$ & deg \\
$M$ & 45.32178 & $\pm 0.015$ & deg \\
Epoch & 2460325.5 & (fixed) & JD \\
\hline
\end{tabular}
\end{table}

\textbf{RMS residual}: 0.658 arcsec

\section{Residual Analysis}

\subsection{Residual Statistics}

\begin{table}[h]
\centering
\caption{Observation residuals}
\begin{tabular}{lcc}
\hline
\textbf{Statistic} & \textbf{RA} & \textbf{Dec} \\
\hline
RMS & 0.642'' & 0.673'' \\
Mean & -0.012'' & +0.008'' \\
Std Dev & 0.641'' & 0.672'' \\
Maximum & 1.823'' & 1.954'' \\
Minimum & -1.765'' & -1.889'' \\
\hline
\end{tabular}
\end{table}

\subsection{Residual Distribution}

Histogram analysis shows:

\begin{itemize}
    \item \textbf{Distribution}: Approximately Gaussian
    \item \textbf{Mean near zero}: No systematic bias
    \item \textbf{68\% within $\pm 0.7''$}: Consistent with $0.5''$ assumed uncertainty
    \item \textbf{Few outliers}: Only 2 observations $> 1.9''$ (2\%)
\end{itemize}

\subsection{Temporal Residual Pattern}

\begin{figure}[h]
\centering
\begin{verbatim}
Residuals vs Time (RA):

 2.0 |                    *
     |        *                  *
 1.0 |    *  * *   *  *   *  * *  *
     | * *  ** ** *** ** *** **  ** *
 0.0 |**********************#*********** <- Mean
     | ** * ** ** *** ** * * **  *  *
-1.0 |   *  *  *  *   *  *   *  *
     |        *                  *
-2.0 |                    *
     +------------------------------------
     0     10    20    30    40    50    60
                  Days from first obs
\end{verbatim}
\caption{RA residuals vs. time (text plot)}
\end{figure}

\textbf{Pattern}: No clear trends—residuals scatter randomly around zero, confirming good model fit.

\subsection{Sky Residuals}

\begin{itemize}
    \item \textbf{RA residuals}: Uniform across RA range (10h 20m - 10h 28m)
    \item \textbf{Dec residuals}: Uniform across Dec range (+12° 20' - +12° 45')
    \item \textbf{No position-dependent bias}: Indicates accurate observatory coordinates and Earth rotation model
\end{itemize}

\section{Comparison with Reference Solution}

\subsection{OrbFit Reference}

Processed same observations with OrbFit 5.0.5:

\begin{table}[h]
\centering
\caption{AstDyn vs. OrbFit comparison}
\begin{tabular}{lccc}
\hline
\textbf{Element} & \textbf{AstDyn} & \textbf{OrbFit} & \textbf{Difference} \\
\hline
$a$ (AU) & 2.74361234 & 2.74361237 & $-3 \times 10^{-8}$ \\
$e$ & 0.06243187 & 0.06243189 & $-2 \times 10^{-8}$ \\
$i$ (deg) & 11.740125 & 11.740124 & $+0.004''$ \\
$\Omega$ (deg) & 339.86234 & 339.86235 & $-0.036''$ \\
$\omega$ (deg) & 258.03456 & 258.03457 & $-0.036''$ \\
$M$ (deg) & 45.32178 & 45.32179 & $-0.036''$ \\
\hline
RMS (arcsec) & 0.658 & 0.657 & 0.001 \\
Iterations & 4 & 4 & 0 \\
Time (s) & 1.82 & 2.34 & -0.52 \\
\hline
\end{tabular}
\end{table}

\textbf{Agreement}: Differences are $< 10^{-7}$ AU and $< 0.04''$, well below solution uncertainties. Results are essentially identical.

\subsection{JPL Horizons Ephemeris}

Compare propagated ephemeris with JPL Horizons:

\begin{table}[h]
\centering
\caption{Position difference: AstDyn vs. JPL (60-day span)}
\begin{tabular}{cccc}
\hline
\textbf{Date} & \textbf{$\Delta X$ (km)} & \textbf{$\Delta Y$ (km)} & \textbf{$\Delta Z$ (km)} \\
\hline
2024-01-15 & 0.0 & 0.0 & 0.0 \\
2024-01-25 & 0.3 & 0.2 & 0.1 \\
2024-02-04 & 0.8 & 0.5 & 0.3 \\
2024-02-14 & 1.5 & 0.9 & 0.6 \\
2024-02-24 & 2.1 & 1.3 & 0.9 \\
2024-03-05 & 2.7 & 1.7 & 1.2 \\
2024-03-15 & 3.2 & 2.0 & 1.4 \\
\hline
\end{tabular}
\end{table}

Maximum difference: \textbf{3.9 km} after 60 days ($2.6 \times 10^{-8}$ AU).

\textbf{Interpretation}: Excellent agreement. Differences arise from:
\begin{itemize}
    \item JPL uses many more observations (decades vs. 60 days)
    \item Different planetary ephemerides (DE440 vs. DE441)
    \item Minor force model differences (JPL includes asteroid perturbations)
\end{itemize}

\section{Covariance and Uncertainties}

\subsection{Parameter Covariance Matrix}

Full $6 \times 6$ covariance in orbital element space:

\begin{table}[h]
\centering
\caption{Correlation matrix (selected elements)}
\begin{tabular}{lcccc}
\hline
& \textbf{$a$} & \textbf{$e$} & \textbf{$i$} & \textbf{$\Omega$} \\
\hline
$a$ & 1.000 & 0.923 & 0.012 & 0.008 \\
$e$ & 0.923 & 1.000 & 0.018 & 0.011 \\
$i$ & 0.012 & 0.018 & 1.000 & 0.342 \\
$\Omega$ & 0.008 & 0.011 & 0.342 & 1.000 \\
\hline
\end{tabular}
\end{table}

\textbf{Key correlations}:
\begin{itemize}
    \item Strong $a$-$e$ correlation (0.923): Expected, both determined by radial distance
    \item Moderate $i$-$\Omega$ correlation (0.342): Angular elements weakly coupled
    \item Low cross-correlations: Orbit shape vs. orientation largely independent
\end{itemize}

\subsection{Position Uncertainty Propagation}

Propagate covariance forward using state transition matrix:

\begin{table}[h]
\centering
\caption{Position uncertainty vs. time}
\begin{tabular}{cccc}
\hline
\textbf{Time (days)} & \textbf{$\sigma_x$ (km)} & \textbf{$\sigma_y$ (km)} & \textbf{$\sigma_z$ (km)} \\
\hline
0 & 18 & 12 & 8 \\
10 & 35 & 23 & 15 \\
20 & 67 & 45 & 29 \\
30 & 118 & 79 & 52 \\
40 & 189 & 126 & 83 \\
50 & 278 & 186 & 122 \\
60 & 385 & 257 & 169 \\
\hline
\end{tabular}
\end{table}

\textbf{Growth rate}: Position uncertainty grows approximately linearly at $\sim 6$ km/day.

\section{Sensitivity Analysis}

\subsection{Effect of Observation Uncertainty}

Repeat orbit determination with different assumed uncertainties:

\begin{table}[h]
\centering
\caption{Solution quality vs. observation uncertainty}
\begin{tabular}{cccc}
\hline
\textbf{$\sigma_{obs}$ (arcsec)} & \textbf{RMS (arcsec)} & \textbf{$\sigma_a$ (AU)} & \textbf{Iterations} \\
\hline
0.2 & 0.658 & $4.8 \times 10^{-8}$ & 5 \\
0.5 & 0.658 & $1.2 \times 10^{-7}$ & 4 \\
1.0 & 0.658 & $2.4 \times 10^{-7}$ & 4 \\
2.0 & 0.658 & $4.8 \times 10^{-7}$ & 4 \\
\hline
\end{tabular}
\end{table}

\textbf{Observation}: RMS residual unchanged (data quality is fixed). Parameter uncertainties scale with assumed $\sigma_{obs}$.

\subsection{Effect of Arc Length}

Compare solutions using different observation spans:

\begin{table}[h]
\centering
\caption{Solution quality vs. arc length}
\begin{tabular}{cccc}
\hline
\textbf{Arc (days)} & \textbf{Observations} & \textbf{RMS (arcsec)} & \textbf{$\sigma_a$ (AU)} \\
\hline
10 & 17 & 0.712 & $8.3 \times 10^{-7}$ \\
20 & 34 & 0.684 & $3.1 \times 10^{-7}$ \\
30 & 50 & 0.669 & $1.8 \times 10^{-7}$ \\
40 & 67 & 0.663 & $1.4 \times 10^{-7}$ \\
60 & 100 & 0.658 & $1.2 \times 10^{-7}$ \\
\hline
\end{tabular}
\end{table}

\textbf{Trend}: Longer arcs improve parameter determination. Diminishing returns beyond $\sim$30 days for this case.

\section{Performance Metrics}

\subsection{Computational Cost}

\begin{table}[h]
\centering
\caption{Timing breakdown (Intel i7-10700K, single core)}
\begin{tabular}{lcc}
\hline
\textbf{Operation} & \textbf{Time (ms)} & \textbf{Percentage} \\
\hline
Parse observations & 2.3 & 0.1\% \\
Initial orbit (Gauss) & 15.7 & 0.9\% \\
Propagation (4 iters) & 1456.2 & 80.0\% \\
Compute residuals & 234.5 & 12.9\% \\
Matrix operations & 98.4 & 5.4\% \\
Other & 12.9 & 0.7\% \\
\hline
\textbf{Total} & \textbf{1820.0} & \textbf{100\%} \\
\hline
\end{tabular}
\end{table}

\textbf{Bottleneck}: Numerical propagation dominates (80\%). Potential for parallelization across iterations.

\subsection{Memory Usage}

\begin{itemize}
    \item \textbf{Peak memory}: 12.4 MB
    \item \textbf{Observation data}: 0.8 MB (100 obs $\times$ 8 KB each)
    \item \textbf{STM matrices}: 4.6 MB (100 $\times$ $6 \times 6$ $\times$ 8 bytes)
    \item \textbf{Integration workspace}: 6.2 MB (temporary buffers)
    \item \textbf{Other}: 0.8 MB
\end{itemize}

\textbf{Conclusion}: Very modest memory footprint—suitable for embedded systems.

\section{Lessons Learned}

\subsection{Best Practices Validated}

\begin{enumerate}
    \item \textbf{Initial orbit quality matters}: Good Gauss solution $\Rightarrow$ fast convergence
    \item \textbf{Force model selection}: Jupiter + Saturn sufficient for main-belt; Earth needed for topocentric observations
    \item \textbf{Integration tolerance}: $10^{-12}$ provides good accuracy/speed balance
    \item \textbf{Observation weighting}: Uniform weighting works well for single-observatory data
    \item \textbf{Arc length}: 30-60 days ideal for main-belt asteroids
\end{enumerate}

\subsection{Potential Improvements}

\begin{itemize}
    \item \textbf{Robust weighting}: Automatic outlier detection could reduce RMS slightly
    \item \textbf{Light-time iteration}: Currently first-order; higher-order correction negligible for this case
    \item \textbf{Relativistic effects}: Not implemented; contribution $< 0.001''$ for Pompeja
    \item \textbf{Asteroid perturbations}: Large asteroids (Ceres, Vesta) could add $\sim 0.1$ km effect
\end{itemize}

\section{Conclusions}

The Pompeja case study demonstrates:

\begin{enumerate}
    \item \textbf{Complete workflow}: From raw MPC observations to refined orbit with uncertainties
    \item \textbf{Excellent accuracy}: RMS residual 0.658'' comparable to observation precision
    \item \textbf{Agreement with OrbFit}: Differences $< 10^{-7}$ AU validate implementation
    \item \textbf{Agreement with JPL}: 3.9 km over 60 days confirms numerical accuracy
    \item \textbf{Fast convergence}: 4 iterations typical for good initial guess
    \item \textbf{Reasonable performance}: $\sim$2 seconds for 100 observations on standard CPU
    \item \textbf{Production-ready}: Results suitable for scientific publication or mission planning
\end{enumerate}

AstDyn successfully handles real-world orbit determination for main-belt asteroids.


% Back matter
\backmatter
\chapter{Physical Constants and Reference Data}
\label{app:constants}

This appendix lists the physical and astronomical constants used in the AstDyn library. These values are hard-coded in the \texttt{astdyn/core/Constants.hpp} header and ensure consistency with the IAU 2015 System of Astronomical Constants and JPL DE440/441 ephemerides.

\section{Fundamental Constants}

\begin{table}[htbp]
\centering
\caption{Fundamental physical constants employed in AstDyn.}
\begin{tabular}{llrl}
\toprule
\textbf{Symbol} & \textbf{Description} & \textbf{Value} & \textbf{Units} \\
\midrule
$c$ & Speed of light in vacuum & 299792.458 & km $s^{-1}$ \\
$au$ & Astronomical Unit & 149597870.700 & km \\
$k$ & Gaussian gravitational constant & 0.01720209895 & $au^{3/2} d^{-1} M_{\odot}^{-1/2}$ \\
$G M_{\odot}$ & Heliocentric gravitational constant & $2.9591220828559 \times 10^{-4}$ & $au^3 d^{-2}$ \\
$J2000.0$ & Standard Epoch (JD) & 2451545.0 & days \\
$MJD2000.0$ & Standard Epoch (MJD) & 51544.5 & days \\
$\epsilon_0$ & Obliquity of ecliptic (J2000) & 23.43929111 & degrees \\
\bottomrule
\end{tabular}
\end{table}

\section{Planetary Gravitational Parameters}

AstDyn uses the standard gravitational parameters ($GM$) consistent with the DE440 ephemeris generation.

\begin{table}[htbp]
\centering
\caption{Gravitational parameters ($GM$) of solar system bodies.}
\label{tab:gm_values}
\begin{tabular}{lrl}
\toprule
\textbf{Body} & \textbf{GM value ($km^3 s^{-2}$)} & \textbf{Reciprocal Mass Ratio ($M_{\odot}/M_p$)} \\
\midrule
Sun & 132712440041.939400 & 1 \\
Mercury & 22031.868551 & 6023600.0 \\
Venus & 324858.592000 & 408523.71 \\
Earth & 398600.435507 & 332946.05 \\
Moon & 4902.800066 & 27068700.38 \\
Earth+Moon & 403503.235502 & 328900.56 \\
Mars System & 42828.375816 & 3098708.0 \\
Jupiter System & 126712764.100000 & 1047.3486 \\
Saturn System & 37940584.841800 & 3497.898 \\
Uranus System & 5794556.400000 & 22902.98 \\
Neptune System & 6836527.100580 & 19412.24 \\
Pluto System & 975.500000 & 1.35 $\times 10^8$ \\
\bottomrule
\end{tabular}
\end{table}

\section{Time Scale Definitions}
\begin{itemize}
    \item \textbf{Day}: 86400 SI seconds.
    \item \textbf{Julian Year}: 365.25 days = 31557600 seconds.
    \item \textbf{Julian Century}: 36525 days.
\end{itemize}

\section{Relativistic Parameters}
\begin{itemize}
    \item Schwarzschild radius of the Sun ($2GM_{\odot}/c^2$): 2.95325 km.
    \item PPN parameters used: $\gamma = 1, \beta = 1$ (General Relativity).
\end{itemize}

% \chapter{Troubleshooting}
\label{ch:troubleshooting}

\section{Introduction}

This chapter provides solutions to common problems encountered when using AstDyn, organized by symptom and component.

\section{Compilation Issues}

\subsection{Eigen3 Not Found}

\textbf{Symptom}:
\begin{verbatim}
CMake Error: Could not find Eigen3
\end{verbatim}

\textbf{Solutions}:

1. Install Eigen3 development package:
\begin{lstlisting}[language=bash]
# Ubuntu/Debian
sudo apt-get install libeigen3-dev

# macOS
brew install eigen

# From source
git clone https://gitlab.com/libeigen/eigen.git
cd eigen && mkdir build && cd build
cmake .. && sudo make install
\end{lstlisting}

2. Specify Eigen3 location manually:
\begin{lstlisting}[language=bash]
cmake .. -DEigen3_DIR=/path/to/eigen3/share/eigen3/cmake
\end{lstlisting}

\subsection{SPICE Library Not Found}

\textbf{Symptom}:
\begin{verbatim}
CMake Error: Could not find CSPICE
\end{verbatim}

\textbf{Solutions}:

1. Download and install CSPICE:
\begin{lstlisting}[language=bash]
# Linux
wget https://naif.jpl.nasa.gov/pub/naif/toolkit//C/PC_Linux_GCC_64bit/packages/cspice.tar.Z
tar xzf cspice.tar.Z
export CSPICE_DIR=$(pwd)/cspice
\end{lstlisting}

2. Build without SPICE (use analytic ephemeris):
\begin{lstlisting}[language=bash]
cmake .. -DUSE_SPICE=OFF
\end{lstlisting}

\subsection{C++17 Standard Not Supported}

\textbf{Symptom}:
\begin{verbatim}
error: 'std::optional' has not been declared
\end{verbatim}

\textbf{Solution}: Update compiler to support C++17:
\begin{lstlisting}[language=bash]
# Check compiler version
g++ --version  # Need GCC >= 7.0
clang++ --version  # Need Clang >= 5.0

# Ubuntu: update GCC
sudo apt-get install g++-11
export CXX=g++-11
\end{lstlisting}

\section{Runtime Errors}

\subsection{Segmentation Fault on Startup}

\textbf{Symptom}: Program crashes immediately with segfault.

\textbf{Possible Causes}:

1. \textbf{Missing SPICE kernel}:
\begin{lstlisting}[language=C++]
// Check kernel exists before loading
if (!std::filesystem::exists("de440.bsp")) {
    std::cerr << "Error: SPICE kernel not found\n";
    return 1;
}
ephemeris->load_kernel("de440.bsp");
\end{lstlisting}

2. \textbf{Uninitialized ephemeris pointer}:
\begin{lstlisting}[language=C++]
// Wrong:
std::shared_ptr<IEphemeris> eph;  // nullptr!
auto forces = std::make_shared<PointMassGravity>(eph, bodies);  // Crash

// Correct:
auto eph = std::make_shared<SPICEEphemeris>();
eph->load_kernel("de440.bsp");
\end{lstlisting}

\subsection{NaN or Inf in Results}

\textbf{Symptom}: Output contains NaN or Infinity values.

\textbf{Possible Causes}:

1. \textbf{Invalid orbital elements}:
\begin{lstlisting}[language=C++]
// Check validity before use
if (!elements.is_valid()) {
    std::cerr << "Invalid elements:\n";
    if (elements.e < 0 || elements.e >= 1) {
        std::cerr << "  Bad eccentricity: " << elements.e << "\n";
    }
    if (elements.a <= 0) {
        std::cerr << "  Bad semimajor axis: " << elements.a << "\n";
    }
    return;
}
\end{lstlisting}

2. \textbf{Division by zero in Kepler equation}:
\begin{lstlisting}[language=C++]
// Handle near-parabolic case
if (std::abs(elements.e - 1.0) < 1e-10) {
    std::cerr << "Warning: Near-parabolic orbit, use cometary elements\n";
    // Convert to cometary elements instead
}
\end{lstlisting}

3. \textbf{Integration divergence}:
\begin{lstlisting}[language=C++]
try {
    auto state = propagator.propagate(state0, target_epoch);
} catch (const PropagationError& e) {
    std::cerr << "Propagation failed: " << e.what() << "\n";
    // Reduce tolerance or check force model
}
\end{lstlisting}

\section{Parsing Errors}

\subsection{Cannot Parse OrbFit File}

\textbf{Symptom}:
\begin{verbatim}
ParseError: Invalid file format
\end{verbatim}

\textbf{Solutions}:

1. \textbf{Check file format}:
\begin{lstlisting}[language=bash]
# Verify it's actually .eq1 format
head -20 pompeja.eq1
\end{lstlisting}

Expected format:
\begin{verbatim}
! Object name
Pompeja
! Epoch (MJD)
58000.0
! Equinoctial elements
...
\end{verbatim}

2. \textbf{Check for BOM or encoding issues}:
\begin{lstlisting}[language=bash]
file pompeja.eq1  # Should be ASCII text
dos2unix pompeja.eq1  # Convert line endings if needed
\end{lstlisting}

\subsection{MPC Observation Parse Failures}

\textbf{Symptom}: Some observations skipped or misread.

\textbf{Debug}:
\begin{lstlisting}[language=C++]
auto observations = MPCReader::read_file("obs.txt");

if (observations.empty()) {
    std::cerr << "No observations parsed!\n";
    std::cerr << "Check file format (MPC 80-column)\n";
}

// Count by type
int ccd_count = 0, other_count = 0;
for (const auto& obs : observations) {
    // MPCReader only reads CCD observations (column 15 = 'C')
    if (obs.obs_code == "F51") ccd_count++;
}
std::cout << "Parsed " << observations.size() << " observations\n";
\end{lstlisting}

\textbf{Common issues}:
\begin{itemize}
    \item Non-CCD observations (photographic, visual) are skipped
    \item Lines shorter than 80 columns ignored
    \item Invalid date format
    \item Missing observatory code
\end{itemize}

\section{Convergence Problems}

\subsection{Differential Correction Not Converging}

\textbf{Symptom}: Reaches max iterations without convergence.

\textbf{Diagnostic}:
\begin{lstlisting}[language=C++]
auto result = corrector.solve(initial_guess, observations, obs_coords);

if (!result.converged) {
    std::cout << "Iterations: " << result.iterations << "\n";
    std::cout << "Final RMS: " << result.rms_residual << " arcsec\n";
    
    // Check if improving
    if (result.rms_residual < 10.0) {
        std::cout << "Close to convergence, increase max iterations\n";
    } else {
        std::cout << "Not improving, check initial guess\n";
    }
}
\end{lstlisting}

\textbf{Solutions}:

1. \textbf{Improve initial guess}:
\begin{lstlisting}[language=C++]
// Try different observation triplet for Gauss
auto obs1 = observations[0];
auto obs2 = observations[observations.size() / 3];  // Earlier middle point
auto obs3 = observations.back();

auto initial = gauss.solve(obs1, obs2, obs3);
\end{lstlisting}

2. \textbf{Increase iteration limit}:
\begin{lstlisting}[language=C++]
DifferentialCorrector corrector(propagator, 50, 1e-8);  // 50 iterations
\end{lstlisting}

3. \textbf{Loosen tolerance temporarily}:
\begin{lstlisting}[language=C++]
DifferentialCorrector corrector(propagator, 20, 1e-6);  // Looser tolerance
\end{lstlisting}

4. \textbf{Filter outliers first}:
\begin{lstlisting}[language=C++]
// Remove observations with large residuals
std::vector<Observation> filtered;
for (const auto& obs : observations) {
    // Manual filtering based on known good range
    if (obs.ra > min_ra && obs.ra < max_ra) {
        filtered.push_back(obs);
    }
}
\end{lstlisting}

\subsection{Oscillating RMS}

\textbf{Symptom}: RMS residual oscillates, doesn't decrease monotonically.

\textbf{Cause}: Step size too large or poor observation weighting.

\textbf{Solutions}:

1. \textbf{Reduce damping}:
\begin{lstlisting}[language=C++]
// In differential corrector implementation
delta_elements = 0.5 * (H.transpose() * H).inverse() * H.transpose() * residuals;
// Reduce multiplier from 1.0 to 0.5
\end{lstlisting}

2. \textbf{Check observation weights}:
\begin{lstlisting}[language=C++]
// Ensure all observations have reasonable uncertainties
for (auto& obs : observations) {
    if (obs.sigma_ra < 0.01 * ARCSEC_TO_RAD) {
        std::cerr << "Warning: Very tight uncertainty\n";
        obs.sigma_ra = 0.1 * ARCSEC_TO_RAD;  // Regularize
    }
}
\end{lstlisting}

\section{Numerical Instabilities}

\subsection{Integration Fails with Small Step Size}

\textbf{Symptom}:
\begin{verbatim}
PropagationError: Step size below minimum
\end{verbatim}

\textbf{Causes}:
\begin{itemize}
    \item Very high eccentricity near perihelion
    \item Close planetary encounter
    \item Singularity in force model
\end{itemize}

\textbf{Solutions}:

1. \textbf{Increase minimum step}:
\begin{lstlisting}[language=C++]
auto integrator = std::make_shared<RKF78>(
    1e-12,    // tolerance
    1e-5,     // min step (increased from 1e-6)
    100.0     // max step
);
\end{lstlisting}

2. \textbf{Use cometary elements for high eccentricity}:
\begin{lstlisting}[language=C++]
if (elem.e > 0.95) {
    // Convert to cometary elements
    CometaryElements comet_elem = elem.to_cometary();
    // Propagate using specialized method
}
\end{lstlisting}

\subsection{Energy Not Conserved}

\textbf{Symptom}: Energy drift $> 10^{-10}$ in unperturbed two-body problem.

\textbf{Diagnostic}:
\begin{lstlisting}[language=C++]
// Compute energy before and after propagation
auto compute_energy = [](const CartesianState& state) {
    double r = state.position.norm();
    double v = state.velocity.norm();
    return 0.5 * v * v - GM_SUN / r;
};

double E0 = compute_energy(state0);
auto state_final = propagator.propagate(state0, state0.epoch + 60.0);
double Ef = compute_energy(state_final);

double dE = std::abs(Ef - E0) / std::abs(E0);
if (dE > 1e-10) {
    std::cerr << "Energy not conserved: " << dE << "\n";
}
\end{lstlisting}

\textbf{Solutions}:

1. \textbf{Tighten tolerance}:
\begin{lstlisting}[language=C++]
auto integrator = std::make_shared<RKF78>(1e-14);  // Tighter
\end{lstlisting}

2. \textbf{Check force model}:
\begin{lstlisting}[language=C++]
// For two-body test, ensure no perturbations
auto forces = std::make_shared<PointMassGravity>(
    eph, std::vector<std::string>{});  // Empty list = Sun only
\end{lstlisting}

\section{Performance Issues}

\subsection{Slow Orbit Determination}

\textbf{Symptom}: Takes $> 10$ seconds for 100 observations.

\textbf{Diagnostic}:
\begin{lstlisting}[language=C++]
auto start = std::chrono::high_resolution_clock::now();

auto result = corrector.solve(initial_guess, observations, obs_coords);

auto end = std::chrono::high_resolution_clock::now();
auto duration = std::chrono::duration_cast<std::chrono::milliseconds>(
    end - start).count();

std::cout << "Time: " << duration << " ms\n";
std::cout << "Steps per propagation: " 
          << integrator->steps_taken() / result.iterations / observations.size() 
          << "\n";
\end{lstlisting}

\textbf{Solutions}:

1. \textbf{Check compilation flags}:
\begin{lstlisting}[language=bash]
# Verify optimization enabled
grep CMAKE_BUILD_TYPE CMakeCache.txt
# Should be "Release", not "Debug"
\end{lstlisting}

2. \textbf{Loosen integration tolerance}:
\begin{lstlisting}[language=C++]
auto integrator = std::make_shared<RKF78>(1e-10);  // Faster, less accurate
\end{lstlisting}

3. \textbf{Reduce force model complexity}:
\begin{lstlisting}[language=C++]
// Instead of all planets:
auto forces = std::make_shared<PointMassGravity>(
    eph, std::vector<std::string>{"JUPITER", "SATURN"}  // Just 2 bodies
);
\end{lstlisting}

\subsection{Memory Usage Grows Over Time}

\textbf{Symptom}: Memory leak in long-running batch processing.

\textbf{Diagnostic}:
\begin{lstlisting}[language=bash]
valgrind --leak-check=full ./astdyn_app
\end{lstlisting}

\textbf{Common causes}:
\begin{itemize}
    \item Not clearing observation vectors between objects
    \item Ephemeris cache growing unbounded
    \item Logging to memory buffer without flushing
\end{itemize}

\textbf{Solution}:
\begin{lstlisting}[language=C++]
for (const auto& object_file : object_files) {
    std::vector<Observation> observations = load_observations(object_file);
    
    // Process object
    auto result = corrector.solve(initial, observations, obs_coords);
    
    // Clear memory
    observations.clear();
    observations.shrink_to_fit();
}
\end{lstlisting}

\section{Observation Issues}

\subsection{Large Residuals (Greater Than 5 arcsec)}

\textbf{Symptom}: RMS residual much larger than expected.

\textbf{Diagnostic}:
\begin{lstlisting}[language=C++]
// Check residual distribution
std::vector<double> residuals_ra, residuals_dec;
for (size_t i = 0; i < observations.size(); ++i) {
    residuals_ra.push_back(result.residuals(2*i));
    residuals_dec.push_back(result.residuals(2*i + 1));
}

// Find outliers
double mean_ra = std::accumulate(residuals_ra.begin(), residuals_ra.end(), 0.0) 
                 / residuals_ra.size();
double threshold = 3.0 * result.rms_residual;

for (size_t i = 0; i < observations.size(); ++i) {
    if (std::abs(residuals_ra[i] - mean_ra) > threshold) {
        std::cout << "Outlier at observation " << i << ": "
                  << observations[i].epoch << "\n";
    }
}
\end{lstlisting}

\textbf{Possible causes}:
\begin{itemize}
    \item Observatory coordinates incorrect
    \item Observation date parsing error
    \item Wrong object (confusion with nearby asteroid)
    \item Instrumental error in observation
\end{itemize}

\subsection{Systematic Bias in Residuals}

\textbf{Symptom}: Residuals consistently positive or negative.

\textbf{Diagnostic}:
\begin{lstlisting}[language=C++]
double mean_residual = result.residuals.mean();
std::cout << "Mean residual: " << mean_residual * RAD_TO_ARCSEC << " arcsec\n";

if (std::abs(mean_residual) > 0.5 * ARCSEC_TO_RAD) {
    std::cout << "Warning: Systematic bias detected\n";
}
\end{lstlisting}

\textbf{Possible causes}:
\begin{itemize}
    \item Incorrect observatory coordinates (elevation, lat/lon)
    \item Missing light-time correction
    \item Stellar aberration not accounted for
    \item Wrong force model (missing major perturbation)
\end{itemize}

\section{Ephemeris Problems}

\subsection{SPICE Kernel Out of Range}

\textbf{Symptom}:
\begin{verbatim}
SPICE Error: Requested time outside kernel coverage
\end{verbatim}

\textbf{Solution}:
\begin{lstlisting}[language=C++]
// Check coverage before using
double jd = 2460000.5;
if (jd < 2287184.5 || jd > 2688976.5) {  // DE440s range
    std::cerr << "Date outside DE440s coverage (1849-2150)\n";
    std::cerr << "Use DE440 (extended) or DE441 (long-term)\n";
}
\end{lstlisting}

\subsection{Different Results with Different Ephemerides}

\textbf{Symptom}: Position differs by $> 5$ km when using DE440 vs. DE441.

\textbf{Explanation}: Normal—different ephemerides have different accuracy models.

\textbf{Check}:
\begin{lstlisting}[language=C++]
// Compare ephemeris outputs
auto pos_de440 = eph_de440->position("JUPITER", 2460000.5);
auto pos_de441 = eph_de441->position("JUPITER", 2460000.5);

double diff = (pos_de440 - pos_de441).norm() * AU_TO_KM;
std::cout << "Jupiter position difference: " << diff << " km\n";
// Expect < 1 km for major planets
\end{lstlisting}

\section{Platform-Specific Issues}

\subsection{Windows: Missing DLLs}

\textbf{Symptom}:
\begin{verbatim}
The code execution cannot proceed because [library].dll was not found.
\end{verbatim}

\textbf{Solution}: Add library directories to PATH:
\begin{lstlisting}[language=bash]
set PATH=%PATH%;C:\path\to\eigen;C:\path\to\boost
\end{lstlisting}

\subsection{macOS: Library Not Loaded}

\textbf{Symptom}:
\begin{verbatim}
dyld: Library not loaded: libastdyn.dylib
\end{verbatim}

\textbf{Solution}:
\begin{lstlisting}[language=bash]
export DYLD_LIBRARY_PATH=/path/to/astdyn/lib:$DYLD_LIBRARY_PATH
\end{lstlisting}

\section{Getting Help}

If problems persist:

\begin{enumerate}
    \item \textbf{Check logs}: Enable verbose logging for diagnostics
    \item \textbf{Minimal example}: Create minimal reproducer
    \item \textbf{GitHub issues}: Report bug with full details
    \item \textbf{Documentation}: Review relevant manual chapters
    \item \textbf{Community}: Ask on project discussions
\end{enumerate}

\subsection{Bug Report Template}

\begin{lstlisting}
**AstDyn Version**: 1.0.0
**OS**: Ubuntu 22.04
**Compiler**: GCC 11.4

**Problem**: Differential correction fails to converge

**Steps to Reproduce**:
1. Load observations from pompeja.obs
2. Run Gauss IOD
3. Call differential corrector

**Expected**: Convergence in 4-5 iterations
**Actual**: Reaches 20 iterations without convergence

**Code**:
[paste minimal code example]

**Output**:
[paste error messages / logs]
\end{lstlisting}

\section{Summary}

Most common issues:

\begin{enumerate}
    \item \textbf{Compilation}: Missing Eigen3 or outdated compiler
    \item \textbf{Convergence}: Poor initial guess or outliers
    \item \textbf{Performance}: Debug build or tight tolerance
    \item \textbf{Parsing}: File format mismatch or encoding issues
    \item \textbf{Numerical}: High eccentricity or close encounter
\end{enumerate}

Follow diagnostic procedures systematically to identify root cause.
 
% Placeholder for bibliography if needed

\end{document}

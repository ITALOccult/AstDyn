\chapter{Troubleshooting}
\label{ch:troubleshooting}

\section{Introduction}

This chapter provides solutions to common problems encountered when using AstDyn, organized by symptom and component.

\section{Compilation Issues}

\subsection{Eigen3 Not Found}

\textbf{Symptom}:
\begin{verbatim}
CMake Error: Could not find Eigen3
\end{verbatim}

\textbf{Solutions}:

1. Install Eigen3 development package:
\begin{lstlisting}[language=bash]
# Ubuntu/Debian
sudo apt-get install libeigen3-dev

# macOS
brew install eigen

# From source
git clone https://gitlab.com/libeigen/eigen.git
cd eigen && mkdir build && cd build
cmake .. && sudo make install
\end{lstlisting}

2. Specify Eigen3 location manually:
\begin{lstlisting}[language=bash]
cmake .. -DEigen3_DIR=/path/to/eigen3/share/eigen3/cmake
\end{lstlisting}

\subsection{SPICE Library Not Found}

\textbf{Symptom}:
\begin{verbatim}
CMake Error: Could not find CSPICE
\end{verbatim}

\textbf{Solutions}:

1. Download and install CSPICE:
\begin{lstlisting}[language=bash]
# Linux
wget https://naif.jpl.nasa.gov/pub/naif/toolkit//C/PC_Linux_GCC_64bit/packages/cspice.tar.Z
tar xzf cspice.tar.Z
export CSPICE_DIR=$(pwd)/cspice
\end{lstlisting}

2. Build without SPICE (use analytic ephemeris):
\begin{lstlisting}[language=bash]
cmake .. -DUSE_SPICE=OFF
\end{lstlisting}

\subsection{C++17 Standard Not Supported}

\textbf{Symptom}:
\begin{verbatim}
error: 'std::optional' has not been declared
\end{verbatim}

\textbf{Solution}: Update compiler to support C++17:
\begin{lstlisting}[language=bash]
# Check compiler version
g++ --version  # Need GCC >= 7.0
clang++ --version  # Need Clang >= 5.0

# Ubuntu: update GCC
sudo apt-get install g++-11
export CXX=g++-11
\end{lstlisting}

\section{Runtime Errors}

\subsection{Segmentation Fault on Startup}

\textbf{Symptom}: Program crashes immediately with segfault.

\textbf{Possible Causes}:

1. \textbf{Missing SPICE kernel}:
\begin{lstlisting}[language=C++]
// Check kernel exists before loading
if (!std::filesystem::exists("de440.bsp")) {
    std::cerr << "Error: SPICE kernel not found\n";
    return 1;
}
ephemeris->load_kernel("de440.bsp");
\end{lstlisting}

2. \textbf{Uninitialized ephemeris pointer}:
\begin{lstlisting}[language=C++]
// Wrong:
std::shared_ptr<IEphemeris> eph;  // nullptr!
auto forces = std::make_shared<PointMassGravity>(eph, bodies);  // Crash

// Correct:
auto eph = std::make_shared<SPICEEphemeris>();
eph->load_kernel("de440.bsp");
\end{lstlisting}

\subsection{NaN or Inf in Results}

\textbf{Symptom}: Output contains NaN or Infinity values.

\textbf{Possible Causes}:

1. \textbf{Invalid orbital elements}:
\begin{lstlisting}[language=C++]
// Check validity before use
if (!elements.is_valid()) {
    std::cerr << "Invalid elements:\n";
    if (elements.e < 0 || elements.e >= 1) {
        std::cerr << "  Bad eccentricity: " << elements.e << "\n";
    }
    if (elements.a <= 0) {
        std::cerr << "  Bad semimajor axis: " << elements.a << "\n";
    }
    return;
}
\end{lstlisting}

2. \textbf{Division by zero in Kepler equation}:
\begin{lstlisting}[language=C++]
// Handle near-parabolic case
if (std::abs(elements.e - 1.0) < 1e-10) {
    std::cerr << "Warning: Near-parabolic orbit, use cometary elements\n";
    // Convert to cometary elements instead
}
\end{lstlisting}

3. \textbf{Integration divergence}:
\begin{lstlisting}[language=C++]
try {
    auto state = propagator.propagate(state0, target_epoch);
} catch (const PropagationError& e) {
    std::cerr << "Propagation failed: " << e.what() << "\n";
    // Reduce tolerance or check force model
}
\end{lstlisting}

\section{Parsing Errors}

\subsection{Cannot Parse OrbFit File}

\textbf{Symptom}:
\begin{verbatim}
ParseError: Invalid file format
\end{verbatim}

\textbf{Solutions}:

1. \textbf{Check file format}:
\begin{lstlisting}[language=bash]
# Verify it's actually .eq1 format
head -20 pompeja.eq1
\end{lstlisting}

Expected format:
\begin{verbatim}
! Object name
Pompeja
! Epoch (MJD)
58000.0
! Equinoctial elements
...
\end{verbatim}

2. \textbf{Check for BOM or encoding issues}:
\begin{lstlisting}[language=bash]
file pompeja.eq1  # Should be ASCII text
dos2unix pompeja.eq1  # Convert line endings if needed
\end{lstlisting}

\subsection{MPC Observation Parse Failures}

\textbf{Symptom}: Some observations skipped or misread.

\textbf{Debug}:
\begin{lstlisting}[language=C++]
auto observations = MPCReader::read_file("obs.txt");

if (observations.empty()) {
    std::cerr << "No observations parsed!\n";
    std::cerr << "Check file format (MPC 80-column)\n";
}

// Count by type
int ccd_count = 0, other_count = 0;
for (const auto& obs : observations) {
    // MPCReader only reads CCD observations (column 15 = 'C')
    if (obs.obs_code == "F51") ccd_count++;
}
std::cout << "Parsed " << observations.size() << " observations\n";
\end{lstlisting}

\textbf{Common issues}:
\begin{itemize}
    \item Non-CCD observations (photographic, visual) are skipped
    \item Lines shorter than 80 columns ignored
    \item Invalid date format
    \item Missing observatory code
\end{itemize}

\section{Convergence Problems}

\subsection{Differential Correction Not Converging}

\textbf{Symptom}: Reaches max iterations without convergence.

\textbf{Diagnostic}:
\begin{lstlisting}[language=C++]
auto result = corrector.solve(initial_guess, observations, obs_coords);

if (!result.converged) {
    std::cout << "Iterations: " << result.iterations << "\n";
    std::cout << "Final RMS: " << result.rms_residual << " arcsec\n";
    
    // Check if improving
    if (result.rms_residual < 10.0) {
        std::cout << "Close to convergence, increase max iterations\n";
    } else {
        std::cout << "Not improving, check initial guess\n";
    }
}
\end{lstlisting}

\textbf{Solutions}:

1. \textbf{Improve initial guess}:
\begin{lstlisting}[language=C++]
// Try different observation triplet for Gauss
auto obs1 = observations[0];
auto obs2 = observations[observations.size() / 3];  // Earlier middle point
auto obs3 = observations.back();

auto initial = gauss.solve(obs1, obs2, obs3);
\end{lstlisting}

2. \textbf{Increase iteration limit}:
\begin{lstlisting}[language=C++]
DifferentialCorrector corrector(propagator, 50, 1e-8);  // 50 iterations
\end{lstlisting}

3. \textbf{Loosen tolerance temporarily}:
\begin{lstlisting}[language=C++]
DifferentialCorrector corrector(propagator, 20, 1e-6);  // Looser tolerance
\end{lstlisting}

4. \textbf{Filter outliers first}:
\begin{lstlisting}[language=C++]
// Remove observations with large residuals
std::vector<Observation> filtered;
for (const auto& obs : observations) {
    // Manual filtering based on known good range
    if (obs.ra > min_ra && obs.ra < max_ra) {
        filtered.push_back(obs);
    }
}
\end{lstlisting}

\subsection{Oscillating RMS}

\textbf{Symptom}: RMS residual oscillates, doesn't decrease monotonically.

\textbf{Cause}: Step size too large or poor observation weighting.

\textbf{Solutions}:

1. \textbf{Reduce damping}:
\begin{lstlisting}[language=C++]
// In differential corrector implementation
delta_elements = 0.5 * (H.transpose() * H).inverse() * H.transpose() * residuals;
// Reduce multiplier from 1.0 to 0.5
\end{lstlisting}

2. \textbf{Check observation weights}:
\begin{lstlisting}[language=C++]
// Ensure all observations have reasonable uncertainties
for (auto& obs : observations) {
    if (obs.sigma_ra < 0.01 * ARCSEC_TO_RAD) {
        std::cerr << "Warning: Very tight uncertainty\n";
        obs.sigma_ra = 0.1 * ARCSEC_TO_RAD;  // Regularize
    }
}
\end{lstlisting}

\section{Numerical Instabilities}

\subsection{Integration Fails with Small Step Size}

\textbf{Symptom}:
\begin{verbatim}
PropagationError: Step size below minimum
\end{verbatim}

\textbf{Causes}:
\begin{itemize}
    \item Very high eccentricity near perihelion
    \item Close planetary encounter
    \item Singularity in force model
\end{itemize}

\textbf{Solutions}:

1. \textbf{Increase minimum step}:
\begin{lstlisting}[language=C++]
auto integrator = std::make_shared<RKF78>(
    1e-12,    // tolerance
    1e-5,     // min step (increased from 1e-6)
    100.0     // max step
);
\end{lstlisting}

2. \textbf{Use cometary elements for high eccentricity}:
\begin{lstlisting}[language=C++]
if (elem.e > 0.95) {
    // Convert to cometary elements
    CometaryElements comet_elem = elem.to_cometary();
    // Propagate using specialized method
}
\end{lstlisting}

\subsection{Energy Not Conserved}

\textbf{Symptom}: Energy drift $> 10^{-10}$ in unperturbed two-body problem.

\textbf{Diagnostic}:
\begin{lstlisting}[language=C++]
// Compute energy before and after propagation
auto compute_energy = [](const CartesianState& state) {
    double r = state.position.norm();
    double v = state.velocity.norm();
    return 0.5 * v * v - GM_SUN / r;
};

double E0 = compute_energy(state0);
auto state_final = propagator.propagate(state0, state0.epoch + 60.0);
double Ef = compute_energy(state_final);

double dE = std::abs(Ef - E0) / std::abs(E0);
if (dE > 1e-10) {
    std::cerr << "Energy not conserved: " << dE << "\n";
}
\end{lstlisting}

\textbf{Solutions}:

1. \textbf{Tighten tolerance}:
\begin{lstlisting}[language=C++]
auto integrator = std::make_shared<RKF78>(1e-14);  // Tighter
\end{lstlisting}

2. \textbf{Check force model}:
\begin{lstlisting}[language=C++]
// For two-body test, ensure no perturbations
auto forces = std::make_shared<PointMassGravity>(
    eph, std::vector<std::string>{});  // Empty list = Sun only
\end{lstlisting}

\section{Performance Issues}

\subsection{Slow Orbit Determination}

\textbf{Symptom}: Takes $> 10$ seconds for 100 observations.

\textbf{Diagnostic}:
\begin{lstlisting}[language=C++]
auto start = std::chrono::high_resolution_clock::now();

auto result = corrector.solve(initial_guess, observations, obs_coords);

auto end = std::chrono::high_resolution_clock::now();
auto duration = std::chrono::duration_cast<std::chrono::milliseconds>(
    end - start).count();

std::cout << "Time: " << duration << " ms\n";
std::cout << "Steps per propagation: " 
          << integrator->steps_taken() / result.iterations / observations.size() 
          << "\n";
\end{lstlisting}

\textbf{Solutions}:

1. \textbf{Check compilation flags}:
\begin{lstlisting}[language=bash]
# Verify optimization enabled
grep CMAKE_BUILD_TYPE CMakeCache.txt
# Should be "Release", not "Debug"
\end{lstlisting}

2. \textbf{Loosen integration tolerance}:
\begin{lstlisting}[language=C++]
auto integrator = std::make_shared<RKF78>(1e-10);  // Faster, less accurate
\end{lstlisting}

3. \textbf{Reduce force model complexity}:
\begin{lstlisting}[language=C++]
// Instead of all planets:
auto forces = std::make_shared<PointMassGravity>(
    eph, std::vector<std::string>{"JUPITER", "SATURN"}  // Just 2 bodies
);
\end{lstlisting}

\subsection{Memory Usage Grows Over Time}

\textbf{Symptom}: Memory leak in long-running batch processing.

\textbf{Diagnostic}:
\begin{lstlisting}[language=bash]
valgrind --leak-check=full ./astdyn_app
\end{lstlisting}

\textbf{Common causes}:
\begin{itemize}
    \item Not clearing observation vectors between objects
    \item Ephemeris cache growing unbounded
    \item Logging to memory buffer without flushing
\end{itemize}

\textbf{Solution}:
\begin{lstlisting}[language=C++]
for (const auto& object_file : object_files) {
    std::vector<Observation> observations = load_observations(object_file);
    
    // Process object
    auto result = corrector.solve(initial, observations, obs_coords);
    
    // Clear memory
    observations.clear();
    observations.shrink_to_fit();
}
\end{lstlisting}

\section{Observation Issues}

\subsection{Large Residuals (Greater Than 5 arcsec)}

\textbf{Symptom}: RMS residual much larger than expected.

\textbf{Diagnostic}:
\begin{lstlisting}[language=C++]
// Check residual distribution
std::vector<double> residuals_ra, residuals_dec;
for (size_t i = 0; i < observations.size(); ++i) {
    residuals_ra.push_back(result.residuals(2*i));
    residuals_dec.push_back(result.residuals(2*i + 1));
}

// Find outliers
double mean_ra = std::accumulate(residuals_ra.begin(), residuals_ra.end(), 0.0) 
                 / residuals_ra.size();
double threshold = 3.0 * result.rms_residual;

for (size_t i = 0; i < observations.size(); ++i) {
    if (std::abs(residuals_ra[i] - mean_ra) > threshold) {
        std::cout << "Outlier at observation " << i << ": "
                  << observations[i].epoch << "\n";
    }
}
\end{lstlisting}

\textbf{Possible causes}:
\begin{itemize}
    \item Observatory coordinates incorrect
    \item Observation date parsing error
    \item Wrong object (confusion with nearby asteroid)
    \item Instrumental error in observation
\end{itemize}

\subsection{Systematic Bias in Residuals}

\textbf{Symptom}: Residuals consistently positive or negative.

\textbf{Diagnostic}:
\begin{lstlisting}[language=C++]
double mean_residual = result.residuals.mean();
std::cout << "Mean residual: " << mean_residual * RAD_TO_ARCSEC << " arcsec\n";

if (std::abs(mean_residual) > 0.5 * ARCSEC_TO_RAD) {
    std::cout << "Warning: Systematic bias detected\n";
}
\end{lstlisting}

\textbf{Possible causes}:
\begin{itemize}
    \item Incorrect observatory coordinates (elevation, lat/lon)
    \item Missing light-time correction
    \item Stellar aberration not accounted for
    \item Wrong force model (missing major perturbation)
\end{itemize}

\section{Ephemeris Problems}

\subsection{SPICE Kernel Out of Range}

\textbf{Symptom}:
\begin{verbatim}
SPICE Error: Requested time outside kernel coverage
\end{verbatim}

\textbf{Solution}:
\begin{lstlisting}[language=C++]
// Check coverage before using
double jd = 2460000.5;
if (jd < 2287184.5 || jd > 2688976.5) {  // DE440s range
    std::cerr << "Date outside DE440s coverage (1849-2150)\n";
    std::cerr << "Use DE440 (extended) or DE441 (long-term)\n";
}
\end{lstlisting}

\subsection{Different Results with Different Ephemerides}

\textbf{Symptom}: Position differs by $> 5$ km when using DE440 vs. DE441.

\textbf{Explanation}: Normal—different ephemerides have different accuracy models.

\textbf{Check}:
\begin{lstlisting}[language=C++]
// Compare ephemeris outputs
auto pos_de440 = eph_de440->position("JUPITER", 2460000.5);
auto pos_de441 = eph_de441->position("JUPITER", 2460000.5);

double diff = (pos_de440 - pos_de441).norm() * AU_TO_KM;
std::cout << "Jupiter position difference: " << diff << " km\n";
// Expect < 1 km for major planets
\end{lstlisting}

\section{Platform-Specific Issues}

\subsection{Windows: Missing DLLs}

\textbf{Symptom}:
\begin{verbatim}
The code execution cannot proceed because [library].dll was not found.
\end{verbatim}

\textbf{Solution}: Add library directories to PATH:
\begin{lstlisting}[language=bash]
set PATH=%PATH%;C:\path\to\eigen;C:\path\to\boost
\end{lstlisting}

\subsection{macOS: Library Not Loaded}

\textbf{Symptom}:
\begin{verbatim}
dyld: Library not loaded: libastdyn.dylib
\end{verbatim}

\textbf{Solution}:
\begin{lstlisting}[language=bash]
export DYLD_LIBRARY_PATH=/path/to/astdyn/lib:$DYLD_LIBRARY_PATH
\end{lstlisting}

\section{Getting Help}

If problems persist:

\begin{enumerate}
    \item \textbf{Check logs}: Enable verbose logging for diagnostics
    \item \textbf{Minimal example}: Create minimal reproducer
    \item \textbf{GitHub issues}: Report bug with full details
    \item \textbf{Documentation}: Review relevant manual chapters
    \item \textbf{Community}: Ask on project discussions
\end{enumerate}

\subsection{Bug Report Template}

\begin{lstlisting}
**AstDyn Version**: 1.0.0
**OS**: Ubuntu 22.04
**Compiler**: GCC 11.4

**Problem**: Differential correction fails to converge

**Steps to Reproduce**:
1. Load observations from pompeja.obs
2. Run Gauss IOD
3. Call differential corrector

**Expected**: Convergence in 4-5 iterations
**Actual**: Reaches 20 iterations without convergence

**Code**:
[paste minimal code example]

**Output**:
[paste error messages / logs]
\end{lstlisting}

\section{Summary}

Most common issues:

\begin{enumerate}
    \item \textbf{Compilation}: Missing Eigen3 or outdated compiler
    \item \textbf{Convergence}: Poor initial guess or outliers
    \item \textbf{Performance}: Debug build or tight tolerance
    \item \textbf{Parsing}: File format mismatch or encoding issues
    \item \textbf{Numerical}: High eccentricity or close encounter
\end{enumerate}

Follow diagnostic procedures systematically to identify root cause.

\documentclass[12pt]{article}
\usepackage[paperwidth=326.8mm, paperheight=234.6mm, margin=0mm]{geometry}
\usepackage[utf8]{inputenc}
\usepackage[T1]{fontenc}
\usepackage{tikz}
\usepackage{mathpazo} 
\usepackage{anyfontsize}
\usepackage{xcolor}

% Color Definition
\definecolor{deepnavy}{HTML}{0B1021}
\definecolor{accentblue}{HTML}{1E90FF}
\definecolor{stargold}{HTML}{FFD700}

% Force Background Color
\pagecolor{deepnavy}

\begin{document}
\pagestyle{empty}

\begin{tikzpicture}[remember picture, overlay]
    % Anchor to bottom-left
    \begin{scope}[shift={(current page.south west)}, x=1mm, y=1mm]
        
        % --- STAR FIELD ---
        \foreach \i in {1,...,150} {
             \fill[white, opacity=0.4] ({random(5,320)}, {random(5,229)}) circle ({0.1+random(0,2)/10});
        }

        % ==========================================================================================
        % FRONT COVER (Right Side) - Center X = 247.6 mm
        % ==========================================================================================
        
        % Title Block
        \node[anchor=north, align=center] at (247.6, 210) {
            {\fontsize{40}{48}\selectfont \bfseries \textcolor{white}{AstDyn}}\\[6mm]
            {\fontsize{22}{26}\selectfont \bfseries \textcolor{accentblue}{Scientific Reference Manual}}\\[4mm]
            {\fontsize{18}{22}\selectfont \bfseries \textcolor{white}{of the C++ Library}}\\[8mm]
            {\fontsize{14}{17}\selectfont \itshape \textcolor{white!80}{The ITALOccult Framework for High-Precision}}\\
            {\fontsize{14}{17}\selectfont \itshape \textcolor{white!80}{Asteroid Dynamics \& Occultation Prediction}}
        };

        % Graphics
        \begin{scope}[shift={(247.6, 95)}, scale=10.0]
            \shade[inner color=stargold!50!white, outer color=deepnavy] (0,0) circle (2.0);
            \fill[stargold] (0,0) circle (0.15);
            \draw[accentblue!40, thick, rotate=15] (0,0) ellipse (3.5 and 1.2);
            \draw[accentblue!80, very thick, rotate=-10] (0,0) ellipse (2.0 and 1.8);
            \draw[white!20, thin, dashed] (0,0) circle (2.8);
            \shade[ball color=accentblue] (1.9, 0.4) circle (0.12); 
            \shade[ball color=red] (-3.2, -0.6) circle (0.08); 
            \draw[white!40, thin, ->, >=stealth] (0,0) -- node[midway, fill=deepnavy, text=white, font=\tiny, scale=0.2] {$\mathbf{r}$} (-3.2, -0.6);
        \end{scope}

        % Author
        \node[anchor=south] at (247.6, 15) {
            {\fontsize{18}{22}\selectfont \bfseries \textcolor{white}{Michele Bigi}}
        };

        % ==========================================================================================
        % SPINE (Center) - X = 163.4 mm (Width 16mm)
        % ==========================================================================================
        \node[rotate=-90, anchor=center] at (163.4, 190) {
            {\fontsize{14}{17}\selectfont \bfseries \textcolor{white}{AstDyn}}
        };
        % Subtitle might be too long for 234mm height? No, it fits.
        \node[rotate=-90, anchor=center] at (163.4, 117) {
            {\fontsize{10}{12}\selectfont \bfseries \textcolor{accentblue}{Scientific Reference Manual}}
        };
        \node[rotate=-90, anchor=center] at (163.4, 40) {
            {\fontsize{10}{12}\selectfont \bfseries \textcolor{white}{Michele Bigi}}
        };

        % ==========================================================================================
        % BACK COVER (Left Side) - Center X = 79.2 mm
        % ==========================================================================================
        \node[anchor=north west, align=left] at (25, 210) {
            {\fontsize{16}{19}\selectfont \bfseries \textcolor{accentblue}{About the Library}}
        };

        % Abstract Text
        \node[anchor=north west, align=justify, text width=105mm] at (25, 195) {
            \color{white!90}
            \fontsize{10}{12}\selectfont
            \textbf{AstDyn (Asteroid Dynamics)} is a high-fidelity C++ library designed for the precise orbit determination and propagation of celestial bodies, specifically focusing on Main Belt asteroids and Near-Earth Objects (NEOs).
            
            \vspace{0.8em}
            Developed within the \textbf{ITALOccult project}, this software represents the state-of-the-art in computational astrodynamics, aimed at reducing ephemeris uncertainties to sub-milliarcsecond levels.
            
            \vspace{0.8em}
            \textbf{Key Features:}
            \begin{itemize}
                \item[--] \textbf{Rigorous Dynamics}: Relativistic models (EIH) \& JPL DE441.
                \item[--] \textbf{High-Order Integration}: Adaptive RKF78.
                \item[--] \textbf{Precision}: Validated against NASA/JPL Horizons.
            \end{itemize}
        };

        % Bottom Info
        \node[anchor=south west, text width=80mm] at (25, 15) {
            \color{white!60}
            \fontsize{9}{11}\selectfont
            \textbf{ITALOccult Project}\\
            Open Source Scientific Software
        };

        % Barcode Box (White only)
        \fill[white] (100, 15) rectangle (140, 45);

    \end{scope}
\end{tikzpicture}
\end{document}

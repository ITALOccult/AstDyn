\chapter*{Prefazione}
\addcontentsline{toc}{chapter}{Prefazione}

Questo manuale fornisce una guida completa alla libreria AstDyn, un toolkit C++ moderno per calcoli di meccanica celeste e determinazione orbitale. AstDyn è progettato per essere sia educativo che pratico, fornendo implementazioni chiare di algoritmi classici insieme a funzionalità avanzate per applicazioni di ricerca.

\section*{Motivazione}

Lo studio della meccanica celeste richiede una solida comprensione sia della teoria matematica che delle tecniche numeriche. Mentre esistono molti testi eccellenti sulla teoria, c'è spesso un divario tra le equazioni nei libri e il codice funzionante. AstDyn mira a colmare questo divario fornendo:

\begin{itemize}
    \item Implementazioni ben documentate di algoritmi classici
    \item Architettura software moderna e manutenibile
    \item Esempi pratici e casi di validazione
    \item Strumenti per applicazioni reali di ricerca
\end{itemize}

\section*{Struttura del Manuale}

Il manuale è organizzato in cinque parti:

\begin{description}
    \item[Parte I: Fondamenti Teorici] copre i concetti fondamentali: sistemi di tempo e coordinate, elementi orbitali, il problema dei due corpi e teoria delle perturbazioni.
    
    \item[Parte II: Metodi Numerici] descrive l'integrazione numerica, la propagazione orbitale, matrici di transizione di stato e generazione di effemeridi.
    
    \item[Parte III: Determinazione Orbitale] spiega come determinare orbite da osservazioni usando metodi classici e moderni.
    
    \item[Parte IV: Implementazione] documenta l'architettura della libreria, i moduli core, il sistema di parser configurabile e l'API completa.
    
    \item[Parte V: Validazione e Applicazioni] presenta casi di studio reali, analisi di performance e best practices.
\end{description}

\section*{Pubblico di Riferimento}

Questo manuale è rivolto a:

\begin{itemize}
    \item Studenti di laurea magistrale e dottorato in astrofisica o ingegneria aerospaziale
    \item Ricercatori che necessitano di strumenti affidabili per determinazione orbitale
    \item Sviluppatori di software che lavorano su applicazioni di meccanica celeste
    \item Chiunque sia interessato a comprendere gli algoritmi dietro i calcoli orbitali
\end{itemize}

Si assume familiarità con il calcolo multivariabile, l'algebra lineare e la programmazione C++. La conoscenza della meccanica classica è utile ma non strettamente necessaria.

\section*{Convenzioni}

\begin{itemize}
    \item I \textbf{vettori} sono indicati in grassetto: $\mathbf{r}$, $\mathbf{v}$
    \item Le \textbf{matrici} sono indicate in maiuscolo grassetto: $\mathbf{A}$, $\mathbf{H}$
    \item Gli \textbf{scalari} sono in corsivo normale: $a$, $e$, $\mu$
    \item Le unità SI sono utilizzate salvo diversa indicazione
    \item Il codice C++ è mostrato con syntax highlighting
\end{itemize}

\section*{Risorse Online}

\begin{itemize}
    \item Repository GitHub: \url{https://github.com/your-org/astdyn}
    \item Documentazione API: \url{https://astdyn.readthedocs.io}
    \item Issues e supporto: \url{https://github.com/your-org/astdyn/issues}
\end{itemize}

\vspace{1cm}

\noindent
Buona lettura e buoni calcoli orbitali!

\vspace{0.5cm}

\noindent
\textit{Gli Autori}


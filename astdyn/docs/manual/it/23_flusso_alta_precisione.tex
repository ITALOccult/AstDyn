\chapter{Flusso di Lavoro ad Alta Precisione}
\label{ch:high_precision}

Questo capitolo descrive il flusso di lavoro raccomandato per ottenere risultati astrometrici di alta precisione utilizzando le funzionalità avanzate di AstDyn.

\section{Astrometria Rigorosa}

Per garantire residui sub-arcosecondo e orbite di alta fedeltà, AstDyn implementa un modello astrometrico rigoroso che include:

\begin{itemize}
    \item \textbf{Sistemi di Riferimento Coerenti}: Tutte le integrazioni avvengono nel sistema baricentrico (ICRF), ma la posizione dell'osservatore (Terra) viene attentamente riferita al Sole per calcolare correttamente i vettori eliocentrici quando necessario.
    \item \textbf{Aberrazione Annua}: Viene applicata utilizzando la velocità della Terra rispetto al Baricentro del Sistema Solare (SSB).
    \item \textbf{Correzione Tempo di Luce}: Calcolata iterativamente per tenere conto del tempo impiegato dai fotoni per raggiungere l'osservatore.
\end{itemize}

\subsection{Configurazione del Propagatore}

Per attivare la modalità ad alta precisione, configurare il propagatore come segue:

\begin{lstlisting}[style=cpp,caption={Configurazione per Alta Precisione}]
PropagationSettings settings;
settings.include_planets = true;     // Perturbazioni planetarie (DE441)
settings.include_asteroids = true;   // Perturbazioni dei 300+ asteroidi maggiori
settings.include_relativity = true;  // Relatività Generale (PPN)
settings.time_scale = TimeScale::TDB; // Utilizzo rigoroso del Tempo Dinamico
\end{lstlisting}

\section{Gestione del Tempo (TimeScale)}

La precisione temporale è critica. Un errore di 1 secondo può corrispondere a decine di chilometri di errore orbitale.

\subsection{Caricamento Dati IERS}

Per la massima accuratezza nella rotazione terrestre (necessaria per osservatori da terra), è indispensabile caricare i prodotti IERS (International Earth Rotation Service):

\begin{lstlisting}[style=cpp,caption={Caricamento EOP}]
// Carica i dati Earth Orientation Parameters (UT1-UTC, Polar Motion)
bool ok = astdyn::time::load_dut1_data("finals.all");
if (!ok) {
    std::cerr << "Attenzione: EOP non caricati. Usata approssimazione." << std::endl;
}
\end{lstlisting}

\subsection{API di Conversione}

Il modulo \texttt{TimeScale} offre funzioni collaudate per le conversioni:

\begin{itemize}
    \item \texttt{utc\_to\_tdb(mjd\_utc)}: Da usare per preparare i tempi di integrazione.
    \item \texttt{utc\_to\_ut1(mjd\_utc)}: Usato internamente per il calcolo del Tempo Siderale (GMST/LAST).
\end{itemize}

\section{Best Practices per l'Analisi}

\begin{enumerate}
    \item \textbf{Inizializzazione}: Se possibile, inizializzare l'orbita con un vettore di stato Cartesiano (Posizione/Velocità) non singolare, idealmente ottenuto da JPL Horizons, per evitare le approssimazioni intrinseche degli elementi osculatori Kepleriani in epoche lontane.
    \item \textbf{Effemeridi}: Utilizzare sempre file SPK binari (es. \texttt{de441.bsp}) per la posizione dei pianeti. Le formule analitiche approssimate non sono sufficienti per l'astrometria di precisione moderna.
    \item \textbf{Verifica}: Controllare sempre che l'RMS post-fit sia compatibile con l'errore del catalogo stellare utilizzato (tipicamente $< 0.5$ arcsec per Gaia DR3).
\end{enumerate}

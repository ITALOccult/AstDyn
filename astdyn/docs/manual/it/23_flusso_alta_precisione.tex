\chapter{Flusso di Lavoro ad Alta Precisione}
\label{ch:high_precision}

Questo capitolo descrive il flusso di lavoro raccomandato per ottenere risultati astrometrici di alta precisione utilizzando le funzionalità avanzate di AstDyn.

\section{Astrometria Rigorosa}

Per garantire residui sub-arcosecondo e orbite di alta fedeltà, AstDyn implementa un modello astrometrico rigoroso che include:

\begin{itemize}
    \item \textbf{Sistemi di Riferimento Coerenti}: Tutte le integrazioni avvengono nel sistema baricentrico (ICRF), ma la posizione dell'osservatore (Terra) viene attentamente riferita al Sole per calcolare correttamente i vettori eliocentrici quando necessario.
    \item \textbf{Aberrazione Annua}: Viene applicata utilizzando la velocità della Terra rispetto al Baricentro del Sistema Solare (SSB).
    \item \textbf{Correzione Tempo di Luce}: Calcolata iterativamente per tenere conto del tempo impiegato dai fotoni per raggiungere l'osservatore.
\end{itemize}

\subsection{Configurazione del Propagatore}

Per attivare la modalità ad alta precisione, configurare il propagatore come segue:

\begin{lstlisting}[style=cpp,caption={Configurazione per Alta Precisione}]
PropagationSettings settings;
settings.include_planets = true;     // Perturbazioni planetarie (DE441)
settings.include_asteroids = true;   // Perturbazioni dei 300+ asteroidi maggiori
settings.include_relativity = true;  // Relatività Generale (PPN)
settings.time_scale = TimeScale::TDB; // Utilizzo rigoroso del Tempo Dinamico
\end{lstlisting}

\section{Gestione del Tempo (TimeScale)}

La precisione temporale è critica. Un errore di 1 secondo può corrispondere a decine di chilometri di errore orbitale.

\subsection{Caricamento Dati IERS}

Per la massima accuratezza nella rotazione terrestre (necessaria per osservatori da terra), è indispensabile caricare i prodotti IERS (International Earth Rotation Service):

\begin{lstlisting}[style=cpp,caption={Caricamento EOP}]
// Carica i dati Earth Orientation Parameters (UT1-UTC, Polar Motion)
bool ok = astdyn::time::load_dut1_data("finals.all");
if (!ok) {
    std::cerr << "Attenzione: EOP non caricati. Usata approssimazione." << std::endl;
}
\end{lstlisting}

\subsection{API di Conversione}

Il modulo \texttt{TimeScale} offre funzioni collaudate per le conversioni:

\begin{itemize}
    \item \texttt{utc\_to\_tdb(mjd\_utc)}: Da usare per preparare i tempi di integrazione.
    \item \texttt{utc\_to\_ut1(mjd\_utc)}: Usato internamente per il calcolo del Tempo Siderale (GMST/LAST).
\end{itemize}

\section{API di Alto Livello: HighPrecisionPropagator}
\label{sec:high_precision_api}

Per semplificare il flusso di lavoro rigoroso, AstDyn fornisce la classe \texttt{HighPrecisionPropagator}. Questa API di alto livello incapsula le impostazioni N-body, la correzione del tempo di luce e la riduzione geocentrica in un'unica chiamata.

\begin{lstlisting}[style=cpp, caption={Utilizzo di HighPrecisionPropagator}]
using namespace astdyn::propagation;

// 1. Configurazione per la massima precisione
HighPrecisionPropagator::Config config;
config.de441_path = "/percorso/per/de441.bsp";
config.perturbations_planets = true;
config.relativity = true;

HighPrecisionPropagator propagator(config);

// 2. Calcolo della posizione geocentrica (RA/Dec) con correzione tempo di luce
auto result = propagator.calculateGeocentricObservation(elements, target_jd);

std::cout << "RA: " << result.ra_deg << " deg" << std::endl;
std::cout << "Dec: " << result.dec_deg << " deg" << std::endl;
\end{lstlisting}

\section{Integrazione Semplificata: ITALOccultLibrary}
\label{sec:italoccultlib_integration}

Per rendere accessibili queste funzionalità nel progetto \texttt{IOccultCalc}, è stata creata la libreria \texttt{ITALOccultLibrary}. Questa agisce come un wrapper che automatizza i passaggi critici.

\subsection{Conversione Automatica Mean $\to$ Osculating}
Uno degli errori più comuni è l'utilizzo diretto degli elementi medi di AstDyS (\texttt{.eq1}) in un propagatore numerico N-Body. Questo introduce un errore sistematico di circa $360$ arcsec. Il wrapper \texttt{AstDynWrapper} risolve questo problema applicando automaticamente la conversione:

\begin{lstlisting}[style=cpp, caption={AstDynWrapper per Alta Precisione}]
#include <italoccultlib/astdyn_wrapper.h>

ioccultcalc::AstDynWrapper wrapper;
// Carica e converte automaticamente Mean (Ecl) -> Osculating (Equatorial)
wrapper.loadFromEQ1File("34713.eq1");

// Calcola osservazione geocentrica con DE441 part-2
auto obs = wrapper.calculateObservation(61050.0);
std::cout << "RA: " << obs.ra_deg << " deg" << std::endl;
\end{lstlisting}

\section{Coerenza del Sistema di Riferimento}
\label{sec:frame_consistency}

Una fonte comune di errore sistematico è la confusione tra sistemi \textbf{Eliocentrici} e \textbf{Baricentrici}.
\begin{itemize}
    \item \textbf{Propagatore AstDyn}: Integra le equazioni del moto in un sistema \textit{Eliocentrico} (origine nel centro del Sole).
    \item \textbf{JPL DE441}: Fornisce posizioni relative al \textit{Baricentro del Sistema Solare} (SSB).
\end{itemize}

L'offset tra il Sole e l'SSB può arrivare a $\sim 0.01$ AU, principalmente a causa di Giove. Trascurare questo offset introduce un errore sistematico di diversi arcominuti.

\subsection{Logica di Correzione del Frame}
La classe \texttt{HighPrecisionPropagator} gestisce queste trasformazioni internamente. Se si implementa una riduzione personalizzata, assicurarsi che la posizione della Terra venga convertita nel sistema Eliocentrico sottraendo la posizione del Sole rispetto all'SSB:

\begin{equation}
    \mathbf{r}_{earth}^{helio} = \mathbf{r}_{earth}^{bary}(SSB) - \mathbf{r}_{sun}^{bary}(SSB)
\end{equation}

Questo garantisce che sia l'asteroide (propagato eliocentricamente) che l'osservatore (Terra) siano definiti nello stesso sistema di coordinate.

\chapter{Appendici}
\label{ch:appendices}

\section{Costanti Fisiche}
\label{app:constants}

\subsection{Costanti Fondamentali}

\begin{table}[H]
\centering
\caption{Costanti fisiche fondamentali}
\begin{tabular}{llc}
\hline
\textbf{Costante} & \textbf{Valore} & \textbf{Unit\`a} \\
\hline
Velocita' luce $c$ & $299{,}792{,}458$ & m/s \\
Costante gravitazionale $G$ & $6.67430 \times 10^{-11}$ & m$^3$ kg$^{-1}$ s$^{-2}$ \\
Unita' astronomica AU & $1.495978707 \times 10^{11}$ & m \\
Parsec pc & $3.0857 \times 10^{16}$ & m \\
Anno giuliano & $365.25$ & giorni \\
Secolo giuliano & $36{,}525$ & giorni \\
\hline
\end{tabular}
\end{table}

\subsection{Parametri Sistema Solare}

\begin{table}[H]
\centering
\caption{Masse planetarie (parametri GM in AU$^3$/giorno$^2$)}
\begin{tabular}{lcc}
\hline
\textbf{Corpo} & \textbf{GM} & \textbf{Massa (M$_\odot$)} \\
\hline
Sole & $2.959122 \times 10^{-4}$ & $1.0$ \\
Mercurio & $4.9125 \times 10^{-11}$ & $1.66 \times 10^{-7}$ \\
Venere & $7.2435 \times 10^{-10}$ & $2.45 \times 10^{-6}$ \\
Terra & $8.8877 \times 10^{-10}$ & $3.00 \times 10^{-6}$ \\
Luna & $1.0931 \times 10^{-11}$ & $3.69 \times 10^{-8}$ \\
Marte & $9.5496 \times 10^{-11}$ & $3.23 \times 10^{-7}$ \\
Giove & $2.8253 \times 10^{-7}$ & $9.55 \times 10^{-4}$ \\
Saturno & $8.4597 \times 10^{-8}$ & $2.86 \times 10^{-4}$ \\
Urano & $1.2920 \times 10^{-8}$ & $4.37 \times 10^{-5}$ \\
Nettuno & $1.5244 \times 10^{-8}$ & $5.15 \times 10^{-5}$ \\
\hline
\end{tabular}
\end{table}

\section{Elementi Orbitali Planetari}
\label{app:planet_elements}

\subsection{Elementi Medi J2000.0}

\begin{table}[H]
\centering
\caption{Elementi orbitali planetari (epoca J2000.0)}
\begin{tabular}{lcccccc}
\hline
\textbf{Pianeta} & $a$ (AU) & $e$ & $i$ (°) & $\Omega$ (°) & $\omega$ (°) & $L$ (°) \\
\hline
Mercurio & 0.387 & 0.206 & 7.00 & 48.3 & 29.1 & 252.3 \\
Venere & 0.723 & 0.007 & 3.39 & 76.7 & 54.9 & 181.9 \\
Terra & 1.000 & 0.017 & 0.00 & — & 288.1 & 100.5 \\
Marte & 1.524 & 0.093 & 1.85 & 49.6 & 286.5 & 355.4 \\
Giove & 5.203 & 0.048 & 1.30 & 100.5 & 273.9 & 34.4 \\
Saturno & 9.537 & 0.054 & 2.49 & 113.7 & 339.4 & 50.1 \\
Urano & 19.191 & 0.047 & 0.77 & 74.0 & 96.7 & 314.1 \\
Nettuno & 30.069 & 0.009 & 1.77 & 131.8 & 273.2 & 304.3 \\
\hline
\end{tabular}
\end{table}

\section{Conversioni Unit\`a}
\label{app:conversions}

\subsection{Conversioni Angolari}

\begin{table}[H]
\centering
\caption{Fattori conversione angolare}
\begin{tabular}{lll}
\hline
\textbf{Da} & \textbf{A} & \textbf{Fattore} \\
\hline
radianti & gradi & $180/\pi = 57.2958$ \\
gradi & radianti & $\pi/180 = 0.0174533$ \\
radianti & arcosecondi & $206{,}265$ \\
arcosecondi & radianti & $4.8481 \times 10^{-6}$ \\
ore & gradi & $15$ \\
gradi & ore & $1/15 = 0.0666667$ \\
\hline
\end{tabular}
\end{table}

\subsection{Conversioni Temporali}

\begin{table}[H]
\centering
\caption{Fattori conversione temporale}
\begin{tabular}{lll}
\hline
\textbf{Da} & \textbf{A} & \textbf{Fattore} \\
\hline
giorni & secondi & $86{,}400$ \\
secondi & giorni & $1.15741 \times 10^{-5}$ \\
anni giuliani & giorni & $365.25$ \\
secoli giuliani & giorni & $36{,}525$ \\
MJD & JD & $+2{,}400{,}000.5$ \\
\hline
\end{tabular}
\end{table}

\subsection{Conversioni Distanza}

\begin{table}[H]
\centering
\caption{Fattori conversione distanza}
\begin{tabular}{lll}
\hline
\textbf{Da} & \textbf{A} & \textbf{Fattore} \\
\hline
AU & km & $149{,}597{,}871$ \\
AU & m & $1.496 \times 10^{11}$ \\
km & AU & $6.6846 \times 10^{-9}$ \\
parsec & AU & $206{,}265$ \\
anno luce & AU & $63{,}241$ \\
\hline
\end{tabular}
\end{table}

\section{Formule Utili}
\label{app:formulas}

\subsection{Problema Due Corpi}

Energia specifica:
$$\varepsilon = \frac{v^2}{2} - \frac{\mu}{r} = -\frac{\mu}{2a}$$

Momento angolare specifico:
$$h = r v \cos \gamma = \sqrt{\mu a (1 - e^2)}$$

Periodo orbitale:
$$P = 2\pi \sqrt{\frac{a^3}{\mu}}$$

Velocita' circolare:
$$v_c = \sqrt{\frac{\mu}{r}}$$

Velocita' di fuga:
$$v_e = \sqrt{\frac{2\mu}{r}}$$

\subsection{Anomalie}

Anomalia media:
$$M = n (t - t_0) = \sqrt{\frac{\mu}{a^3}} (t - t_0)$$

Equazione Keplero:
$$M = E - e \sin E$$

Anomalia vera da eccentrica:
$$\tan \frac{f}{2} = \sqrt{\frac{1+e}{1-e}} \tan \frac{E}{2}$$

\subsection{Trasformazioni Coordinate}

Rotazione asse x:
$$R_x(\theta) = \begin{pmatrix} 1 & 0 & 0 \\ 0 & \cos\theta & -\sin\theta \\ 0 & \sin\theta & \cos\theta \end{pmatrix}$$

Rotazione asse z:
$$R_z(\theta) = \begin{pmatrix} \cos\theta & -\sin\theta & 0 \\ \sin\theta & \cos\theta & 0 \\ 0 & 0 & 1 \end{pmatrix}$$

\section{Codici Osservatorio MPC}
\label{app:obs_codes}

\subsection{Osservatori Principali}

\begin{table}[H]
\centering
\caption{Codici osservatorio MPC selezionati}
\begin{tabular}{lll}
\hline
\textbf{Codice} & \textbf{Nome} & \textbf{Localita'} \\
\hline
F51 & Pan-STARRS 1 & Haleakala, Hawaii, USA \\
G96 & Mt. Lemmon Survey & Arizona, USA \\
703 & Catalina Sky Survey & Arizona, USA \\
691 & Steward Observatory & Arizona, USA \\
568 & Mauna Kea & Hawaii, USA \\
J75 & OCA-DLR Survey & Germania \\
C51 & Pan-STARRS 2 & Haleakala, Hawaii, USA \\
I41 & Palomar Mountain ZTF & California, USA \\
V00 & ATLAS-MLO & Mauna Loa, Hawaii, USA \\
\hline
\end{tabular}
\end{table}

\section{Acronimi e Abbreviazioni}
\label{app:acronyms}

\begin{table}[H]
\centering
\caption{Acronimi comuni}
\begin{tabular}{ll}
\hline
\textbf{Acronimo} & \textbf{Significato} \\
\hline
API & Application Programming Interface \\
AU & Astronomical Unit (Unita' Astronomica) \\
CCD & Charge-Coupled Device \\
DE & Development Ephemeris (JPL) \\
EDO & Equazioni Differenziali Ordinarie \\
IAU & International Astronomical Union \\
ICRF & International Celestial Reference Frame \\
IOD & Initial Orbit Determination \\
JPL & Jet Propulsion Laboratory (NASA) \\
JD & Julian Day (Giorno Giuliano) \\
MJD & Modified Julian Day \\
MPC & Minor Planet Center \\
NEA & Near-Earth Asteroid \\
RMS & Root Mean Square \\
SPICE & Spacecraft Planet Instrument C-matrix Events \\
STM & State Transition Matrix \\
TAI & Temps Atomique International \\
TDB & Barycentric Dynamical Time \\
TT & Terrestrial Time \\
UTC & Coordinated Universal Time \\
\hline
\end{tabular}
\end{table}

\section{Convenzioni Notazione}
\label{app:notation}

\subsection{Vettori e Matrici}

\begin{itemize}
    \item Vettori: $\mathbf{r}$, $\mathbf{v}$ (grassetto minuscolo)
    \item Matrici: $\mathbf{R}$, $\mathbf{M}$ (grassetto maiuscolo)
    \item Versori: $\hat{\mathbf{x}}$, $\hat{\mathbf{e}}_i$ (cappello)
    \item Componenti: $r_x$, $v_i$ (pedice)
    \item Norma: $|\mathbf{r}| = r$ (barre verticali o senza grassetto)
\end{itemize}

\subsection{Elementi Orbitali}

\begin{itemize}
    \item $a$: Semiasse maggiore
    \item $e$: Eccentricita'
    \item $i$: Inclinazione
    \item $\Omega$: Longitudine nodo ascendente
    \item $\omega$: Argomento perielio
    \item $M$: Anomalia media
    \item $E$: Anomalia eccentrica
    \item $f$: Anomalia vera
\end{itemize}

\subsection{Sistemi Coordinate}

\begin{itemize}
    \item ICRF: International Celestial Reference Frame
    \item Eclittica: Piano orbitale Terra
    \item Equatore: Piano equatoriale Terra
    \item $(x, y, z)$: Coordinate cartesiane
    \item $(\alpha, \delta)$: Ascensione retta, declinazione
    \item $(\lambda, \beta)$: Longitudine, latitudine eclittica
\end{itemize}

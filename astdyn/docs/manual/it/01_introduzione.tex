\chapter{Introduzione}
\label{ch:introduction}

\section{Cos'è la Meccanica Celeste?}

La meccanica celeste è il ramo dell'astronomia che studia i moti dei corpi celesti sotto l'influenza delle forze gravitazionali. Fornisce il quadro matematico e fisico per comprendere:

\begin{itemize}
    \item Le orbite di pianeti, lune, asteroidi e comete
    \item La progettazione di traiettorie spaziali e l'analisi di missioni
    \item La stabilità a lungo termine del sistema solare
    \item Gli effetti mareali e la dinamica rotazionale
    \item La formazione ed evoluzione dei sistemi planetari
\end{itemize}

Il campo ha una storia illustre, che inizia con le leggi empiriche del moto planetario di Johannes Kepler (1609-1619) e la legge di gravitazione universale di Isaac Newton (1687). Newton dimostrò che le leggi di Kepler potevano essere derivate da principi fisici fondamentali, segnando la nascita della meccanica celeste teorica.

\subsection{Il Problema dei Due Corpi}

La pietra angolare della meccanica celeste è il \textit{problema dei due corpi}: determinare il moto di due masse puntiformi che interagiscono esclusivamente attraverso l'attrazione gravitazionale reciproca. Questo problema ha un'elegante soluzione analitica, espressa in termini di sei \textit{elementi orbitali} che specificano completamente l'orbita.

Consideriamo due corpi con masse $m_1$ e $m_2$, separati da una distanza $r$. La legge di gravitazione di Newton afferma:

\begin{equation}
F = G\frac{m_1 m_2}{r^2}
\end{equation}

dove $G = 6.67430 \times 10^{-11}$ m$^3$ kg$^{-1}$ s$^{-2}$ è la costante gravitazionale.

Per un corpo piccolo di massa $m$ che orbita attorno a un corpo molto più grande di massa $M$ (come un asteroide che orbita attorno al Sole), possiamo approssimare il sistema come un problema a un corpo con il corpo massiccio all'origine. L'equazione del moto diventa:

\begin{equation}
\ddot{\mathbf{r}} = -\frac{\mu}{r^3}\mathbf{r}
\end{equation}

dove $\mu = GM$ è il parametro gravitazionale e $\mathbf{r}$ è il vettore posizione del corpo piccolo.

\subsection{Il Problema degli N Corpi}

In realtà, i corpi celesti esistono in sistemi con molteplici oggetti gravitanti. Il sistema solare, per esempio, contiene il Sole, otto pianeti maggiori, numerose lune, asteroidi e comete—tutti che esercitano forze gravitazionali gli uni sugli altri. Questo è il \textit{problema degli N corpi}.

A differenza del problema dei due corpi, il problema degli N corpi non ha una soluzione analitica generale per $N \geq 3$. Invece, dobbiamo ricorrere a:

\begin{enumerate}
    \item \textbf{Teoria delle perturbazioni}: Trattare le forze aggiuntive come piccole correzioni a una soluzione a due corpi
    \item \textbf{Integrazione numerica}: Calcolare le orbite passo dopo passo usando i computer
    \item \textbf{Soluzioni speciali}: Risultati analitici per casi ristretti (es. punti di Lagrange)
\end{enumerate}

La libreria AstDyn implementa tutti e tre gli approcci, con enfasi sulla teoria delle perturbazioni e sull'integrazione numerica ad alta precisione.

\section{Panoramica della Libreria AstDyn}

\subsection{Filosofia di Progettazione}

La libreria AstDyn è costruita su diversi principi fondamentali:

\begin{description}
    \item[Accuratezza] I metodi numerici sono scelti e calibrati per alta precisione, validati contro software consolidati
    \item[Modularità] I componenti sono debolmente accoppiati, permettendo agli utenti di impiegare solo le funzionalità necessarie
    \item[Chiarezza] Il codice è documentato con riferimenti a formulazioni matematiche e letteratura
    \item[Prestazioni] Gli algoritmi sono ottimizzati usando caratteristiche moderne di C++ senza sacrificare la leggibilità
    \item[Estensibilità] L'architettura supporta l'aggiunta di nuovi integratori, modelli di forza e tipi di osservazione
\end{description}

\subsection{Caratteristiche Principali}

La libreria fornisce:

\begin{itemize}
    \item \textbf{Sistemi temporali}: Conversioni tra UTC, TAI, TT, TDB con modelli accurati di $\Delta T$
    \item \textbf{Sistemi di coordinate}: Trasformazioni tra sistemi eclittico, equatoriale e planetario
    \item \textbf{Elementi orbitali}: Rappresentazioni kepleriane, cartesiane, equinoziali e di Delaunay
    \item \textbf{Integrazione numerica}: Runge-Kutta, Adams-Bashforth-Moulton e metodi adattativi
    \item \textbf{Modelli di forza}: Gravitazione a N corpi, perturbazioni asteroidali, effetti relativistici
    \item \textbf{Propagazione orbitale}: Integrazione avanti/indietro con matrice di transizione di stato
    \item \textbf{Determinazione dell'orbita iniziale}: Metodo di Gauss per tre osservazioni
    \item \textbf{Correzione differenziale}: Adattamento dell'orbita ai minimi quadrati alle osservazioni astrometriche
    \item \textbf{Effemeridi}: Posizioni planetarie usando VSOP87 e DE440/441
    \item \textbf{I/O dati}: Parser per OrbFit (.eq1, .rwo), MPC e formati personalizzati
\end{itemize}

\subsection{Architettura Software}

La Figura~\ref{fig:architecture} illustra l'architettura di alto livello:

\begin{figure}[H]
\centering
\begin{tikzpicture}[
    node distance=1.5cm,
    box/.style={rectangle, draw, fill=blue!10, text width=3cm, align=center, rounded corners, minimum height=1cm},
    arrow/.style={->, >=stealth, thick}
]
    % Top layer
    \node[box] (engine) {AstDynEngine\\{\small API di alto livello}};
    
    % Second layer
    \node[box, below left=of engine] (prop) {Propagazione};
    \node[box, below right=of engine] (od) {Determinazione\\Orbita};
    
    % Third layer
    \node[box, below=of prop] (integrator) {Integrazione\\Numerica};
    \node[box, below=of od] (obs) {Osservazioni\\\& Residui};
    
    % Fourth layer
    \node[box, below=of integrator] (coord) {Coordinate\\\& Tempo};
    \node[box, below=of obs] (io) {I/O Dati\\Parser};
    
    % Bottom layer
    \node[box, below=2cm of coord] (math) {Matematica \& Algebra Lineare\\{\small Eigen3}};
    
    % Arrows
    \draw[arrow] (engine) -- (prop);
    \draw[arrow] (engine) -- (od);
    \draw[arrow] (prop) -- (integrator);
    \draw[arrow] (od) -- (obs);
    \draw[arrow] (integrator) -- (coord);
    \draw[arrow] (obs) -- (io);
    \draw[arrow] (prop) -- (coord);
    \draw[arrow] (od) -- (coord);
    \draw[arrow] (integrator) -- (math);
    \draw[arrow] (obs) -- (math);
\end{tikzpicture}
\caption{Architettura della libreria AstDyn con design stratificato}
\label{fig:architecture}
\end{figure}

L'architettura segue un design stratificato:

\begin{enumerate}
    \item \textbf{Livello fondamentale}: Utilità matematiche e algebra lineare (Eigen3)
    \item \textbf{Livello core}: Sistemi temporali, trasformazioni di coordinate, elementi orbitali
    \item \textbf{Livello algoritmico}: Integrazione numerica, gestione osservazioni
    \item \textbf{Livello applicativo}: Propagazione orbitale, determinazione orbita
    \item \textbf{Livello interfaccia}: API di alto livello (AstDynEngine), parser dati
\end{enumerate}

\subsection{Dipendenze}

AstDyn si basa su librerie consolidate:

\begin{description}
    \item[Eigen3] Operazioni di algebra lineare (matrici, vettori, decomposizioni)
    \item[Boost] Filesystem, date-time, opzioni programma
    \item[GoogleTest] Framework per unit testing (opzionale)
\end{description}

Tutte le dipendenze sono ampiamente disponibili e attivamente mantenute.

\section{Applicazioni}

La libreria AstDyn supporta varie applicazioni:

\subsection{Determinazione di Orbite Asteroidali}

Date osservazioni astrometriche (ascensione retta e declinazione) di un asteroide da telescopi terrestri, determinare la sua orbita eliocentrica. Questo è cruciale per:

\begin{itemize}
    \item Prevedere posizioni future per campagne osservative
    \item Valutare il rischio di collisione con la Terra
    \item Pianificare missioni spaziali
    \item Comprendere popolazioni e dinamica degli asteroidi
\end{itemize}

Esempio: Il Capitolo~\ref{ch:case_studies} presenta un'analisi completa dell'asteroide 203 Pompeja utilizzando 100 osservazioni recenti, ottenendo residui RMS di 0.66 arcosecondi.

\subsection{Analisi di Traiettorie Spaziali}

Progettare e analizzare traiettorie spaziali per:

\begin{itemize}
    \item Trasferimenti interplanetari
    \item Manovre orbitali
    \item Operazioni di station-keeping
    \item Analisi di avvicinamenti ravvicinati
\end{itemize}

La propagazione ad alta precisione della libreria e la capacità di calcolare matrici di transizione di stato la rendono adatta per la progettazione preliminare di missioni.

\subsection{Evoluzione Orbitale a Lungo Termine}

Studiare il comportamento a lungo termine dei corpi minori sotto perturbazioni planetarie:

\begin{itemize}
    \item Evoluzione secolare degli elementi orbitali
    \item Identificazione di risonanze
    \item Analisi di caos e stabilità
    \item Stima della probabilità di impatto
\end{itemize}

\subsection{Strumento Educativo}

La libreria serve come risorsa educativa per studenti che apprendono:

\begin{itemize}
    \item Implementazione pratica di algoritmi da manuale
    \item Metodi numerici in astrodinamica
    \item Ingegneria del software per calcolo scientifico
    \item Tecniche moderne di programmazione C++
\end{itemize}

\section{Validazione e Accuratezza}

Un punto di forza chiave di AstDyn è la rigorosa validazione contro software consolidati:

\begin{itemize}
    \item \textbf{OrbFit}: Il confronto dei risultati di determinazione orbitale per l'asteroide 203 Pompeja mostra un accordo di $\Delta a = 578$ km, $\Delta e = 0.0006$, $\Delta i = 5''$
    \item \textbf{JPL Horizons}: I confronti di effemeridi validano i modelli di perturbazione planetaria
    \item \textbf{Soluzioni analitiche}: La propagazione a due corpi è testata contro le formule kepleriane
\end{itemize}

Studi di validazione dettagliati sono presentati nel Capitolo~\ref{ch:validation}.

\section{Per Iniziare}

\subsection{Installazione}

La libreria può essere compilata usando CMake:

\begin{lstlisting}[style=cpp,caption={Compilazione di AstDyn}]
git clone https://github.com/manvalan/ITALOccultLibrary.git
cd ITALOccultLibrary/astdyn
mkdir build && cd build
cmake .. -DCMAKE_BUILD_TYPE=Release
make -j8
\end{lstlisting}

Questo produce:
\begin{itemize}
    \item \texttt{libastdyn.a} (libreria statica, 1.5 MB, 1232 simboli)
    \item \texttt{libastdyn.dylib} (libreria condivisa, 877 KB)
\end{itemize}

\subsection{Esempio Rapido}

Un esempio minimale di propagazione orbitale:

\begin{lstlisting}[style=cpp,caption={Propagazione orbitale di base}]
#include <astdyn/AstDyn.hpp>
using namespace astdyn;

int main() {
    // Definire elementi orbitali (asteroide in AU, radianti)
    propagation::KeplerianElements orbit;
    orbit.epoch = 61000.0;  // MJD TDB
    orbit.a = 2.7;          // semiasse maggiore (AU)
    orbit.e = 0.15;         // eccentricita
    orbit.i = 10.0 * constants::DEG_TO_RAD;
    orbit.Omega = 80.0 * constants::DEG_TO_RAD;
    orbit.omega = 73.0 * constants::DEG_TO_RAD;
    orbit.M = 45.0 * constants::DEG_TO_RAD;
    orbit.gm = constants::GMS;  // GM del Sole
    
    // Creare propagatore
    propagation::Propagator prop;
    
    // Propagare 1 anno in avanti
    double target_mjd = orbit.epoch + 365.25;
    auto result = prop.propagate_keplerian(orbit, target_mjd);
    
    // Stampare risultati
    std::cout << "Posizione: " << result.position.transpose() << " AU\n";
    std::cout << "Velocita: " << result.velocity.transpose() << " AU/giorno\n";
    
    return 0;
}
\end{lstlisting}

Esempi più completi sono forniti nel Capitolo~\ref{ch:examples}.

\section{Organizzazione dei Capitoli Rimanenti}

Il resto di questo manuale è organizzato come segue:

\textbf{Capitoli 2-7} (Parte I) stabiliscono le fondamenta teoriche: sistemi temporali, coordinate, elementi orbitali, dinamica a due corpi e perturbazioni.

\textbf{Capitoli 8-11} (Parte II) descrivono metodi numerici: algoritmi di integrazione, propagazione, matrici di transizione di stato e calcolo di effemeridi.

\textbf{Capitoli 12-15} (Parte III) coprono la determinazione orbitale: modelli di osservazione, determinazione dell'orbita iniziale, correzione differenziale e analisi dei residui.

\textbf{Capitoli 16-20} (Parte IV) documentano l'implementazione della libreria: architettura, moduli core, parser, riferimento API ed esempi.

\textbf{Capitoli 21-23} (Parte V) presentano studi di validazione, applicazioni del mondo reale e benchmark delle prestazioni.

Ogni capitolo include derivazioni matematiche, note di implementazione ed esempi di codice funzionanti per collegare teoria e pratica.
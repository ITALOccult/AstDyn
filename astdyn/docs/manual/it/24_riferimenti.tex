\chapter{Riferimenti}
\label{ch:references}

\section{Libri di Testo}

\subsection{Meccanica Celeste}

\begin{itemize}
    \item \textbf{Murray \& Dermott} (1999). \textit{Solar System Dynamics}. Cambridge University Press. ISBN: 978-0521575973. Trattazione completa dinamica sistema solare.
    
    \item \textbf{Danby, J.M.A.} (1988). \textit{Fundamentals of Celestial Mechanics}. Willmann-Bell. ISBN: 978-0943396200. Introduzione classica alla meccanica celeste.
    
    \item \textbf{Bate, Mueller, White} (1971). \textit{Fundamentals of Astrodynamics}. Dover Publications. ISBN: 978-0486600610. Testo standard per ingegneria astronautica.
    
    \item \textbf{Valtonen \& Karttunen} (2006). \textit{The Three-Body Problem}. Cambridge University Press. ISBN: 978-0521852241. Trattazione approfondita problema N-corpi.
\end{itemize}

\subsection{Metodi Numerici}

\begin{itemize}
    \item \textbf{Hairer, Norsett, Wanner} (1993). \textit{Solving Ordinary Differential Equations I}. Springer. ISBN: 978-3540566700. Riferimento per integrazione numerica EDO.
    
    \item \textbf{Press et al.} (2007). \textit{Numerical Recipes: The Art of Scientific Computing}, 3rd ed. Cambridge University Press. ISBN: 978-0521880688.
\end{itemize}

\section{Articoli Scientifici}

\subsection{Determinazione Orbitale}

\begin{itemize}
    \item \textbf{Milani \& Gronchi} (2010). \textit{Theory of Orbit Determination}. Cambridge University Press. ISBN: 978-0521873895. Trattazione teorica completa.
    
    \item \textbf{Gauss, C.F.} (1809). \textit{Theoria Motus Corporum Coelestium}. Metodo originale Gauss per IOD.
    
    \item \textbf{Carpino, Milani, Chesley} (2003). "Error statistics of asteroid optical astrometric observations". \textit{Icarus} 166(2):248-270.
\end{itemize}

\subsection{Sistemi Temporali}

\begin{itemize}
    \item \textbf{Seidelmann, P.K.} (1992). \textit{Explanatory Supplement to the Astronomical Almanac}. University Science Books. ISBN: 978-0935702682.
    
    \item \textbf{IAU SOFA} (2021). "Standards of Fundamental Astronomy". \texttt{http://www.iausofa.org/}
\end{itemize}

\section{Software e Strumenti}

\subsection{Software Determinazione Orbitale}

\begin{itemize}
    \item \textbf{OrbFit}: \texttt{http://adams.dm.unipi.it/orbfit/} - Software determinazione orbitale by Milani et al.
    
    \item \textbf{Find\_Orb}: \texttt{https://www.projectpluto.com/find\_orb.htm} - Software IOD e orbit determination by Project Pluto.
    
    \item \textbf{REBOUND}: \texttt{https://rebound.readthedocs.io/} - Framework integratore N-corpi.
\end{itemize}

\subsection{Effemeridi}

\begin{itemize}
    \item \textbf{JPL Horizons}: \texttt{https://ssd.jpl.nasa.gov/horizons/} - Sistema effemeridi online NASA JPL.
    
    \item \textbf{SPICE Toolkit}: \texttt{https://naif.jpl.nasa.gov/naif/toolkit.html} - Libreria NASA per effemeridi e geometria spaziale.
    
    \item \textbf{JPL DE440/DE441}: Effemeridi planetarie ad alta precisione (2021).
\end{itemize}

\subsection{Osservazioni}

\begin{itemize}
    \item \textbf{Minor Planet Center}: \texttt{https://minorplanetcenter.net/} - Database centrale osservazioni asteroidi.
    
    \item \textbf{AstDyS}: \texttt{https://newton.spacedys.com/astdys/} - Sistema dinamica asteroidi Pisa.
\end{itemize}

\section{Standard e Convenzioni}

\begin{itemize}
    \item \textbf{IAU Resolutions}: Risoluzioni International Astronomical Union per sistemi riferimento e costanti.
    
    \item \textbf{IERS Conventions} (2010). "IERS Conventions (2010)". IERS Technical Note 36.
    
    \item \textbf{MPC Observation Format}: Formato standard 80-colonne per osservazioni astrometriche.
\end{itemize}

\section{Risorse Online}

\begin{itemize}
    \item \textbf{AstDyn Documentation}: \texttt{https://github.com/user/astdyn} - Documentazione completa e esempi.
    
    \item \textbf{Eigen Library}: \texttt{https://eigen.tuxfamily.org/} - Libreria algebra lineare C++.
    
    \item \textbf{Boost Libraries}: \texttt{https://www.boost.org/} - Librerie utility C++.
\end{itemize}
